% \CheckSum{284}
% \iffalse
% ======================================================================
% scrtime.dtx 
% Copyright (c) Markus Kohm, 1995-2011
%
% This file is part of the LaTeX2e KOMA-Script bundle.
%
% This work may be distributed and/or modified under the conditions of
% the LaTeX Project Public License, version 1.3c of the license.
% The latest version of this license is in
%   http://www.latex-project.org/lppl.txt
% and version 1.3c or later is part of all distributions of LaTeX 
% version 2005/12/01 or later and of this work.
%
% This work has the LPPL maintenance status "author-maintained".
%
% The Current Maintainer and author of this work is Markus Kohm.
%
% This work consists of all files listed in manifest.txt.
% ----------------------------------------------------------------------
% scrtime.dtx
% Copyright (c) Markus Kohm, 1995-2011
%
% Dieses Werk darf nach den Bedingungen der LaTeX Project Public Lizenz,
% Version 1.3c, verteilt und/oder veraendert werden.
% Die neuste Version dieser Lizenz ist
%   http://www.latex-project.org/lppl.txt
% und Version 1.3c ist Teil aller Verteilungen von LaTeX
% Version 2005/12/01 oder spaeter und dieses Werks.
%
% Dieses Werk hat den LPPL-Verwaltungs-Status "author-maintained"
% (allein durch den Autor verwaltet).
%
% Der Aktuelle Verwalter und Autor dieses Werkes ist Markus Kohm.
% 
% Dieses Werk besteht aus den in manifest.txt aufgefuehrten Dateien.
% ======================================================================
% \fi
%
% \CharacterTable
%  {Upper-case    \A\B\C\D\E\F\G\H\I\J\K\L\M\N\O\P\Q\R\S\T\U\V\W\X\Y\Z
%   Lower-case    \a\b\c\d\e\f\g\h\i\j\k\l\m\n\o\p\q\r\s\t\u\v\w\x\y\z
%   Digits        \0\1\2\3\4\5\6\7\8\9
%   Exclamation   \!     Double quote  \"     Hash (number) \#
%   Dollar        \$     Percent       \%     Ampersand     \&
%   Acute accent  \'     Left paren    \(     Right paren   \)
%   Asterisk      \*     Plus          \+     Comma         \,
%   Minus         \-     Point         \.     Solidus       \/
%   Colon         \:     Semicolon     \;     Less than     \<
%   Equals        \=     Greater than  \>     Question mark \?
%   Commercial at \@     Left bracket  \[     Backslash     \\
%   Right bracket \]     Circumflex    \^     Underscore    \_
%   Grave accent  \`     Left brace    \{     Vertical bar  \|
%   Right brace   \}     Tilde         \~}
%
% \iffalse
%%% From File: scrtime.dtx
%<*driver>
% \fi
\ProvidesFile{scrtime.dtx}[2011/02/22 v3.08b KOMA-Script 
  (packages scrtime and scrdate)]
% \iffalse
%</driver>
%<package&identify>\NeedsTeXFormat{LaTeX2e}[1995/12/01]
%<package&identify&scrtime>\ProvidesPackage{scrtime}[%
%<package&identify&scrdate>\ProvidesPackage{scrdate}[%
%<package&identify>  \KOMAScriptVersion\space package (%
%<package&identify&scrtime>  time of LaTeX run)]
%<package&identify&scrdate> day of the week)]
%<*driver>
\documentclass{scrdoc}
\usepackage[german,english]{babel}
\usepackage[latin1]{inputenc}
\CodelineIndex
\RecordChanges
\GetFileInfo{scrtime.dtx}
\title{The \textsf{KOMA}-timedate-bundle\thanks{This file has
    version number \fileversion, last revised \filedate.}}
\author{Markus Kohm}
\date{\filedate}
\begin{document}
  \maketitle
  \begin{abstract}
    This bundle includes a package \texttt{scrtime} defining some macros to
    handle compilation-time.  It's a very simple implementation similar to
    \texttt{time.sty}. I've tried to use no additional register.

    The second package \texttt{scrdate} defines some macros to handle the name
    of the day!
  \end{abstract}
  \tableofcontents
  \DocInput{scrtime.dtx}
\end{document}
%</driver>
% \fi
%
% \tableofcontents
%
% \section{Introduction}
%
% See the \KOMAScript{} guide for informations about how to use the package.
%
% \StopEventually{\PrintIndex\PrintChanges}
%
% \section{Implementation}
%
% \subsection{Options}
%
% \iffalse
%<*option>
% \fi
% Since version~1.2 both packages use \textsf{scrkbase} for options and
% additional features.
%
%
% \subsubsection{Options of \textsf{scrtime}}
%
% \iffalse
%<*scrtime>
% \fi
%
% \begin{option}{12h}
%   \changes{v1.1b}{1995/02/15}{Option \texttt{12h} added.}
%   \changes{v1.2}{2010/03/10}{Option uses \textsf{scrkbase}.}
% \begin{option}{24h}
%   \changes{v1.1b}{1995/02/15}{Option \texttt{24h} added.}
%   \changes{v1.2}{2010/03/10}{Option is deprecated.}
% \begin{macro}{\if@Hxii}
%   \changes{v1.1b}{1995/02/15}{new switch}
% There are two the two Options |24h| and |12h|. We need a switch to
% distinguish.
%    \begin{macrocode}
\newif\if@Hxii
%    \end{macrocode}
% \end{macro}
%
% So the Options are simple.
%    \begin{macrocode}
\KOMA@ifkey{12h}{@Hxii}
\KOMA@DeclareDeprecatedOption[scrtime]{24h}{12h=false}
%    \end{macrocode}
% \end{option}
% \end{option}
%
% Currently only \textsf{scrtime} uses options, so only \textsf{scrtime} needs
% to process them.
%    \begin{macrocode}
\KOMAProcessOptions\relax
%    \end{macrocode}
%
% \iffalse
%</scrtime>
%</option>
% \fi
%
% \subsection{Macros}
%
% \iffalse
%<*body>
% \fi
%
% \subsubsection{Macros of \textsf{scrtime}}
% \iffalse
%<*scrtime>
% \fi
% We use |\@tempcnta| and |\@tempcntb| but we know, that this is not
% a very good idea.
%
%  \begin{macro}{\thistime}
% \changes{v1.1b}{1995/02/15}{{\cmd\thistime*} added.}
% \changes{v1.1b}{1995/02/15}{{\cmd\thistime} works.}
% First we have to decide, is it a star-version ore not.
%    \begin{macrocode}
\def\thistime{%
  \@ifstar
    {\let\@tempif\iffalse\@thistime}
    {\let\@tempif\iftrue\@thistime}}
%    \end{macrocode}
% Know we have to calculate the hours and minutes. |\@tempcnta| are the
% hours and |\@tempcntb| are the minutes.
%    \begin{macrocode}
\newcommand*{\@thistime}[1][:]{%
  \begingroup
    \@tempcnta\time\divide\@tempcnta60\multiply\@tempcnta60
    \@tempcntb\time\advance\@tempcntb-\@tempcnta
    \divide\@tempcnta60
%    \end{macrocode}
% If we use 12h-mode, we have to cut the houres.
% \changes{v1.1d}{1996/01/14}{Space added at \cs{@thistime} between -12
%                             and \cs{fi} (Martin Schroeder).}
%    \begin{macrocode}
    \if@Hxii\ifnum\@tempcnta>12 \advance\@tempcnta-12 \fi\fi
%    \end{macrocode}
% Know we have to compose the value. If the minutes are less than 10
% maybe there has to be an additional 0.
%    \begin{macrocode}
    \the\@tempcnta{#1}\@tempif\ifnum\@tempcntb<10 0\fi\fi\the\@tempcntb%
  \endgroup}
%    \end{macrocode}
%  \end{macro}
%
%  \begin{macro}{\settime}
% \changes{v1.1b}{1995/02/15}{{\cmd\settime} redefines {\cmd\@thistime}}
% \changes{v1.1c}{1995/05/24}{missing macrocode-environment inserted}
% We simply have to set |\thistime|.
%    \begin{macrocode}
\newcommand*{\settime}[1]{\renewcommand*{\@thistime}[1][]{#1}}
%    \end{macrocode}
%  \end{macro}
%
%
% \iffalse
%</scrtime>
% \fi
%
%
% \subsubsection{Macros of \textsf{scrdate}}
%
% \iffalse
%<*scrdate>
% \fi
%
% \changes{v1.1a}{1995/02/12}{Changed simply all but the user-interface.}
% \changes{v3.05a}{2010/03/10}{Changed simply everything.}
% Since version~3.05a \textsf{scrdate} was rewritten.  First step was to make
% more macros full expandable to provide \cs{MakeUppercase} and
% \cs{MakeLowercase}. Second was to extend the user interface by some new
% functionality.
%
% \begin{macro}{\CenturyPart}
%   \changes{v3.05a}{2010/03/10}{New}
% This is the century part of a year number and so only a shortcut to
% |\XdivY{...}{100}|, that is defined at \textsf{scrbase}.
%    \begin{macrocode}
\newcommand*{\CenturyPart}[1]{\XdivY{#1}{100}}
%    \end{macrocode}
% \end{macro}
%
% \begin{macro}{\DecadePart}
%   \changes{v3.05a}{2010/03/10}{New}
% This is the year number withoud the century part and therefrso only a
% shortcut to |\XmodY{...}{100}|, that is defined at \textsf{scrbase}.
%    \begin{macrocode}
\newcommand*{\DecadePart}[1]{\XmodY{#1}{100}}
%    \end{macrocode}
% \end{macro}
%
% \begin{macro}{\@GaussYear}
%   \changes{v3.05a}{2010/03/10}{New (internal)}
% At the Gauss calculation of the day of the week January and February relates
% to the year before. This macro does the correction for any date.
%    \begin{macrocode}
\newcommand*{\@GaussYear}[3]{%
  \ifcase #2
    \PackageError{scrdate}{month out of range}{%
      You've asked for the Gauss year of ISO date #1-#2-#3,\MessageBreak
      this means, that month hat invalid value '#2'.}%
  \or
    \numexpr #1 - 1\relax
  \or
    \numexpr #1 - 1\relax
  \else
    #1
  \fi
}
%    \end{macrocode}
% \end{macro}
%
% \begin{macro}{\DayNumber}
%   \changes{v3.05a}{2010/03/10}{New}
% Returns the numerical value of the day of week. Note, that Sunday is 0,
% Monday is 1, \dots, Saturday is 6. We use the Gauss calculation of the day
% of the week. First argument is the year, second the month and last the day
% of the month.
%    \begin{macrocode}
\newcommand*{\DayNumber}[3]{%
  \XmodY{%
    \numexpr #3 
           + \ifcase #2
               \PackageError{scrdate}{month out of range}{%
                 You've asked for the dayname of ISO date #1-#2-#3,\MessageBreak
                 this means, that month hat invalid value '#2'.}%
             \or 28 \or 31 \or 2 \or 5 \or 7 \or 10 \or 12 \or 15 \or 18 
             \or 20 \or 23 \or 25
             \else
               \PackageError{scrdate}{month out of range}{%
                 You've asked for the dayname of ISO date #1-#2-#3,\MessageBreak
                 this means, that month hat invalid value '#2'.}%
             \fi
           + \DecadePart{\@GaussYear{#1}{#2}{#3}}
           + \XdivY{\DecadePart{\@GaussYear{#1}{#2}{#3}}}{4}
           + \XdivY{\CenturyPart{\@GaussYear{#1}{#2}{#3}}}{4}
           - 2 * \CenturyPart{\@GaussYear{#1}{#2}{#3}} \relax
  }{7}%
}
%    \end{macrocode}
% \end{macro}
% \begin{macro}{\ISODayNumber}
%   \changes{v3.05a}{2010/03/10}{New}
% Das gleiche wie \cs{DayNumber} aber mit einem ISO-Datum als Argument.
%    \begin{macrocode}
\newcommand*{\ISODayNumber}[1]{\expandafter\@IsoDayNumber#1\@nil}
%    \end{macrocode}
% \begin{macro}{\@IsoDayNumber}
%   \changes{v3.05a}{2010/03/10}{New (internal)}
%    \begin{macrocode}
\newcommand*{\@IsoDayNumber}{}
\def\@IsoDayNumber#1-#2-#3\@nil{\DayNumber{#1}{#2}{#3}}
%    \end{macrocode}
% \end{macro}
% \end{macro}
%
% \begin{macro}{\DayName}
%   \changes{v3.05a}{2010/03/10}{New}
% Returns the name of the day of the week. Arguments like \cs{DayNumber}.
%    \begin{macrocode}
\newcommand*{\DayName}[3]{\@dayname{\DayNumber{#1}{#2}{#3}}}
%    \end{macrocode}
% \end{macro}
% \begin{macro}{\ISODayName}
%   \changes{v3.05a}{2010/03/10}{New}
% Das gleiche wie \cs{DayName} aber mit einem ISO-Datum als Argument.
%    \begin{macrocode}
\newcommand*{\ISODayName}[1]{\@dayname{\ISODayNumber{#1}}}
%    \end{macrocode}
% \end{macro}
%
% \begin{macro}{\DayNameByNumber}
%   \changes{v3.05a}{2010/03/10}{New}
% Returns the name of the day of the week. The argument is a number that will
% be transposed to the range 0..6.
%    \begin{macrocode}
\newcommand*{\DayNameByNumber}[1]{%
  \@dayname{\XmodY{#1}{7}}%
}
%    \end{macrocode}
% \end{macro}
%
% \begin{macro}{\ISOToday}
%   \changes{v3.05a}{2010/03/10}{New}
% Returns the ISO date.
%    \begin{macrocode}
\newcommand*{\ISOToday}{%
  \the\year-\ifnum \month<10 0\fi\the\month-\ifnum \day<10 0\fi\the\day%
}
%    \end{macrocode}
% \end{macro}
%
% \begin{macro}{\IsoToday}
%   \changes{v3.05a}{2010/03/10}{New}
% Returns the ISO date.
%    \begin{macrocode}
\newcommand*{\IsoToday}{%
  \the\year-\the\month-\the\day%
}
%    \end{macrocode}
% \end{macro}
%
% \begin{macro}{\todaysname}
%   \changes{v3.05a}{2010/03/10}{Rewritten}
% Using \cs{DayName} this is very, very simple.
%    \begin{macrocode}
\newcommand*{\todaysname}{\DayName{\year}{\month}{\day}}
%    \end{macrocode}
% \end{macro}
%
% \begin{macro}{\todaysnumber}
%   \changes{v3.05a}{2010/03/11}{New}
% Using \cs{DayNumber} this is very, very simple.
%    \begin{macrocode}
\newcommand*{\todaysnumber}{\DayNumber{\year}{\month}{\day}}
%    \end{macrocode}
% \end{macro}
%
% \begin{macro}{\nameday}
%   \changes{v3.05a}{2010/03/10}{Not longer \cs{long}}
% We simply have to redefine |\todaysname|
%    \begin{macrocode}
\newcommand\nameday[1]{\renewcommand*{\todaysname}{#1}}
%    \end{macrocode}
% \end{macro}
%
% \begin{macro}{\newdaylanguage}
%   \changes{v3.05a}{2010/03/10}{Sunday is 0}
% We write a macro to define the name of the days.
%   \begin{macro}{\scrdate@languagenamewarning}
% But before this, we have to define a once only warning.
%    \begin{macrocode}
\newcommand*\scrdate@languagenamewarning{%
  \PackageWarningNoLine{scrdate}
    {\string\languagename\space not
     defined, using \string\language.\MessageBreak
     This may result in use of wrong language!\MessageBreak
     You should use a compatible language
     package\MessageBreak
     (e.g. `Babel', `german', `french', ...)}%
  \let\scrdate@languagenamewarning\relax}
%    \end{macrocode}
%  \end{macro}
%    \begin{macrocode}
\newcommand\newdaylanguage[8]{%
%    \end{macrocode}
% First we check, if the language is defined at the format, the user uses.
% If it is not defined, we do not define the name of the days and warn.
%    \begin{macrocode}
  \scr@ifundefinedorrelax{l@#1}{%
    \PackageInfo{scrdate}{Language #1\space not defined.\MessageBreak
                          \protect\dayname@#1\space skipped!}%
%    \end{macrocode}
% \changes{v1.1c}{1995/05/24}{missing \cs{end\{macrocode\}} added.}
% If it is defined, we define the name-selection-macro
% |\dayname@|\emph{language}.
% First we define the new macro |\dayname@|\emph{language}:
% \changes{v3.05a}{2010/03/10}{Group removed.}
%    \begin{macrocode}
  }{%
    \@namedef{dayname@#1}##1{%
        \ifcase ##1
          #8\or #2\or #3\or #4\or #5\or #6\or #7\fi%
    }%
%    \end{macrocode}
% Then we define, what to do at |\begin{document}|:
%    \begin{macrocode}
    \AtBeginDocument{%
%    \end{macrocode}
% There we first have to test, if |\date|\emph{language} is defined
% (e.g. using |german.sty|). If not, we have to warn once more.
%    \begin{macrocode}
      \scr@ifundefinedorrelax{date#1}{%
        \PackageWarningNoLine{scrdate}
                             {\protect\date#1\space not defined.\MessageBreak
                              \protect\todaysname maybe can't use
                              \protect\dayname@#1!}%
%    \end{macrocode}
% But if it is defined, we can use it
%    \begin{macrocode}
      }{%
%    \end{macrocode}
% There we first save |\date|\emph{language} as |\D@date|\emph{language}.
% This is a bit tricky, but I think, you'll understand it.
%    \begin{macrocode}
        \expandafter\let\csname D@date#1\expandafter\endcsname
                        \csname date#1\endcsname
%    \end{macrocode}
% Now we redefine |\date|\emph{language}. It first defines |\@dayname| and
% then calls saved macro. If you've understand the definition before, you'll
% understand this tricky one, too.
%    \begin{macrocode}
        \@namedef{date#1}{%
          \expandafter\let\expandafter\@dayname\csname dayname@#1\endcsname
          \@nameuse{D@date#1}}%
%    \end{macrocode}
% Last we have to select this new |\date|\emph{language}.
% \changes{v1.1j}{2000/01/20}{use of \cs{languagename} if defined}
%    \begin{macrocode}
        \@ifundefined{languagename}{
          \scrdate@languagenamewarning
          \ifnum\language=\@nameuse{l@#1}
            \@nameuse{date#1}%
          \fi}{%
          \@ifundefined{date\languagename}%
            {}%
            {\@nameuse{date\languagename}}%
        }%
      }%
    }%
  }%
}
%    \end{macrocode}
% \end{macro}
%
%  \begin{macro}{\@dayname}
% This should be named selecting the language. Since I changed the definitions
% |german.sty| and equal may be loaded before or after |scrdate.sty| or not.
%
% First we define the usual languages using |\newdaylanguage|:
%  \begin{macro}{\dayname@german}
%    \begin{macrocode}
\newdaylanguage{german}{Montag}{Dienstag}{Mittwoch}
               {Donnerstag}{Freitag}{Samstag}{Sonntag}
%    \end{macrocode}
%  \end{macro}
%  \begin{macro}{\dayname@ngerman}
% \changes{v1.1i}{1999/10/09}{new language ``ngerman''.}
%    \begin{macrocode}
\newdaylanguage{ngerman}{Montag}{Dienstag}{Mittwoch}
               {Donnerstag}{Freitag}{Samstag}{Sonntag}
%    \end{macrocode}
%  \end{macro}
%  \begin{macro}{\dayname@naustrian}
% \changes{v3.08b}{2011/02/22}{new language ``naustrian''.}
%    \begin{macrocode}
\newdaylanguage{naustrian}{Montag}{Dienstag}{Mittwoch}
               {Donnerstag}{Freitag}{Samstag}{Sonntag}
%    \end{macrocode}
%  \end{macro}
%  \begin{macro}{\dayname@austrian}
% \changes{v3.08b}{2011/02/22}{new language ``austrian''.}
%    \begin{macrocode}
\newdaylanguage{austrian}{Montag}{Dienstag}{Mittwoch}
               {Donnerstag}{Freitag}{Samstag}{Sonntag}
%    \end{macrocode}
%  \end{macro}
%  \begin{macro}{\dayname@english}
% \changes{v1.1g}{1997/06/21}{correct name is ``tuesday''.}
%    \begin{macrocode}
\newdaylanguage{english}{Monday}{Tuesday}{Wednesday}
               {Thursday}{Friday}{Saturday}{Sunday}
%    \end{macrocode}
%  \end{macro}
%  \begin{macro}{\dayname@USenglish}
% \changes{v1.1g}{1997/06/21}{correct name is ``tuesday''.}
%    \begin{macrocode}
\newdaylanguage{USenglish}{Monday}{Tuesday}{Wednesday}
               {Thursday}{Friday}{Saturday}{Sunday}
%    \end{macrocode}
%  \end{macro}
%  \begin{macro}{\dayname@french}
%    \begin{macrocode}
\newdaylanguage{french}{Lundi}{Mardi}{Mercredi}
               {Jeudi}{Vendredi}{Samedi}{Dimanche}
%    \end{macrocode}
%  \end{macro}
%  \begin{macro}{\dayname@italian}
% \changes{v1.1f}{1997/06/06}{New (thanks to Lorenzo M.\ Catucci)}
%    \begin{macrocode}
\newdaylanguage{italian}{Luned\`\i}{Marted\`\i}{Mercoled\`\i}
               {Gioved\`\i}{Venerd\`\i}{Sabato}{Domenica}
%    \end{macrocode}
%  \end{macro}
%  \begin{macro}{\dayname@spanish}
% \changes{v1.1h}{1997/07/26}{New (thanks to Ralph J.\ Hangleiter)}
%    \begin{macrocode}
\newdaylanguage{spanish}{Lunes}{Martes}{Mi\'ercoles}
               {Jueves}{Viernes}{S\'abado}{Domingo}
%    \end{macrocode}
%  \end{macro}
%  \begin{macro}{\dayname@croatian}
% \changes{v1.1l}{2001/10/05}{New (thanks to Branka Lon\v{c}arevi\'{c})}
%  \begin{macrocode}
\newdaylanguage{croatian}{Ponedjeljak}{Utorak}{Srijeda}
               {\v{C}etvrtak}{Petak}{Subota}{Nedjelja}
%    \end{macrocode}
%  \end{macro}
%  \begin{macro}{\dayname@dutch}
%    \changes{v1.1m}{2002/02/02}{New (thanks to Henk Jongbloets)}
%    \changes{v1.1p}{2009/01/01}{fixed to upper case}
%    \begin{macrocode}
\newdaylanguage{dutch}{Maandag}{Dinsdag}{Woensdag}
               {Donderdag}{Vrijdag}{Zaterdag}{Zondag}
%    \end{macrocode}
%  \end{macro}
%
%  \begin{macro}{\dayname@finnish}
%    \changes{v1.1n}{2005/02/07}{New (thanks to Hannu V\"ais\"anen)}
%    \begin{macrocode}
\newdaylanguage{finnish}{Maanantai}{Tiistai}{Keskiviikko}
               {Torstai}{Perjantai}{Lauantai}{Sunnuntai}
%    \end{macrocode}
%  \end{macro}
%
% \begin{macro}{\dayname@norsk}
%   \changes{v1.1p}{2009/01/01}{New (thank to Sveinung Heggen)}
%    \begin{macrocode}
\newdaylanguage{norsk}{Mandag}{Tirsdag}{Onsdag}
               {Torsdag}{Fredag}{L\o{}rdag}{S\o{}ndag}
%    \end{macrocode}
% \end{macro}
%
% \begin{macro}{\dayname@danish}
%   \changes{v3.08}{2011/01/18}{New (thanks to Benjamin Hell)}
%    \begin{macrocode}
\newdaylanguage{danish}{Mandag}{Tirsdag}{Onsdag}
               {Torsdag}{Fredag}{L\o{}rdag}{S\o{}ndag}
%    \end{macrocode}
% \end{macro}
%
% \begin{macro}{\dayname@swedish}
%   \changes{v3.08}{2011/01/18}{New (thanks to Benjamin Hell)}
%    \begin{macrocode}
\newdaylanguage{swedish}{M\aa{}ndag}{Tisdag}{Onsdag}
               {Torsdag}{Fredag}{L\"ordag}{S\"ondag}
%    \end{macrocode}
% \end{macro}
%
% If there are no language-definitions, we simply want the US-english names
% of the days.
% \changes{v1.1g}{1997/06/21}{correct name is ``tuesday''.}
%    \begin{macrocode}
\def\@dayname#1{%
  \ifcase #1
    Monday\or Tuesday\or Wednesday\or Thursday\or
    Friday\or Saturday\or Sunday\fi%
}
%    \end{macrocode}
%  \end{macro}
%
% Last but not least file \texttt{scrdate.cfg} has to be included,
% if it exists.
%    \begin{macrocode}
\InputIfFileExists{scrdate.cfg}
           {\typeout{*************************************^^J%
                     * Local config file scrdate.cfg used^^J%
                     *************************************}}
           {}
%    \end{macrocode}
%
% \iffalse
%</scrdate>
%</body>
% \fi
%
% \Finale
%
\endinput
%
% Ende der Datei `scrtime.dtx'
%%% Local Variables:
%%% mode: doctex
%%% TeX-master: t
%%% End:
