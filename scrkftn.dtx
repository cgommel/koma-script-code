% \CheckSum{111}
% \iffalse meta-comment
% ======================================================================
% scrkftn.dtx
% Copyright (c) Markus Kohm, 2002-2006
%
% This file is part of the LaTeX2e KOMA-Script bundle.
%
% This work may be distributed and/or modified under the conditions of
% the LaTeX Project Public License, version 1.3b of the license.
% The latest version of this license is in
%   http://www.latex-project.org/lppl.txt
% and version 1.3b or later is part of all distributions of LaTeX 
% version 2005/12/01 or later and of this work.
%
% This work has the LPPL maintenance status "author-maintained".
%
% The Current Maintainer and author of this work is Markus Kohm.
%
% This work consists of all files listed in manifest.txt.
% ----------------------------------------------------------------------
% scrkftn.dtx
% Copyright (c) Markus Kohm, 2002-2006
%
% Dieses Werk darf nach den Bedingungen der LaTeX Project Public Lizenz,
% Version 1.3b, verteilt und/oder veraendert werden.
% Die neuste Version dieser Lizenz ist
%   http://www.latex-project.org/lppl.txt
% und Version 1.3b ist Teil aller Verteilungen von LaTeX
% Version 2005/12/01 oder spaeter und dieses Werks.
%
% Dieses Werk hat den LPPL-Verwaltungs-Status "author-maintained"
% (allein durch den Autor verwaltet).
%
% Der Aktuelle Verwalter und Autor dieses Werkes ist Markus Kohm.
% 
% Dieses Werk besteht aus den in manifest.txt aufgefuehrten Dateien.
% ======================================================================
% \fi
%
% \CharacterTable
%  {Upper-case    \A\B\C\D\E\F\G\H\I\J\K\L\M\N\O\P\Q\R\S\T\U\V\W\X\Y\Z
%   Lower-case    \a\b\c\d\e\f\g\h\i\j\k\l\m\n\o\p\q\r\s\t\u\v\w\x\y\z
%   Digits        \0\1\2\3\4\5\6\7\8\9
%   Exclamation   \!     Double quote  \"     Hash (number) \#
%   Dollar        \$     Percent       \%     Ampersand     \&
%   Acute accent  \'     Left paren    \(     Right paren   \)
%   Asterisk      \*     Plus          \+     Comma         \,
%   Minus         \-     Point         \.     Solidus       \/
%   Colon         \:     Semicolon     \;     Less than     \<
%   Equals        \=     Greater than  \>     Question mark \?
%   Commercial at \@     Left bracket  \[     Backslash     \\
%   Right bracket \]     Circumflex    \^     Underscore    \_
%   Grave accent  \`     Left brace    \{     Vertical bar  \|
%   Right brace   \}     Tilde         \~}
%
% \iffalse
%%% From File: scrkftn.dtx
%<*driver>
% \fi
\ProvidesFile{scrkftn.dtx}[2006/03/28 v2.95 KOMA-Script (footnotes)]
% \iffalse
\documentclass[halfparskip-]{scrdoc}
\usepackage[english,ngerman]{babel}
\usepackage[latin1]{inputenc}
\CodelineIndex
\RecordChanges
\GetFileInfo{scrkftn.dtx}
\title{\KOMAScript{} \partname\ \texttt{\filename}%
  \footnote{Dies ist Version \fileversion\ von Datei
    \texttt{\filename}.}}
\date{\filedate}
\author{Markus Kohm}

\begin{document}
  \maketitle
  \tableofcontents
  \DocInput{\filename}
\end{document}
%</driver>
% \fi
%
% \selectlanguage{ngerman}
%
% \changes{v2.95}{2002/07/01}{%
%   erste Version aus der Aufteilung von \texttt{scrclass.dtx}}
%
% \section{Fu�noten}
%
% Die Fu�notengestaltung in \KOMAScript{} basiert auf einem Hinweis
% aus dem dokumentierten \LaTeX-Kern.
%
% \StopEventually{\PrintIndex\PrintChanges}
%
% \iffalse
%<*option>
% \fi
%
% \subsection{Option}
% Die Fu�noten werden nicht unmittelbar von Optionen beeinflusst.
%
%
% \iffalse
%</option>
%<*body>
% \fi
%
% \subsection{Definitionen f�r Fu�noten}
%
%  \begin{macro}{\footnoterule}
%    \changes{v2.3a}{1995/07/08}{\cs{@width} verwendet}
%    \changes{v2.8q}{2002/02/06}{\cs{raggedbottom}-Verbesserung
%      eingef�gt}
%    \changes{v2.95}{2002/01/07}{\texttt{.05fil} statt
%      \texttt{.005fil}}
% Im Fall, dass \cs{raggedbottom} verwendet wird, die Fu�notenlinie
% nach unten gedr�ckt. Dies funktioniert nur dann nicht, wenn
% Abbildungen oder Tabellen mit der Option "`\texttt{b}"' an das Ende
% der Seite gesetzt werden. \cs{raggedbottom} wird daran erkannt, dass
% \cs{@textbottom} nicht \cs{relax} ist. Das kann nat�rlich auch
% einmal schief gehen, ist dann aber auch nicht ganz so tragisch.
%    \begin{macrocode}
\renewcommand*\footnoterule{%
  \kern-3\p@
  \ifx\@textbottom\relax\else\vskip \z@ \@plus.05fil\fi
  \hrule\@width.4\columnwidth
  \kern 2.6\p@}
%    \end{macrocode}
%  \end{macro}
%
%  \begin{Counter}{footnote}
% Der Fu�notenz�hler wird mit \cs{chapter} zur�ckgesetzt, die Fu�noten
% werden also kapitelweise nummeriert.
%    \begin{macrocode}
%<book|report>\@addtoreset{footnote}{chapter}
%    \end{macrocode}
%  \end{Counter}
%
%  \begin{macro}{\deffootnote}
%    \changes{v2.4l}{1997/02/06}{neu}
%    \changes{v2.95}{2002/07/09}{Absatzeinzug korrigiert}
%    \changes{v2.9l}{2003/01/28}{\cs{edef}\cs{@tempa} ersetzt}
%    \changes{v2.9q}{2004/01/31}{\cs{ftn@font} erlaubt aus Gr"unden der
%      Kompatibilit�t mit dem \textsl{footnote} Paket kein Argument mehr}
% Dieses Makro zur Definition der Gestalt von Fu�noten erlaubt einen
% optionalen und erwartet drei weitere Parameter. Der erste, optionale
% gibt den Einzug der ersten Zeile des Fu�notentextes vom linken Rand
% an. Die Fu�notenmarkierungen werden rechtsb�ndig in diesen Einzug
% gesetzt. Der zweite, also erste nicht optionale Parameter gibt den
% Einzug jeder weiteren Zeile des Fu�notentextes vom linken Rand
% an. Fehlt der optionale Parameter so ist er gleich diesem. Der
% dritte, also zweite nicht optionale Parameter gibt den zus�tzlichen
% Einzug jedes weiteren Absatzes einer Fu�note an. Der letzte
% Parameter schlie�lich bestimmt die Ausgabe der Fu�notenmarkierung in
% der Fu�note. Diese wird zus�tzlich in eine \cs{hbox} gesetzt.
%    \begin{macrocode}
\newcommand\deffootnote[4][]{%
  \long\def\@makefntext##1{%
    \setlength{\@tempdimc}{#3}%
    \def\@tempa{#1}\ifx\@tempa\@empty
      \@setpar{\@@par
        \@tempdima = \hsize
        \addtolength{\@tempdima}{-#2}%
        \parshape \@ne #2 \@tempdima}%
    \else
      \addtolength{\@tempdimc}{#2}%
      \addtolength{\@tempdimc}{-#1}%
      \@setpar{\@@par
        \@tempdima = \hsize
        \addtolength{\@tempdima}{-#1}%
        \@tempdimb = \hsize
        \addtolength{\@tempdimb}{-#2}%
        \parshape \tw@ #1 \@tempdima #2 \@tempdimb
      }%
    \fi
    \par
    \parindent\@tempdimc\noindent
    \ftn@font\hbox to \z@{\hss\@@makefnmark}##1%
  }%
%    \end{macrocode}
%  \begin{macro}{\@@makefnmark}
%    \changes{v2.4l}{1997/02/06}{neu}
% Makro zum Setzen der Fu�notenmarkierung in der Fu�note:
%    \begin{macrocode}
  \def\@@makefnmark{\hbox{\ftnm@font{#4}}}%
%    \end{macrocode}
%  \end{macro}
%    \begin{macrocode}
}
%    \end{macrocode}
%  \end{macro}
%
%  \begin{macro}{\ftn@font}
%    \changes{v2.8q}{2001/11/16}{neu (intern)}
%  \begin{macro}{\scr@fnt@footnote}
%    \changes{v2.8q}{2001/11/16}{neues Element \texttt{footnote}}
%  \begin{macro}{\ftnm@font}
%    \changes{v2.8q}{2001/11/16}{neu (intern)}
%  \begin{macro}{\scr@fnt@footnotenumber}
%    \changes{v2.8q}{2001/11/16}{neues Element
%      \texttt{footnotenumber}}
%  \begin{macro}{\scr@fnt@footnotenlabel}
%    \changes{v2.8q}{2001/11/16}{neues Element \texttt{footnotelabel}}
%  \begin{macro}{\scr@fnt@instead@footnotetext}
%    \changes{v2.8q}{2001/11/16}{neuer Ersatz f�r Element
%    \texttt{footnotetext}}
% Das erste Element gibt die Schrift an, in der die Fu�note gesetzt
% wird. Davon abweichend kann mit dem zweiten Element die Schriftart
% f�r die Fu�notennummer getrennt angegeben werden. Aufgrund der
% Definition kann auch ein Makro verwendet werden, das ein Argument
% erwartet.
%    \begin{macrocode}
\newcommand*{\ftn@font}{\normalfont}
\newcommand*{\scr@fnt@footnote}{\ftn@font}
\newcommand*{\ftnm@font}{}
\newcommand*{\scr@fnt@footnotenumber}{\ftnm@font}
\newcommand*{\scr@fnt@footnotelabel}{\ftnm@font}
\newcommand*{\scr@fnt@instead@footnotetext}{footnote}
%    \end{macrocode}
%  \end{macro}
%  \end{macro}
%  \end{macro}
%  \end{macro}
%  \end{macro}
%  \end{macro}
%
%  \begin{macro}{\deffootnotemark}
%    \changes{v2.4l}{1997/02/06}{neu}
% Makro zur Definition der Fu�notenmarkierung im Text:
%    \begin{macrocode}
\newcommand*\deffootnotemark[1]{%
  \def\@makefnmark{\hbox{\ftntm@font{#1}}}%
}
%    \end{macrocode}
%
%  \begin{macro}{\ftntm@font}
%    \changes{v2.8q}{2001/11/16}{neu (intern)}
%  \begin{macro}{\scr@fnt@footnotereference}
%    \changes{v2.8q}{2001/11/16}{neues Element
%      \texttt{footnotereference}}
%  \begin{macro}{\scr@fnt@footnoteref}
%    \changes{v2.8q}{2001/11/16}{neues Element \texttt{footnoteref}}
% Dies ist die Schriftart der Fu�notenreferenz im Text. Auch hier kann
% ggf. ein Makro verwendet werden, das ein Argument erwartet.
%    \begin{macrocode}
\newcommand*{\ftntm@font}{}
\newcommand*{\scr@fnt@footnotereference}{\ftntm@font}
\newcommand*{\scr@fnt@footnoteref}{\ftntm@font}
%    \end{macrocode}
%  \end{macro}
%  \end{macro}
%  \end{macro}
%  \end{macro}
%
%  \begin{macro}{\thefootnotemark}
%    \changes{v2.4l}{1997/02/06}{neu}
% Makro, damit \cs{@thefnmark} auf Anwenderebene verf�gbar wird:
%    \begin{macrocode}
\newcommand*{\thefootnotemark}{\@thefnmark}
%    \end{macrocode}
%  \end{macro}
%
% \changes{v2.4l}{1997/02/06}{Verwendung der neuen Makros zur
%   Fu�notengestaltgebung}
%  \begin{macro}{\@makefnmark}
%    \begin{macrocode}
\deffootnote[1em]{1.5em}{1em}{\textsuperscript{\thefootnotemark}}
\deffootnotemark{\textsuperscript{\thefootnotemark}}
%    \end{macrocode}
%  \end{macro}
%
%
% \iffalse
%</body>
% \fi
%
% \Finale
%
\endinput
%
% end of file `scrkftn.dtx'
%%% Local Variables:
%%% mode: doctex
%%% TeX-master: t
%%% End:
