% \CheckSum{365}
% \iffalse meta-comment
% ======================================================================
% scrhack.dtx
% Copyright (c) Markus Kohm, 2009
%
% This file is part of the LaTeX2e KOMA-Script bundle.
%
% This work may be distributed and/or modified under the conditions of
% the LaTeX Project Public License, version 1.3b of the license.
% The latest version of this license is in
%   http://www.latex-project.org/lppl.txt
% and version 1.3b or later is part of all distributions of LaTeX
% version 2005/12/01 or later and of this work.
%
% This work has the LPPL maintenance status "author-maintained".
%
% The Current Maintainer and author of this work is Markus Kohm.
%
% This work consists of all files listed in manifest.txt.
%
% To create `scrhack.sty' run `tex scrhack.dtx'.  Using LaTeX instead
% of TeX would generate the implementation documentation.
% ----------------------------------------------------------------------
% scrhack.dtx
% Copyright (c) Markus Kohm, 2008-2009
%
% Dieses Werk darf nach den Bedingungen der LaTeX Project Public Lizenz,
% Version 1.3b, verteilt und/oder veraendert werden.
% Die neuste Version dieser Lizenz ist
%   http://www.latex-project.org/lppl.txt
% und Version 1.3b ist Teil aller Verteilungen von LaTeX
% Version 2005/12/01 oder spaeter und dieses Werks.
%
% Dieses Werk hat den LPPL-Verwaltungs-Status "author-maintained"
% (allein durch den Autor verwaltet).
%
% Der Aktuelle Verwalter und Autor dieses Werkes ist Markus Kohm.
%
% Dieses Werk besteht aus den in manifest.txt aufgefuehrten Dateien.
%
% `scrhack.sty' kann durch den Aufruf `tex scrhack.dtx' erzeugt 
% werden. Bei Verwendung von LaTeX statt TeX wird hingegen die
% Implementierungsdokumentation erzeugt.
% ======================================================================
% \fi
%
% \CharacterTable
%  {Upper-case    \A\B\C\D\E\F\G\H\I\J\K\L\M\N\O\P\Q\R\S\T\U\V\W\X\Y\Z
%   Lower-case    \a\b\c\d\e\f\g\h\i\j\k\l\m\n\o\p\q\r\s\t\u\v\w\x\y\z
%   Digits        \0\1\2\3\4\5\6\7\8\9
%   Exclamation   \!     Double quote  \"     Hash (number) \#
%   Dollar        \$     Percent       \%     Ampersand     \&
%   Acute accent  \'     Left paren    \(     Right paren   \)
%   Asterisk      \*     Plus          \+     Comma         \,
%   Minus         \-     Point         \.     Solidus       \/
%   Colon         \:     Semicolon     \;     Less than     \<
%   Equals        \=     Greater than  \>     Question mark \?
%   Commercial at \@     Left bracket  \[     Backslash     \\
%   Right bracket \]     Circumflex    \^     Underscore    \_
%   Grave accent  \`     Left brace    \{     Vertical bar  \|
%   Right brace   \}     Tilde         \~}
%
% \iffalse
%%% From File: tocbasic.dtx
%<package&identify>%%% using: package,identify
%<package&option>%%% using: package,option
%<package&body>%%% using: package,body
%<package&identity>\NeedsTeXFormat{LaTeX2e}[1995/06/01]
%<manual>\ProvidesFile{scrhack.tex}
%<*dtx>
\ProvidesFile{scrhack.dtx}
%</dtx>
%<*dtx|manual>
             [2009/03/16 v3.03 KOMA-Script (hacking other packages)]
%</dtx|manual>
%<package&identify>\ProvidesPackage{scrhack}
%<hyperref&identify>\ProvidesFile{hyperref.hck}
%<float&identify>\ProvidesFile{float.hck}
%<identify>                [\KOMAScriptVersion\space
%<package&identify>                  package (hacking other packages)]
%<hack&identify>                  hacking package
%<hyperref&identify>                  hyperref]
%<float&identify>                  float]
%<*dtx>
\documentclass[halfparskip-]{scrdoc}
\usepackage[latin1]{inputenc}
\usepackage[english,ngerman]{babel}
\usepackage[T1]{fontenc}
\usepackage{lmodern}
\CodelineIndex
\RecordChanges
\begin{document}
\GetFileInfo{scrhack.dtx}
\DocInput{scrhack.dtx}
\end{document}
%</dtx>
% \fi
%
%
% \selectlanguage{english}
%
% \newcommand*{\Package}[1]{\textsf{\mbox{#1}}}
% \let\File\Package
% \let\Macro\cs
% \let\PName\meta
% \let\Parameter\marg
% \newcommand*{\PParameter}[1]{\texttt{\{#1\}}}
% \newcommand*{\counter}[1]{\texttt{\mbox{#1}}}
% \newcommand*{\OptionValue}[2]{\texttt{\mbox{#1=}\linebreak[3]\mbox{#2}}}
% \let\PValue\texttt
%
% \title{\KOMAScript{} \partname\ \texttt{\filename}%
%   \thanks{This file is version \fileversion\ of \texttt{\filename}.}}
% \date{\filedate}
% \author{Markus Kohm\thanks{mailto:komascript(at)gmx.info}}
% \maketitle
% \begin{abstract}
% \iffalse
%<*manual|dtx>
%<*manual>
\chapter{Hacks for Third-Party Packages by Package \Package{scrhack}}
\labelbase{scrhack}
%</manual>
% \fi
Some packages from other authors may have problems with \KOMAScript{}.  In my
opinion some packages could be improved. With some packages this makes only
sense, if \KOMAScript{} was used. With some other packages the package author
has another opinion. Sometimes proposals was never answered. Package
\Package{scrhack} contains all those improvement proposals for other
packages. This means, \Package{scrhack} redefines macros of packages from
other authors! The redefinitions are only activated, if those packages were
loaded. Users may prevent \Package{scrhack} from redefining macros of
individual packages.
%\iffalse
%<*dtx>
%\fi
% \end{abstract}
% \tableofcontents
%\iffalse
%</dtx>
%\fi

\section{The \Package{hyperref} hack}
\label{sec:scrhack.hyperref}

Package \Package{hyperref} does behave different at part, chapter, and section
headings that get no number. If they get no number, because of to low counter
\counter{secnumdepth} \Package{hyperref} sets an anchor for links and
bookmarks before the heading. Same would be, if the headings have a
number. But if the headings get no number because of usage of the star version
of the commands, e.g., \Macro{part*}, \Macro{chapter*} or \Macro{section*},
the anchor for links and bookmarks are set after the headings. The anchors for
numbered headings are always set before the headings.

Package \Package{scrhack} redefines some macros of some hyperref driver files,
e.g., \File{hpdftex.def}, while execution of
\Macro{begin}\PParameter{document}. With this redefinitions the anchor of not
numbered headings will be set always before the headings, too.

% \iffalse
\begin{Declaration}
  \KOption{hyperref}{switch}
\end{Declaration}
% \fi
You may switch of the \Package{hyperref} hack loading package
\Package{scrhack} with option \OptionValue{hyperref}{false}. You may also
switch of the \Package{hyperref} hack using
\Macro{KOMAoptions}\PParameter{hyperref=false} or
\Macro{KOMAoption}\PParameter{hyperref}\PParameter{false} somewhere after
loading package \Package{scrhack}, but before
\Macro{begin}\PParameter{document}.

\section{The \Package{float} hack}
\label{sec:scrhack.float}

Package \Package{float} uses macros \Macro{float@listhead} to set the headings
of a float listing and \Macro{float@addtolists} to add informations to all
float listings. These macros where proposed by the \KOMAScript{} author for
some years. In theory those macros may be used by several class and package
authors to deligate some parts of the creation of a float listing to the
class. This would increase the compatiblity of packages and classes. But
unfortunaltly some package authors, even the author of package
\Package{float}, impemented the commands in such a way, that these packages
will become incompatible to each other.

Because of this \KOMAScript\ stopped support for \Macro{float@addtolists} and
\Macro{float@listhead} with version 3. Instead of this \KOMAScript\ supports
several improvements for package authors using \KOMAScript\ package
\Package{tocbasic}.

Package \Package{scrhack} redefines some macros of package \Package{float} to
not longer use \Macro{float@addtolists} and \Macro{float@listhead} but use the
interface of package \Package{tocbasic}. This does not only improve the
compatibility of \KOMAScript\ and package \Package{float}, but also improves
the compatibility of packages \Package{babel} and \Package{float}.

Note: A significant change with \Package{scrhack} is, that \KOMAScript{} options
like \OptionValue{lists=totoc} or \OptionValue{lists=totocnumbered} does only
change the behaviour of float listings, that are already defined using
\Macro{newfloat} \emph{before} using such an option!

% \iffalse
\begin{Declaration}
  \KOption{float}{switch}
\end{Declaration}
% \fi
You may switch of the \Package{float} hack loading package \Package{scrhack}
with option \OptionValue{float}{false}. You may also switch of the
\Package{float} hack using \Macro{KOMAoptions}\PParameter{float=false} or
\Macro{KOMAoption}\PParameter{float}\PParameter{false} somewhere after loading
package \Package{scrhack}, but before loading package \Package{float}.

% \iffalse
%</manual|dtx>
% \fi
%
% \selectlanguage{ngerman}
% \StopEventually{\PrintIndex\PrintChanges}
% \changes{v3.03}{2009/03/12}{erste Version des Pakets}
%
% \section{Implementation of \Package{scrhack}}
%
% \subsection{Optionen}
%
% \iffalse
%<*package&option>
% \fi
%
% Das Paket bedient sich \cs{KOMAoptions} etc. aus \textsf{scrkbase}. 
%
% Per Option kann gew�hlt werden, welche Manipulationen geladen werden
% sollen. Alle diese Optionen k�nnen jedoch nur bis zum Laden des
% entsprechenden Pakets gesetzt werden. Anschlie�end sind sie wirkungslos.
%
% \subsection{Der \textsf{hyperref}-Hack}
%
% \textsf{hyperref} setzt den Anker zu der Stern-Variante einer �berschrift
% hinter die �berschrift, w�hrend es bei der nicht Stern-Variante den Anker
% auch dann vor die �berschrift setzt, wenn die �berschrift aufgrund von
% \texttt{secnumdepth} nicht nummeriert wird. Der Hack setzt den Anker
% einheitlich vor die �berschrift.
%
% \begin{option}{hyperref}
%    \begin{macrocode}
\KOMA@ifkey{hyperref}{@scrhack@hyperref}%
\@scrhack@hyperreftrue
\AtBeginDocument{%
  \KOMA@key[.scrhack.sty]{hyperref}{%
    \PackageWarning{scrhack}{option `hyperref=#1' ignored}%
  }%
  \if@scrhack@hyperref\scr@hack@load\@pkgextension{hyperref}\fi
}
%    \end{macrocode}
% \end{option}
%
% \iffalse
%</package&option>
%<*hyperref&body>
% \fi
%
% \begin{macro}{\@schapter}
% \begin{macro}{\@spart}
% \begin{macro}{\@ssect}
% Eigentlich wird hier gar nicht \texttt{hyperref.sty} ver�ndert, sondern
% diverse Treiberdateien. Sobald das Paket \textsf{hyperref} geladen ist, ist
% auch die passende Treiberdatei geladen und au�erdem sind alle
% Treiberdateien, die entsprechende Definitionen vornehmen, gleicherma�en
% betroffen. Also kann der entsprechende Patch einfach erfolgen, wenn hyperref
% geladen ist (was bereits von \cs{scr@hack@load} getestet wurde). Es muss
% also nur noch sichergestellt werden, dass die umzudefinierenden Macros
% derzeit den erwarteten Inhalt haben.
%    \begin{macrocode}
\scr@ifexpected\@schapter{%
  \def\@schapter#1{%
    \H@old@schapter{#1}%
    \begingroup
      \let\@mkboth\@gobbletwo
      \Hy@GlobalStepCount\Hy@linkcounter
      \xdef\@currentHref{\Hy@chapapp*.\the\Hy@linkcounter}%
      \Hy@raisedlink{%
        \hyper@anchorstart{\@currentHref}\hyper@anchorend
      }%
    \endgroup
  }%
}{%
  \PackageInfo{scrhack}{redefining \string\@schapter}%
  \def\@schapter#1{%
    \begingroup
      \let\@mkboth\@gobbletwo
      \Hy@GlobalStepCount\Hy@linkcounter
      \xdef\@currentHref{\Hy@chapapp*.\the\Hy@linkcounter}%
      \Hy@raisedlink{%
        \hyper@anchorstart{\@currentHref}\hyper@anchorend
      }%
    \endgroup
    \H@old@schapter{#1}%
  }%
}{%
  \scr@ifexpected\@schapter{%
    \def\@schapter#1{%
      \begingroup
        \let\@mkboth\@gobbletwo
        \Hy@GlobalStepCount\Hy@linkcounter
        \xdef\@currentHref{\Hy@chapapp*.\the\Hy@linkcounter}%
        \Hy@raisedlink{%
          \hyper@anchorstart{\@currentHref}\hyper@anchorend
        }%
      \endgroup
      \H@old@schapter{#1}%
    }%
  }{}{%
    \PackageWarningNoLine{scrhack}{unknown \string\@schapter\space
      definition found!\MessageBreak
      Maybe you are using a unsupported hyperref version}%
  }%
}

\scr@ifexpected\@spart{%
  \def\@spart#1{%
    \H@old@spart{#1}%
    \Hy@GlobalStepCount\Hy@linkcounter
    \xdef\@currentHref{part*.\the\Hy@linkcounter}%
    \Hy@raisedlink{%
      \hyper@anchorstart{\@currentHref}\hyper@anchorend
    }%
  }%
}{%
  \PackageInfo{scrhack}{redefining \string\@spart}%
  \def\@spart#1{%
    \Hy@GlobalStepCount\Hy@linkcounter
    \xdef\@currentHref{part*.\the\Hy@linkcounter}%
    \Hy@raisedlink{%
      \hyper@anchorstart{\@currentHref}\hyper@anchorend
    }%
    \H@old@spart{#1}%
  }%
}{%
  \scr@ifexpected\@spart{%
    \def\@spart#1{%
      \Hy@GlobalStepCount\Hy@linkcounter
      \xdef\@currentHref{part*.\the\Hy@linkcounter}%
      \Hy@raisedlink{%
        \hyper@anchorstart{\@currentHref}\hyper@anchorend
      }%
      \H@old@spart{#1}%
    }%
  }{}{%
    \PackageWarningNoLine{scrhack}{unknown \string\@spart\space
      definition found!\MessageBreak
      Maybe you are using a unsupported hyperref version}%
  }%
}

\scr@ifexpected\@ssect{%
  \def\@ssect#1#2#3#4#5{%
    \H@old@ssect{#1}{#2}{#3}{#4}{#5}%
    \phantomsection
  }%
}{%
  \PackageInfo{scrhack}{redefining \string\@ssect}%
  \def\@ssect#1#2#3#4#5{%
    \H@old@ssect{#1}{#2}{#3}{#4}{\phantomsection\ignorespaces#5}%
  }%
}{%
  \scr@ifexpected\@ssect{%
    \def\@ssect#1#2#3#4#5{%
      \H@old@ssect{#1}{#2}{#3}{#4}{\phantomsection\ignorespaces#5}%
    }%
  }{}{%
    \PackageWarningNoLine{scrhack}{unknown \string\@ssect\space
      definition found!\MessageBreak
      Maybe you are using a unsupported hyperref version}%
  }%
}
%    \end{macrocode}
% \end{macro}
% \end{macro}
% \end{macro}
%
% \iffalse
%</hyperref&body>
%<*package&option>
% \fi
%
% \subsection{Der \textsf{float}-Hack}
%
% Das \textsf{float}-Paket verwendet das Makro \cs{float@listhead} zum
% Setzen der �berschriften. Dies wird seit \KOMAScript~3 nicht mehr empfohlen
% und fliegt demn�chst komplett aus der Unterst�tzung. Stattdessen wird
% empfohlen, dass Pakete \textsf{tocbasic} unterst�tzen. Der Aufwand daf�r ist
% sehr gering und wird mit vielen neuen M�glichkeiten belohnt.
%
% Dieser Hack r�stet die \textsf{tocbasic}-Unterst�tzung f�r \textsf{float}
% nach.
%
% \begin{option}{float}
%    \begin{macrocode}
\KOMA@ifkey{float}{@scrhack@float}%
\@scrhack@floattrue
\AfterPackage*{float}{%
  \KOMA@key[.scrhack.sty]{float}{%
    \PackageWarning{float}{option `float' ignored}%
  }%
  \if@scrhack@float\scr@hack@load\@pkgextension{float}\fi
}
%    \end{macrocode}
% \end{option}
%
% \iffalse
%</package&option>
%<*float&body>
% \fi
%
% \begin{macro}{\newfloat}
% �ber die Anweisung \cs{newfloat} wird eine neue Gleitumgebung
% definiert. Hier muss die neue Erweiterung aus dem dritten Argument
% \textsf{tocbasic} bekannt gemacht werden.
% \begin{macro}{\listof}
% �ber die Anweisung \cs{listof} wird ein Verzeichnis f�r Gleitumgebungen
% ausgegeben. Hier muss schlicht die entsprechende Anweisung von
% \textsf{tocbasic} verwendet werden.
% \begin{macro}{\float@addtolists}
% Diese Anweisung wird nicht l�nger ben�tigt und daher auf die urspr�ngliche
% Definition zur�ckgesetzt.
%    \begin{macrocode}
\scr@ifexpected{\newfloat}{%
  \long\def\newfloat#1#2#3{\@namedef{ext@#1}{#3}
    \let\float@do=\relax
    \xdef\@tempa{\noexpand\float@exts{\the\float@exts \float@do{#3}}}%
    \@tempa
    \floatplacement{#1}{#2}%
    \@ifundefined{fname@#1}{\floatname{#1}{#1}}{}
    \expandafter\edef\csname ftype@#1\endcsname{\value{float@type}}%
    \addtocounter{float@type}{\value{float@type}}
    \restylefloat{#1}%
    \expandafter\edef\csname fnum@#1\endcsname%
    {\expandafter\noexpand\csname fname@#1\endcsname{}
      \expandafter\noexpand\csname the#1\endcsname}
    \@ifnextchar[%]
    {\float@newx{#1}}%
    {\@ifundefined{c@#1}{\newcounter{#1}\@namedef{the#1}{\arabic{#1}}}%
      {}}}%
}{%
  \scr@ifexpected{\listof}{%
    \def\listof#1#2{%  
      \@ifundefined{ext@#1}{\float@error{#1}}{%
        \@namedef{l@#1}{\@dottedtocline{1}{1.5em}{2.3em}}%
        \float@listhead{#2}%
        \begingroup\setlength{\parskip}{\z@}%
        \@starttoc{\@nameuse{ext@#1}}%
        \endgroup}}%
  }{%
    \PackageInfo{scrhack}{redefining \string\newfloat}%
    \renewcommand\newfloat[3]{%
      \ifattoclist{#3}{%
        \PackageError{scrhack}{extension `#3' already in use}{%
          Each extension may be used only once.\MessageBreak
          You, the class, or another package already uses extension
          `#3'.\MessageBreak
          \string\newfloat\space command will be ignored!}%
      }{%
        \addtotoclist[float]{#3}%
        \setuptoc{#3}{chapteratlist}%
        \@namedef{ext@#1}{#3}%
        \let\float@do=\relax
        \xdef\@tempa{\noexpand\float@exts{\the\float@exts \float@do{#3}}}%
        \@tempa
        \floatplacement{#1}{#2}%
        \@ifundefined{fname@#1}{\floatname{#1}{#1}}{}%
        \expandafter\edef\csname ftype@#1\endcsname{\value{float@type}}%
        \addtocounter{float@type}{\value{float@type}}%
        \restylefloat{#1}%
        \expandafter\edef\csname fnum@#1\endcsname%
        {\expandafter\noexpand\csname fname@#1\endcsname{}%
          \expandafter\noexpand\csname the#1\endcsname}%
        \@ifnextchar[%]
        {\float@newx{#1}}%
        {\@ifundefined{c@#1}{\newcounter{#1}\@namedef{the#1}{\arabic{#1}}}%
          {}}}%
    }%
    \PackageInfo{scrhack}{redefining \string\listof}%
    \renewcommand*\listof[2]{%
      \@ifundefined{ext@#1}{\float@error{#1}}{%
        \@ifundefined{l@#1}{\expandafter\let\csname l@#1\endcsname\l@figure
          \@ifundefined{l@#1}{%
            \@namedef{l@#1}{\@dottedtocline{1}{1.5em}{2.3em}}}{}%
        }{}%
        \listoftoc[{#2}]{\csname ext@#1\endcsname}%
      }%
    }%
    \scr@ifexpected{\float@addtolists}{%
      \long\def\float@addtolists#1{%
        \def\float@do##1{\addtocontents{##1}{#1}} \the\float@exts}%
    }{%
      \PackageInfo{scrhack}{undefining \string\float@addtolists}%
      \let\float@addtolists\relax
    }{%
      \PackageWarningNoLine{scrhack}{unkown \string\float@addtolists\space
        definition found!\MessageBreak
        Maybe you are using a unsupported float version}%
    }%
  }{%
    \PackageWarningNoLine{scrhack}{unknown \string\listof\space
      definition found!\MessageBreak
      Maybe you are using a unsupported float version}%
  }%
}{%
  \PackageWarningNoLine{scrhack}{unknown \string\newfloat\space
    definition found!\MessageBreak
    Maybe you are using a unsupported float version}%
}
%    \end{macrocode}
% \end{macro}
% \end{macro}
% \end{macro}
% 
% \iffalse
%</float&body>
%<*package&option>
% \fi
%
%
% \subsection{Optionen ausf�hren}
%
% Zum Schluss noch die Optionen ausf�hren.
%    \begin{macrocode}
\KOMAProcessOptions\relax
%    \end{macrocode}
%
% \iffalse
%</package&option>
% \fi
%
% \subsection{Verwendete Anweisungen}
%
% \iffalse
%<*package&body>
% \fi
%
% \begin{macro}{\scr@ifexpected}
% Wenn die im ersten Argument angegebene Anweisung nach Ausf�hrung der im
% zweiten Argument angegebenen Anweisungen unver�ndert ist, dann soll das
% dritte Argument ausgef�hrt werden, sonst das vierte.
%    \begin{macrocode}
\newcommand{\scr@ifexpected}[2]{%
  \begingroup
    \let\@tempa#1
    #2
    \ifx\@tempa#1
      \aftergroup\@firstoftwo
    \else
      \aftergroup\@secondoftwo
    \fi
  \endgroup
}
%    \end{macrocode}
% \end{macro}
%
% \begin{macro}{\scr@hack@load}
% Wenn die Datei mit dem Namen des zweiten Arguments und der Endung des ersten
% Arguments so geladen wurde, dass \LaTeX{} eine Versionsinfo dazu gespeichert
% hat, dann soll zus�tzlich der entsprechende Hack geladen werden.
%    \begin{macrocode}
\newcommand*{\scr@hack@load}[2]{%
  \expandafter\ifx\csname ver@#2.#1\endcsname\relax
    \expandafter\@secondoftwo
  \else
    \expandafter\@firstoftwo
  \fi
  {\PackageInfo{scrhack}{loading #2 hack}%
    \makeatletter\input{#2.hak}\makeatother}%
  {\PackageInfo{scrhack}{ignorring #2 hack}}%
}
%    \end{macrocode}
% \end{macro}
%
% \iffalse
%</package&body>
% \fi
%
% \Finale
%
\endinput
%
% end of file `scrhack.dtx'
%%% Local Variables:
%%% mode: doctex
%%% coding: iso-latin-1
%%% TeX-master: t
%%% End:
