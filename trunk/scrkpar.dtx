% \CheckSum{424}
% \iffalse meta-comment
% ======================================================================
% scrkpar.dtx
% Copyright (c) Markus Kohm, 2002-2006
%
% This file is part of the LaTeX2e KOMA-Script bundle.
%
% This work may be distributed and/or modified under the conditions of
% the LaTeX Project Public License, version 1.3b of the license.
% The latest version of this license is in
%   http://www.latex-project.org/lppl.txt
% and version 1.3b or later is part of all distributions of LaTeX 
% version 2005/12/01 and of this work.
%
% This work has the LPPL maintenance status "author-maintained".
%
% The Current Maintainer and author of this work is Markus Kohm.
%
% This work consists of all files listed in manifest.txt.
% ----------------------------------------------------------------------
% scrkpar.dtx
% Copyright (c) Markus Kohm, 2002-2006
%
% Dieses Werk darf nach den Bedingungen der LaTeX Project Public Lizenz,
% Version 1.3b.
% Die neuste Version dieser Lizenz ist
%   http://www.latex-project.org/lppl.txt
% und Version 1.3b ist Teil aller Verteilungen von LaTeX
% Version 2005/12/01 und dieses Werks.
%
% Dieses Werk hat den LPPL-Verwaltungs-Status "author-maintained"
% (allein durch den Autor verwaltet).
%
% Der Aktuelle Verwalter und Autor dieses Werkes ist Markus Kohm.
% 
% Dieses Werk besteht aus den in manifest.txt aufgefuehrten Dateien.
% ======================================================================
% \fi
% \iffalse
%%% From File: scrkpar.dtx
%<*driver>
% \fi
\ProvidesFile{scrkpar.dtx}[2006/03/08 v3.0 KOMA-Script (paragraphs)]
% \iffalse
\documentclass[halfparskip-]{scrdoc}
\usepackage[english,german]{babel}
\usepackage[latin1]{inputenc}
\CodelineIndex
\RecordChanges
\GetFileInfo{scrkpar.dtx}
\title{\KOMAScript{} \partname\ \texttt{\filename}%
  \footnote{Dies ist Version \fileversion\ von Datei \texttt{\filename}.}}
\date{\filedate}
\author{Markus Kohm}

\begin{document}
  \maketitle
  \tableofcontents
  \DocInput{\filename}
\end{document}
%</driver>
% \fi
%
% \selectlanguage{german}
%
% \changes{v3.0}{2002/06/25}{%
%   erste Version aus der Aufteilung von \texttt{scrclass.dtx}}
%
% \section{Absatzformatierung und -umbruch}
%
% In diesen Bereich geh�rt alles, was die Absatzformatierung selbst
% betrifft. Dazu kommen dann noch einige Kleinigkeiten aus dem
% Seitenumbruch.
%
% \StopEventually{\PrintIndex\PrintChanges}
%
%
% \iffalse
%<*option>
% \fi
%
% \subsection{Optionen f�r das Absatzlayout}
%
% Ab Version 2.8i k�nnen wir auch mit Absatzlayouts umgehen, die einen
% Absatzabstand statt einem Absatzeinzug verlangen. Gesteuert wird
% dies �ber Optionen und Schalter.
%
%  \begin{macro}{\setparsizes}
%    \changes{v3.0}{2004/11/05}{neues Macro}
% �ber dieses Makro wird die �nderung der Absatzparameter |\parskip|,
% |\parindent|, |\parfillskip| gesetzt. Das erste Argument ist der Einzug, das
% zweite der Abstand und das dritte die F�llung. Aktiviert werden die
% �nderungen wie bei |\fontsize| �ber |\selectfont|. Obwohl in der
% Voreinstellung absolut gearbeitet wird, wird hier intern
% |\par@updaterelative| verwendet.
%    \begin{macrocode}
\newcommand*{\setparsizes}[3]{%
  \edef\f@parindent{\the\parindent}%
  \edef\f@parskip{\the\parskip}%
  \edef\f@parfillskip{\the\parfillskip}%
  \def\scr@parindent{#1}%
  \def\scr@parskip{#2}%
  \def\scr@parfillskip{#3}%
  \def\par@update{\let\par@update\default@par@update\par@updaterelative}%
}
%    \end{macrocode}
%  \begin{macro}{\f@parindent}
%    \changes{v3.0}{2004/11/05}{neu (intern)}
% Eingestellter Absatzeinzug.
%  \begin{macro}{\f@parskip}
%    \changes{v2.8i}{2001/07/22}{neu (intern)}
%    \changes{v3.0}{2004/11/05}{Bedeutung ge�ndert}
% Eingestellter Absatzabstand.
%  \begin{macro}{\f@parfillskip}
%    \changes{v2.8i}{2001/07/22}{neu (intern)}
%    \changes{v3.0}{2004/11/05}{Bedeutung ge�ndert}
% Eingestellte Absatzf�llung.
%
% Diese drei Werte werden automatisch bei der Font-Initialisierung eingestellt
% und sind vorher ung�ltig!
%    \begin{macrocode}
\newcommand*{\f@parindent}{\the\parindent}
\newcommand*{\f@parskip}{\the\parskip}
\newcommand*{\f@parfillskip}{\the\parfillskip}
\AtEndOfClass{%
  \edef\f@parindent{\the\parindent}%
  \edef\f@parskip{\the\parskip}%
  \edef\f@parfillskip{\the\parfillskip}%
}
%    \end{macrocode}
%  \end{macro}
%  \end{macro}
%  \end{macro}
%  \end{macro}
%
%  \begin{macro}{\par@update}
%    \changes{v3.0}{2004/11/05}{neues internes Macro}
% Dieses Makro wird sp�ter in |\selectfont| die �nderung vornehmen.
%  \begin{macro}{\default@par@update}
%    \changes{v3.0}{2004/11/05}{neues internes Macro}
% In der Voreinstellung findet keine �nderung statt. Dies wird jedoch durch
% die Auswahl einer entsprechenden Option ge�ndert.
%    \begin{macrocode}
\newcommand*{\par@update}{}
\let\par@update\relax
\newcommand*{\default@par@update}{}
\let\default@par@update\relax
%    \end{macrocode}
%  \end{macro}
%  \end{macro}
%
%  \begin{option}{parskip}
%    \changes{v2.8i}{2001/07/22}{neue Option}
%  \begin{option}{parskip-}
%    \changes{v2.8l}{2001/08/16}{neue Option}
%  \begin{option}{parskip+}
%    \changes{v2.8i}{2001/07/22}{neue Option}
%  \begin{option}{parskip*}
%    \changes{v2.8i}{2001/07/22}{neue Option}
%  \begin{option}{halfparskip}
%    \changes{v2.8i}{2001/07/22}{neue Option}
%  \begin{option}{halfparskip-}
%    \changes{v2.8l}{2001/08/16}{neue Option}
%  \begin{option}{halfparskip+}
%    \changes{v2.8i}{2001/07/22}{neue Option}
%  \begin{option}{halfparskip*}
%    \changes{v2.8i}{2001/07/22}{neue Option}
%  \begin{option}{parindent}
%    \changes{v2.8i}{2001/07/22}{neue Option}
% Diese neun Optionen steuern die Umschaltung zwischen den Modi. Dabei
% schalten alle \texttt{parskip}-Optionen einen Absatzabstand ein,
% wohingegen die \texttt{parindent}-Option den Absatzeinzug
% einschaltet. Die \texttt{+}-Variante sorgt au�erdem daf�r, dass
% die letzte Zeile eines Absatzes maximal zu zwei Dritteln gef�llt
% wird. Entsprechend sorgt die \texttt{*}-Variante f�r eine maximale
% F�llung von drei Vierteln. Die normale Variante sorgt lediglich
% f�r einen freien Raum von 1\,em. Die \texttt{-}-Variante sorgt f�r
% �berhaupt nichts.
%  \begin{macro}{\scr@parindent}
%    \changes{v3.0}{2004/11/05}{neu (intern)}
% Der einzustellende Absatzeinzug.
%  \begin{macro}{\scr@parskip}
%    \changes{v2.8i}{2001/07/22}{neu (intern)}
%    \changes{v3.0}{2004/11/05}{Bedeutung ge�ndert}
% Der einzustellende Absatzabstand.
%  \begin{macro}{\scr@parfillskip}
%    \changes{v2.8i}{2001/07/22}{neu (intern)}
%    \changes{v3.0}{2004/11/05}{Bedeutung ge�ndert}
% Die einzustellende Absatzf�llung.
%    \begin{macrocode}
\newcommand*{\scr@parindent}{1em}
\newcommand*{\scr@parskip}{\z@}
\newcommand*{\scr@parfillskip}{\z@ \@plus 1fil}
%    \end{macrocode}
%  \end{macro}
%  \end{macro}
%  \end{macro}
%    \begin{macrocode}
\DeclareOption{parskip}{%
  \setparsizes{\z@}{\baselineskip \@plus .1\baselineskip}{%
    1em \@plus 1fil}%
}
\DeclareOption{parskip-}{%
  \setparsizes{\z@}{\baselineskip \@plus .1\baselineskip}{%
    \z@ \@plus 1fil}%
}
\DeclareOption{parskip+}{%
  \setparsizes{\z@}{\baselineskip \@plus .1\baselineskip}{%
    .3333\linewidth\@plus 1fil}%
}
\DeclareOption{parskip*}{%
  \setparsizes{\z@}{\baselineskip \@plus .1\baselineskip}{%
    .25\linewidth \@plus 1fil}%
}
\DeclareOption{halfparskip}{%
  \setparsizes{\z@}{.5\baselineskip \@plus .5\baselineskip}{%
    1em \@plus 1fil}%
}
\DeclareOption{halfparskip-}{%
  \setparsizes{\z@}{.5\baselineskip \@plus .5\baselineskip}{%
    \z@ \@plus 1fil}%
}
\DeclareOption{halfparskip+}{%
  \setparsizes{\z@}{.5\baselineskip \@plus .5\baselineskip}{%
    .3333\linewidth \@plus 1fil}%
}
\DeclareOption{halfparskip*}{%
  \setparsizes{\z@}{.5\baselineskip \@plus .5\baselineskip}{%
    .25\linewidth \@plus 1fil}%
}
\DeclareOption{parindent}{%
  \setparsizes{1em}{\z@ \@plus \p@}{\z@ \@plus 1fil}%
}
%    \end{macrocode}
%  \end{option}
%  \end{option}
%  \end{option}
%  \end{option}
%  \end{option}
%  \end{option}
%  \end{option}
%  \end{option}
%  \end{option}
%
% Hierf�r\marginline{Geplant!} sollte eine keyval-Option mit Werten definiert
% werden. Dazu dann neue Werte f�r absolutes oder relatives Verhalten.
%
% \iffalse
%</option>
%<*body>
% \fi
%
%
% \subsection{Abatzformatierung}
%
% \changes{v2.8i}{2001/07/22}{\cs{baselinestretch} wird nicht
%   umdefiniert}
%  \begin{Length}{lineskip}
%  \begin{Length}{normallineskip}
% Minimaler Zeilenabstand:
%    \begin{macrocode}
\setlength{\lineskip}{\p@}
\setlength{\normallineskip}{\p@}
%    \end{macrocode}
%  \end{Length}
%  \end{Length}
%
%  \begin{Length}{columnsep}
%  \begin{Length}{columnseprule}
% Spaltenabstand und Spaltentrennlinie:
%    \begin{macrocode}
\setlength{\columnsep}{10\p@}
\setlength{\columnseprule}{\z@}
%    \end{macrocode}
%  \end{Length}
%  \end{Length}
%
%  \begin{macro}{\selectfont}
%    \changes{v3.0}{2004/11/05}{neue �nderung}
% Ab Version~3.0 soll die M�glichkeit bestehen, |\parskip|, |\parindent| und
% |\parfillskip| mit der Schriftgr��e automatisch anzupassen. Dazu muss
% |\selectfont| entsprechend erweitert werden.
%    \begin{macrocode}
\expandafter\g@addto@macro\csname selectfont \endcsname{\par@update}
%    \end{macrocode}
%  \begin{macro}{\par@updaterelative}
%    \changes{v3.0}{2004/11/05}{neues internes Macro}
% Die eigentliche �nderung verbirgt sich in |\par@updaterelative|. Ggf. wird
% |\par@update| zu |\par@updaterelative|.
%    \begin{macrocode}
\newcommand*{\par@updaterelative}{%
%    \end{macrocode}
% Die neuen Werte werden nur gesetzt, wenn die bisherigen Werten den
% erwarteten Werten entsprechen. Sonst lassen wir lieber die Finger davon,
% weil wir dann davon ausgehen, dass der Anwender die so setzen wollte.
%    \begin{macrocode}
  \begingroup
    \edef\@tempa{\the\parindent}\ifx\@tempa\f@parindent
      \aftergroup\parindent@update
%<*trace>
    \else
      \ClassInfo{\KOMAClassName}{\string\parindent\space not changed}%
%</trace>
    \fi
    \edef\@tempa{\the\parskip}\ifx\@tempa\f@parskip
      \aftergroup\parskip@update
%<*trace>
    \else
      \ClassInfo{\KOMAClassName}{\string\parskip\space not changed}%
%</trace>
    \fi
    \edef\@tempa{\the\parfillskip}\ifx\@tempa\f@parfillskip
      \aftergroup\parfillskip@update
%<*trace>
    \else
      \ClassInfo{\KOMAClassName}{\string\parfillskip\space not changed}%
%</trace>
    \fi
  \endgroup
}
%    \end{macrocode}
%  \begin{macro}{\parindent@update}
%    \changes{v3.0}{2004/11/05}{neues internes Macro}
%  \begin{macro}{\parskip@update}
%    \changes{v3.0}{2004/11/05}{neues internes Macro}
%  \begin{macro}{\parfillskip@update}
%    \changes{v3.0}{2004/11/05}{neues internes Macro}
% Ein paar Hilfsmakros.
%    \begin{macrocode}
\newcommand*{\parindent@update}{%
  \scr@defaultunits\parindent\scr@parindent
  \begingroup
    \let\@tempb\endgroup
    \edef\@tempa{\the\parindent}\ifx\@tempa\f@parindent\else
      \def\@tempb{\endgroup\edef\f@parindent{\the\parindent}}%
%<trace>      \ClassInfo{\KOMAClassName}{\string\parindent=\f@parindent}%
    \fi
  \@tempb
}
\newcommand*{\parskip@update}{%
  \scr@defaultunits\parskip\scr@parskip
  \begingroup
    \let\@tempb\endgroup
    \edef\@tempa{\the\parskip}\ifx\@tempa\f@parskip\else
      \def\@tempb{\endgroup\edef\f@parskip{\the\parskip}}%
%<trace>      \ClassInfo{\KOMAClassName}{\string\parskip=\f@parskip}%
    \fi
  \@tempb
}
\newcommand*{\parfillskip@update}{%
  \scr@defaultunits\parfillskip\scr@parfillskip
  \begingroup
    \let\@tempb\endgroup
    \edef\@tempa{\the\parfillskip}\ifx\@tempa\f@parfillskip\else
      \def\@tempb{\endgroup\edef\f@parfillskip{\the\parfillskip}}%
%<trace>      \ClassInfo{\KOMAClassName}{\string\parfillskip=\f@parfillskip}%
    \fi
  \@tempb
}
%    \end{macrocode}
%  \end{macro}
%  \end{macro}
%  \end{macro}
%
%  \begin{macro}{\scr@defaultunits}
%    \changes{v3.0}{2004/11/05}{neues internes Macro}
%  \begin{macro}{\scr@@defaultunits}
%    \changes{v3.0}{2004/11/05}{neues internes Macro}
%  \begin{macro}{\scr@@@defaultunits}
%    \changes{v3.0}{2004/11/05}{neues internes Macro}
% Damit |\par@updaterelative| �berhaupt funktionieren kann, wird
% |\scr@defaultunits| ben�tigt. Dieses Makro arbeitet prinzipiell wie
% |\@defaultunits| bekommt aber Dimension bzw. Skip als erstes und den Wert
% als zweites Argument. Als Besonderheit d�rften im Wert auch andere
% Dimensions bzw. Skips vor und nach \texttt{plus} und \texttt{minus}
% verwendet werden. Es sind also auch Angaben der Art "`\texttt{12 plus 1
% minus 2}"' sowie
% "`\texttt{\string\baselineskip\string\@plus.1\string\baselineskip}"'
% g�ltig.
%    \begin{macrocode}
\newcommand*{\scr@defaultunits}[2]{%
  \begingroup
    \edef\@tempa{#2}%
    \expandafter\scr@@defaultunits\expandafter#1\@tempa plusplus\@nnil
    \edef\@tempa{\noexpand\endgroup\noexpand\setlength{\noexpand#1}{\the#1}}%
  \@tempa
}
\newcommand*{\scr@@defaultunits}{}
\def\scr@@defaultunits#1#2plus#3plus#4\@nnil{%
  \ifx\relax#3\relax
    \scr@@@defaultunits#1{}#2minusminus\@nnil
  \else
    \scr@@@defaultunits#1{#2}#3minusminus\@nnil
  \fi
}
\newcommand*{\scr@@@defaultunits}{}
\def\scr@@@defaultunits#1#2#3minus#4minus#5\@nnil{%
  \ifx\relax#2\relax
    \@defaultunits\@tempskipa#3pt\relax\@nnil
    \setlength{#1}{\@tempskipa}%
  \else
    \@defaultunits\@tempskipa\z@\@plus#3pt\relax\@nnil
    \setlength{#1}{\@tempskipa}%
    \@defaultunits\@tempskipa#2pt\relax\@nnil
    \addtolength{#1}{\@tempskipa}%
  \fi
  \ifx\relax#4\relax\else
    \@defaultunits\@tempskipa\z@\@minus #4pt\relax\@nnil
    \addtolength{#1}{\@tempskipa}%
  \fi
}
%    \end{macrocode}
%  \end{macro}
%  \end{macro}
%  \end{macro}
%  \end{macro}
%  \end{macro}
%
% Absatzabstand und Absatzeinzug:
%  \begin{macro}{\@listi}
%  \begin{macro}{\@listI}
%  \begin{macro}{\@listii}
%  \begin{macro}{\@listiii}
%  \begin{macro}{\@list@extra}
%    \changes{v2.8q}{2001/11/06}{neu (intern)}
%  \begin{macro}{\add@extra@listi}
%    \changes{v2.9h}{2002/09/03}{neu (intern)}
%  \begin{macro}{\footnotesize}
%    \changes{v2.9h}{2002/09/03}{etwas robuster}
%  \begin{macro}{\small}
%    \changes{v2.9h}{2002/09/03}{etwas robuster}
% Ab Version 2.8i wird hier optionsabh�ngig gearbeitet. Dabei m�ssen
% auch die Befehle bei der Umschaltung der Schriftgr��e f�r \cs{small}
% und \cs{footnotesize} ge�ndert werden. 
%    \begin{macrocode}
\g@addto@macro{\@listi}{\@list@extra}
\let\@listI=\@listi
\g@addto@macro{\@listii}{\@list@extra}
\g@addto@macro{\@listiii}{\@list@extra}
\newcommand*{\add@extra@listi}[1]{%
  \expandafter\let\csname #1@listi\endcsname=\@listi
  \def\@listi{\csname #1@listi\endcsname\@list@extra}%
}
\g@addto@macro{\footnotesize}{\protect\add@extra@listi{ftns}}
\g@addto@macro{\small}{\protect\add@extra@listi{sml}}
\newcommand*{\@list@extra}{%
  \ifdim\parskip>\z@\topsep\z@\parskip\parskip\itemsep\z@\fi
}
%    \end{macrocode}
%  \end{macro}
%  \end{macro}
%  \end{macro}
%  \end{macro}
%  \end{macro}
%  \end{macro}
%  \end{macro}
%  \end{macro}
%
%
% \subsection{Umbruchsteuerung}
%
% F�r die Umbruchsteuerung sind einige Penalties zust�ndig. Diese sind
% im \LaTeX-Kern definiert. Leider sind \cs{@lowpenalty},
% \cs{@medpenalty} und \cs{@highpenalty} aber nicht mit
% Voreinstellungen versehen, besitzen einheitlich die Voreinstellung
% 0. Hier werden deshalb die Werte aus den Standardklassen �bernommen:
%    \begin{macrocode}
\@lowpenalty  = 51
\@medpenalty  =151
\@highpenalty =301
%    \end{macrocode}
%
%
% \iffalse
%</body>
% \fi
%
% \Finale
%
\endinput
%
% end of file `scrkpar.dtx'
%%% Local Variables:
%%% mode: doctex
%%% TeX-master: t
%%% End:
