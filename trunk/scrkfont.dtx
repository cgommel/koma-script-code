% \CheckSum{240}
% \iffalse meta-comment
% ======================================================================
% scrkfont.dtx
% Copyright (c) Markus Kohm, 2002-2006
%
% This file is part of the LaTeX2e KOMA-Script bundle.
%
% This work may be distributed and/or modified under the conditions of
% the LaTeX Project Public License, version 1.3b of the license.
% The latest version of this license is in
%   http://www.latex-project.org/lppl.txt
% and version 1.3b or later is part of all distributions of LaTeX 
% version 2005/12/01 and of this work.
%
% This work has the LPPL maintenance status "author-maintained".
%
% The Current Maintainer and author of this work is Markus Kohm.
%
% This work consists of all files listed in manifest.txt.
% ----------------------------------------------------------------------
% scrkfont.dtx
% Copyright (c) Markus Kohm, 2002-2006
%
% Dieses Werk darf nach den Bedingungen der LaTeX Project Public Lizenz,
% Version 1.3b.
% Die neuste Version dieser Lizenz ist
%   http://www.latex-project.org/lppl.txt
% und Version 1.3b ist Teil aller Verteilungen von LaTeX
% Version 2005/12/01 und dieses Werks.
%
% Dieses Werk hat den LPPL-Verwaltungs-Status "author-maintained"
% (allein durch den Autor verwaltet).
%
% Der Aktuelle Verwalter und Autor dieses Werkes ist Markus Kohm.
% 
% Dieses Werk besteht aus den in manifest.txt aufgefuehrten Dateien.
% ======================================================================
% \fi
% \iffalse
%%% From File: scrkfont.dtx
%<*driver>
% \fi
\ProvidesFile{scrkfont.dtx}[2004/07/21 v3.0 KOMA-Script (font selection)]
% \iffalse
\documentclass[halfparskip-]{scrdoc}
\usepackage[english,german]{babel}
\usepackage[latin1]{inputenc}
\CodelineIndex
\RecordChanges
\GetFileInfo{scrkfont.dtx}
\title{\KOMAScript{} \partname\ \texttt{\filename}%
  \footnote{Dies ist Version \fileversion\ von Datei \texttt{\filename}.}}
\date{\filedate}
\author{Markus Kohm}

\begin{document}
  \maketitle
  \tableofcontents
  \DocInput{\filename}
\end{document}
%</driver>
% \fi
%
% \selectlanguage{german}
%
% \changes{v3.0}{2002/06/25}{%
%   erste Version aus der Aufteilung von \texttt{scrclass.dtx}}
%
% \section{Schriftauswahl}
%
% Dieser Bereich befasst sich mit allem, was zur Schriftauswahl
% geh�rt. Die einzelnen Elemente, f�r die eine Schriftauswahl m�glich
% ist, sind jedoch in den Dateien definiert, in denen diese Elemente
% auftreten.
%
% \StopEventually{\PrintIndex\PrintChanges}
%
% \iffalse
%<*option>
% \fi
%
%
% \subsection{Option zur Auswahl der Gr��e der Grundschrift}
%
%  \begin{option}{10pt}
%  \begin{option}{11pt}
%  \begin{option}{12pt}
% Diese Optionen existieren nicht mehr als explizite Optionen, sondern
% werden in \texttt{scrtarea.dtx} �ber \cs{DeclareOption*}
% ausgewertet.
%  \begin{macro}{\@ptsize}
% Wir definieren \cs{@ptsize} als Grundschriftgr��ee minus 10\,pt.
% Hier wird die Standardeinstellung 11\,pt gew�hlt.
%    \begin{macrocode}
\newcommand*\@ptsize{1}
%    \end{macrocode}
%  \end{macro}
%  \end{option}
%  \end{option}
%  \end{option}
%
%  \begin{macro}{\@pt@scan}
%    \changes{v2.6}{2000/01/04}{neu (intern)}
%  \begin{macro}{\@pt@@scan}
%    \changes{v2.6}{2000/01/04}{neu (intern)}
% Das Macro \cs{@pt@scan} wird ben�tigt, um die Schriftgr��e nach der alten
% Methode zu erkennen.
%    \begin{macrocode}
\newcommand*{\@pt@scan}{%
  \expandafter\@pt@@scan\CurrentOption pt\@pt@@scan%
}
\newcommand*{\@pt@@scan}{}
\def\@pt@@scan #1pt#2\@pt@@scan{%
  \ifstr{#2}{pt}{%
    \ifnumber{#1}{%
      \KOMA@UseObsolete{option}{\PackageInfo}{#1#2}{fontsize=#1#2}%
    }{\@headlines}%
  }{\@headlines}%
}
%    \end{macrocode}
%  \end{macro}
%  \end{macro}
%
% \begin{option}{fontsize}
% Hier nun die Option, die tats�chlich verwendet wird. Es sei darauf
% hingewiesen, dass damit auch Schriftgr��en wie \texttt{10.3pt} denkbar
% w�ren.
%    \begin{macrocode}
\KOMA@key{fontsize}{%
  \begingroup
    \setlength{\@tempdima}{#1}%
    \addtolength{\@tempdima}{-10\p@}%
    \edef\@tempa{\endgroup\noexpand\renewcommand*\noexpand\@ptsize{%
        \strip@pt\@tempdima}}%
  \@tempa
}
\@onlypreamble\KV@KOMA@fontsize
%    \end{macrocode}
% \end{option}
%
%
%
% \iffalse
%</option>
%<*body>
% \fi
%
% \subsection{Einlesen der Schriftgr��endatei}
%
% \changes{v2.6}{2000/01/04}{Einlesen der Schriftgr��en-Options-Datei
%   funktioniert nun mit diversen Gr��en}
%
% Dies\marginline{Geplant!} soll in einer zuk�nftigen Version von
% \KOMAScript{} komplett �berfl�ssig werden. Stattdessen werden
% Standardeinstellung berechnet, wobei die M�glichkeit bestehen wird,
% Abweichungen davon in einer Konfigurationsdatei abzulegen.
%    \begin{macrocode}
\@tempcnta=10
\advance\@tempcnta by\@ptsize
\xdef\@tempa{%
%<!book>  size%
%<book>  bk%
  \the\@tempcnta.clo}
\InputIfFileExists{\@tempa}{}{%
%<*book>
  \xdef\@tempb{size\the\@tempcnta.clo}%
  \InputIfFileExists{\@tempb}{%
    \ClassWarning{\KOMAClassName}{%
      File `\@tempa' not found!\MessageBreak
      Used `\@tempb' instead%
    }%
  }{%
%</book>
    \ClassError{\KOMAClassName}{%
      File `\@tempa' not found!\MessageBreak
      This error is fatal!%
    }{%
      To every fontsize you want to select by class
      option,\MessageBreak
      you need a corresponding
%<book>       bk-class-option-file or\MessageBreak at least a
      size-class-option-file declaring all\MessageBreak
      the macros based on fontsize.\MessageBreak
      Ask a wizzard for more information or use another
      fontsize.\MessageBreak
      You cannot continue!%
    }%
    \batchmode \csname @@end\endcsname \end
%<book>  }%
}
%    \end{macrocode}
% 
%
% \subsection{"`Alte"' Font-Auswahlbefehle}
%
%  \begin{macro}{\rm}
%  \begin{macro}{\sf}
%  \begin{macro}{\tt}
%  \begin{macro}{\bf}
%  \begin{macro}{\it}
%  \begin{macro}{\sl}
%  \begin{macro}{\sc}
%  \begin{macro}{\sfb}
%    \changes{v2.2c}{1995/05/25}{nicht mehr mathematisch}
%    \changes{v2.3a}{1995/07/08}{keine Unterscheidung mehr f�r den
%      Kompatibilit�tsmodus}
% Um die Umstellung von \LaTeX\ auf \LaTeXe\ zu erleichtern, gibt es
% die alten Font-Auswahlbefehle \cs{rm}, \cs{sf}, \cs{tt}, \cs{bf},
% \cs{it}, \cs{sl} und \cs{sc} auch in den Klassen. Es ist zu
% beachten, dass diese Befehle hier nach dem alten
% Fontauswahlverfahren arbeiten. Es werden also immer alle Parameter
% zugleich ge�ndert. Somit ist zu empfehlen, dass statt dieser Befehle
% zuk�nftig in der Regel die neuen \cs{text\dots}-Befehle verwendet
% werden.
%
% Der aus der \textsf{Script 2.0}-Familie bekannte Befehl \cs{sfb} ist
% hier ebenfalls definiert.
%    \begin{macrocode}
\DeclareOldFontCommand{\rm}{\normalfont\rmfamily}{\mathrm}
\DeclareOldFontCommand{\sf}{\normalfont\sffamily}{\mathsf}
\DeclareOldFontCommand{\tt}{\normalfont\ttfamily}{\mathtt}
\DeclareOldFontCommand{\bf}{\normalfont\bfseries}{\mathbf}
\DeclareOldFontCommand{\it}{\normalfont\itshape}{\mathit}
\DeclareOldFontCommand{\sl}{\normalfont\slshape}{\@nomath\sl}
\DeclareOldFontCommand{\sc}{\normalfont\scshape}{\@nomath\sc}
\DeclareOldFontCommand{\sfb}{\normalfont\sffamily\bfseries}{%
  \@nomath\sfb}
%    \end{macrocode}
%  \end{macro}
%  \end{macro}
%  \end{macro}
%  \end{macro}
%  \end{macro}
%  \end{macro}
%  \end{macro}
%  \end{macro}
%
%  \begin{macro}{\cal}
%    \changes{v2.3}{1995/06/25}{hier statt im \LaTeX-Kern}
% Der Befehl ist in \LaTeX{} nicht mehr definiert. Deshalb wird er nun
% neu und gleich robust deklariert. Daf�r f�llt \cs{pcal} weg.
%    \changes{v2.3g}{1996/01/14}{�berfl�ssige Klammerpaare entfernt}
%  \begin{macro}{\mit}
%    \changes{v2.3}{1995/06/25}{hier statt im \LaTeX-Kern}
% Der Befehl ist in \LaTeX{} nicht mehr definiert. Deshalb wird er nun
% neu und gleich robust deklariert. Daf�r f�llt \cs{pmit} weg.
%   \changes{v2.3g}{1996/01/14}{�berfl�ssige Klammerpaare entfernt}
%
% Die beiden Befehle \cs{cal} und \cs{mit} war bis zur \LaTeX-Version
% vom 1.12.1994 patch level 3 noch im Kernal definiert. Jetzt sind die
% beiden den Klassen �berlassen. Hier ist die Definition aus den
% Standard-classes �bernommen.
%    \begin{macrocode}
\DeclareRobustCommand*{\cal}{\@fontswitch\relax\mathcal}
\DeclareRobustCommand*{\mit}{\@fontswitch\relax\mathnormal}
%    \end{macrocode}
%  \end{macro}
%  \end{macro}
%
%
% \subsection{Setzen der Schrift eines Elements}
%
%  \begin{macro}{\setkomafont}
%    \changes{v2.8o}{2001/09/14}{neu}
% Mit Hilfe dieses Makros kann die Schriftart von
% \KOMAScript-Elementen gesetzt werden.
%    \begin{macrocode}
\newcommand*{\setkomafont}[2]{%
  \@ifundefined{scr@fnt@#1}{%
    \@ifundefined{scr@fnt@instead@#1}{%
      \ClassError{\KOMAClassName}{%
        font of element `#1' can't be set}{%
        You've told me to redefine the font selection of the
        element,\MessageBreak%
        but either no such element is known by
        KOMA-Script\MessageBreak%
        or the element does not use a special font selection%
      }%
    }{%
      \ClassError{\KOMAClassName}{%
        font of element `#1' can't be set%
      }{%
        You've told me to redefine the font selection of the
        element,\MessageBreak%
        but the element does not use a special font
        selection.\MessageBeak%
        The font of `#1' may be changed with element `\csname
        scr@fnt@instead@#1\endcsname'%
      }%
    }%
  }{%
    \expandafter\expandafter\expandafter\renewcommand
    \csname scr@fnt@#1\endcsname{#2}%
    \@ifundefined{scr@fnt@wrn@#1}{}{%
      \ClassWarning{\KOMAClassName}{%
        \csname scr@fnt@wrn@#1\endcsname{#1}%
      }%
    }%
  }%
}
%    \end{macrocode}
%  \end{macro}
%  \begin{macro}{\addtokomafont}
%    \changes{v2.8p}{2001/09/22}{neu}
% Dieses Makro funktioniert fast wie obiges, allerdings f�gt es der
% vorhandenen Definition etwas an.
%  \begin{macro}{\l@addto@macro}
%    \changes{v2.8p}{2001/09/22}{neu (intern)}
% Damit dies lokal m�glich ist, wird eine lokal arbeitende Variante
% von \cs{g@addto@macro} ben�tigt, wie sie hier vorgestellt wird.
%    \begin{macrocode}
\newcommand{\l@addto@macro}[2]{%
  \begingroup%
    \toks@\expandafter{#1#2}%
    \edef\@tempa{\endgroup\def\noexpand#1{\the\toks@}}%
  \@tempa
}
%    \end{macrocode}
%  \end{macro}
%    \begin{macrocode}
\newcommand*{\addtokomafont}[2]{%
  \@ifundefined{scr@fnt@#1}{%
    \@ifundefined{scr@fnt@instead@#1}{%
      \ClassError{\KOMAClassName}{%
        font of element `#1' can't be extended%
      }{%
        You've told me to extend the font selection of the
        element,\MessageBreak
        but either no such element is known by
        KOMA-Script\MessageBreak
        or the element does not use a special font selection%
      }%
    }{%
      \ClassError{\KOMAClassName}{%
        font of element `#1' can't be extended%
      }{%
        You've told me to extend the font selection of the
        element,\MessageBreak
        but the element does not use a special font
        selection.\MessageBeak
        The font of `#1' may be changed with element `\csname
        scr@fnt@instead@#1\endcsname'%
      }%
    }%
  }{%
    \expandafter\expandafter\expandafter\l@addto@macro
    \csname scr@fnt@#1\endcsname{#2}%
    \@ifundefined{scr@fnt@wrn@#1}{}{%
      \ClassWarning{\KOMAClassName}{%
        \csname scr@fnt@wrn@#1\endcsname{#1}%
      }%
    }%
  }%
}
%    \end{macrocode}
%  \end{macro}
%
%  \begin{macro}{\usekomafont}
%    \changes{v2.8p}{2001/09/28}{neu}
% Mit Hilfe dieses Makros kann auf die Schriftart von
% \KOMAScript-Elementen umgeschaltet werden.
%    \begin{macrocode}
\newcommand*{\usekomafont}[1]{%
  \@ifundefined{scr@fnt@#1}{%
    \@ifundefined{scr@fnt@instead@#1}{%
      \ClassError{\KOMAClassName}{%
        font of element `#1' can't be used%
      }{%
        You've told me to use the font selection of the
        element,\MessageBreak
        but either no such element is known by
        KOMA-Script\MessageBreak
        or the element does not use a special font selection%
      }%
    }{%
      \ClassWarning{\KOMAClassName}{%
        You've told me to use the font selection of the
        element,\MessageBreak
        but the element does not use a special font
        selection. The\MessageBeak
        The font of `#1' may be selected with element `\csname
        scr@fnt@instead@#1\endcsname'.\MessageBreak
        I fix this using font of `\csname
        scr@fnt@instead@#1\endcsname'%
      }%
    }%
    \@nameuse{scr@fnt@instead@#1}%
  }{%
    \@nameuse{scr@fnt@#1}%
  }%
}
%    \end{macrocode}
%  \end{macro}
%
% Um dann ein Element zu definieren, definiert man zun�chst ein
% Makro, das die Schriftart enth�lt. Dann definiert man
% \cs{scr@fnt@\emph{Elementname}}. Wobei der Inhalt dieses Makros das
% Makro ist, das die Schriftart speichert. Damit ist das Element
% definiert. Soll ein Element durch ein anderes Element gesteuert
% werden, so definiert man stattdessen
% \cs{scr@fnt@instead@\emph{Elementname}}, wobei der Inhalt des Makros
% dann der Name jenes anderen Elements ist. Soll beim �ndern der
% Schrift eines Elements eine zus�tzliche Warnung ausgegeben werden,
% so kann der Text dieser Warnung in
% \cs{scr@fnt@wrn@\emph{Elementname}} abgelegt werden. Also alles
% eigentlich ganz einfach. Damit es noch einfacher wird, ein paar 
% Hilfsmakros:
%
%  \begin{macro}{\newkomafont}
%    \changes{v3.0}{2004/07/21}{neu}
% Mit \cs{newkomafont} wird ein neues Element definiert. Das erste optionale
% Argument ist dabei die optionale Warnung. Ist dieses nicht gesetzt oder
% \cs{relax}, so wird keine Warnung definiert. Das erste obligatorische
% Argument ist der Name des Elements. Als letztes folgt die Voreinstellung f�r
% die Schrift dieses Elements. Als Fontmakro wird �brigens
% \cs{@\emph{Elementname}font} definiert.
%    \begin{macrocode}
\newcommand*{\newkomafont}[3][\relax]{%
  \expandafter\newcommand\expandafter*\csname @#2font\endcsname{#3}%
  \expandafter\newcommand\expandafter*\csname scr@fnt@#2\endcsname{%
    \csname @#2font\endcsname%
  }%
  \ifx\relax#1\relax\else
    \expandafter\newcommand\expandafter*\csname scr@fnt@wrn@#2\endcsname{#1}%
  \fi
}
%    \end{macrocode}
%  \begin{macro}{\aliaskomafont}
%    \changes{v3.0}{2004/07/21}{neu}
% Mit \cs{aliaskomafont} wird hingegen einfach ein Alias definiert. Dieser
% Befehlt kennt zwei Argumente: der Name des Elements und der Names des
% Elements, das stattdessen verwendet werden soll.
%    \begin{macrocode}
\newcommand*{\aliadkomafont}[2]{%
  \expandafter\newcommand\expandafter*\csname scr@fnt@instead@#1\endcsname{%
    #2%
  }%
}
%    \end{macrocode}
%  \end{macro}
%  \end{macro}
%
% \iffalse
%</body>
% \fi
%
% \Finale
%
\endinput
%
% end of file `scrkfont.dtx'
%%% Local Variables:
%%% mode: doctex
%%% TeX-master: t
%%% End:
