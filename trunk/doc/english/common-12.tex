% ======================================================================
% common-12.tex
% Copyright (c) Markus Kohm, 2001-2015
%
% This file is part of the LaTeX2e KOMA-Script bundle.
%
% This work may be distributed and/or modified under the conditions of
% the LaTeX Project Public License, version 1.3c of the license.
% The latest version of this license is in
%   http://www.latex-project.org/lppl.txt
% and version 1.3c or later is part of all distributions of LaTeX 
% version 2005/12/01 or later and of this work.
%
% This work has the LPPL maintenance status "author-maintained".
%
% The Current Maintainer and author of this work is Markus Kohm.
%
% This work consists of all files listed in manifest.txt.
% ----------------------------------------------------------------------
% common-12.tex
% Copyright (c) Markus Kohm, 2001-2015
%
% Dieses Werk darf nach den Bedingungen der LaTeX Project Public Lizenz,
% Version 1.3c, verteilt und/oder veraendert werden.
% Die neuste Version dieser Lizenz ist
%   http://www.latex-project.org/lppl.txt
% und Version 1.3c ist Teil aller Verteilungen von LaTeX
% Version 2005/12/01 oder spaeter und dieses Werks.
%
% Dieses Werk hat den LPPL-Verwaltungs-Status "author-maintained"
% (allein durch den Autor verwaltet).
%
% Der Aktuelle Verwalter und Autor dieses Werkes ist Markus Kohm.
% 
% Dieses Werk besteht aus den in manifest.txt aufgefuehrten Dateien.
% ======================================================================
%
% Paragraphs that are common for several chapters of the KOMA-Script guide
% Maintained by Markus Kohm
%
% ----------------------------------------------------------------------
%
% Absaetze, die mehreren Kapiteln der KOMA-Script-Anleitung gemeinsam sind
% Verwaltet von Markus Kohm
%
% ======================================================================

\KOMAProvidesFile{common-12.tex}
                 [$Date$
                  KOMA-Script guide (common paragraphs)]
\translator{Gernot Hassenpflug\and Markus Kohm\and Krickette Murabayashi}

% Date of the translated German file: 2015-03-31

\makeatletter
\@ifundefined{ifCommonmaincls}{\newif\ifCommonmaincls}{}%
\@ifundefined{ifCommonscrextend}{\newif\ifCommonscrextend}{}%
\@ifundefined{ifCommonscrlttr}{\newif\ifCommonscrlttr}{}%
\@ifundefined{ifIgnoreThis}{\newif\ifIgnoreThis}{}%
\makeatother


\section{Lists}
\seclabel{lists}%
\ifshortversion\IgnoreThisfalse\IfNotCommon{maincls}{\IgnoreThistrue}\fi%
\ifIgnoreThis %+++++++++++++++++++++++++++++++++++++++++++++ nicht maincls +
What is described in
\autoref{sec:maincls.lists} applies, mutatis mutandis.
\IfCommon{scrextend}{%
  However, \Package{scrextend} does support only the environments
  \Environment{labeling}, \Environment{addmargin} and
  \Environment{addmargin*}. All the other list environments may be supported
  and controlled by the used class.}
\else %------------------------------------------------------- nur maincls -
\BeginIndex{}{lists}%

Both {\LaTeX} and the standard classes offer different environments for
lists. Though slightly changed or extended all these list are of course
offered in {\KOMAScript} as well. In general, all lists\,---\,even of different
kind\,---\,can be nested up to four levels. From a typographical view,
anything more would make no sense, as more than three levels can no longer be
easily perceived. The recommended\textnote{Hint!} procedure in such a case is
to split the large list into several smaller ones.

\IfCommon{scrlttr2}{%
  Because lists are standard elements of \LaTeX{} this section abandons on
  examples. Nevertheless, you may find examples either in
  \autoref{sec:maincls.lists} from \autopageref{sec:maincls.lists} or in
  almost every introduction to \LaTeX.}

\ifCommonscrextend\else
\begin{Declaration}
  \XMacro{begin}\PParameter{\Environment{itemize}}\\
  \quad\Macro{item}\\
  \quad\dots\\
  \XMacro{end}\PParameter{itemize}\\
  \Macro{labelitemi}\\
  \Macro{labelitemii}\\
  \Macro{labelitemiii}\\
  \Macro{labelitemiv}
\end{Declaration}%
\BeginIndex{Env}{itemize}%
\BeginIndex{Cmd}{item}%
\BeginIndex{Cmd}{labelitemi}%
\BeginIndex{Cmd}{labelitemii}%
\BeginIndex{Cmd}{labelitemiii}%
\BeginIndex{Cmd}{labelitemiv}%
The simplest form of a list is an \Environment{itemize} list.
\iffalse % Umbruckoptimierungstext
The users of a certain disliked word processing package often refer to
this form of a list as \emph{bulletpoints}.  Presumably, these users
are unable to envisage that, depending on the level, a different
symbol from a large dot could be used to introduce each
point. %
\fi%
Depending on the level, {\KOMAScript} uses the following marks:
``{\labelitemi}'', ``{\labelitemii}'', ``{\labelitemiii}'' and
``{\labelitemiv}''. The definition of these symbols is specified in
the macros \Macro{labelitemi}, \Macro{labelitemii},
\Macro{labelitemiii} and \Macro{labelitemiv}, all of which can be
redefined using \Macro{renewcommand}. Every item is introduced with
\Macro{item}.%
%
\ifCommonmaincls
\begin{Example}
  \phantomsection\label{desc:maincls.env.itemize.example}%
  You have a simple list which is nested in several levels. You write
  for example:
\begin{lstcode}
  \minisec{Vehicles}
  \begin{itemize}
    \item aeroplanes
    \begin{itemize}
      \item biplane
      \item jets
      \item transport planes
      \begin{itemize}
        \item single-engined
        \begin{itemize}
          \item jet-driven
          \item propeller-driven
        \end{itemize}
        \item multi-engined
      \end{itemize}
      \item helicopters
    \end{itemize}
    \item automobiles
    \begin{itemize}
      \item racing cars
      \item private cars
      \item lorries
    \end{itemize}
    \item bicycles
  \end{itemize}
\end{lstcode}
  As output you get:
  \begin{ShowOutput}[\baselineskip]
  \minisec{Vehicles}
  \begin{itemize}
    \item aeroplanes
    \begin{itemize}
      \item biplanes
      \item jets
      \item transport planes
      \begin{itemize}
        \item single-engined
        \begin{itemize}
          \item jet-driven
          \item propeller-driven
        \end{itemize}
        \item multi-engined
      \end{itemize}
      \item helicopters
    \end{itemize}
% Falls fuer die Umbruchoptimierung erforderlich, kann die Liste
% beispielsweise so erweitert werden:
%    \item motorcycles
%    \begin{itemize}
%      \item historically accurate
%      \item futuristic, not real
%    \end{itemize}
    \item automobiles
    \begin{itemize}
      \item racing cars
      \item private cars
      \item lorries
    \end{itemize}
    \item bicycles
  \end{itemize}
  \end{ShowOutput}
\end{Example}
\fi% \ifCommonmaincls
%
\EndIndex{Env}{itemize}%
\EndIndex{Cmd}{item}%
\EndIndex{Cmd}{labelitemi}%
\EndIndex{Cmd}{labelitemii}%
\EndIndex{Cmd}{labelitemiii}%
\EndIndex{Cmd}{labelitemiv}%


\begin{Declaration}
  \XMacro{begin}\PParameter{\Environment{enumerate}}\\
  \quad\XMacro{item}\\
  \quad\dots\\
  \XMacro{end}\PParameter{enumerate}\\
  \Macro{theenumi}\\
  \Macro{theenumii}\\
  \Macro{theenumiii}\\
  \Macro{theenumiv}\\
  \Macro{labelenumi}\\
  \Macro{labelenumii}\\
  \Macro{labelenumiii}\\
  \Macro{labelenumiv}
\end{Declaration}%
\BeginIndex{Env}{enumerate}%
\BeginIndex{Cmd}{item}%
\BeginIndex{Cmd}{theenumi}%
\BeginIndex{Cmd}{theenumii}%
\BeginIndex{Cmd}{theenumiii}%
\BeginIndex{Cmd}{theenumiv}%
\BeginIndex{Cmd}{labelenumi}%
\BeginIndex{Cmd}{labelenumii}%
\BeginIndex{Cmd}{labelenumiii}%
\BeginIndex{Cmd}{labelenumiv}%
Another form of a list often used is a numbered list which is already
implemented by the {\LaTeX} kernel. Depending on the level, the
numbering\Index{numbering} uses the following characters: Arabic numbers,
small letters, small roman numerals, and capital letters. The kind of numbering
is defined with the macros \Macro{theenumi} down to \Macro{theenumiv}. The
output format is determined by the macros \Macro{labelenumi} to
\Macro{labelenumiv}. While the small letter of the second level is followed by
a round parenthesis, the values of all other levels are followed by a
dot. Every item is introduced with \Macro{item}.%
\ifCommonmaincls
\begin{Example}
  \phantomsection\label{desc:maincls.env.enumerate.example}%
  Replacing every occurrence of an \Environment{itemize} environment
  with an \Environment{enumerate} environment in the example above we
  get the following result:
  \begin{ShowOutput}[\baselineskip]
  \minisec{Vehicles}
  \begin{enumerate}
    \item aeroplanes
    \begin{enumerate}
      \item biplanes
      \item jets
      \item transport planes
      \begin{enumerate}
        \item single-engined
        \begin{enumerate}
          \item jet-driven\label{xmp:maincls.jets}
          \item propeller-driven
        \end{enumerate}
        \item multi-engined
      \end{enumerate}
      \item helicopters
    \end{enumerate}
% Wie oben:
%    \item motorcycles
%    \begin{enumerate}
%      \item historically accurate
%      \item futuristic, not real
%    \end{enumerate}
    \item automobiles
    \begin{enumerate}
      \item racing cars
      \item private cars
      \item lorries
    \end{enumerate}
    \item bicycles
  \end{enumerate}
  \end{ShowOutput}
  Using \Macro{label} within a list you can set labels which are
  referenced with \Macro{ref}. In the example above, a label was set
  after the jet-driven, single-engined transport planes with
  \Macro{label}\PParameter{xmp:jets}. The \Macro{ref} value is then
  \ref{xmp:maincls.jets}.
\end{Example}
\fi% \ifCommonmaincls
%
\EndIndex{Env}{enumerate}%
\EndIndex{Cmd}{item}%
\EndIndex{Cmd}{theenumi}%
\EndIndex{Cmd}{theenumii}%
\EndIndex{Cmd}{theenumiii}%
\EndIndex{Cmd}{theenumiv}%
\EndIndex{Cmd}{labelenumi}%
\EndIndex{Cmd}{labelenumii}%
\EndIndex{Cmd}{labelenumiii}%
\EndIndex{Cmd}{labelenumiv}%


\begin{Declaration}
  \XMacro{begin}\PParameter{\Environment{description}}\\
  \quad\XMacro{item}\OParameter{keyword}\\
  \quad\dots\\
  \XMacro{end}\PParameter{description}
\end{Declaration}%
\BeginIndex{Env}{description}%
\BeginIndex{Cmd}{item}%
A further list form is the description list. Its main use is the description
of several items. The item itself is an optional parameter in
\Macro{item}. The font\Index{font>style}\ChangedAt{v2.8p}{%
  \Class{scrbook}\and \Class{scrreprt}\and \Class{scrartcl}}%
 which is responsible for emphasizing the item can be changed with the commands
\Macro{setkomafont} and \Macro{addtokomafont} (see \autoref{sec:\csname
  label@base\endcsname.textmarkup}, \autopageref{desc:\csname
  label@base\endcsname.cmd.setkomafont}) for the element
\FontElement{descriptionlabel}\IndexFontElement{descriptionlabel} (see
\autoref{tab:\csname label@base\endcsname.elementsWithoutText},
\autopageref{tab:\csname label@base\endcsname.elementsWithoutText}). Default
setting is \Macro{sffamily}\linebreak[2]\Macro{bfseries}.%
\ifCommonmaincls
\begin{Example}
  \phantomsection\label{desc:maincls.env.description.example}%
  Instead of items in sans serif and bold, you want them printed in the
  standard font in bold. Using
\begin{lstcode}
  \setkomafont{descriptionlabel}{\normalfont\bfseries}
\end{lstcode}
  you redefine the font accordingly.

  An example for a description list is the output of the page styles
  listed in \autoref{sec:maincls.pagestyle}. The heavily
  abbreviated source could be:
\begin{lstcode}
  \begin{description}
  \item[empty] is the page style without any header or footer.
    \item[plain] is the page style without headings.
    \item[headings] is the page style with running headings.
    \item[myheadings] is the page style for manual headings.
  \end{description}
\end{lstcode}
  This abbreviated version gives:
  \begin{ShowOutput}
    \begin{description}
    \item[empty] is the page style without any header or footer.
    \item[plain] is the page style without headings.
    \item[headings] is the page style with running headings.
    \item[myheadings] is the page style for manual headings.
    \end{description}
  \end{ShowOutput}
\end{Example}
\fi% \ifComminmaincls
%
\EndIndex{Env}{description}%
\EndIndex{Cmd}{item}%
\fi % \ifCommonscrextend\else

\begin{Declaration}
  \XMacro{begin}\PParameter{\Environment{labeling}}%
  \OParameter{delimiter}\Parameter{widest pattern}\\
  \quad\XMacro{item}\OParameter{keyword}\\
  \quad\dots\\
  \XMacro{end}\PParameter{labeling}
\end{Declaration}%
\BeginIndex{Env}{labeling}%
\BeginIndex{Cmd}{item}%
An additional form of a description list is only available in the
{\KOMAScript} classes%
\IfCommon{scrextend}{ and the package \Package{scrextend}}: the
\Environment{labeling} environment. Unlike the
\Environment{description} environment, you can provide a pattern whose length
determines the indentation of all items. Furthermore, you can put an optional
\PName{delimiter} between the item and its description.  The
font\Index{font>style}\ChangedAt{v3.01}{%
  \Class{scrbook}\and \Class{scrreprt}\and \Class{scrartcl}\and
  \Package{scrextend}}%
 which is responsible for emphasizing the item and the separator can be changed
with the commands \Macro{setkomafont} and \Macro{addtokomafont} (see \autoref{sec:\csname
  label@base\endcsname.textmarkup}, \autopageref{desc:\csname
  label@base\endcsname.cmd.setkomafont}) for the element
\FontElement{labelinglabel}\IndexFontElement{labelinglabel} and
\FontElement{labelingseparator}\IndexFontElement{labelingseparator} (see
\ifCommonscrextend
\autoref{tab:maincls.elementsWithoutText},
\autopageref{tab:maincls.elementsWithoutText}
\else
\autoref{tab:\csname label@base\endcsname.elementsWithoutText},
\autopageref{tab:\csname label@base\endcsname.elementsWithoutText}%
\fi).
\ifCommonscrlttr\par\else
\begin{Example}
  \phantomsection\label{desc:\csname
    label@base\endcsname.env.labeling.example}%
  \IfCommon{maincls}{Slightly changing the example from the
    \Environment{description} environment, we could write:}%
  \IfCommon{scrextend}{A small example of a list like this may be written:}%
\begin{lstcode}
  \setkomafont{labelinglabel}{\ttfamily}
  \setkomafont{labelingseparator}{\normalfont}
  \begin{labeling}[~--]{myheadings}
    \item[empty]
      Page style without header and footer
    \item[plain]
      Page style for chapter beginnings without headings
    \item[headings]
      Page style for running headings
    \item[myheadings]
      Page style for manual headings
  \end{labeling}
\end{lstcode}
  As the result we get:
  \begin{ShowOutput}
    \setkomafont{labelinglabel}{\ttfamily}
    \setkomafont{labelingseparator}{\normalfont}
    \begin{labeling}[~--]{myheadings}
    \item[empty]
      Page style without header and footer
    \item[plain]
      Page style for chapter beginnings without headings
    \item[headings]
      Page style for running headings
    \item[myheadings]
      Page style for manual headings
    \end{labeling}
  \end{ShowOutput}
  As can be seen in this example, a font changing command may be set
  in the usual way. But if you do not want the font of the separator to
  be changed in the same way as the font of the label, you have to set
  the font of the separator as well.
\end{Example}
\fi% \ifCommonscrlttr \else
Originally, this environment was implemented for things like ``Precondition,
Assertion, Proof'', or ``Given, Required, Solution'' that are often used in
lecture hand-outs.  By now this environment has found many different
applications. For example, the environment for examples in this guide was
defined with the \Environment{labeling} environment.%
%
\EndIndex{Env}{labeling}%
\EndIndex{Cmd}{item}%


\ifCommonscrextend\else
\begin{Declaration}
  \XMacro{begin}\PParameter{\Environment{verse}}\\
  \quad\dots\\
  \XMacro{end}\PParameter{verse}
\end{Declaration}%
\BeginIndex{Env}{verse}%
Usually the \Environment{verse} environment is not perceived as a list
environment because you do not work with \Macro{item}
commands. Instead, fixed line breaks are used within the
\Environment{flushleft} environment. Yet internally in both the
standard classes as well as {\KOMAScript} it is indeed a list
environment.

In general, the \Environment{verse} environment is used for
poems\Index{poems}.  Lines are indented both left and
right. Individual lines of verse are ended by a fixed line break
\verb|\\|. Verses are set as paragraphs, separated by an empty
line. Often also \Macro{medskip}\IndexCmd{medskip} or
\Macro{bigskip}\IndexCmd{bigskip} is used instead. To avoid a page
break at the end of a line of verse you could, as usual, insert \verb|\\*|
instead of \verb|\\|.
\ifCommonmaincls
\begin{Example}
  \phantomsection\label{desc:maincls.env.verse.example}%
  As an example, the first lines of ``Little Red Riding Hood and the
  Wolf'' by Roald Dahl:
\begin{lstcode}
  \begin{verse}
    As soon as Wolf began to feel\\*
    that he would like a decent meal,\\*
    He went and knocked on Grandma's door.\\*
    When Grandma opened it, she saw\\*
    The sharp white teeth, the horrid grin,\\*
    And Wolfie said, `May I come in?'
  \end{verse}
\end{lstcode}
  The result is as follows:
  \begin{ShowOutput}
  \begin{verse}
    As soon as Wolf began to feel\\*
    That he would like a decent meal,\\*
    He went and knocked on Grandma's door.\\*
    When Grandma opened it, she saw\\*
    The sharp white teeth, the horrid grin,\\*
    And Wolfie said, `May I come in?'
  \end{verse}
  \end{ShowOutput}
  However, if you have very long lines of verse, for instance:
\begin{lstcode}
  \begin{verse}
    Both the philosopher and the house-owner
    have always something to repair.\\
    \bigskip
    Don't trust a man, my son, who tells you
    that he has never lied.
  \end{verse}
\end{lstcode}
  where a line break occurs within a line of verse:
\begin{ShowOutput}
  \begin{verse}
    Both the philosopher and the house-owner
    have always something to repair.\\
    \bigskip
    Don't trust a man, my son, who tells you
    that he has never lied.
  \end{verse}
\end{ShowOutput}
  there \verb|\\*| can not prevent a page break occurring within a verse
  at such a line break. To prevent such a page break, a
  \Macro{nopagebreak}\IndexCmd{nopagebreak} would have to be inserted
  somewhere in the first line:
\begin{lstcode}
  \begin{verse}
    Both the philosopher and the house-owner\nopagebreak
    have always something to repair.\\
    \bigskip
    Don't trust a man, my son, who tells you\nopagebreak
    that he has never lied.
  \end{verse}
\end{lstcode}

  In the above example, \Macro{bigskip} was used to separate the lines
  of verse.
\end{Example}
\fi % \ifCommonmaincls
%
\EndIndex{Env}{verse}%

\begin{Declaration}
  \XMacro{begin}\PParameter{\Environment{quote}}\\
  \quad\dots\\
  \XMacro{end}\PParameter{quote}\\
  \XMacro{begin}\PParameter{\Environment{quotation}}\\
  \quad\dots\\
  \XMacro{end}\PParameter{quotation}
\end{Declaration}%
\BeginIndex{Env}{quote}%
\BeginIndex{Env}{quotation}%
These two environments are also list environments and can be found
both in the standard and the {\KOMAScript} classes. Both environments
use justified text which is indented both on the left and right side.
Usually they are used to separate long citations\Index{citations} from
the main text. The difference between these two lies in the manner in
which paragraphs are typeset. While \Environment{quote} paragraphs are
highlighted by vertical space, in \Environment{quotation} paragraphs
the first line is indented. This is also true for the first line of a
\Environment{quotation} environment. To prevent indentation you have
to insert a \Macro{noindent} command\IndexCmd{noindent} before the
text.
\ifCommonmaincls
\begin{Example}
  \phantomsection\label{desc:maincls.env.quote.example}%
  You want to highlight a short anecdote. You write the following
  \Environment{quotation} environment for this:
  %
\begin{lstcode}
  A small example for a short anecdote:
  \begin{quotation}
    The old year was turning brown; the West Wind was
    calling;
        
    Tom caught the beechen leaf in the forest falling.
    ``I've caught the happy day blown me by the breezes!
    Why wait till morrow-year? I'll take it when me pleases.
    This I'll mend my boat and journey as it chances
    west down the withy-stream, following my fancies!''
    
    Little Bird sat on twig. ``Whillo, Tom! I heed you.
    I've a guess, I've a guess where your fancies lead you.
    Shall I go, shall I go, bring him word to meet you?''
  \end{quotation}
\end{lstcode}
  The result is:
  \begin{ShowOutput}
    A small example for a short anecdote:
    \begin{quotation}
    The old year was turning brown; the West Wind was
    calling;
    
    Tom caught the beechen leaf in the forest falling.
    ``I've caught the happy day blown me by the breezes!
    Why wait till morrow-year? I'll take it when me pleases.
    This I'll mend my boat and journey as it chances
    west down the withy-stream, following my fancies!''
    
    Little Bird sat on twig. ``Whillo, Tom! I heed you.
    I've a guess, I've a guess where your fancies lead you.
    Shall I go, shall I go, bring him word to meet you?''
    \end{quotation}
  \end{ShowOutput}
  %
  Using a \Environment{quote} environment instead you get:
  %
  \begin{ShowOutput}
    A small example for a short anecdote:
     \begin{quote}\setlength{\parskip}{4pt plus 2pt minus 2pt}
    The old year was turning brown; the West Wind was
    calling;

    Tom caught the beechen leaf in the forest falling.
    ``I've caught the happy day blown me by the breezes!
    Why wait till morrow-year? I'll take it when me pleases.
    This I'll mend my boat and journey as it chances
    west down the withy-stream, following my fancies!''
    
    Little Bird sat on twig. ``Whillo, Tom! I heed you.
    I've a guess, I've a guess where your fancies lead you.
    Shall I go, shall I go, bring him word to meet you?''
    \end{quote}
  \end{ShowOutput}
  %
\end{Example}
\fi % \ifCommonmaincls
%
\EndIndex{Env}{quote}%
\EndIndex{Env}{quotation}%
\fi % \ifCommonscrextend\else

\begin{Declaration}
  \XMacro{begin}\PParameter{\Environment{addmargin}}%
  \OParameter{left indentation}\Parameter{indentation}\\
  \quad\dots\\
  \XMacro{end}\PParameter{addmargin}\\
  \XMacro{begin}\PParameter{\Environment{addmargin*}}%
  \OParameter{inner indentation}\Parameter{indentation}\\
  \quad\dots\\
  \XMacro{end}\PParameter{addmargin*}
\end{Declaration}
\BeginIndex{Env}{addmargin}%
\BeginIndex{Env}{addmargin*}%
Similar to \Environment{quote} and \Environment{quotation}%
\IfCommon{scrextend}{ which are available at the standard classes and also the
  \KOMAScript{} classes}%
, the \Environment{addmargin} environment changes the margin\Index{margin}.
In contrast to the first two environments, with \Environment{addmargin} the
user can set the width of the indentation. Besides this, this environment does
not change the indentation of the first line nor the vertical spacing between
paragraphs.

If only the obligatory argument \PName{indentation} is given, both the
left and right margin are expanded by this value. If the optional
argument \PName{left indentation} is given as well, then at the left
margin the value \PName{left indentation} is used instead of
\PName{indentation}.

The starred \Environment{addmargin*} only differs from the normal
version in a two-sided layout. Furthermore, the difference only occurs
if the optional argument \PName{inner indentation} is used. In this
case this value \PName{inner indentation} is added to the normal inner
indentation. For right-hand pages this is the left margin, for
left-hand pages the right margin. Then the value of
\PName{indentation} determines the width of the opposite margin.

Both versions of this environment take also negative values for all
parameters. This has the effect of expanding the environment into the
margin.
\ifCommonscrlttr\else
\begin{Example}
  \phantomsection\label{desc:maincls.env.addmargin.example}%
\begin{lstcode}
  \newenvironment{SourceCodeFrame}{%
    \begin{addmargin*}[1em]{-1em}%
      \begin{minipage}{\linewidth}%
        \rule{\linewidth}{2pt}%
  }{%
      \rule[.25\baselineskip]{\linewidth}{2pt}%
      \end{minipage}%
    \end{addmargin*}%
  }
\end{lstcode}
  If you now put your source code in such an environment it will show
  up as:
  \begin{ShowOutput}
  \newenvironment{SourceCodeFrame}{%
    \begin{addmargin*}[1em]{-1em}%
      \begin{minipage}{\linewidth}%
        \rule{\linewidth}{2pt}%
  }{%
      \rule[.25\baselineskip]{\linewidth}{2pt}%
      \end{minipage}%
    \end{addmargin*}%
  }
  You define yourself the following environment:
  \begin{SourceCodeFrame}
\begin{lstcode}
\newenvironment{\SourceCodeFrame}{%
  \begin{addmargin*}[1em]{-1em}%
    \begin{minipage}{\linewidth}%
      \rule{\linewidth}{2pt}%
}{%
    \rule[.25\baselineskip]{\linewidth}{2pt}%
    \end{minipage}%
  \end{addmargin*}%
}
\end{lstcode}
  \end{SourceCodeFrame}
  This may be feasible or not. In any case it shows the usage of this
  environment.
  \end{ShowOutput}
  The optional argument of the \Environment{addmargin*} environment
  makes sure that the inner margin is extended by 1\Unit{em}. In turn
  the outer margin is decreased by 1\Unit{em}. The result is a shift
  by 1\Unit{em} to the outside.  Instead of \PValue{1em} you can of
  course use a length, for example, \PValue{2\Macro{parindent}}.
\end{Example}

Whether a page is going to be on the left or right side
of the book can not be determined for certain in the first {\LaTeX}
run.  For details please refer to the explanation of the commands
\Macro{ifthispageodd} (\autoref{sec:\csname
  label@base\endcsname.oddOrEven}, \autopageref{desc:\csname
  label@base\endcsname.cmd.ifthispageodd}) and \Macro{ifthispagewasodd}
(\autoref{sec:maincls-experts.addInfos},
\autopageref{desc:maincls-experts.cmd.ifthispageodd}).

\begin{Explain}
  There may be several questions about coexistence of lists and
  paragraphs. Because of this additional information may be found at the
  description of option \Option{parskip} in
  \autoref{sec:maincls-experts.addInfos},
  \autopageref{desc:maincls-experts.option.parskip}. Also at the expert part,
  in \autoref{sec:maincls-experts.addInfos},
  \autopageref{desc:maincls-experts.env.addmargin*}, you may find additional
  information about page breaks inside of \Environment{addmargin*}.%
\end{Explain}%
\fi% \ifCommonscrlttr\else
%
\EndIndex{Env}{addmargin}%
\EndIndex{Env}{addmargin*}%
%
\EndIndex{}{lists}
\fi %**************************************************** Ende nur maincls *


%%% Local Variables:
%%% mode: latex
%%% coding: us-ascii
%%% TeX-master: "../guide"
%%% End:
