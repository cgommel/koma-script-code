% ======================================================================
% scrbase.tex
% Copyright (c) Markus Kohm, 2002-2013
%
% This file is part of the LaTeX2e KOMA-Script bundle.
%
% This work may be distributed and/or modified under the conditions of
% the LaTeX Project Public License, version 1.3c of the license.
% The latest version of this license is in
%   http://www.latex-project.org/lppl.txt
% and version 1.3c or later is part of all distributions of LaTeX 
% version 2005/12/01 or later and of this work.
%
% This work has the LPPL maintenance status "author-maintained".
%
% The Current Maintainer and author of this work is Markus Kohm.
%
% This work consists of all files listed in manifest.txt.
% ----------------------------------------------------------------------
% scrbase.tex
% Copyright (c) Markus Kohm, 2002-2013
%
% Dieses Werk darf nach den Bedingungen der LaTeX Project Public Lizenz,
% Version 1.3c, verteilt und/oder veraendert werden.
% Die neuste Version dieser Lizenz ist
%   http://www.latex-project.org/lppl.txt
% und Version 1.3c ist Teil aller Verteilungen von LaTeX
% Version 2005/12/01 oder spaeter und dieses Werks.
%
% Dieses Werk hat den LPPL-Verwaltungs-Status "author-maintained"
% (allein durch den Autor verwaltet).
%
% Der Aktuelle Verwalter und Autor dieses Werkes ist Markus Kohm.
% 
% Dieses Werk besteht aus den in manifest.txt aufgefuehrten Dateien.
% ======================================================================
%
% Package scrbase for Package and Class Authors
% Maintained by Markus Kohm
%
% ----------------------------------------------------------------------
%
% Paket scrbase fuer Paket- und Klassenautoren
% Verwaltet von Markus Kohm
%
% ======================================================================

\ProvidesFile{scrbase.tex}[2013/03/05 KOMA-Script package scrbase]
\translator{Markus Kohm}

% Date of the translated German file: 2013/03/05

\chapter{Basic Functions at Package \Package{scrbase}}
\labelbase{scrbase}

\BeginIndex{Package}{scrbase}%

The package \Package{scrbase} provides basic features, that are designed and
implemented to be used by authors of packages and classes. Thereby it may not
only used for wrapper classes, that use a \KOMAScript{} class. Authors of
classes, that aren't related to anything else from \KOMAScript{}, may also
benefit from the functionality of \Package{scrbase}.

\section{Loading the Package}
\label{sec:scrbase.loadit}

Whereas users load packages using \Macro{usepackage}, authors of packages or
classes should use \Macro{RequirePackage}\IndexCmd{RequirePackage}. Authors of
wrapper packages may also use \Macro{RequirePackageWithOptions}. Command
\Macro{RequirePackage} has the same optional argument for package options like
\Macro{usepackage}. In opposite to this \Macro{RequirePackageWithOptions}
doesn't have an optional argument but passes all options given while loading
the wrapper package to the required package. See \cite{latex:clsguide} for
more information about these commands.

The package \Package{scrbase} needs the functionality of package
\Package{keyval}\IndexPackage{keyval} internally. This may be provided by
package \Package{xkeyval} alternatively. Package \Package{scrbase} loads
\Package{keyval} as needed.

The package \Package{keyval} provides definition of keys and assignment of
values to this keys. The options provided by \Package{scrbase} also use
\Package{keyval} syntax: \PName{key}\texttt{=}\PName{value}.

\begin{Declaration}
  \KOption{internalonly}\PName{value}
\end{Declaration}
\BeginIndex{Option}{internalonly~=\PName{value}}%
Package \Package{scrbase} provides some commands for conditional
execution. The primary names of those are build like
\Macro{scr@\PName{name}}. With this those are internal commands. \KOMAScript{}
really uses this internal commands internally. Authors of packages and classes
may use those internal commands too, but should not redefine them. Because
some of those commands may be useful for users too, they are also provided as
\Macro{\PName{name}} normally. But eventually other packages provide commands
with the same name but different syntax or different functionality. This would
result in an conflict. So \Package{scrbase} provides to suppress the
definition of the user commands, \Macro{\PName{name}}, only. Using option
\Option{internalonly} without \PName{value} will define only the internal
commands and suppress definition of all the user commands for conditional
execution. Alternatively, the user may give all the commands, that shouldn't
be defined as \PName{value}, but replaces ``\Macro{}'' by ``\texttt{/}''.

Authors of packages and classes normally should not use this option. Users may
use it with or without \PName{value} either as a global option with
\Macro{documentclass} or using \Macro{PassOptionsToPackage}.
\begin{Example}
  The user doesn't want \Package{scrbase} to define commands \Macro{ifVTeX}
  and \Macro{ifundefinedorrelax}. Because of this, user uses:
\begin{lstcode}
  \documentclass%
    [internalonly=/ifVTeX/ifundefinedorrelax]%
    {foo}
\end{lstcode}
  to load the class. Class name \Class{foo} has been used as an placeholder
  for any class in this example. The meanings of commands \Macro{ifVTeX} and
  \Macro{ifundefinedorrelax} and many more commands for conditional execution
  may be found in \autoref{sec:scrbase.if}.
\end{Example}
%
\EndIndex{Option}{internalonly~=\PName{value}}%


\section{Keys as Attributes of Families and their Members}
\label{sec:scrbase.keyvalue}

As already mentioned in \autoref{sec:scrbase.loadit}, \Package{scrbase} uses
package \Package{keyval} for keys and values of keys. Nevertheless
\Package{scrbase} extends the functionality of \Package{keyval}. Whereas only
one family owns all keys of \Package{keyval}, \Package{scrbase} knows also
family members. Now, a key may be owned by a family or by one or more family
members. Additionally a value may be assigned to the key of a family member,
of a family or of all family members.

\begin{Declaration}
  \Macro{DefineFamily}\Parameter{family}\\
  \Macro{DefineFamilyMember}\OParameter{family member}\Parameter{family}%
\end{Declaration}
\BeginIndex{Cmd}{DefineFamily}%
\BeginIndex{Cmd}{DefineFamilyMember}%
\Package{scrbase} needs to know the members of a family for different
reasons. So it's necessary first to define a new family using
\Macro{DefineFamily}, that produces also an empty member list. If the family
has already been defined nothing would happen. Nothing means also, that an
already existing member list would not be overwritten.

A new member may be added to the family using \Macro{DefineFamilyMember}. If
the family doesn't exist, this would result in an error message. If the member
already exists, nothing happens. If the member is omitted, the member won't
stay empty, but ``\texttt{.}\Macro{@currname}\texttt{.}\Macro{@currext}'' will
be assumed. While loading a class or package \Macro{@currname} and
\Macro{@currext} together represent the file name of the class or package.

Theoretically is would be possible, to define a member without a name using an
empty optional argument \PName{family member}. But this would be same like the
family itself. It is recommended to use only letters and digits at the
\PName{family} an start the \PName{family member} with another char like a
period. Otherwise it could happen, that members of one family are same like
members of another family.

\Package{scrbase} itself defines family ``\PValue{KOMA}'' and adds member
``\PValue{.scrbase.sty}'' to is. Generally family ``\PValue{KOMA}'' is
reserved to \KOMAScript{}. For your own packages it is recommended to use the
name of the bundle as \PName{family} and the name of the package as
\PName{family member} of that \PName{family}.
%
\begin{Example}
  Assumed you're writing a bundle ``master butcher''. Within that bundle you
  have packages \File{salami.sty}, \File{liversausage.sty}, and
  \File{kielbasa.sty}. Therefor you decide to use family name
  ``\PValue{butcher}'' and add the lines
\begin{lstcode}
  \DefineFamily{butcher}
  \DefineFamilyMember{butcher}
\end{lstcode}
  to each of the package files. While loading the three packages this will
  all the members ``\PValue{.salami.sty}'', ``\PValue{.liversausage.sty}'',
  and ``\PValue{.kielbasa.sty}'' to the family ``\PValue{butcher}''. After
  loading all three packages, all three member will be defined.
\end{Example}
%
\EndIndex{Cmd}{DefineFamilyMember}%
\EndIndex{Cmd}{DefineFamily}%


\begin{Declaration}
  \Macro{DefineFamilyKey}\OParameter{family member}\Parameter{family}%
  \Parameter{key}\OParameter{default}\\
  \hphantom{\Macro{DefineFamilyKey}}\Parameter{action}\\
  \Macro{FamilyKeyState}\\
  \Macro{FamilyKeyStateUnknown}\\
  \Macro{FamilyKeyStateProcessed}\\
  \Macro{FamilyKeyStateUnknownValue}\\
  \Macro{FamilyKeyStateNeedValue}
\end{Declaration}
\BeginIndex{Cmd}{DefineFamilyKey}%
\BeginIndex{Cmd}{FamilyKeyState}%
\BeginIndex{Cmd}{FamilyKeyStateUnknown}%
\BeginIndex{Cmd}{FamilyKeyStateProcessed}%
\BeginIndex{Cmd}{FamilyKeyStateUnknownValue}%
\BeginIndex{Cmd}{FamilyKeyStateNeedValue}%
This command defines a \PName{key}. If a \PName{family member} is given, the
\PName{key} will become an attribute of that member of the also given
\PName{family}. If no \PName{family member} is given, the member
``\texttt{.}\Macro{@currname}\texttt{.}\Macro{@currext}'' will be assumed. If
later a value will be assigned to the \PName{key}, the \PName{action} will be
executed and the value will be an argument of this. So inside of
\PName{action} ``\lstinline{#1}'' would be that value. If the value will be
omitted, the \PName{default} will be used instead. If there's no
\PName{default}, the \PName{key} can be used only with value.

\begin{Explain}
  \phantomsection\label{explain:scrbase.macro.DefineFamilyKey}%
  At least
\begin{lstcode}[escapeinside=`']
  \DefineFamilyKey[`\PName{member}']{`\PName{family}'}{`\PName{key}'}
                  [`\PName{default}']{`\PName{action}'}
\end{lstcode}
  will result in a call of
\begin{lstcode}[moretexcs={define@key},escapeinside=`']
  \define@key{`\PName{family\,member}'}{`\PName{key}'}
             [`\PName{default}']{`\PName{extended action}'}
\end{lstcode}
  with \Macro{define@key} is provided by package
  \Package{keyval}\IndexPackage{keyval} (see \cite{package:keyval}). However,
  the call of \Macro{define@key} and the \PName{action} will in fact be
  extended by additional arrangements.
\end{Explain}

Success\ChangedAt{v3.12}{\Package{scrbase}} or failure of the execution of the
\PName{action} should be reported back to \Package{scrbase} by
\Macro{FamilyKeyState}. The package itself will take care for additional
procedures if needed. You should not report errors by yourself! The default
state before execution of \PName{action} is
\Macro{FamilyKeyStateUnknown}. This signals, that it is not known whether or
not the execution is successful. If this state doesn't change until end of
execution of the \PName{action}, \Package{scrbase} will write a message into
the \File{log} file and supposes state \Macro{FamilyKeyStateProcessed} during
the further procedure.

State \Macro{FamilyKeyStateProcessed} signals, that the option and the value
assignment to the option are completely and successfully finished. You may
switch to this state by using \Macro{FamilyKeyStateProcessed} itself simply.

State \Macro{FamilyKeyStateUnknownValue} indicates, that the option was
handled, but the value, that should be assigned to the key, was unknown or not
allowed. You may use \Macro{FamilyKeyStateUnknownValue} itself to switch to
this state.

State \Macro{FamilyKeyStateNeedValue} signals, that the option couldn't be
set, because it needs a value, but not value have been assigned to the
key. This state will be used automatically, whenever an option, that has been
defined without \PName{default} value, will be used without value
assignment. You could also set the state using \Macro{FamilyKeyStateNeedValue}
yourself, but you shouldn't.

Last but not least you may switch to additional failure states, simply
re-defining \Macro{FamilyKeyState} with a very short message
text. Generally the four predefined states should be sufficient.

\begin{Example}
  Assumed, each of the three packages from the previous example should get a
  key \PValue{coldcuts}. If this is used, a switch should be set at each of
  the packages. For package \Package{salami} this may be, e.\,g.,
\begin{lstcode}
  \newif\if@Salami@Aufschnitt
  \DefineFamilyKey{butcher}%
                  {coldcut}[true]{%
    \expandafter\let\expandafter\if@salami@coldcut
    \csname if#1\endcsname
    \FamilyKeyStateProcessed
  }
\end{lstcode}
  Available values for the key are \PValue{true} or \PValue{false} in this
  case. Instead of testing on inappropriate values, success will be signaled
  in any case in this example. If the key will be used later, this has to be
  done with one of the allowed values or without assignment. In the second
  case the default \PName{true} will be used.

  The definitions in the other packages are similar. Only ``\texttt{salami}''
  has to be replaced by the corresponding names.
\end{Example}
%
\EndIndex{Cmd}{FamilyKeyStateNeedValue}%
\EndIndex{Cmd}{FamilyKeyStateUnknownValue}%
\EndIndex{Cmd}{FamilyKeyStateProcessed}%
\EndIndex{Cmd}{FamilyKeyStateUnknown}%
\EndIndex{Cmd}{FamilyKeyState}%
\EndIndex{Cmd}{DefineFamilyKey}%


\begin{Declaration}
  \Macro{FamilyProcessOptions}\OParameter{family member}\Parameter{family}
\end{Declaration}
\BeginIndex{Cmd}{FamilyProcessOptions}%
Generally the extension of keys of families to keys of families and family
members was mentioned to use keys or key-value settings as class or package
options. This command therefor is an extension of \Macro{ProcessOption*} from
\LaTeX{} kernel (see \cite{latex:clsguide}. Thereby the command processes not
only options that has been declared using \Macro{DeclareOption}. It processes
even all keys of the given family member. If the optional argument
\PName{family member} is omitted, family member
``\texttt{.}\Macro{@currname}\texttt{.}\Macro{@currext}'' will be used.

Somehow special are keys, that are not attached to a family member, but to a
family. These are keys with an empty family member. Such keys will be set also
and before the keys of the family members.
\begin{Example}
  If a package in the previous example would be extended by the line
\begin{lstcode}
  \FamilyProcessOptions{butcher}
\end{lstcode}
  then the user may select the option \Option{coldcut} while loading the
  package. If the option will be used globally, this means at the optional
  argument of \Macro{documentclass}, then the option would be passed
  automatically to all three packages, if all three packages will be loaded
  later.
\end{Example}
Please note\textnote{Attention!} that packages process global options always
before local options, that has been assigned locally to the package. In
opposite to unknown options while processing of global options, that will only
result in an information in the \File{log}-file, unknown options assigned to
the package result in error messages.

\Macro{FamilyProcessOptions} may be interpreted either as an extension of
\Macro{ProcessOption*} or as an extension of the \PName{key=value} mechanism
of \Package{keyval}. Finally \PName{key=value} pairs become options with help
of \Macro{FamilyProcessOptions}.%
%
\EndIndex{Cmd}{FamilyProcessOptions}%


\begin{Declaration}
  \Macro{FamilyExecuteOptions}\OParameter{family member}\Parameter{family}%
  \Parameter{options list}
\end{Declaration}
\BeginIndex{Cmd}{FamilyExecuteOptions}%
This command is an extension of \Macro{ExecuteOptions} from the \LaTeX{}
kernel (see \cite{latex:clsguide}). Thereby the command processes not only
options, that has been defined using \Macro{DeclareOption}. Also all keys of
the given family member will be processed. If the optional argument
\Macro{family member} is omitted, then
``\texttt{.}\Macro{@currname}\texttt{.}\Macro{@currext}'' is used.

Somehow special are keys, that are not attached to a family member, but to a
family. These are keys with an empty family member. Such keys will be set also
and before the keys of the family members.
\begin{Example}
  Assumed, option \Option{coldcut} should be set by default already in the
  previous example. In this case only line
\begin{lstcode}
  \FamilyExecuteOptions{butcher}{coldcut}
\end{lstcode}
  has to be added.
\end{Example}
%
\EndIndex{Cmd}{FamilyExecuteOptions}%


\begin{Declaration}
  \Macro{FamilyOptions}\Parameter{family}\Parameter{options list}%
\end{Declaration}
\BeginIndex{Cmd}{FamilyOptions}%
Thereby \PName{options list} is like:
\begin{flushleft}\vskip\dp\strutbox\begin{tabular}{l}
    \PName{key}\texttt{=}\PName{value}\texttt{,}%
    \PName{key}\texttt{=}\PName{value}\dots
\end{tabular}\vskip\dp\strutbox\end{flushleft}
whereby the value assignment may be omitted for \PName{key}s, that have a
defined default.

In opposite to average options, that has been defined using
\Macro{DeclareOption}, the \PName{key}s may also be set after loading a class
or package. For this the user uses \Macro{FamilyOptions}. Thereby the
\PName{key}s of all members of the given family will be set. If a \PName{key}
also exists as an family attribute, then the family key will be set
first. After this the member keys will follow in the order, the members has
been defined. If a given \PName{key} does exist neither in the family nor in
any member of the family, then \Macro{FamilyOptions} will result in an
error. Package \Package{scrbase} reports an error also, if there are members
with key \PName{key}, but all those members signal failure via
\Macro{FamilyKeyState}.
\begin{Example}
  You extend your butcher project by a package \Package{sausagesalad}. If this
  package has been loaded, all sausage package should generate cold cut:
\begin{lstcode}
  \ProvidesPackage{sausagesalad}%
                  [2008/05/06 nonsense package]
  \DefineFamily{butcher}
  \DefineFamilyMember{butcher}
  \FamilyProcessOptions{butcher}\relax
  \FamilyOptions{butcher}{coldcut}
\end{lstcode}
  If currently non of the sausage packages has been loaded, then an error
  message would result because of undefined option ``\Option{coldcut}''. Dies
  may be avoided changing the last line of the previous code into:
\begin{lstcode}[moretexcs={Family@Options}]
  \Family@Options{butcher}{coldcut}{}
\end{lstcode}
  Nevertheless, sausage packages, that will be loaded after
  \Package{sausagesalad}, won't produce cold cut. This may be changed
  additionally, by changing the last line again:
\begin{lstcode}[moretexcs={Family@Options}]
  \AtBeginDocument{%
    \Family@Options{butcher}{coldcut}{%
      \PackageWarning{sausagesalad}{%
        sausage salad needs at least 
        one sausage package}%
    }%
  }%
\end{lstcode}
  This will also throw a warning message, if non of the sausage packages will
  be loaded.
\end{Example}
%
\EndIndex{Cmd}{FamilyOptions}%


\begin{Declaration}
  \Macro{FamilyOption}\Parameter{family}%
                      \Parameter{option}\Parameter{values list}%
\end{Declaration}
\BeginIndex{Cmd}{FamilyOption}%
Beside options that have concurrently excluding values, there may be options,
that have several values at the same time. Using \Macro{FamilyOptions} for
that kind of options would result in using the same option several times with
different value assignments. Instead of this, \Macro{FamilyOption} may be used
to assign a whole \PName{values list} to the same \PName{option}. The
\PName{values list} is a comma separated list of values:
\begin{flushleft}\begin{tabular}{l}
    \PName{value}\texttt{,}\PName{value}\dots
\end{tabular}\end{flushleft}
By the way please note, that usage of a comma inside a value may be done only,
if the value is put into braces. The general functionality of this command
is the same like that of the previous command \Macro{FamilyOptions}.
\begin{Example}
  Package \Package{sausagesalad} should have one mire option, to add
  additional ingredients. Each of the ingredients will set a switch like done
  before for the cold cut.
\begin{lstcode}
  \newif\if@saladwith@onions
  \newif\if@saladwith@gherkins
  \newif\if@saladwith@chillies
  \DefineFamilyKey{butcher}{ingredient}{%
    \csname @saladwith@#1true\endcsname
  }
\end{lstcode}
  Here the three ingredients ``onions'', ``gherkins'', and ``chillies'' has
  been defined. An error message for not defined ingredients doesn't exist.

  For a salad with onions and gherkins the user may use
\begin{lstcode}
  \FamilyOptions{butcher}{%
    ingredient=onions,ingredient=gherkins}
\end{lstcode}
  or shorter
\begin{lstcode}
  \FamilyOption{butcher}
               {ingredient}{onions,gherkins}
\end{lstcode}
\end{Example}
%
\EndIndex{Cmd}{FamilyOption}%


\begin{Declaration}
  \Macro{AtEndOfFamilyOptions}\Parameter{action}%
\end{Declaration}
\BeginIndex{Cmd}{AtEndOfFamilyOptions}%
Sometimes\ChangedAt{v3.12}{\Package{scrbase}} it's useful to delay the
execution of an \PName{action}, that is part of a value assignment to a key,
until all assignments inside one \Macro{FamilyProcessOptions},
\Macro{FamilyExecuteOptions}, \Macro{FamilyOptions}, or \Macro{FamilyOption}
has been finished. This may be done, using \Macro{AtEndOfFamilyOptions} inside
an option definition. Reporting failure states of \PName{action} isn't
possible in this case. Furthermore, the command shouldn't be used outside an
option definition.
%
\EndIndex{Cmd}{AtEndOfFamilyOptions}%


\begin{Declaration}
  \Macro{FamilyBoolKey}\OParameter{family member}\Parameter{family}%
                       \Parameter{key}\Parameter{switch name}\\
  \Macro{FamilySetBool}\Parameter{family}%
                       \Parameter{key}\Parameter{switch name}\Parameter{value}
\end{Declaration}
\BeginIndex{Cmd}{FamilyBoolKey}%
\BeginIndex{Cmd}{FamilySetBool}%
In the previous examples boolean switches often have been used. In the example
with option \Option{coldcut} is was necessary, that the user assigns either
value \PValue{true} or value \PValue{false}. No error message existed on wrong
value assignment. Because of such boolean switches are an often needed
feature, \Package{scrbase} provides \Macro{FamilyBoolKey} for definition of
such options. Thereby the arguments \PName{family member}, \PName{family},
and \PName{key} are same like used by \Macro{DefineFamilyKey} (see
\autopageref{desc:scrbase.cmd.DefineFamilyKey}). Argument \PName{switch name}
is the name of the switch without the prefix \Macro{if}. If a switch with this
name doesn't exist already, \Macro{FamilyBoolKey} will define it and
initialize it to \PName{false}. Internally \Macro{FamilyBooKey} uses
\Macro{FamilySetBool} as \PName{action} of \Macro{DefineFamilyKey}. The
\PName{default} for those options is always \PValue{true}.

\Macro{FamilySetBool} on the other side understands \PName{value} \PValue{on}
and \PValue{yes} beside \PName{true} for switching on and \PName{off} and
\PName{no} beside \PName{false} for switching off. Unknown values will result
in a call of \Macro{FamilyUnknownKeyValue} with the arguments \PName{family},
\PName{key}, and \PName{value}, which results in setting of
\Macro{FamilyKeyState}. Depending on the used command and other family
members, may this result in an error message about unknown value assignment
(see also \autopageref{desc:scrbase.cmd.FamilyUnkownKeyValue} and
\autopageref{desc:scrbase.cmd.FamilyKeyState}).
\begin{Example}
  The key \Option{coldcut} should be declared somehow more robust. Additionally
  all sausage packages should use the same key. So either all or none of them
  will produce cold cut.
\begin{lstcode}
  \FamilyBoolKey{butcher}{coldcut}%
                         {@coldcut}
\end{lstcode}
  A test, whether or not to produce cold cut, may be:
\begin{lstcode}
  \if@coldcut
     ...
  \else
     ...
  \fi
\end{lstcode}
  This would be the same in each of the three sausage packages. With this the
  attribute ``coldcut'' may be defined as a family option:
\begin{lstcode}[moretexcs={define@key}]
  \@ifundefined{if@coldcut}{%
    \expandafter\newif\csname if@coldcut\endcsname
  }{}%
  \DefineFamilyKey[]{butcher}{coldcut}[true]{%
    \FamilySetBool{butcher}{coldcut}%
                           {@coldcut}%
                           {#1}%
  }
\end{lstcode}
  or shorter:
\begin{lstcode}
  \FamilyBoolKey[]{butcher}{coldcut}%
                           {@coldcut}
\end{lstcode}
  using the additional information at
  \autopageref{explain:scrbase.macro.DefineFamilyKey}, that's not only valid
  for \Macro{DefineFamilyKey} but for \Macro{FamilyBoolKey} too.
\end{Example}
%
\EndIndex{Cmd}{FamilySetBool}%
\EndIndex{Cmd}{FamilyBoolKey}


\begin{Declaration}
  \Macro{FamilyNumericalKey}\OParameter{family member}\Parameter{family}%
                            \Parameter{key}\\
  \hphantom{\XMacro{FamilyNumericalKey}}%
                            \OParameter{default}\Parameter{command}%
                            \Parameter{values list}\\
  \Macro{FamilySetNumerical}\Parameter{family}\Parameter{key}\\
  \hphantom{\XMacro{FamilySetNumerical}}%
                            \Parameter{command}\Parameter{values list}%
                            \Parameter{value}
\end{Declaration}
\BeginIndex{Cmd}{FamilyNumericalKey}%
\BeginIndex{Cmd}{FamilySetNumerical}%
In opposite to switches that may be either true or false, also key exists,
that accept several values. For example an alignment may not only be left or
not left, but, e.\,g., left, centered, or right. Internally such
differentiations are often made using \Macro{ifcase}. This \TeX{} command
expects a numerical value. Because of this the command to define a macro by
a \PName{key} has been named \Macro{FamilyNumericalKey} in
\Package{scrbase}. The \PName{values list} thereby has the form:
\begin{flushleft}\vskip\dp\strutbox\begin{tabular}{l}
    \Parameter{value}\Parameter{definition}\texttt{,}%
    \Parameter{value}\Parameter{definition}\texttt{,}%
    \dots
\end{tabular}\vskip\dp\strutbox\end{flushleft}
So the \PName{values list} does not only define the supported values to the
\PName{key}. For each of the \PName{value}s it also gives the
\PName{definition} of macro \Macro{\PName{command}}. Usually
\PName{definition} is just a numerical value. Nevertheless other content is
possible and allowed. Currently the only limitation is, that
\PName{definition} has to be fully expandable and will be expanded while the
assignment.
\begin{Example}
  The sausage may be cut different. For example the cold cut may stay uncut or
  will by cut roughly or may be cut thinly. This information should be stored
  in command \Macro{cuthow}.
\begin{lstcode}
  \FamilyNumericalKey{butcher}%
                     {saladcut}{cuthow}{%
                       {none}{none},{no}{none},{not}{none}%
                       {rough}{rough},%
                       {thin}{thin}%
                     }
\end{lstcode}
  Not cutting anything may be selected either by
\begin{lstcode}
  \FamilyOptions{butcher}{saladcut=none}
\end{lstcode}
  or
\begin{lstcode}
  \FamilyOptions{butcher}{saladcut=no}
\end{lstcode}
  or
\begin{lstcode}
  \FamilyOptions{butcher}{saladcut=not}
\end{lstcode}
  In all three cases \Macro{cuthow} would be defined with content
  \PValue{none}. It may be very useful to provide several values for the same
  result like shown in this example.

  Now, it's most likely, that the kind of cutting will not be printed, but
  should be evaluated later. In this case a textual definition wouldn't be
  useful. If the key is defined like this:
\begin{lstcode}
  \FamilyNumericalKey{butcher}%
                     {saladcut}{cuthow}{%
                       {none}{0},{no}{0},{not}{0}%
                       {rough}{1},%
                       {thin}{2}%
                     }
\end{lstcode}
  then a condition like this:
\begin{lstcode}
  \ifcase\cuthow
    % no cut
  \or
    % rough cut
  \else
    % thin cut
  \fi
\end{lstcode}
  may be used.
\end{Example}
Internally \Macro{FamilyNumericalKey} uses \Macro{FamilySetNumerical} as
\PName{action} of \Macro{DefineFamilyKey}. If a unknown value is assigned to
such a key, \Macro{FamilySetNumerical} will call \Macro{FamilyUnkownKeyValue}
with the arguments \PName{family}, \PName{key} and \PName{value}. This will
normally result in an error message about assigning an unknown value.
%
\EndIndex{Cmd}{FamilySetNumerical}%
\EndIndex{Cmd}{FamilyNumericalKey}%


\begin{Declaration}
  \Macro{FamilyStringKey}\OParameter{family member}\Parameter{family}%
                         \Parameter{key}\\
  \hphantom{\XMacro{FamilyStringKey}}%
                         \OParameter{default}\Parameter{command}
\end{Declaration}
\BeginIndex{Cmd}{FamilyStringKey}%
This defines\ChangedAt{v3.08}{\Package{scrbase}} a \PName{key}, that accepts
every value. The value will be stored into the given \Macro{command}. If their
is no optional argument for the \PName{default}, \Macro{FamilyStringKey} is
the same like
\begin{quote}
\Macro{DefineFamilyKey}\OParameter{family member}\Parameter{family}%
\Parameter{key}\\
\hphantom{\Macro{DefineFamilyKey}}%
\PParameter{\Macro{def}\PName{command}\string{\#1\string}%
\Macro{FamilyKeyStateProcessed}}.
\end{quote}
If an optional argument \PName{default} has been used, \Macro{FamilyStringKey}
is the same like
\begin{quote}
\Macro{DefineFamilyKey}\OParameter{family member}\Parameter{family}%
\Parameter{key}\\
\hphantom{\Macro{DefineFamilyKey}}%
\OParameter{default}%
\PParameter{\Macro{def}\PName{command}\string{\#1\string}}.
\end{quote}
If \PName{command} hasn't been defined already, an empty macro will be
defined.
\begin{Example}
  By default an amount of 250\Unit{g} sausage salad should be produced. The
  amount should be configurable by an option. The wanted amount will be stored
  in the macro \Macro{saladweight}. The option should be named
  \PValue{saladweight} too:
\begin{lstcode}
  \newcommand*{\saladweight}{250g}
  \FamilyStringKey{butcher}%
                  {saladweight}[250g]{\saladweight}
\end{lstcode}
  To switch back to the default weight after changing it, the user may simply
  use the option without weight:
\begin{lstcode}
  \FamilyOptions{butcher}{saladweight}
\end{lstcode}
  That may be done, because the default weight has been set as default at the
  definition of the option above.
\end{Example}
In this case there are no unknown values, because all values are simply used
for a macro definition. Nevertheless, you should note, that paragraph breaks
at the value assignment to the key are not allowed.
%
\EndIndex{Cmd}{FamilyStringKey}%


\begin{Declaration}
  \Macro{FamilyUnkownKeyValue}\Parameter{family}\Parameter{key}%
                              \Parameter{value}\Parameter{values list}%
\end{Declaration}
\BeginIndex{Cmd}{FamilyUnkownKeyValue}%
\BeginIndex{Cmd}{FamilyElseValues}%
The command \Macro{FamilyUnknownKeyValue} throws and error message about
unknown values assigned to a known key. The \PName{values list} is a comma
separated list of allowed values in the form:
\begin{flushleft}\vskip\dp\strutbox\begin{tabular}{l}
    `\PName{value}'\texttt{,} `\PName{value}' \dots
\end{tabular}\vskip\dp\strutbox\end{flushleft}
Certainly\ChangedAt{v3.12}{\Package{scrbase}} \PName{values list} is currently
not used by \Package{scrbase}.
\begin{Example}
  Now, for the cold cut the choices should be cut or no cut and in case of cut
  rough or thin. Rough should be the default for cutting.
\begin{lstcode}
  \@ifundefined{if@thincut}{%
    \expandafter
    \newif\csname if@thincut\endcsname}{}%
  \@ifundefined{if@coldcut}{%
    \expandafter
    \newif\csname if@coldcut\endcsname}{}
  \DefineFamilyKey{butcher}%
                  {coldcut}[true]{%
    \FamilySetBool{butcher}{coldcut}%
                           {@coldcut}%
                           {#1}%
    \ifx\FamilyKeyState\FamilyKeyStateProcessed
      \@thincutfalse
    \else
      \ifstr{#1}{thin}{%
        \@coldcuttrue
        \@thincuttrue
        \FamilyKeyStateProcessed
      }{}%
    \fi
  }%
\end{lstcode}
  Let's have a look at the definition of \Option{butcher}:
  First of all, we try to set the boolean switch of cold cut using
  \Macro{FamilySetBool}. After this command we test, whether or not
  \Macro{FamilyKeyState} signals the success of the command with state
  \Macro{FamilyKeyStateProcessed}. If so, only the thin cut has to be
  deactivated. In the other case, the value will be tested to be equal to
  \PValue{thin}. In that case, cold cut and thin cut are activated and the
  state will be switched to \Macro{FamilyKeyStateProcessed}. If the last test
  failed also, the failure state of \Macro{FamilySetBool} is still valid at
  the end of the execution.

  Command \Macro{ifstr}, that has been used for the thin test, will be
  described at \autopageref{desc:scrbase.cmd.ifstr} in
  \autoref{sec:scrbase.if}.
\end{Example}
%
\EndIndex{Cmd}{FamilyUnkownKeyValue}%

\begin{Declaration}
  \Macro{FamilyElseValues}
\end{Declaration}
\BeginIndex{Cmd}{FamilyElseValues}%
With former releases\ChangedAt{v3.12}{\Package{scrbase}} of \Package{scrbase}
additional allowed values reported by \Macro{FamilyUnknownKeyValue} could be
set re-defining \Macro{FamilyElseValues} in the form:
\begin{flushleft}\vskip\dp\strutbox\begin{tabular}{l}
    \texttt{,} `\PName{value}'\texttt{,} `\PName{value}' \dots
\end{tabular}\vskip\dp\strutbox\end{flushleft}
Since release~3.12 \Macro{FamilyUnknownValue} doesn't report errors itself but
signals them using \Macro{FamilyKeyState}. Therefore \Macro{FamilyElseValues}
became deprecated. Nevertheless, it's former usage will be recognized by
\Package{scrbase} and results in a code change demand message.
%
\EndIndex{Cmd}{FamilyElseValues}%


\section{Conditional Execution}
\label{sec:scrbase.if}

The package \Package{scrbase} provides several commands for conditional
execution. Thereby not the \TeX{} syntax of conditionals, e.\,g.,
% Umbruchkorrektur
\begin{lstcode}[belowskip=\dp\strutbox]
  \iftrue
    ...
  \else
    ...
  \fi
\end{lstcode}
but the \LaTeX{} syntax also known from \LaTeX{} commands like
\Macro{IfFileExists}, \Macro{@ifundefined}, \Macro{@ifpackageloaded}, and many
others will be used. Nevertheless, some package authors prefer to use the
\TeX{} syntax even for users at the \LaTeX{} interface level. In fact the
conditionals of \Package{scrbase} are very basic conditionals, this could
result in conditionals with the same name but different syntax and though in
naming conflicts. Because of this \Package{scrbase} firstly defines these
conditionals as internal commands with prefix \Macro{scr@}. Additional user
commands without this prefix are additionally defined. But the definition of
the user commands may be suppressed with option \Option{internalonly} (see
\autoref{sec:scrbase.loadit}, \autopageref{desc:scrbase.option.internalonly}).

Authors of packages and classes should use the internal commands like
\KOMAScript{} itself does. Nevertheless, for completeness the user commands
are described additionally.


\begin{Declaration}
  \Macro{scr@ifundefinedorrelax}%
  \Parameter{name}\Parameter{then instructions}\Parameter{else instructions}\\
  \Macro{ifundefinedorrelax}%
  \Parameter{name}\Parameter{then instructions}\Parameter{else instructions}
\end{Declaration}
\BeginIndex{Cmd}{scr@ifundefinedorrelax}%
\BeginIndex{Cmd}{ifundefinedorrelax}%
This command has almost the same functionality as \Macro{@ifundefined} from
the \LaTeX{} kernel (see \cite{latex:source2e}). So the \PName{then
  instructions} will be executed, if \PName{name} is the name of a command,
that is currently either not defined or \Macro{relax}. Otherwise the
\PName{else instructions} will be executed. In opposite to
\Macro{@ifundefined} \Macro{\PName{name}} won't be defined to be \Macro{relax}
in the case it wasn't defined before. Using \eTeX{} this case won't at least
consume any hash memory.%
%
\EndIndex{Cmd}{ifundefinedorrelax}%
\EndIndex{Cmd}{scr@ifundefinedorrelax}%


\begin{Declaration}
  \Macro{scr@ifpdftex}%
  \Parameter{then instructions}\Parameter{else instructions}\\
  \Macro{ifpdftex}%
  \Parameter{then instructions}\Parameter{else instructions}
\end{Declaration}
\BeginIndex{Cmd}{scr@ifpdftex}%
\BeginIndex{Cmd}{ifpdftex}%
If pdf\TeX{} has been used, the \PName{then instructions} will be executed,
otherwise the \PName{else instructions}. Whether or not a PDF-file will be
out, doesn't matter. Nevertheless, this pdf\TeX{} test seems so make sense
seldom only. Generally it's recommended to test for the wanted commands
instead (see previous \Macro{scr@ifundefinedorrelax} and
\Macro{ifundefinedorrelax}).
%
\EndIndex{Cmd}{ifpdftex}%
\EndIndex{Cmd}{scr@ifpdftex}%


\begin{Declaration}
  \Macro{scr@ifVTeX}%
  \Parameter{then instructions}\Parameter{else instructions}\\
  \Macro{ifVTeX}%
  \Parameter{then instructions}\Parameter{else instructions}
\end{Declaration}
\BeginIndex{Cmd}{scr@ifVTeX}%
\BeginIndex{Cmd}{ifVTeX}%
If V\TeX{} has been used, the \PName{then instructions} will be executed,
otherwise the \PName{else instructions}. This test seems so make sense
seldom only. Generally it's recommended to test for the wanted commands instead
(see previous \Macro{scr@ifundefinedorrelax} and \Macro{ifundefinedorrelax}).
%
\EndIndex{Cmd}{ifVTeX}%
\EndIndex{Cmd}{scr@ifVTeX}%


\begin{Declaration}
  \Macro{scr@ifpdfoutput}%
  \Parameter{then instructions}\Parameter{else instructions}\\
  \Macro{ifpdfoutput}%
  \Parameter{then instructions}\Parameter{else instructions}
\end{Declaration}
\BeginIndex{Cmd}{scr@ifpdfoutput}%
\BeginIndex{Cmd}{ifpdfoutput}%
While PDF-file generation the \PName{then instructions} will be executed,
otherwise the \PName{else instructions}. Whether, e.\,g., pdf\TeX{} or V\TeX{}
is used to generate the PDF-file doesn't matter.
%
\EndIndex{Cmd}{ifpdfoutput}%
\EndIndex{Cmd}{scr@ifpdfoutput}%


\begin{Declaration}
  \Macro{scr@ifpsoutput}%
  \Parameter{then instructions}\Parameter{else instructions}\\
  \Macro{ifpsoutput}%
  \Parameter{then instructions}\Parameter{else instructions}
\end{Declaration}
\BeginIndex{Cmd}{scr@ifpsoutput}%
\BeginIndex{Cmd}{ifpsoutput}%
While PostScript-file generation the \PName{then instructions} will be executed,
otherwise the \PName{else instructions}. V\TeX{} provides direct PostScript
generation, that will be recognized here. If V\TeX{} is not used but a switch
\Macro{if@dvips} has been defined, the decision depends on that
switch. \KOMAScript{}, e.\,g., provides \Macro{if@dvips} in
\Package{typearea}.
%
\EndIndex{Cmd}{ifpsoutput}%
\EndIndex{Cmd}{scr@ifpsoutput}%


\begin{Declaration}
  \Macro{scr@ifdvioutput}%
  \Parameter{then instructions}\Parameter{else instructions}\\
  \Macro{ifdvioutput}%
  \Parameter{then instructions}\Parameter{else instructions}
\end{Declaration}
\BeginIndex{Cmd}{scr@ifdvioutput}%
\BeginIndex{Cmd}{ifdvioutput}%
While DVI-file generation the \PName{then instructions} will be executed,
otherwise the \PName{else instructions}. If neither a direct PDF-file
generation nor a direct PostScript-file generation has been detected, DVI-file
generation is assumed.
%
\EndIndex{Cmd}{ifdvioutput}%
\EndIndex{Cmd}{scr@ifdvioutput}%


\begin{Declaration}
  \Macro{ifnotundefined}\Parameter{name}%
  \Parameter{then instructions}\Parameter{else instructions}
\end{Declaration}
\BeginIndex{Cmd}{ifnotundefined}%
\eTeX{} will be used to test, whether or not a command with the given
\PName{name} has already been defined. The \PName{then instructions} will be
executed, if the command is defined, the \PName{else instructions}
otherwise. There's no corresponding internal command for this.
%
\EndIndex{Cmd}{ifnotundefined}%


\begin{Declaration}
  \Macro{ifstr}\Parameter{string}\Parameter{string}%
  \Parameter{then instructions}\Parameter{else instructions}
\end{Declaration}
\BeginIndex{Cmd}{ifstr}%
Both \PName{string} arguments will be expanded and afterward compared. If the
expansions are same, the \PName{then instructions} will be
executed, otherwise the \PName{else instructions}. There's no corresponding
internal command for this.
%
\EndIndex{Cmd}{ifstr}%


\begin{Declaration}
  \Macro{ifnumber}\Parameter{string}%
  \Parameter{then instructions}\Parameter{else instructions}
\end{Declaration}
\BeginIndex{Cmd}{ifnumber}%
Note, that this does not compare numbers.  The \PName{then instructions} will
be executed, if the expansion of \PName{string} consists of digits
only. Otherwise the \PName{else instructions} will be used. There's no
corresponding internal command for this.
%
\EndIndex{Cmd}{ifnumber}%


\begin{Declaration}
  \Macro{ifdimen}\Parameter{string}%
  \Parameter{then instructions}\Parameter{else instructions}
\end{Declaration}
\BeginIndex{Cmd}{ifdimen}%
Note, that this does not compare dimensions. The \PName{then instructions} will
be executed, if the expansion of \PName{string} consists of digits and a valid
unit of length. Otherwise the \PName{else instructions} will be used. There's no
corresponding internal command for this.
%
\EndIndex{Cmd}{ifdimen}%


\begin{Declaration}
  \Macro{if@atdocument}\ \PName{the instructions} 
  \Macro{else}\ \PName{else instructions} \Macro{fi}
\end{Declaration}
\BeginIndex{Cmd}{if@atdocument}%
This command exists intentionally as internal command only. In the document
preamble \Macro{if@atdocument} is same like \Macro{iffalse}. After
\Macro{begin}\PParameter{document} it's same like \Macro{iftrue}. Authors of
classes and packages may use this, if a command should work somehow different
depending in whether it has been used in the preamble or inside the documents
body. Please note\textnote{Attention!}, that this is a condition in \TeX{}
syntax not in \LaTeX{} syntax!
%
\EndIndex{Cmd}{if@atdocument}%


\section{Definition of Language-Dependent Terms}
\label{sec:scrbase.languageSupport}
\BeginIndex{}{language>definition}
\index{language>dependent terms|see{language definition}}
\index{terms>language-dependent|see{language definition}}

Normally one has to change or define language-dependent terms like
\Macro{captionsenglish} in such a way that in addition to the available terms
the new or redefined terms are defined. This is made more difficult by the
fact that some packages like \Package{german}\IndexPackage{german} or
\Package{ngerman}\IndexPackage{ngerman} redefine those settings when the
packages are loaded. This definitions unfortunately occurs in such a manner as
to destroy all previous private settings. That is also the reason why it makes
sense to delay own changes with \Macro{AtBeginDocument} until
\Macro{begin}\PParameter{document}, that is, after package loading is
completed. The user can also use \Macro{AtBeginDocument}, or redefine the
language-dependent terms after \Macro{begin}\PParameter{document}, that is,
not put them in the preamble at all. The package \Package{scrbase}
provides some additional commands for defining language-dependent terms.


\begin{Declaration}
  \Macro{providecaptionname}%
    \Parameter{language}\Parameter{term}\Parameter{definition}\\
  \Macro{newcaptionname}%
    \Parameter{language}\Parameter{term}\Parameter{definition}\\
  \Macro{renewcaptionname}%
    \Parameter{language}\Parameter{term}\Parameter{definition}
\end{Declaration}
\BeginIndex{Cmd}{providecaptionname}%
\BeginIndex{Cmd}{newcaptionname}%
\BeginIndex{Cmd}{renewcaptionname}%
Using one of these commands, the user can assign a \PName{definition} for a
particular \PName{language} to a \PName{term}.  The \PName{term} is always a
macro. The commands differ dependent on whether a given \PName{language} or a
\PName{term} within a given \PName{language} are already defined or not at the
time the command is called.

If \PName{language} is not defined, then \Macro{providecaptionname} does
nothing other than writes a message in the log file. This happens only once
for each language.  If \PName{language} is defined but \PName{term} is not yet
defined for it, then it will be defined using \PName{definition}.  The
\PName{term} will not be redefined if the \PName{language} already has such a
definition; instead, an appropriate message is written to the log file.

The command \Macro{newcaptionname} has a slightly different behaviour.  If the
\PName{language} is not yet defined, then a new language command will be
created and a message written to the log file. For \PName{language}
\PValue{USenglish}, e.\,g.\, this would be the language command
\Macro{captionsUSenglish}. If \PName{term} is not yet defined in
\PName{language}, then it will be defined using \PName{definition}.  If
\PName{term} already exists in \PName{language}, then this results in an error
message.

The command \Macro{renewcaptionname} again behaves differently. It requires an
existing definition of \PName{term} in \PName{language}.  If neither
\PName{language} nor \PName{term} exist or \PName{term} is unknown in a
defined language then a error message will be given. Otherwise, the
\PName{term} for \PName{language} will be redefined according to
\PName{definition}.

\KOMAScript{} itself employs \Macro{providecaptionname} in order to define the
commands in \autoref{sec:scrlttr2-experts.languages}.

\begin{Example}
  If you prefer ``fig.'' instead of ``figure'', you may achieve this using:
\begin{lstcode}
  \renewcaptionname{USenglish}{\figurename}{fig.}
\end{lstcode}
\end{Example}
  
Since\textnote{Attention!} only existing terms in available languages can be
redefined using \Macro{renewcaptionname}, you have to put the command after
\Macro{begin}\PParameter{document} or delay the command by using
\Macro{AtBeginDocument}. Furthermore, you will get an error message if there
is no package used that switches language selection to, e.\,g.,
\PName{USenglish} in the example above.

Language dependent terms usually defined by classes and language
packages are listed and described in \autoref{tab:scrbase.commonNames}.

\begin{desclist}
  \renewcommand*{\abovecaptionskipcorrection}{-\normalbaselineskip}%
  \desccaption[{%
    Overview of usual language dependent terms%
  }]{%
    Overview of language dependent terms of usual language packages%
    \label{tab:scrbase.commonNames}%
  }{%
    Overview of usual language dependent terms
    (\emph{continuation})%
  }%
  \entry{\Macro{abstractname}}{%
    heading of the abstract%
    \IndexCmd{abstractname}%
  }%
  \entry{\Macro{alsoname}}{%
    ``see also'' in additional cross references of the index%
    \IndexCmd{alsoname}%
  }%
  \entry{\Macro{appendixname}}{%
    ``appendix'' in the heading of an appendix chapter%
    \IndexCmd{appendixname}%
  }%
  \entry{\Macro{bibname}}{%
    heading of the bibliography%
    \IndexCmd{bibname}%
  }%
  \entry{\Macro{ccname}}{%
    prefix heading for the distribution list of a letter%
    \IndexCmd{ccname}%
  }%
  \entry{\Macro{chaptername}}{%
    ``chapter'' in the heading of a chapter%
    \IndexCmd{chaptername}%
  }%
  \entry{\Macro{contentsname}}{%
    heading of the table of contents%
    \IndexCmd{contentsname}%
  }%
  \entry{\Macro{enclname}}{%
    prefix heading for the enclosure of a letter%
    \IndexCmd{enclname}%
  }%
  \entry{\Macro{figurename}}{%
    prefix heading of figure captions%
    \IndexCmd{figurename}%
  }%
  \entry{\Macro{glossaryname}}{%
    heading of the glossary%
    \IndexCmd{glossaryname}%
  }%
  \entry{\Macro{headtoname}}{%
    ``to'' in header of letter pages%
    \IndexCmd{headtoname}%
  }%
  \entry{\Macro{indexname}}{%
    heading of the index%
    \IndexCmd{indexname}%
  }%
  \entry{\Macro{listfigurename}}{%
    heading of the list of figures%
    \IndexCmd{listfigurename}%
  }%
  \entry{\Macro{listtablename}}{%
    heading of the list of tables%
    \IndexCmd{listtablename}%
  }%
  \entry{\Macro{pagename}}{%
    ``page'' in the pagination of letters%
    \IndexCmd{pagename}%
  }%
  \entry{\Macro{partname}}{%
    ``part'' in the heading of a part%
    \IndexCmd{partname}%
  }%
  \entry{\Macro{prefacename}}{%
    heading of the preface%
    \IndexCmd{prefacename}%
  }%
  \entry{\Macro{proofname}}{%
    prefix heading of mathematical proofs%
    \IndexCmd{proofname}%
  }%
  \entry{\Macro{refname}}{%
    heading of the list of references%
    \IndexCmd{refname}%
  }%
  \entry{\Macro{seename}}{%
    ``see'' in cross references of the index%
    \IndexCmd{seename}%
  }%
  \entry{\Macro{tablename}}{%
    prefix heading at table captions%
    \IndexCmd{tablename}%
  }%
\end{desclist}
%
\EndIndex{Cmd}{providecaptionname}%
\EndIndex{Cmd}{newcaptionname}%
\EndIndex{Cmd}{renewcaptionname}%
%
\EndIndex{}{language>definition}


\section{Identification of \KOMAScript}
\label{sec:scrbase.identify}

\Package{scrbase} may be used independent of \KOMAScript{} with
other packages and classes, nevertheless it's a \KOMAScript{} package. For
this is also provides commands to identify \KOMAScript{} and itself as a
\KOMAScript{} package.

\begin{Declaration}
  \Macro{KOMAScript}
\end{Declaration}
\BeginIndex{Cmd}{KOMAScript}%
This command only sets the word ``\KOMAScript'' with sans-serif font and
a little bit tracking of the capitals. Bay the way: All \KOMAScript{} classes
and packages define this command, if it hasn't been defined already. The
definition is robust using \Macro{DeclareRobustCommand}.
%
\EndIndex{Cmd}{KOMAScript}


\begin{Declaration}
  \Macro{KOMAScriptVersion}
\end{Declaration}
\BeginIndex{Cmd}{KOMAScriptVersion}%
\KOMAScript{} defines the main version of \KOMAScript{} in this command. It
has the form ``\PName{date} \PName{version} \texttt{KOMA-Script}''. This main
version is same for all \KOMAScript{} classes and the \KOMAScript{} packages,
that are essential for the classes. Because of this, it may be inquired after
loading \Package{scrbase} too. This document has been made, e.\,g., using
\KOMAScript{} version ``\KOMAScriptVersion''.
%
\EndIndex{Cmd}{KOMAScriptVersion}%


\section{Extension of the \LaTeX{} Kernel}
\label{sec:scrbase:latexkernel}

Sometimes the \LaTeX{} kernel itself provides commands, but lacks of other,
similar commands, that would often be useful too. Some of those commands for
authors of packages and classes are provided by \Package{scrbase}.

\begin{Declaration}
  \Macro{ClassInfoNoLine}\Parameter{class name}\Parameter{information}\\
  \Macro{PackageInfoNoLine}\Parameter{package name}\Parameter{information}%
\end{Declaration}%
\BeginIndex{Cmd}{PackageInfoNoLine}%
\BeginIndex{Cmd}{ClassInfoNoLine}%
For authors of classes and package the \LaTeX{} kernel already provides
commands like \Macro{ClassInfo} and \Macro{PackageInfo} to write information
together with the current line number into the log-file. Beside
\Macro{PackageWarning} and \Macro{ClassWarning} to throw warning messages with
line numbers, it also provides \Macro{PackageWarningNoLine} and
\Macro{ClassWarningNoLine} for warning messages without line
numbers. Nevertheless, the commands \Macro{ClassInfoNoLine} and
\Macro{PackageInfoNoLine}, to write information without line numbers into the
log-file, are missing. Package \Package{scrbase} provides them.
%
\EndIndex{Cmd}{ClassInfoNoLine}%
\EndIndex{Cmd}{PackageInfoNoLine}


\begin{Declaration}
  \Macro{l@addto@macro}\Parameter{command}\Parameter{extension}%
\end{Declaration}%
\BeginIndex{Cmd}{l@addto@macro}%
The \LaTeX{} kernel already provides an internal command \Macro{g@addto@macro}
to extend the definition of macro \Macro{command} by \PName{extension}
globally. This may be used only for macros that have no
arguments. Nevertheless, sometimes a command like this, that works locally to
a group instead of globally, could be useful. Package \Package{scrbase}
provides such a command with \Macro{l@addto@macro}. An alternative may be
usage of package \Package{etoolbox}\IndexPackage{etoolbox}, that provides
several of such commands for different purpose (see \cite{package:etoolbox}).
%
\EndIndex{Cmd}{l@addto@macro}


\section{Extension of the Mathematical Features of \eTeX}
\label{sec:scrbase:etex}

\eTeX{}, that is meanwhile used by \LaTeX{} and needed by \KOMAScript{},
provides with \Macro{numexpr}\IndexCmd{numexpr} an extended feature for
calculation of simple arithmetic with \TeX{} counters and
integers. The four basic arithmetic operations and brackets are
supported. Correct rounding is done while division. Sometimes additional
operators would be useful.

\begin{Declaration}
  \Macro{XdivY}\Parameter{dividend}\Parameter{divisor}\\
  \Macro{XmodY}\Parameter{dividend}\Parameter{divisor}%
\end{Declaration}%
\BeginIndex{Cmd}{XdivY}%
\BeginIndex{Cmd}{XmodY}%
Having a division with remainder, command\ChangedAt{v3.05a}{\Package{scrbase}}
\Macro{XdivY} gives the value of the integer quotient, command \Macro{XmodY}
the value of the remainder. This kind of division is defined:
\[
\textit{dividend} = \textit{divisor} \cdot
\textit{integer quotient} + \textit{remainder}
\]
with \textit{dividend} and \textit{remainder} are integer, \textit{remainder}
is greater or equal to \textit{divisor}, and \textit{divisor} is a natural
number greater than 0.

The value may be used for assignment to a counter or directly in the
expression of \Macro{numexpr}. For output the value as an Arabic number is has
to be prefixed by \Macro{the}.%
%
\EndIndex{Cmd}{XmodY}%
\EndIndex{Cmd}{XdivY}%
%
\EndIndex{Package}{scrbase}%

\endinput

%%% Local Variables:
%%% mode: latex
%%% mode: flyspell
%%% ispell-local-dictionary: "english"
%%% coding: us-ascii
%%% TeX-master: "../guide.tex"
%%% End:
