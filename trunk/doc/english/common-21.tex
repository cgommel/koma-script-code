% ======================================================================
% common-21.tex
% Copyright (c) Markus Kohm, 2013
%
% This file is part of the LaTeX2e KOMA-Script bundle.
%
% This work may be distributed and/or modified under the conditions of
% the LaTeX Project Public License, version 1.3c of the license.
% The latest version of this license is in
%   http://www.latex-project.org/lppl.txt
% and version 1.3c or later is part of all distributions of LaTeX 
% version 2005/12/01 or later and of this work.
%
% This work has the LPPL maintenance status "author-maintained".
%
% The Current Maintainer and author of this work is Markus Kohm.
%
% This work consists of all files listed in manifest.txt.
% ----------------------------------------------------------------------
% common-21.tex
% Copyright (c) Markus Kohm, 2013
%
% Dieses Werk darf nach den Bedingungen der LaTeX Project Public Lizenz,
% Version 1.3c, verteilt und/oder veraendert werden.
% Die neuste Version dieser Lizenz ist
%   http://www.latex-project.org/lppl.txt
% und Version 1.3c ist Teil aller Verteilungen von LaTeX
% Version 2005/12/01 oder spaeter und dieses Werks.
%
% Dieses Werk hat den LPPL-Verwaltungs-Status "author-maintained"
% (allein durch den Autor verwaltet).
%
% Der Aktuelle Verwalter und Autor dieses Werkes ist Markus Kohm.
% 
% Dieses Werk besteht aus den in manifest.txt aufgefuehrten Dateien.
% ======================================================================
%
% Text that is common for several chapters of the KOMA-Script guide
% Maintained by Markus Kohm
%
% ----------------------------------------------------------------------
%
% Absaetze, die mehreren Kapitels in der KOMA-Script-Anleitung gemeinsam sind
% Verwaltet von Markus Kohm
%
% ============================================================================

\KOMAProvidesFile{common-21.tex}
                 [$Date: 2013-10-22 11:17:09 +0200 (Di, 22 Okt 2013) $
                  KOMA-Script guide (common paragraph: 
                                     Manipulation of defined page styles)]
\translator{Markus Kohm}

% Date version of the translated file: 0000-00-00

\makeatletter
\@ifundefined{ifCommonscrlayer}{\newif\ifCommonscrlayer}{}%
\@ifundefined{ifCommoonscrlayer-scrpage}{\expandafter\newif\csname ifCommonscrlayer-scrpage\endcsname}{}%
\@ifundefined{ifIgnoreThis}{\newif\ifIgnoreThis}{}%
\makeatother

\section{Manipulation of Defined Page Styles}
\label{sec:\csname label@base\endcsname.pagestyle.content}

\IfCommon{scrlayer}{%
  Even though \Package{scrlayer} itself doesn't define a concrete page style
  with content\,---\,the already mentioned page style \Pagestyle{@everystyle@}
  and \Pagestyle{empty} initially are defined without any level and so empty
  and without content\,---\,, it provides some options and commands for the
  manipulation of the contents.%
}

\IfCommon{scrlayer-scrpage}{%
  In \autoref{sec:scrlayer-scrpage.predefined.pagestyles} you've been told,
  how the page styles \Pagestyle{scrheadings} and
  \Pagestyle{plain.scrheadings} habe been predefined by default and how you
  can change these default generally. But information about the generation of,
  i.\,e., the running heads, the manipulation of the widths of page head and
  footer, and how you can get horizontal lines printed above or below the page
  head or footer are still missing. Even though all these are features of
  package \Package{scrlayer}, they will be described consecutively, because
  these basic features of \Package{scrlayer} are also an evident part of the
  features of \Package{scrlayer-scrpage}.%
}

\ifshortversion\IgnoreThistrue\IfCommon{scrlayer-scrpage}{\IgnoreThisfalse}\fi
\ifIgnoreThis %++++++++++++++++++++++++++++++++++++++++ not scrlayer-scrpage +
\else %----------------------------------------------- only scrlayer-scrpage -

\begin{Declaration}
  \Macro{automark}\OParameter{section level of the right mark}%
                  \Parameter{section level of the left mark}\\
  \Macro{automark*}\OParameter{section level of the right mark}%
                  \Parameter{section level of the left mark}\\
  \Macro{manualmark}\\
  \Option{automark}\\
  \KOption{autooneside}\PName{simple switch}\\
  \Option{manualmark}
\end{Declaration}
\BeginIndex{Cmd}{automark}%
\BeginIndex{Cmd}{manualmark}%
\BeginIndex{Option}{automark}%
\BeginIndex{Option}{manualmark}%
\BeginIndex{Option}{autooneside~=\PName{simple switch}}%
With the \LaTeX{} standard classes or the \KOMAScript{} classes the decision
whether using automatic running heads\Index{running heads>automatic}, or
static or manual running heads\Index{running heads>static}\Index{running
  heads>manual} would be done using either page style \Pagestyle{headings} or
\Pagestyle{myheadings}. Automatic running heads are replications of a
significant, characterizing text of the page mostly inside the page head,
sometimes in the page footer.

The article classes\OnlyAt{\Class{article}\and \Class{scrartcl}}
\Class{article} or \Class{scrartcl} with page style
\Pagestyle{headings}\IndexPagestyle{headings} use the section heading, which
is either the mandatory or the optional argument of \Macro{section}, for the
automatic running head of single side documents. Double side documents use
this section heading as the \emph{left mark} and at once use the subsection
heading as the \emph{right mark}. The left mark will be printed on left pages,
which founds the name \emph{left mark}. The right mark will be printed on
right\,---\,in single side mode this means every\,---\,page. The classes by
default also remove the right mark whenever they put the section heading into
the left mark.

The report and book classes \OnlyAt{\Class{report}\and
  \Class{scrreprt}\and \Class{book}\and \Class{scrbook}} start one level
higher. So they use the chapter heading to be the right mark in single side
layout. In double side layout the use the chapter heading to be the left mark
and the section heading to be the right mark.

If you are using \Pagestyle{myheadings} instead of
\Pagestyle{headings}\IndexPagestyle{myheadings}, the marks in the page header
still exists and would be printed same like using \Pagestyle{headings}. But
section commands won't automatically set the marks any longer. So you can fill
them up only using the commands \Macro{markright} and \Macro{markboth}, that
will be described later in this section.

This difference between two page style has been abolished with
\Package{scrpage2} and also with \Package{scrlayer}. Instead of distinguish
between automatic and manual running head by the selection of a page style,
two new commands, \Macro{automark} and \Macro{manualmark}, are
provided. Thereby \Macro{manualmark} switches to manual marks and deactivates
the automatic filling of the marks. In contrast to this \Macro{automark} and
\Macro{automark*} can be used to define, which section levels should be used
for the automatic mark filling. The optional argument is the \PName{section
  level of the right mark}, the mandatory argument the \PName{section level of
  the left mark}. The arguments always should be the name of a section level
like  \PValue{part}, \PValue{chapter}, \PValue{section}, \PValue{subsection},
\PValue{subsubsection}, \PValue{paragraph}, or \PValue{subparagraph}.

Generally the higher level should be used for the left mark and the lower
level for the right mark. This is only a convention and not a need, but it
makes sense.

The difference in \Macro{automark} and \Macro{automark*} is, that
\Macro{automark} deleted all prior usages of \Macro{automark} or
\Macro{automark*}, while \Macro{automark*} changes only the behaviour of the
section levels of it's arguments. So you can even build more complex cases.

\begin{Example}
  Assumed you want the chapter heading to be the running head of even pages
  and the section heading to be the running head of odd pages like this is
  usual for books. But on odd pages you also want the chapter headings as a
  running head as long as the first section appears. To do so, you first have
  to load \Package{scrlayer-scrpage} and select page style
  \Pagestyle{scrheadings}. So you're document starts with:
\begin{lstcode}
  \documentclass{scrbook}
  \usepackage{scrlayer-scrpage}
  \pagestyle{scrheadings}
\end{lstcode}
  Next you will setup the chapter heading to be not only the left but also the
  right mark:
\begin{lstcode}
  \automark[chapter]{chapter}
\end{lstcode}
  Then the section heading should also become a right mark:
\begin{lstcode}
   \automark*[section]{}
\end{lstcode}
  Here the star version is used, because the prior \Macro{automark} command
  should be still valid. Additionally the mandatory argument for the
  \PName{section level of the left mark} is empty, because this mark should be
  unchanged.

  Now you just need some document contents to see a result:
\begin{lstcode}
  \usepackage{lipsum}
  \begin{document}
  \chapter{Chapter Heading}
  \lipsum[1-20]
  \section{Section Heading}
  \lipsum[21-40]
  \end{document}
\end{lstcode}
  We once again use package \Package{lipsum}\IndexPackage{lipsum} to generate
  some dummy text with command \Macro{lipsum}\IndexCmd{lipsum}. The package is
  really useful.

  If you'd test the example, you'd see, that the chapter start page will be
  without running head as usual. This is, because it automatically uses the
  \Pagestyle{plain} page style \Pagestyle{plain.scrheadings}. Pages~2--4 have
  the chapter headings in the running head. After the section heading on
  page~4 the running head of page~5 changes into this section heading. From
  this page to the end the running head alternates between chapter and section
  heading from page to page.
\end{Example}

Instead of the commands you may also use the options \Option{manualmark} and
\Option{automark} to switch between manual and automatic running
heads. Thereby \Option{automark} always uses the default
\begin{lstcode}[belowskip=\dp\strutbox]
  \automark[section]{chapter}
\end{lstcode}
for classes with \Macro{chapter} and
\begin{lstcode}[belowskip=\dp\strutbox]
  \automark[subsection]{section}
\end{lstcode}
for classes without \Macro{chapter}.

But normally in single side mode you don't want, that the lower level
influences the right mark. Instead of this you want the higher level, that
will fill only the left mark in double side layout, to be the running head of
all pages. The default option \Option{autooneside} corresponds with this
behaviour. The option understands the values for simple switches, that can be
found in  \autoref{tab:truefalseswitch} on
\autopageref{tab:truefalseswitch}. If you'd deactivate the option, in single
side layout the optional and the obligatory arguments of \Macro{automark} and
\Macro{automark*} will influence the running head again.%
%
\begin{Example}
  Assumed, you have a single sided report but want similar running heads like
  for the book in the example before. Concretely the chapter headings should
  be used as a running head as long as the first section appears. From the
  first section on, the section heading should be used. So we modify the
  example before a little bit:
\begin{lstcode}
  \documentclass{scrreprt}
  \usepackage[autooneside=false]{scrlayer-scrpage}
  \pagestyle{scrheadings}
  \automark[section]{chapter}
  \usepackage{lipsum}
  \begin{document}
  \chapter{Chapter Heading}
  \lipsum[1-20]
  \section{Section Heading}
  \lipsum[21-40]
  \end{document}
\end{lstcode}
  You can see, that we don't need a \Macro{automark*} command in this
  case. Please try the example also with \Option{autooneside} set to
  \PValue{true} or remove the option and it's value. You should find a
  difference at the running head in the pages' head from page~4 on.
\end{Example}

Please note\textnote{Attention!}, that loading the package only doesn't have
any effect on the fact whether automatic or manual running heads are used or
what kind of section headings do fill up the marks. Only using an explicit
option \Option{automark} or \Option{manualmark} or one of the commands
\Macro{automark} or \Macro{manualmark} can reach a well defined state.%
\EndIndex{Option}{autooneside~=\PName{simple switch}}%
\EndIndex{Option}{manualmark}%
\EndIndex{Option}{automark}%
\EndIndex{Cmd}{manualmark}%
\EndIndex{Cmd}{automark}%
\fi % ******************************************** End only scrlayer-scrpage *

\ifshortversion\IgnoreThistrue\IfCommon{scrlayer}{\IgnoreThisfalse}\fi
\ifIgnoreThis %++++++++++++++++++++++++++++++++++++++++++++++++ not scrlayer +
\else %------------------------------------------------------- only scrlayer -

\begin{Declaration}
  \KOption{draft}\PName{simple switch}
\end{Declaration}
\BeginIndex{Option}{draft~=\PName{simple switch}}%
%
\EndIndex{Cmd}{MakeMarkcase}%
\fi %***************************************************** End only scrlayer *

\ifshortversion\IgnoreThistrue\IfCommon{scrlayer-scrpage}{\IgnoreThisfalse}\fi
\ifIgnoreThis %++++++++++++++++++++++++++++++++++++++++ not scrlayer-scrpage +
\else %----------------------------------------------- only scrlayer-scrpage -
\begin{Declaration}
  \KOption{markcase}\PName{Wert}
\end{Declaration}
\BeginIndex{Option}{markcase~=\PName{Wert}}%
%
\EndIndex{Option}{markcase~=\PName{Wert}}%

\begin{Declaration}
  \Macro{leftmark}\\
  \Macro{rightmark}\\
  \Macro{headmark}\\
  \Macro{pagemark}
\end{Declaration}
\BeginIndex{Cmd}{leftmark}%
\BeginIndex{Cmd}{rightmark}%
\BeginIndex{Cmd}{headmark}%
\BeginIndex{Cmd}{pagemark}%
%
\EndIndex{Cmd}{pagemark}%
\EndIndex{Cmd}{headmark}%
\EndIndex{Cmd}{rightmark}%
\EndIndex{Cmd}{leftmark}%

\begin{Declaration}
  \Macro{partmarkformat}\\
  \Macro{chaptermarkformat}\\
  \Macro{sectionmarkformat}\\
  \Macro{subsectionmarkformat}\\
  \Macro{subsubsectionmarkformat}\\
  \Macro{paragraphmarkformat}\\
  \Macro{subparagraphmarkformat}
\end{Declaration}
\BeginIndex{Cmd}{partmarkformat}%
\BeginIndex{Cmd}{chaptermarkformat}%
\BeginIndex{Cmd}{sectionmarkformat}%
\BeginIndex{Cmd}{subsectionmarkformat}%
\BeginIndex{Cmd}{subsubsectionmarkformat}%
\BeginIndex{Cmd}{paragraphmarkformat}%
\BeginIndex{Cmd}{subparagraphmarkformat}%
%
\EndIndex{Cmd}{subparagraphmarkformat}%
\EndIndex{Cmd}{paragraphmarkformat}%
\EndIndex{Cmd}{subsubsectionmarkformat}%
\EndIndex{Cmd}{subsectionmarkformat}%
\EndIndex{Cmd}{sectionmarkformat}%
\EndIndex{Cmd}{chaptermarkformat}%
\EndIndex{Cmd}{partmarkformat}%

\begin{Declaration}
  \Macro{partmark}\Parameter{Text}\\
  \Macro{chaptermark}\Parameter{Text}\\
  \Macro{sectionmark}\Parameter{Text}\\
  \Macro{subsectionmark}\Parameter{Text}\\
  \Macro{subsubsectionmark}\Parameter{Text}\\
  \Macro{paragraphmark}\Parameter{Text}\\
  \Macro{subparagraphmark}\Parameter{Text}
\end{Declaration}
\BeginIndex{Cmd}{partmark}%
\BeginIndex{Cmd}{chaptermark}%
\BeginIndex{Cmd}{sectionmark}%
\BeginIndex{Cmd}{subsectionmark}%
\BeginIndex{Cmd}{subsubsectionmark}%
\BeginIndex{Cmd}{paragraphmark}%
\BeginIndex{Cmd}{subparagraphmark}%
%
\EndIndex{Cmd}{subparagraphmark}%
\EndIndex{Cmd}{paragraphmark}%
\EndIndex{Cmd}{subsubsectionmark}%
\EndIndex{Cmd}{subsectionmark}%
\EndIndex{Cmd}{sectionmark}%
\EndIndex{Cmd}{chaptermark}%
\EndIndex{Cmd}{partmark}%

\begin{Declaration}
  \Macro{markleft}\Parameter{left mark}\\
  \Macro{markright}\Parameter{right mark}\\
  \Macro{markboth}\Parameter{left mark}\Parameter{right mark}
\end{Declaration}
\BeginIndex{Cmd}{markleft}%
\BeginIndex{Cmd}{markright}%
\BeginIndex{Cmd}{markboth}%

%
\EndIndex{Cmd}{markboth}%
\EndIndex{Cmd}{markright}%
\EndIndex{Cmd}{markleft}%

\fi %********************************************* End only scrlayer-scrpage *


\ifshortversion\IgnoreThistrue\IfCommon{scrlayer}{\IgnoreThisfalse}\fi
\ifIgnoreThis %++++++++++++++++++++++++++++++++++++++++++++++++ not scrlayer +
\else %------------------------------------------------------- only scrlayer -

\begin{Declaration}
  \Macro{GenericMarkFormat}\Parameter{name of the section level}
\end{Declaration}
\BeginIndex{Cmd}{GenericMarkFormat}%
% ToDo: Add an example!
\EndIndex{Cmd}{GenericMarkFormat}%

\begin{Declaration}
  \Macro{@mkleft}\Parameter{left mark}\\
  \Macro{@mkright}\Parameter{right mark}\\
  \Macro{@mkdouble}\Parameter{mark}\\
  \Macro{@mkboth}\Parameter{left mark}\Parameter{right mark}
\end{Declaration}
\BeginIndex{Cmd}{@mkleft}%
\BeginIndex{Cmd}{@mkright}%
\BeginIndex{Cmd}{@mkdouble}%
\BeginIndex{Cmd}{@mkboth}%
%
\EndIndex{Cmd}{@mkboth}%
\EndIndex{Cmd}{@mkdouble}%
\EndIndex{Cmd}{@mkright}%
\EndIndex{Cmd}{@mkleft}%


\fi %***************************************************** End only scrlayer *


\ifshortversion%
\IfCommon{scrlayer}{%
}%
\fi

\emph{Sorry, this section has been translated not yet. Please see
  \autoref{sec:scrpage.basics} for at least some
  information. Additional information may be found in the implementation
  manual \File{scrlayer-scrpage.pdf}.}

%%% Local Variables:
%%% mode: latex
%%% mode: flyspell
%%% coding: iso-latin-1
%%% ispell-local-dictionary: "en_GB"
%%% TeX-master: "scrlayer-de.tex"
%%% TeX-PDF-mode: t
%%% End: 
