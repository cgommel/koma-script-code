% ======================================================================
% introduction.tex
% Copyright (c) Markus Kohm, 2001-2006
%
% This file is part of the LaTeX2e KOMA-Script bundle.
%
% This work may be distributed and/or modified under the conditions of
% the LaTeX Project Public License, version 1.3b of the license.
% The latest version of this license is in
%   http://www.latex-project.org/lppl.txt
% and version 1.3b or later is part of all distributions of LaTeX 
% version 2005/12/01 and of this work.
%
% This work has the LPPL maintenance status "author-maintained".
%
% The Current Maintainer and author of this work is Markus Kohm.
%
% This work consists of all files listed in manifest.txt.
% ----------------------------------------------------------------------
% introduction.tex
% Copyright (c) Markus Kohm, 2001-2006
%
% Dieses Werk darf nach den Bedingungen der LaTeX Project Public Lizenz,
% Version 1.3b.
% Die neuste Version dieser Lizenz ist
%   http://www.latex-project.org/lppl.txt
% und Version 1.3b ist Teil aller Verteilungen von LaTeX
% Version 2005/12/01 und dieses Werks.
%
% Dieses Werk hat den LPPL-Verwaltungs-Status "author-maintained"
% (allein durch den Autor verwaltet).
%
% Der Aktuelle Verwalter und Autor dieses Werkes ist Markus Kohm.
% 
% Dieses Werk besteht aus den in manifest.txt aufgefuehrten Dateien.
% ======================================================================
%
% Introduction of the KOMA-Script guide
% Maintained by Markus Kohm
%
% ----------------------------------------------------------------------
%
% Einleitung der KOMA-Script-Anleitung
% Verwaltet von Markus Kohm
%
% ======================================================================

\ProvidesFile{introduction.tex}[2005/11/26 KOMA-Script guide introduction]
% Date of translated german file: 2005-11-26

\chapter{Introduction}
\labelbase{introduction}

\section{Preface}\label{sec:introduction.preface}

{\KOMAScript} is a very complex bundle. You may see this, because it
is not only one class or one package but a bundle of many classes and
packages. The classes are counterparts to the standard classes (see
\autoref{cha:maincls}) but never they come with only the same
commands, environments, options and optional possibilities like the
standard classes nor they result in the same look-a-like. More about
the history of \KOMAScript{} you will find at
\autoref{sec:introduction.history}.

\KOMAScript{} comes with a lot of classes, packages, commands,
environments and possibilities. Some of these you may find also at the
standard classes, many of them you wouldn't. Some are even supplements to
the \LaTeX{} kernel.

Because of all this \KOMAScript{} guide has to be large. Also there's
no school for \KOMAScript. So there's no teacher knowing everyone of
his students. Such a teacher should be able to change the lessons
dependent on the kind of students he has to teach. It would be easy to
write a guide which is good for one kind of a reader. But it is nearly
impossible to write a guide which is good for every reader. Never the
less the authors tried to do so. The result must be a
compromise. Please remebers this, if you don't like the guide or
parts of it, before you'll write a report about. We hope the
compromise is a good one. Never the less, if you have solutions of
improvements write us.

If you habe any problem with \KOMAScript{} please read the guide
despite the size of it. Apart from the guide this means also all the
other text files of the bundle. You will find a list of these as
\File{readme.txt}.


\section{Structure of the Guide}\label{sec:introduction.structure}

There's only less information for \LaTeX{} beginners at the guide.
Beginners should first read documents like \cite{lshort} or
\cite{latex:usrguide}. It would also be good to read a book about
\LaTeX. You'll find literature at almost every faq like \cite{UK:FAQ}.
You should also have a look to these faqs, if you have a problem with
\LaTeX{}.

\begin{Explain}
  Informations about understanding \LaTeX{} or \KOMAScript{} are
  written like this paragraph. You do not need this informations to
  use \KOMAScript{}, but if you have problems in using \KOMAScript you
  should read this parts too. So read them before writing a bug
  report.
\end{Explain}

Some of you do not want to read the whole guide but only informations
about one class or package. This should be possible, because you will
find almost all informations about one package or one class at one
chapter. Only the three main classes are described together at one
chapter, because most informations about are valid for all of them. If
something is only valid for one or two of the three you will find a
note at the margin. If for example something is only valid for
\Class{scrartcl}\OnlyAt{\Class{scrartcl}} this is marked at the
margin like here.


\section{History of \KOMAScript}\label{sec:introduction.history}

\begin{Explain}
  At the early 90th Frank Neukam has to write a student script. At
  that time \LaTeX2.09 was \LaTeX. At \LaTeX2.09 classes and packages
  has been all the same and where called \emph{styles}. Frank thought
  the standard document styles not to be good enough for his work. He
  needed more and other commands and environments.

  At the same time Frank was very interested in typography. After
  reading \cite{JTsch87} he decided to write his own document style
  not only to write the one script but also to have a style family
  especialy for european and german typography. \Script{} was born.

  I, Markus Kohm, found {\Script} at december 1992. While Frank used
  A4 paper format, I often had to use A5. At 1992/1993 the standard
  styles and \Script{} had no options for A5 paper. So I`ve changed
  \Script{}. These and other changes where also found at \ScriptII{}
  released by Frank at december 1993. 
  
  At the middle of 1994 \LaTeXe{} could be found at CTAN. Most users
  of \ScriptII{} not only wanted to use these styles at compatibility
  mode of \LaTeXe{} but have a native \LaTeXe{} \Script{} --- me too.
  So I have made new \LaTeXe{} classes a first package and released all
  together named \KOMAScript{} at july~7th~1994. Some month later
  Frank declared \KOMAScript{} to be the new \Script. At this time
  there was no letter class at \KOMAScript{}. Axel Kielhorn wrote it
  and so since december~1994 it is part of \KOMAScript{}. Some time
  later Axel Sommerfeldt has written the first real scrguide. The
  old english scrguide was made by Werner Lemberg.

  A lot of time is gone. There where minor changes at \LaTeX{} but a
  lot of changes at the surroundings. There are many new packages and
  classes. \KOMAScript{} isn't the \KOMAScript{} from 1994, it's much
  more. While it was first made to have good classes for german
  authors, now it made to have classes with more flexibility than the
  standard classes. I've got a lot email from all over the
  world. Because of this I've written lots of new macros. They all
  needs documentation. Some day we needed simply a new scrguide. 
\end{Explain}

This new scrguide is not only made for A5 paper. Because of the
history the implementation is german. The primary documentation is in
german too. Maybe this isn't easy to understand. But see: Almost
everything around \LaTeX{} is english. \KOMAScript{} is german. But
don't worry, the commands, environments, options etc. are
english. Sometimes my english was not good enough to find the correct
english name, never the less I tried. As result you'll find options
called \Option{pointlessnumbers} which means only ``numbers without
dot at the end'' and should be called e.g.\ \Option{dotlessnumbers}.


\section{Thanks}\label{sec:introduction.thanks}

In fact the thanks-giving is part of the addendum, but my thanks are not
primarily for the authors of this guide.  These thanks should be given by the
readers!  I give my personal thanks to Frank Neukam. Without his \Script{}
family \KOMAScript{} wouldn't be.  I am indebted to the persons who have
contributed to {\KOMAScript} in many different fashions.  Vicariously for all
those contributers I would like mention Jens-Uwe Morawski and Torsten
Kr\"uger.  The english translation of the guide is , besides many other
things, obliged to Jens' untiring engagement.  Torsten was the best
beta-tester I ever had. Especially his work has enhanced the usabillity of
\Class{scrlttr2} und \Class{scrpage2}.  Many thanks to all who have encouraged
me to go on, make things better, less error-prone or to implement some
additional features.

Thanks to DANTE, Deutschsprachige Anwendervereinigung \TeX~e.V\kern-.18em.,
without the DANTE server, \KOMAScript{} couldn't be released and distributed.
Thanks to everybody at the \TeX{} news groups and mailing lists, who answer
questions and help me to support \KOMAScript{}.

\section{Legal Notes}\label{sec:introduction.legal}

{\KOMAScript} was released under {\LaTeX} Project Public Licence. You
will find it at the file \File{LEGAL.TXT}. Germans will also find a
inofficial translation at \File{LEGALDE.TXT}. This german file is
valid at every country with german being an official language.

\iffree{There's no warranty for any documentation or any other part of the
\KOMAScript{} bundle.}%
{But this printed version of the guide is not free under the conditions of the
  LaTeX{} Project Public Licence. If you need a free version of this guide,
  use the version that is part of the \KOMAScript{} bundle.}

\section{Installation}\label{sec:introduction.installation}
You'll find the documentation about installation at the files
\File{readme.txt} and \File{INSTALL.TXT}. Read also the documentation
of the \TeX{} distribution you're using.

\section{Bugreports and Other Requests}
\label{sec:introduction.errors}

If you think, you've found a bug at the documentation, one of the
\KOMAScript{} classes, one of the \KOMAScript{} packages or another
part of \KOMAScript{} please do the following. First have a look at
CTAN to see, if there is a new version of the part with the bug. In
this case install the new part and try it again.

If you've found an error not at the documentation or a bug, which is
still at the updated file, please write a short example \LaTeX{}
document to show the problem. At this example you should only use
these packages and definitions needed to demonstrate the
problem. Don't use unusual packages.

Writing this example you will often find out, if the problem is a bug
at \KOMAScript{} or not. Report only bugs at \KOMAScript{} to the
author of \KOMAScript{}. Please use \File{komabug.tex} for the
generation of the bug report. \File{komabug.tex} is an interactive
\LaTeX{} document.

If you want to ask your question at a news group or mainling list, you
should also write such a example as part of your question. But you
need not use \File{komabug.tex} in this case. To get the versions of
all used packages, simply put \Macro{listfiles} in the preamble of
your example and read the end of the \File{log}-file.


\section{Additional Informations}
\label{sec:introduction.moreinfos}

After becoming an advanced user you`ll may wanted for advanced examples. Such
examples impart more than basic knowledge. For this they are not
part\iffree{}{ of the main part} of this guide. You may find some additional
examples and informations at the internet homepage of the \KOMAScript{}
Documentation Project \cite{hoempage}. Note that the main language of this
site is german. But never the less english is welcome.

\endinput
%%% Local Variables: 
%%% mode: latex
%%% TeX-master: "../guide.tex"
%%% End: 
