% ======================================================================
% scrlttr2-experts.tex
% Copyright (c) Markus Kohm, 2002-2012
%
% This file is part of the LaTeX2e KOMA-Script bundle.
%
% This work may be distributed and/or modified under the conditions of
% the LaTeX Project Public License, version 1.3c of the license.
% The latest version of this license is in
%   http://www.latex-project.org/lppl.txt
% and version 1.3c or later is part of all distributions of LaTeX 
% version 2005/12/01 or later and of this work.
%
% This work has the LPPL maintenance status "author-maintained".
%
% The Current Maintainer and author of this work is Markus Kohm.
%
% This work consists of all files listed in manifest.txt.
% ----------------------------------------------------------------------
% scrlttr2-experts.tex
% Copyright (c) Markus Kohm, 2002-2012
%
% Dieses Werk darf nach den Bedingungen der LaTeX Project Public Lizenz,
% Version 1.3c, verteilt und/oder veraendert werden.
% Die neuste Version dieser Lizenz ist
%   http://www.latex-project.org/lppl.txt
% und Version 1.3c ist Teil aller Verteilungen von LaTeX
% Version 2005/12/01 oder spaeter und dieses Werks.
%
% Dieses Werk hat den LPPL-Verwaltungs-Status "author-maintained"
% (allein durch den Autor verwaltet).
%
% Der Aktuelle Verwalter und Autor dieses Werkes ist Markus Kohm.
% 
% Dieses Werk besteht aus den in manifest.txt aufgefuehrten Dateien.
% ======================================================================
%
% Chapter about scrlttr2 of the KOMA-Script guide expert part
% Maintained by Markus Kohm
%
% ----------------------------------------------------------------------
%
% Kapitel ueber scrlttr2 im Experten-Teil der KOMA-Script-Anleitung
% Verwaltet von Markus Kohm
%
% ============================================================================

\ProvidesFile{scrlttr2-experts.tex}[2012/02/16 KOMA-Script guide (chapter:
scrlttr2 for experts)]

\translator{Markus Kohm and Gernot Hassenpflug}

\chapter{Additional Information about the Letter Class
  \Class{scrlttr2}\protect\footnote{This section is still missing. Translators from German to English would be welcome! }}
\labelbase{scrlttr2-experts}

Sorry, guys but the translation of this chapter is still missing. Translators
from German to English would be welcome!

\section{Information from the old English Guide\protect\footnote{This
    section is deprecated and should be transformed into the new
    structure. Translators from German to English would be welcome!}}

Nevertheless you may find here some information, that have been moved from the
old English guide and will be placed here until the translation of the
corresponding German chapter will be complete. But you should note, that this
information is not structured or ordered and may be out-dated, deprecated and
even wrong!

% ============================================================================
% Verschoben von scrlttr2.tex:

\begin{Declaration}
  \Macro{usekomavar}\OParameter{command}\Parameter{name}\\
  \Macro{usekomavar*}\OParameter{command}\Parameter{name}
\end{Declaration}
\BeginIndex{Cmd}{usekomavar}\BeginIndex{Cmd}{usekomavar*}%
\begin{Explain}
  The commands \Macro{usekomavar} and \Macro{usekomavar*} are,
  similarly to all commands where a starred version exists or which
  can take an optional argument, not fully expandable. Nevertheless,
  if used within \Macro{markboth}\IndexCmd{markboth},
  \Macro{markright}\IndexCmd{markright} or similar commands, you need
  not insert a \Macro{protect}\IndexCmd{protect} before using them.
  Of course this is also true for \Macro{markleft}\IndexCmd{markleft}
  if using package \Package{scrpage2}. However, these kinds of
  commands can not be used within commands like
  \Macro{MakeUppercase}\IndexCmd{MakeUppercase} which directly
  influence their argument.
  \Macro{MakeUppercase}\PParameter{\Macro{usekomavar}\Parameter{name}}
  would result in \Macro{usekomavar}\Parameter{NAME}. To avoid this
  problem you may use commands like \Macro{MakeUppercase} as an
  optional argument to \Macro{usekomavar} or \Macro{usekomavar*}. Then
  you will get the uppercase content of a variable using
  \Macro{usekomavar}\PValue{[\Macro{MakeUppercase}]}\Parameter{name}.
\end{Explain}
%
\EndIndex{Cmd}{usekomavar}\EndIndex{Cmd}{usekomavar*}%


\begin{Declaration}
  \Macro{ifkomavarempty}\Parameter{name}\Parameter{true}\Parameter{false}\\
  \Macro{ifkomavarempty*}\Parameter{name}\Parameter{true}\Parameter{false}
\end{Declaration}
\BeginIndex{Cmd}{ifkomavarempty}%
\BeginIndex{Cmd}{ifkomavarempty*}%
\begin{Explain}
  It is important to know that the content or description of the
  variable will be expanded as far as this is possible with
  \Macro{edef}. If this results in spaces or unexpandable macros like
  \Macro{relax}, the result will be not empty even where the use of
  the variable would not result in any visible output.
  
  Both variants of the command also must not be used as the argument
  of \Macro{MakeUppercase}\IndexCmd{MakeUppercase} or other commands
  which have similar effects to their arguments (see the description
  of \Macro{usekomavar} above for more information about using
  commands like \Macro{usekomavar} or \Macro{ifkomavarempty} at the
  argument of \Macro{MakeUppercase}). However, they are robust enough
  to be used as the argument of, e.\,g., \Macro{markboth} or
  \Macro{footnote}.
\end{Explain}
\EndIndex{Cmd}{ifkomavarempty*}%
\EndIndex{Cmd}{ifkomavarempty}%



%%% Local Variables:
%%% mode: latex
%%% coding: us-ascii
%%% TeX-master: "guide.tex"
%%% End:
