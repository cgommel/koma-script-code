% ======================================================================
% scrlttr2-experts.tex
% Copyright (c) Markus Kohm, 2002-2012
%
% This file is part of the LaTeX2e KOMA-Script bundle.
%
% This work may be distributed and/or modified under the conditions of
% the LaTeX Project Public License, version 1.3c of the license.
% The latest version of this license is in
%   http://www.latex-project.org/lppl.txt
% and version 1.3c or later is part of all distributions of LaTeX 
% version 2005/12/01 or later and of this work.
%
% This work has the LPPL maintenance status "author-maintained".
%
% The Current Maintainer and author of this work is Markus Kohm.
%
% This work consists of all files listed in manifest.txt.
% ----------------------------------------------------------------------
% scrlttr2-experts.tex
% Copyright (c) Markus Kohm, 2002-2012
%
% Dieses Werk darf nach den Bedingungen der LaTeX Project Public Lizenz,
% Version 1.3c, verteilt und/oder veraendert werden.
% Die neuste Version dieser Lizenz ist
%   http://www.latex-project.org/lppl.txt
% und Version 1.3c ist Teil aller Verteilungen von LaTeX
% Version 2005/12/01 oder spaeter und dieses Werks.
%
% Dieses Werk hat den LPPL-Verwaltungs-Status "author-maintained"
% (allein durch den Autor verwaltet).
%
% Der Aktuelle Verwalter und Autor dieses Werkes ist Markus Kohm.
% 
% Dieses Werk besteht aus den in manifest.txt aufgefuehrten Dateien.
% ======================================================================
%
% Chapter about scrlttr2 of the KOMA-Script guide expert part
% Maintained by Markus Kohm
%
% ----------------------------------------------------------------------
%
% Kapitel ueber scrlttr2 im Experten-Teil der KOMA-Script-Anleitung
% Verwaltet von Markus Kohm
%
% ============================================================================

\ProvidesFile{scrlttr2-experts.tex}[2012/02/16 KOMA-Script guide (chapter:
scrlttr2 for experts)]

\translator{Markus Kohm and Gernot Hassenpflug}

\chapter{Additional Information about the Letter Class
  \Class{scrlttr2}\protect\footnote{This section is still missing. Translators from German to English would be welcome! }}
\labelbase{scrlttr2-experts}

Sorry, guys but the translation of this chapter is still missing. Translators
from German to English would be welcome!

\section{Information from the old English Guide\protect\footnote{This
    section is deprecated and should be transformed into the new
    structure. Translators from German to English would be welcome!}}

Nevertheless you may find here some information, that have been moved from the
old English guide and will be placed here until the translation of the
corresponding German chapter will be complete. But you should note, that this
information is not structured or ordered and may be out-dated, deprecated and
even wrong!

% ============================================================================
% Verschoben von scrlttr2.tex:

\begin{Declaration}
  \Macro{usekomavar}\OParameter{command}\Parameter{name}\\
  \Macro{usekomavar*}\OParameter{command}\Parameter{name}
\end{Declaration}
\BeginIndex{Cmd}{usekomavar}\BeginIndex{Cmd}{usekomavar*}%
\begin{Explain}
  The commands \Macro{usekomavar} and \Macro{usekomavar*} are,
  similarly to all commands where a starred version exists or which
  can take an optional argument, not fully expandable. Nevertheless,
  if used within \Macro{markboth}\IndexCmd{markboth},
  \Macro{markright}\IndexCmd{markright} or similar commands, you need
  not insert a \Macro{protect}\IndexCmd{protect} before using them.
  Of course this is also true for \Macro{markleft}\IndexCmd{markleft}
  if using package \Package{scrpage2}. However, these kinds of
  commands can not be used within commands like
  \Macro{MakeUppercase}\IndexCmd{MakeUppercase} which directly
  influence their argument.
  \Macro{MakeUppercase}\PParameter{\Macro{usekomavar}\Parameter{name}}
  would result in \Macro{usekomavar}\Parameter{NAME}. To avoid this
  problem you may use commands like \Macro{MakeUppercase} as an
  optional argument to \Macro{usekomavar} or \Macro{usekomavar*}. Then
  you will get the uppercase content of a variable using
  \Macro{usekomavar}\PValue{[\Macro{MakeUppercase}]}\Parameter{name}.
\end{Explain}
%
\EndIndex{Cmd}{usekomavar}\EndIndex{Cmd}{usekomavar*}%


\begin{Declaration}
  \Macro{ifkomavarempty}\Parameter{name}\Parameter{true}\Parameter{false}\\
  \Macro{ifkomavarempty*}\Parameter{name}\Parameter{true}\Parameter{false}
\end{Declaration}
\BeginIndex{Cmd}{ifkomavarempty}%
\BeginIndex{Cmd}{ifkomavarempty*}%
\begin{Explain}
  It is important to know that the content or description of the
  variable will be expanded as far as this is possible with
  \Macro{edef}. If this results in spaces or unexpandable macros like
  \Macro{relax}, the result will be not empty even where the use of
  the variable would not result in any visible output.
  
  Both variants of the command also must not be used as the argument
  of \Macro{MakeUppercase}\IndexCmd{MakeUppercase} or other commands
  which have similar effects to their arguments (see the description
  of \Macro{usekomavar} above for more information about using
  commands like \Macro{usekomavar} or \Macro{ifkomavarempty} at the
  argument of \Macro{MakeUppercase}). However, they are robust enough
  to be used as the argument of, e.\,g., \Macro{markboth} or
  \Macro{footnote}.
\end{Explain}
\EndIndex{Cmd}{ifkomavarempty*}%
\EndIndex{Cmd}{ifkomavarempty}%

% Pseudo-lengths:
\subsection{The Pseudo-Lengths}

{\TeX} works with a fixed number of registers. There are registers for tokens,
for boxes, for counters, for skips and for dimensions.  Overall there are 256
registers for each of these categories. For {\LaTeX} lengths, which are
addressed with \Macro{newlength}, skip registers are used. Once all these
registers are in use, you can not define any more additional lengths.  The
letter class \Class{scrlttr2} would normally use up more than 20 of such
registers for the first page alone. {\LaTeX} itself already uses 40 of these
registers. The \Package{typearea} package needs some of them too; thus,
approximately a quarter of the precious registers would already be in
use. That is the reason why lengths specific to letters in \Class{scrlttr2}
are defined with macros instead of lengths. The drawback of this approach is
that computations with macros is somewhat more complicated than with real
lengths.

It can be pointed out that the now recommended {\LaTeX} installation with
{\eTeX} no longer suffers from the above-mentioned limitation. However, that
improvement came too late for \Class{scrlttr2}.

\begin{figure}
  \centering
  \includegraphics{plenDIN}
  \caption{Schematic of the pseudo-lengths for a letter}
  \label{fig:scrlttr2-experts.pseudoLength}
\end{figure}

\begin{desctable}
  \caption{Pseudo-lengths provided by class \Class{scrlttr2}%
    \label{tab:scrlttr2-experts.pseudoLength}}\\
  \Endfirsthead %
  \caption[]{Pseudo-lengths provided by class \Class{scrlttr2}
    (\emph{continued})}\\%
  \Endhead%
  \standardfoot%
  \pentry{backaddrheight}{%
    height of the return address at the upper edge of the address
    field (\autoref{sec:scrlttr2.addressee},
    \autopageref{desc:scrlttr2.plength.backaddrheight})}%
  \pentry{bfoldmarkvpos}{%
    vertical distance of lower foldmark from top paper edge
    (\autoref{sec:scrlttr2.other},
    \autopageref{desc:scrlttr2.plength.bfoldmarkvpos})}%
  \pentry{firstfootvpos}{%
    vertical distance of letterfoot from top paper edge
    (\autoref{sec:scrlttr2.firstFoot},
    \autopageref{desc:scrlttr2.plength.firstfootvpos})}%
  \pentry{firstfootwidth}{%
    width of letterfoot; letterfoot is centered horizontally on letter
    paper (\autoref{sec:scrlttr2.firstFoot},
    \autopageref{desc:scrlttr2.plength.firstfootwidth})}%
  \pentry{firstheadvpos}{%
    vertical distance of letterhead from top paper edge
    (\autoref{sec:scrlttr2.firstHead},
    \autopageref{desc:scrlttr2.plength.firstheadvpos})}%
  \pentry{firstheadwidth}{%
    width of letter head; letterhead is centered horizontally on
    letter paper (\autoref{sec:scrlttr2.firstHead},
    \autopageref{desc:scrlttr2.plength.firstheadwidth})}%
  \pentry{foldmarkhpos}{%
    horizontal distance of all foldmarks from left paper edge
    (\autoref{sec:scrlttr2.other},
    \autopageref{desc:scrlttr2.plength.foldmarkhpos})}%
  \pentry{fromrulethickness}{%
    Thickness of an optional horizontal line in the letterhead
    (\autoref{sec:scrlttr2.firstHead},
    \autopageref{desc:scrlttr2.plength.fromrulethickness})}%
  \pentry{fromrulewidth}{%
    length of an optional horizontal rule in letterhead
    (\autoref{sec:scrlttr2.firstHead},
    \autopageref{desc:scrlttr2.plength.fromrulewidth})}%
  \pentry{locwidth}{%
    width of supplemental data field; for zero value width is
    calculated automatically with respect to option \Option{locfield}
    that is described in \autoref{sec:scrlttr2.headoptions}
    (\autoref{sec:scrlttr2.locationField},
    \autopageref{desc:scrlttr2.plength.locwidth})}%
  \pentry{refaftervskip}{%
    vertical skip below reference fields line
    (\autoref{sec:scrlttr2.refLine},
    \autopageref{desc:scrlttr2.plength.refaftervskip})}%
  \pentry{refhpos}{%
    horizontal distance of reference fields line from left paper edge;
    for zero value reference fields line is centered horizontally on
    letter paper (\autoref{sec:scrlttr2.refLine},
    \autopageref{desc:scrlttr2.plength.refhpos})}%
  \pentry{refvpos}{%
    vertical distance of reference fields line from top paper edge
    (\autoref{sec:scrlttr2.refLine},
    \autopageref{desc:scrlttr2.plength.refvpos})}%
  \pentry{refwidth}{%
    width of reference fields line (\autoref{sec:scrlttr2.refLine},
    \autopageref{desc:scrlttr2.plength.refwidth})}%
  \pentry{sigbeforevskip}{%
    vertical skip between closing and signature
    (\autoref{sec:scrlttr2.closing},
    \autopageref{desc:scrlttr2.plength.sigbeforevskip})}%
  \pentry{sigindent}{%
    indentation of signature with respect to text body
    (\autoref{sec:scrlttr2.closing},
    \autopageref{desc:scrlttr2.plength.sigindent})}%
  \pentry{specialmailindent}{%
    left indentation of mode of dispatch within address field
    (\autoref{sec:scrlttr2.addressee},
    \autopageref{desc:scrlttr2.plength.specialmailindent})}%
  \pentry{specialmailrightindent}{%
    right indentation of mode of dispatch within address field
    (\autoref{sec:scrlttr2.addressee},
    \autopageref{desc:scrlttr2.plength.specialmailrightindent})}%
  \pentry{tfoldmarkvpos}{%
    vertical distance of upper foldmark from top paper edge
    (\autoref{sec:scrlttr2.other},
    \autopageref{desc:scrlttr2.plength.tfoldmarkvpos})}%
  \pentry{toaddrheight}{%
    height of address field (\autoref{sec:scrlttr2.addressee},
    \autopageref{desc:scrlttr2.plength.toaddrheight})}%
  \pentry{toaddrhpos}{%
    horizontal distance of address field from left paper edge, for
    positive values; or negative horizontal distance of address field
    from right paper edge, for negative values
    (\autoref{sec:scrlttr2.addressee},
    \autopageref{desc:scrlttr2.plength.toaddrhpos})}%
  \pentry{toaddrindent}{%
    left and right indentation of address within address field
    (\autoref{sec:scrlttr2.addressee},
    \autopageref{desc:scrlttr2.plength.toaddrindent})}%
  \pentry{toaddrvpos}{%
    vertical distance of address field from top paper edge
    (\autoref{sec:scrlttr2.addressee},
    \autopageref{desc:scrlttr2.plength.toaddrvpos})}%
  \pentry{toaddrwidth}{%
    width of address field (\autoref{sec:scrlttr2.addressee},
    \autopageref{desc:scrlttr2.plength.toaddrwidth})}%
\end{desctable}

\begin{Declaration}
  \Macro{@newplength}\Parameter{name}
\end{Declaration}
\BeginIndex{Cmd}{@newplength}%
This command defines an new pseudo-length. This new pseudo-length is
uniquely identified by its \PName{name}. If with this command a
redefinition of an already existing pseudo-length is attempted, the
commands exits with an error message.

Since the user in general does not define own pseudo-lengths, this
command is not intended as a user command. Thus, it can not be used
within a document, but it can, for example, be used within an
\File{lco} file.
%
\EndIndex{Cmd}{@newplength}%

\begin{Declaration}
  \Macro{@setplength}%
    \OParameter{factor}\Parameter{pseudo-length}\Parameter{value}\\
  \Macro{@addtoplength}%
    \OParameter{factor}\Parameter{pseudo-length}\Parameter{value}
\end{Declaration}
\BeginIndex{Cmd}{@setplength}%
\BeginIndex{Cmd}{@addtoplength}%
Using the command \Macro{@setplength} you can assign the multiple of a
\PName{value} to a \PName{pseudo-length}. The \PName{factor} is given
as an optional argument (see also \Macro{setlengthtoplength}). The
command \Macro{@addtoplength} adds the \PName{value} to a
\PName{pseudo-length}. To assign, or to add the multiple of, one
\PName{pseudo-length} to another pseudo-length, the command
\Macro{useplength} is used within \PName{value}.  To subtract the
value of one pseudo-length from another \PName{pseudo-length} a minus
sign, or \PValue{-1}, is used as the \PName{factor}.

Since the user in general does not define own pseudo-lengths, this
command is not intended as a user command. Thus, it can not be used
within a document, but can, for example, be used within an \File{lco}
file.
%
\EndIndex{Cmd}{@setplength}%
\EndIndex{Cmd}{@addtoplength}%

%%% Local Variables:
%%% mode: latex
%%% coding: us-ascii
%%% TeX-master: "guide.tex"
%%% End:
