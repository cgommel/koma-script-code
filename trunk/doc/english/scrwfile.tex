% ======================================================================
% scrwfile.tex
% Copyright (c) Markus Kohm, 2010-2015
%
% This file is part of the LaTeX2e KOMA-Script bundle.
%
% This work may be distributed and/or modified under the conditions of
% the LaTeX Project Public License, version 1.3c of the license.
% The latest version of this license is in
%   http://www.latex-project.org/lppl.txt
% and version 1.3c or later is part of all distributions of LaTeX 
% version 2005/12/01 or later and of this work.
%
% This work has the LPPL maintenance status "author-maintained".
%
% The Current Maintainer and author of this work is Markus Kohm.
%
% This work consists of all files listed in manifest.txt.
% ----------------------------------------------------------------------
% scrwfile.tex
% Copyright (c) Markus Kohm, 2010-2015
%
% Dieses Werk darf nach den Bedingungen der LaTeX Project Public Lizenz,
% Version 1.3c, verteilt und/oder veraendert werden.
% Die neuste Version dieser Lizenz ist
%   http://www.latex-project.org/lppl.txt
% und Version 1.3c ist Teil aller Verteilungen von LaTeX
% Version 2005/12/01 oder spaeter und dieses Werks.
%
% Dieses Werk hat den LPPL-Verwaltungs-Status "author-maintained"
% (allein durch den Autor verwaltet).
%
% Der Aktuelle Verwalter und Autor dieses Werkes ist Markus Kohm.
% 
% Dieses Werk besteht aus den in manifest.txt aufgefuehrten Dateien.
% ======================================================================
%
% Chapter about scrwfile of the KOMA-Script guide
% Maintained by Markus Kohm
%
% ----------------------------------------------------------------------
%
% Kapitel ueber scrwfile in der KOMA-Script-Anleitung
% Verwaltet von Markus Kohm
%
% ============================================================================

\KOMAProvidesFile{scrwfile.tex}%
                 [$Date$
                  KOMA-Script guide (chapter: scrwfile)]

% Date of translated German file: 2015-03-31

\translator{Markus Kohm\and Jana Schubert}

\chapter[{Economise and Replace Files Using \Package{scrwfile}}]
{Economise and Replace Files Using \Package{scrwfile}%
}
\labelbase{scrwfile}

\BeginIndex{Package}{scrwfile}

\TeX{} supports 18 write handles only. Handle 0 is used by \TeX{} itself (log
file). \LaTeX{} needs at least handle 1 for \Macro{@mainaux}, handle 2 for
\Macro{@partaux}, one handle for \Macro{tableofcontents}, one handle for
\Macro{listoffigures}, one handle for \Macro{listoftables}, one handle for
\Macro{makeindex}. So there are 11 left. Seams a lot and enough. But every new
type of float, every new index and several other packages, e.g.,
\Package{hyperref} need write handles, too.

The bottom line is, that this eventually will result in the error message:
\begin{lstcode}
  ! No room for a new \write .
  \ch@ck ...\else \errmessage {No room for a new #3}
                                                    \fi
\end{lstcode}

There is an additional disadvantage of immediately opening a new write handle
for every table of contents, list of figures, list of tables etc.  These are
not only set by the corresponding commands, they also could not be set once
more, because their helper files are empty after the corresponding commands
until the end of the document.

Package \Package{scrwfile} provides an amendment of the \LaTeX{} kernel,
that solves both problems.

\section{General Modifications of the \LaTeX{} Kernel}
\seclabel{kernelpatches}

To allocate a new file handle eg. for \Macro{listoffigures} or
\Macro{listoftables} \LaTeX{} classes use the \LaTeX{} kernel command
\Macro{@starttoc}\IndexCmd{@starttoc}. This command not only inputs the
associated helper file but also allocates a new write handle for the
associated helper file and opens it for writing. Nevertheless, if afterwards
new entries to these lists of floats are added using \Macro{addcontentsline},
then these file handles are not used immediately, instead \LaTeX{} writes
\Macro{@writefile}\IndexCmd{@writefile} commands into the
\File{aux}-file. Only while reading the \File{aux}-file while the end of the
document, those \Macro{@writefile} commands become real write operations
on the helper files. Additionally \LaTeX{} does not close the helper files
explicitly. Instead \LaTeX{} relies on \TeX{} to close all open files at
the end.

Thus all the helper files are open throughout the entire process. However the
content is written at \Macro{end}\PParameter{document}. The basic idea of
\Package{scrwfile} is tackling this contradiction by redefining
\Macro{@starttoc} and \Macro{@writefile}.

Surely\textnote{Attention!}, changes of the \LaTeX{} kernel may
result in incompatibilities with other packages. In case of
\Package{scrwfile}, clashes my arise with all packages also redefining
\Macro{@starttoc} or \Macro{@writefile}. Sometimes changing the order of
loading the packages may help.

However, only few such problems have been reported eventhough several users
have tested the package for one year before its first release. See
\autoref{sec:scrwfile.incompatible} for more information about known
incompatibilities. If you find such a problem, please contact the
\KOMAScript{} author.

\section{The Single File Feature}
\seclabel{singlefilefeature}

At the point the package is loaded using, e.\,g.,
\begin{lstcode}[belowskip=\dp\strutbox]
  \usepackage{scrwfile}
\end{lstcode}
\Package{scrwfile} will redefine \Macro{@starttoc}\IndexCmd{@starttoc} to not
longer allocate a write handle or open any file for writing. Immediately before
closing the \File{aux}-file in \Macro{end}\PParameter{document} it will
redefine \Macro{@writefile} to no longer write into the usual helper files
but into one single new file with file extension \File{wrt}. After
reading the \File{aux}-file this \File{wrt}-file will be processed once per
helper file. This means, that not all of the helper file have to be open at
the same time, but only one at a time. And this single file will be closed
afterwards and the write handle is not longer needed after it is closed. An
internal write handle of \LaTeX{} is used for this. So \Package{scrwfile}
doesn't need any own write handle.

Because of this, even if only one table of contents should be generated, 
loading \Package{scrwfile} gives one extra write file handle,
e.\,g., for bibliographies, indexes, glossaries and similar, that are not
using \Macro{@starttoc}. Additionally the number of tables of contents and
lists of whatever, that use \Macro{@starttoc}\IndexCmd{@starttoc} is not
limited any longer.

\section{The Clone File Write Feature}
\seclabel{clonefilefeature}

Sometimes it is useful to input a file not only once but several times. As
\Macro{@starttoc} does not open files for writing any more, this can be done
by simply using \Macro{@starttoc} several times with the same extension. But
sometimes you may have additional entries in only some of the content
directories. \Package{scrwfile} allows to copy all entries of a file to
another file, too. We call this cloning.

\begin{Declaration}
  \Macro{TOCclone}%
  \OParameter{heading}\Parameter{source}\Parameter{destination}
\end{Declaration}
\BeginIndex{Cmd}{TOCclone}%
activates the clone feature for files with extensions \PName{source} and
\PName{destination}. All entries to the file
\Macro{jobname}\File{.}\PName{source} will be added to
\Macro{jobname}\File{.}\PName{destination}.

If extension \PName{destination} is a new one, \PName{destination} will be
added to the list of known extensions using the \KOMAScript{} package
\Package{tocbasic}.

If the optional argument \PName{heading} is given, a new list-of macro
\Macro{listof}\PName{destination} is defined. \PName{heading} will be used
as section (or chapter) heading of this list. In this case several
\Package{tocbasic} features of the \PName{source} will be copied to
\PName{destination}, if and only if they have been set up when
\Macro{TOCclone} was used. Feature \PName{nobabel} will always be set,
because the language selection commands are part of the helper file and
would be cloned, anyway.

\begin{Example}
Assumed, you want a short table of contents
with only the chapter level but an additional entry with the table of
contents:
\begin{lstcode}
  \usepackage{scrwfile}
  \TOCclone[Short \contentsname]{toc}{stoc}
\end{lstcode}
This would create a new table of contents with the heading ``Short
Contents''. The new table of contents uses a helper file with extension
\File{stoc}. All entries to the helper file with extension \File{toc} will
also be copied to this new helper file.

The new short table of contents should only have the chapter entries. This may
be done using:
\begin{lstcode}
  \addtocontents{stoc}{\protect\value{tocdepth}=0}
\end{lstcode}
Normally you cannot write into a helper file before
\Macro{begin}\PParameter{document}. But using \Package{scrwfile} changes
this. So the code above is correct already after loading \Package{scrwfile}.

To show the new short contents of helper file extension \File{stoc} we
use
\begin{lstcode}[moretexcs={listofsttoc}]
  \listofstoc
\end{lstcode}
somewhere after \Macro{begin}\PParameter{document}.

If we also want an entry for the table of contents at the short contents, we
cannot use
\begin{lstcode}
  \addtocontents{toc}{% write to the Contents
    \protect\addcontentslinedefault{stoc}% write to Short Contents
      {chapter}% a chapter entry with
      {\contentsname}% the Contents' name
  }
\end{lstcode}
because the \Macro{addcontentsline} command would be copied to \File{stoc}
too. So we cannot add the command to the \File{toc}-file. Package
\Package{tocbasic}\important{\Package{tocbasic}} may be used to solve this:
\csname phantomsection\endcsname\label{example:scrwfile.BeforeStartingTOC}
\begin{lstcode}
  \BeforeStartingTOC[toc]{%
    \addcontentslinedefault{stoc}{chapter}
                    {\protect\contentsname}%
  }
\end{lstcode}
However, this needs, that the file with extension \File{toc} is under control
of package \Package{tocbasic}, which is indeed the case within all
\KOMAScript{} classes.  See \autoref{sec:tocbasic.toc} on
\autopageref{desc:tocbasic.cmd.BeforeStartingTOC} for more information about
\Macro{BeforeStartingTOC}.%
\end{Example}
\EndIndex{Cmd}{TOCclone}

\section{Note on State of Development}
\seclabel{draft}

Eventhough this package has been tested by several users and even is in
productivity usage several times it is still under construction. Therefore,
there might be amendments especially to the internal functionality. Most likely
the package will be extended. Some code for extensions is already in the
package. However, there is currently no user documentation available, as up to
now nobody has requested any of these extensions.

\section{Known Package Incompatibilities}
\seclabel{incompatible}

As mentioned in \autoref{sec:scrwfile.kernelpatches},
\Package{scrwfile} redefines some commands of the \LaTeX{} kernel. This
happens not only while loading the package, but indeed at different times of
processing a document, e.\,g., just before reading the
\File{aux}-file. This\textnote{Attention!} results in incompatibility with
packages that also redefine these commands at run-time.

The \Package{titletoc}\important{Package{titletoc}}\IndexPackage{titletoc}
package is an example for such an incompatibility. That
package redefines \Macro{@writefile} under some conditions at run-time. If you
use both, \Package{scrwfile} and \Package{titletoc}, there is no warranty for
the correct behaviour of neither of them. This is neither an error of
\Package{titletoc} nor of \Package{scrwfile}.

\EndIndex{Package}{scrwfile}


%%% Local Variables:
%%% mode: latex
%%% mode: flyspell
%%% coding: us-ascii
%%% ispell-local-dictionary: "en_GB"
%%% TeX-master: "../guide"
%%% End: 
