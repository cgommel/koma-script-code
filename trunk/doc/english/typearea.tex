% ======================================================================
% typearea.tex
% Copyright (c) Markus Kohm, 2001-2016
%
% This file is part of the LaTeX2e KOMA-Script bundle.
%
% This work may be distributed and/or modified under the conditions of
% the LaTeX Project Public License, version 1.3c of the license.
% The latest version of this license is in
%   http://www.latex-project.org/lppl.txt
% and version 1.3c or later is part of all distributions of LaTeX 
% version 2005/12/01 or later and of this work.
%
% This work has the LPPL maintenance status "author-maintained".
%
% The Current Maintainer and author of this work is Markus Kohm.
%
% This work consists of all files listed in manifest.txt.
% ----------------------------------------------------------------------
% typearea.tex
% Copyright (c) Markus Kohm, 2001-2016
%
% Dieses Werk darf nach den Bedingungen der LaTeX Project Public Lizenz,
% Version 1.3c, verteilt und/oder veraendert werden.
% Die neuste Version dieser Lizenz ist
%   http://www.latex-project.org/lppl.txt
% und Version 1.3c ist Teil aller Verteilungen von LaTeX
% Version 2005/12/01 oder spaeter und dieses Werks.
%
% Dieses Werk hat den LPPL-Verwaltungs-Status "author-maintained"
% (allein durch den Autor verwaltet).
%
% Der Aktuelle Verwalter und Autor dieses Werkes ist Markus Kohm.
% 
% Dieses Werk besteht aus den in manifest.txt aufgefuehrten Dateien.
% ======================================================================
%
% Chapter about typearea of the KOMA-Script guide
% Maintained by Markus Kohm
%
% ----------------------------------------------------------------------
%
% Kapitel ueber typearea in der KOMA-Script-Anleitung
% Verwaltet von Markus Kohm
%
% ======================================================================

\KOMAProvidesFile{typearea.tex}%
                 [$Date$
                  KOMA-Script guide (chapter: typearea)]
\translator{Markus Kohm\and Gernot Hassenpflug\and Krickette Murabayashi}

% Date of the translated German file: 2016-11-14

\chapter{Construction of the Page Layout with \Package{typearea}}
\BeginIndexGroup
\BeginIndex{Package}{typearea}
\labelbase{typearea}

Many {\LaTeX} classes, including the standard classes, present the user
with the largely fixed configuration of margins and typearea. With the
standard classes, the configuration determined is very much dependent
on the chosen font size. There are
separate packages, such as
\Package{geometry}\IndexPackage{geometry} (see
\cite{package:geometry}), which give the user complete control, but
also full responsibility, of the settings of typearea and margins.

\KOMAScript{} takes a somewhat different approach with its
\Package{typearea} package. Here the user is given several
construction setting and automatization possibilities based on
established typography standards in order to help guide him or her in
making a good choice.

It should be noted that the \Package{typearea} package makes use of
the \Package{scrbase} package. The latter is explained in the expert
section \iffree{of this document}{of this book} in
\autoref{cha:scrbase} from \autopageref{cha:scrbase} onwards. The
majority of the rules documented there are however not directed at the
user, but rather at authors of classes and packages. 


\section{Fundamentals of Page Layout}
\seclabel{basics}

\begin{Explain}
  If you look at a single page of a book or other printed materials,
  you will see that it consists of top, bottom, left, and right
  margins, a (running) head area, the text block, and a (running) foot
  area.  There is also a space between the head area and the text
  block, and between the text block and the foot area. The relations
  between these areas are called the \emph{page
    layout}.\Index[indexmain]{page layout}

%%
%% GH: sollte man schon Hinweise zum Buch hineinarbeiten? Auskommentiert?
%%  \ifx\BUCH\undefined\cite{DANTE:TK0402:MJK}\else(siehe
%%   \autoref{cha:typeareaconstruction})\fi
  The literature contains much discussion of different algorithms and
  heuristic approaches for constructing a good page layout%
  \iffree{ \cite{DANTE:TK0402:MJK}}{. A short introduction to the basics may
    be found at \autoref{cha:typeareaconstruction}}%
  .  Often mentioned is an approach which involves diagonals and their
  intersections. The result is a page where the text block proportions are
  related to the proportions of the \emph{page}.  In a single-sided document,
  the left and the right margin should have equal widths.  The relation of the
  upper margin to the lower margin should be 1\(:\)2. In a double-sided
  document\Index{double-sided} (e.\,g. a book) however, the complete inner
  margin (the margin at the spine) should be the same as each of the two outer
  margins; in other words, a single page contributes only half of the inner
  margin.

  In the previous paragraph, we mentioned and emphasized \emph{the
    page}. Erroneously, it is often thought that with the page format
  the page is the same as the paper format.\Index[indexmain]{paper
    format} However, if you look at a bound document, it is obvious
  that part of the paper vanishes in the
  binding\Index[indexmain]{binding} and is no longer part of the
  visible page. For the page layout, it is not the format of the paper
  which is important, it is the impression of the visible page to the
  reader. Therefore, it is clear that the calculation of the page
  layout must account for the ``lost'' paper in the binding and add
  this amount to the width of the inner margin. This is called the
  \emph{binding correction}.\Index[indexmain]{binding correction} The
  binding correction is therefore calculated as part of the
  \emph{gutter}\Index[indexmain]{gutter}, not % however of
  the visible
  inner margin.

  The binding correction depends on the process of actually
  producing the document and thus cannot be calculated in general.
  Every production process needs its own parameter. In professional
  binding, this parameter is not too important since the printing is
  done on oversized paper which is then cropped to the right size.
  The cropping is done in a way so that the relations for the
  visible double-sided page are as explained above.

  Now we know about the relations of the individual parts of a page.
  However, we do not yet know about the width and the height of the
  text block. Once we know one of these values, we can calculate
  all the other values from the paper format and the page format or
  the binding correction.
  \begin{align*}
    \Var{textblock~height} \Index{textblock height} : \Var{textblock~width} &=
    \Var{page~height}\Index{page} : \Var{page~width}\\
    \Var{top~margin}\Index{margin} : \Var{foot~margin} &=
      \text{1} : \text{2} \\
%
    \Var{left~margin} : \Var{right~margin} &=  \text{1} : \text{1} \\
%
    \Var{half~inner~margin} : \Var{outer~margin} &= \text{1} : \text{2} \\
%
   \Var{page~width} &= 
      \Var{paper~width}\Index{paper} - 
      \Var{binding~correction}\Index{binding correction}\\
%
    \Var{top~margin} + \Var{bottom~margin} &=
    \Var{page~height} - \Var{textblock~height} \\
%
    \Var{left~margin} + \Var{right~margin} &=
    \Var{page~width} - \Var{textblock~width} \\
%
    \Var{half~inner~margin} + \Var{outer~margin} &=
    \Var{page~width} - \Var{textblock~width} \\
%
    \Var{half~inner~margin} + \Var{binding~correction} &=
    \Var{gutter}\Index{gutter}
  \end{align*}
  \Index[indexmain]{margin}%
  The values \Var{left~margin} and \Var{right~margin} only exist in a
  single-sided document while \Var{half~inner~margin} and
  \Var{outer~margin} only exist in a double-sided document.  In these
  equations, we work with \Var{half~inner~margin} since the full inner
  margin belongs to a double-page. Thus, one page has only half of the
  inner margin, \Var{half~inner~margin}.

  The question of the width of the textblock is also discussed in
  the literature. The optimum width depends on several factors:
  \begin{itemize}
  \item size, width, type of the font used
  \item line spacing
  \item word length
  \item available room
  \end{itemize}
  The importance of the font becomes clear once you think about the
  meaning of serifs. Serifs\Index[indexmain]{serifs} are fine strokes
  finishing off the lines of the letters. Letters whose main strokes
  run orthogonal to the text line disturb the flow rather than keeping
  and leading the eye along the line. Those letters then have serifs
  at the ends of the vertical strokes so that the horizontal serifs
  can help lead the eye horizontally. In addition, they help the eye
  to find the beginning of the next line more quickly. Thus, the line
  length for a serif font can be slightly longer than for a sans serif
  font.

  With leading\Index[indexmain]{leading} is meant the vertical
  distance between individual lines of text. In {\LaTeX}, the leading
  is set at about 20\% of the font size. With commands like
  \Macro{linespread}\IndexCmd{linespread} or, better, packages like
  \Package{setspace}\IndexPackage{setspace} (see
  \cite{package:setspace}), the leading can be changed. A wider
  leading helps the eye to follow the line. A very wide leading, on
  the other hand, disturbs reading because the eye has to move a wide
  distance between lines. Also, the reader becomes uncomfortable
  because of the visible stripe effect. The uniform gray value of the
  page is thereby spoiled. Still, with a wider leading, the lines can
  be longer.

  The literature gives different values for good line
  lengths\Index[indexmain]{line length}, depending on the author. To
  some extent, this is related to the native language of the
  author. Since the eye jumps from word to word, short words make this
  task easier. Considering all languages and fonts, a line length of
  60 to 70 characters, including spaces and punctuation, forms a
  usable compromise. This requires well-chosen leading, but {\LaTeX}'s
  default is usually good enough. Longer line lengths should only be
  considered for highly-developed readers who spend several hours
  daily reading. However, even for such readers, line lengths greater than 80
  characters are unsuitable. In any case, the leading must be
  appropriately chosen. An extra 5\% to 10\% is recommended as a good
  rule of thumb. With fonts such as Palatino, which require some 5\%
  more leading even at normal line lengths, even more can be
  required.

  Before looking at the actual construction of the page layout, there
  are just some minor things left to know. {\LaTeX} does not start the
  first line in the text block of a page at the upper edge of the text
  block, but sets the baseline at a defined distance from the top of
  the text block. Also, {\LaTeX} knows the commands
  \Macro{raggedbottom}\IndexCmd{raggedbottom} and
  \Macro{flushbottom}\IndexCmd{flushbottom}. \Macro{raggedbottom}
  specifies that the last line of a page should be positioned wherever
  it was calculated. This means that the position of this line can be
  different on each page, up to the height of one line\,---\,in
  combination of the end of the page with titles, figures, tables or
  similar, even more. In double-sided documents this is usually
  undesirable. \Macro{flushbottom} makes sure that the last line is
  always at the lower edge of the text block. To achieve this,
  {\LaTeX} sometimes needs to stretch vertical glue more than
  allowed. Paragraph skip is such a stretchable, vertical glue, even
  when set to zero.  In order to not stretch the paragraph skip on
  normal pages where it is the only stretchable glue, the height of
  the text block should be set to a multiple of the height of the text
  line, including the distance from the upper edge of the text block
  to the first line.

  This concludes the introduction to page layout as handled by
  {\KOMAScript}. Now, we can begin with the actual construction.
\end{Explain}


\section{Page Layout Construction by Dividing}
\seclabel{divConstruction}

\begin{Explain}
  The easiest way to make sure that the text area has the same ratios
  as the page is as follows:
  \begin{itemize}
  \item First, subtract the part \Var{BCOR}, required for the binding
    correction\Index{binding correction}, from the inner edge of the paper,
    and divide the rest of the page vertically into \Var{DIV} rows of equal
    height.
  \item Next, divide the page horizontally into the same number (\Var{DIV}) of
    columns.
  \item Then, take the uppermost row as the upper margin and the two lowermost
    rows as the lower margin. If you are printing double-sided, you similarly
    take the innermost column as the inner margin and the two outermost
    columns as the outer margin.
  \item Then add the binding correction \Var{BCOR} to the inner margin.
  \end{itemize}
  What now remains of the page is the text area.\Index{text area} The width
  and the height of the text area and margins result automatically from the
  number of rows and columns \Var{DIV}. Since the margins always need three
  stripes, \Var{DIV} must be necessarily greater than three. In order that the
  text area occupy at least twice as much space as the margins, \Var{DIV}
  should really be equal to or greater than 9. With this value, the
  construction is also known as the \emph{classical division factor of 9} (see
  \autoref{fig:typearea.nineparts}).

  \begin{figure}
    \centering
    \setlength{\unitlength}{.25mm}
    \definecolor{komalight}{gray}{.75}
    \definecolor{komamed}{gray}{.6}
    \definecolor{komadark}{gray}{.3}
    \begin{picture}(420,297)
      % BCOR
      \put(198,0){\color{komalight}\rule{24\unitlength}{297\unitlength}}
      \multiput(198,2)(0,20){15}{\thinlines\line(0,1){10}}
      \multiput(222,2)(0,20){15}{\thinlines\line(0,1){10}}
      % the paper
      \put(0,0){\thicklines\framebox(420,297){}}
%      \put(210,0){\thicklines\framebox(210,297){}}
      % the page layout
      \put(44,66){\color{komamed}\rule{132\unitlength}{198\unitlength}}
      \put(244,66){\color{komamed}\rule{132\unitlength}{198\unitlength}}
      % helper lines
      \multiput(0,33)(0,33){8}{\thinlines\line(1,0){198}}
      \multiput(222,33)(0,33){8}{\thinlines\line(1,0){198}}
      \multiput(22,0)(22,0){8}{\thinlines\line(0,1){297}}
      \multiput(244,0)(22,0){8}{\thinlines\line(0,1){297}}
      % annotations
      \put(198,0){\color{white}\makebox(24,297)[c]{%
          \rotatebox[origin=c]{-90}{binding correction}}}
      \put(44,66){\color{white}\makebox(132,198)[c]{page layout left}}
      \put(244,66){\color{white}\makebox(132,198)[c]{page layout right}}
      % box numbers
      \makeatletter
      \multiput(1,27)(0,33){9}{\footnotesize\makebox(0,0)[l]{\the\@multicnt}}
      \multiput(177,291)(-22,0){9}{%
        \footnotesize\makebox(0,0)[l]{\the\@multicnt}}
      \multiput(419,27)(0,33){9}{%
        \footnotesize\makebox(0,0)[r]{\the\@multicnt}}
      \multiput(243,291)(22,0){8}{%
        \footnotesize\makebox(0,0)[r]{\the\numexpr\@multicnt+1\relax}}
      \makeatother
    \end{picture}
    \caption{Double-sided layout with the box construction of the classical division factor of 9, after subtraction of a binding correction}
    \label{fig:typearea.nineparts}
  \end{figure}

  In {\KOMAScript}, this kind of construction is implemented in the
  \Package{typearea} package, where the bottom margin may drop any
  fractions of a line in order to conform with the minor condition for
  the text area height mentioned in the previous paragraph, and
  thereby to minimize the mentioned problem with
  \Macro{flushbottom}. For A4 paper, \Var{DIV} is predefined according
  to the font size (see \autoref{tab:typearea.div},
  \autopageref{tab:typearea.div}). If there is no binding correction
  (\(\Var{BCOR} = 0\Unit{pt}\)), the results roughly match the values
  of \autoref{tab:typearea.typearea},
  \autopageref{tab:typearea.typearea}.

  In addition to the predefined values, one can specify \Var{BCOR} and
  \Var{DIV} as options when loading the package (see
  \autoref{sec:typearea.options}, from
  \autopageref{sec:typearea.typearea} onwards). There is also a
  command to explicitly calculate the type area by providing these
  values as parameters (also see \autoref{sec:typearea.options},
  \DescPageRef{typearea.cmd.typearea}).

  The \Package{typearea} package can automatically determine the
  optimal value of \Var{DIV} for the font and leading used.
  Again, see \autoref{sec:typearea.options},
  \DescPageRef{typearea.option.DIV.calc}.
\end{Explain}


\section{Page Layout Construction by Drawing a Circle}
\seclabel{circleConstruction}

\begin{Explain}
  In addition to the page layout construction\Index{page layout}
  method previously described, a somewhat more classical method can be
  found in the literature. The aim of this method is not only to
  obtain identical ratios in the page proportions, but it is
  considered optimal when the height of the text block is the same as
  the width of the page. The exact method is described in
  \cite{JTsch87}.

  A disadvantage of this late Middle Age method is that the width of
  the text area is no longer dependent on the font. Thus, one does not
  choose the text area to match the font, but the author or typesetter
  has to choose the font according to the text area. This can be
  considered a ``must''.

  In the \Package{typearea} package this construction is changed
  slightly. By using a special (normally meaningless) \Var{DIV} value
  or a special package option, a \Var{DIV} value is chosen to match
  the perfect values of the late Middle Age method as closely as
  possible. See also \autoref{sec:typearea.options},
  \DescPageRef{typearea.option.DIV.calc}.
\end{Explain}

\LoadCommonFile{options}% \section{Early or late Selection of Options}

\LoadCommonFile{compatibility}% \section{Compatibility with Earlier Versions of \KOMAScript}

\section{Options and Macros to Influence the Page Layout}
\seclabel{typearea}

The package \Package{typearea} offers two different user interfaces to
influence type area construction. The more important method is to load
the package with options. For information on how to load packages and
to give package options, please refer to the {\LaTeX} literature,
e.\,g.  \cite{lshort} and \cite{latex:usrguide}, or the examples given
here.  Since the \Package{typearea} package is loaded automatically
when using the {\KOMAScript} main classes, the package options can be
given as class options (see \autoref{sec:maincls.options}).

In this section the \Class{protocol} class will be used, not an
existing {\KOMAScript} class but a hypothetical one. This\iffree{
  documentation}{ book} assumes that ideally there exists a class for
every specific task.

\begin{Declaration}
  \OptionVName{BCOR}{correction}
\end{Declaration}%
With the aid of the option
\OptionVName{BCOR}{correction}\ChangedAt{v3.00}{\Package{typearea}}
one may specify the absolute value of the binding
correction\Index{binding correction}, i.\,e., the width of the area
which will be lost from the paper width in the binding process. This
value is then automatically taken into account in the page layout
construction and in the final output is added to the inner (or the
left) margin. For the \PName{correction} specification any measurement
unit understood by \TeX{} is valid.

\begin{Example}
  Assume one is creating a financial report, which should be printed
  out single-sided on A4 paper, and finally kept in a clamp
  folder. The clamp will hide 7.5\Unit{mm}. The stack of pages is very
  thin, thus through paging at most another 0.75\Unit{mm} will be
  lost. Therefore, one may write:
\begin{lstcode}
  \documentclass[a4paper]{report}
  \usepackage[BCOR=8.25mm]{typearea}
\end{lstcode}
  or
\begin{lstcode}
  \documentclass[a4paper,BCOR=8.25mm]{report}
  \usepackage{typearea}
\end{lstcode}
  when using \Option{BCOR} as a global option.

  When using a {\KOMAScript} class, the explicit loading of the
  \Package{typearea} package can be omitted:
\begin{lstcode}
  \documentclass[BCOR=8.25mm]{scrreprt}
\end{lstcode}
  The option \Option{a4paper} could be omitted with \Class{scrreprt},
  since this is a predefined setting for all {\KOMAScript} classes.

  If the option is only later set to a new value, one may then use,
  for example, the following:
\begin{lstcode}
  \documentclass{scrreprt}
  \KOMAoptions{BCOR=8.25mm}
\end{lstcode}
  Thus, at the loading of the \Class{scrreprt} class standard settings
  will be used. When changing the setting with the use of the command
  \DescRef{\LabelBase.cmd.KOMAoptions} or \DescRef{\LabelBase.cmd.KOMAoption} a new page layout with new
  margins will automatically be calculated.
\end{Example}

Please note that when using this option with one of the {\KOMAScript} classes
as in the example above, it must be used either as a class option, or passed
via \DescRef{\LabelBase.cmd.KOMAoptions} or
\DescRef{\LabelBase.cmd.KOMAoption} after loading the class. The
\Package{typearea} package should neither be loaded explicitly with
\DescRef{\LabelBase.cmd.usepackage} when using a {\KOMAScript} class, nor
should the option be given as an optional argument thereto. If the option is
changed via \DescRef{\LabelBase.cmd.KOMAoptions} or
\DescRef{\LabelBase.cmd.KOMAoption} after loading the package, the textblock
and margins are automatically recalculated anew.%
%
\EndIndexGroup


\begin{Declaration}
  \OptionVName{DIV}{factor}
\end{Declaration}%
With the aid of the option
\OptionVName{DIV}{factor}\ChangedAt{v3.00}{\Package{typearea}} the
number of stripes into which the page is divided horizontally and
vertically during the page layout construction is set. The exact
construction method is found in
\autoref{sec:typearea.divConstruction}. Of importance is that the
larger the \PName{factor}, the larger the text block and the smaller
the margins. Any integer value greater than 4 is valid for
\PName{factor}. Please note that large values can lead to
unfulfillment of various minor conditions in the type area, depending
on further options chosen. Thus, in an extreme case, the header may
fall outside of the page. Users applying the option
\OptionVName{DIV}{factor} are themselves responsible for fulfillment
of the marginal conditions and setting of a typographically aesthetic
line length.

In \autoref{tab:typearea.typearea} are found the type area sizes for
several \Var{DIV} factors for an A4 page without binding correction. Here
the minor conditions dependent on font size are not considered.

\begin{table}
  \centering
  \caption[{Type area dimensions dependent on \Var{DIV} for A4}]{Type area
    dimensions dependent on \Var{DIV} for A4 regardless of \Length{topskip}}
  \begin{tabular}{ccccc}
    \toprule
    & 
    \multicolumn{2}{c}{Type area} & \multicolumn{2}{c}{Margins}\\
    %\raisebox{1.5ex}[0pt]{
      \Var{DIV}
    %} 
    & 
    width [mm] & height [mm] & top [mm] & inner [mm] \\
    \midrule
    6  & 105.00 & 148.50 & 49.50 & 35.00 \\
    7  & 120.00 & 169.71 & 42.43 & 30.00 \\
    8  & 131.25 & 185.63 & 37.13 & 26.25 \\
    9  & 140.00 & 198.00 & 33.00 & 23.33 \\
    10 & 147.00 & 207.90 & 29.70 & 21.00 \\
    11 & 152.73 & 216.00 & 27.00 & 19.09 \\
    12 & 157.50 & 222.75 & 24.75 & 17.50 \\
    13 & 161.54 & 228.46 & 22.85 & 16.15 \\
    14 & 165.00 & 233.36 & 21.21 & 15.00 \\
    15 & 168.00 & 237.60 & 19.80 & 14.00 \\
    \bottomrule
  \end{tabular}
  \label{tab:typearea.typearea}
\end{table}

\iffree{}{\clearpage}%
\begin{Example}
  Assume one wants to write a meeting protocol, using the
  \Class{protocol} class. The document should be double-sided. In the
  company 12\Unit{pt} Bookman font is used. This font, which belongs
  to the standard PostScript fonts, is activated in {\LaTeX} with the
  command \verb|\usepackage{bookman}|.  The Bookman font is a very
  wide font, meaning that the individual characters have a large width
  relative to their height. Therefore, the predefined value for
  \Var{DIV} in \Package{typearea} is insufficient. Instead of the
  value of 12 it appears after thorough study of this entire chapter
  that a value of 15 should be most suitable.  The protocol will not
  be bound but punched and kept in a folder. Thus, no binding
  correction is necessary.  One may then write:
\begin{lstcode}
    \documentclass[a4paper,twoside]{protocol}
    \usepackage{bookman}
    \usepackage[DIV=15]{typearea}
\end{lstcode}
  On completion, it is decided that the protocols will from now on be
  collected and bound quarterly into book format. The binding is to be
  a simple glue binding, because it is only done to conform with
  ISO\,9000 and nobody is actually going to read them. For the binding
  including space lost in turning the pages, an average of 12\Unit{mm}
  is required. Thus, one may change the options of the
  \Package{typearea} package accordingly, and use the class for
  protocols conforming to ISO\,9000 regulations:
\begin{lstcode}
  \documentclass[a4paper,twoside]{iso9000p}
  \usepackage{bookman}
  \usepackage[DIV=15,BCOR=12mm]{typearea}
\end{lstcode}
  Of course, it is equally possible to use here a {\KOMAScript} class:
\begin{lstcode}
  \documentclass[twoside,DIV=15,BCOR=12mm]{scrartcl}
  \usepackage{bookman}
\end{lstcode}
  The \Option{a4paper} option can be left out when using the
  \Class{scrartcl} class, as it is predefined in all {\KOMAScript}
  classes.
\end{Example}

Please note that when using the \Option{DIV} option with one of the
{\KOMAScript} classes as in the example above, it must be used either as a
class option, or passed via \DescRef{\LabelBase.cmd.KOMAoptions} or
\DescRef{\LabelBase.cmd.KOMAoption} after loading the class. The
\Package{typearea} package should neither be loaded explicitly with
\DescRef{\LabelBase.cmd.usepackage} when using a {\KOMAScript} class, nor
should the option be given as an optional argument thereto. If the option is
changed via \DescRef{\LabelBase.cmd.KOMAoptions} or
\DescRef{\LabelBase.cmd.KOMAoption} after loading the package, the textblock
and margins are automatically recalculated anew.%
\EndIndexGroup

\begin{Declaration}
  \OptionValue{DIV}{calc}%
  \OptionValue{DIV}{classic}
\end{Declaration}%
As\ChangedAt{v3.00}{\Package{typearea}} already mentioned in
\autoref{sec:typearea.divConstruction}, for A4 paper there are fixed
predefined settings for the \Var{DIV} value. These can be found in
\autoref{tab:typearea.div}. If a different paper format is chosen,
then the \Package{typearea} package independently calculates an
appropriate \Var{DIV} value.  Of course this same calculation can be
applied also to A4. To obtain this result, one simply uses the
\OptionValue{DIV}{calc} option in place of the
\OptionVName{DIV}{factor} option. This option can just as easily be
explicity given for other paper formats. If one desires an automatic
calculation, this also makes good sense, since the possibility exists
to configure different predefined settings in a configuration file
(see \autoref{sec:typearea-experts.cfg}). An explicit passing of the
\OptionValue{DIV}{calc} option then overwrites such configuration
settings.

\begin{table}
  \centering
  \caption{\label{tab:typearea.div}Predefined settings of \PName{DIV} for A4}
  \begin{tabular}{lccc}
    \toprule
    base font size: & 10\Unit{pt} & 11\Unit{pt} & 12\Unit{pt} \\
    \Var{DIV}:           &   8  &  10  &  12  \\
    \bottomrule
  \end{tabular}
\end{table}

The classical page layout construction, the Middle Age book design canon,
mentioned in \autoref{sec:typearea.circleConstruction}, is similarly
selectable. Instead of the
\OptionVName{\DescRef{\LabelBase.option.DIV}}{factor} or
\OptionValue{DIV}{calc} option, one may use the \OptionValue{DIV}{classic}
option. A \Var{DIV} value closest to the Middle Age book design canon is then
chosen.

\begin{Example}
  In the example using the Bookman font with the
  \OptionVName{\DescRef{\LabelBase.option.DIV}}{factor} option, exactly that
  problem of choosing a more appropriate \Var{DIV} value for the font
  arose. As a variation on that example, one could simply leave the choice of
  such a value to the \Package{typearea} package:
\begin{lstcode}
  \documentclass[a4paper,twoside]{protocol}
  \usepackage{bookman}
  \usepackage[DIV=calc]{typearea}
\end{lstcode}
\end{Example}

Please note that when using this option with one of the {\KOMAScript} classes
as in the example above, it must be used either as a class option, or passed
via \DescRef{\LabelBase.cmd.KOMAoptions} or
\DescRef{\LabelBase.cmd.KOMAoption} after loading the class. The
\Package{typearea} package should neither be loaded explicitly with
\DescRef{\LabelBase.cmd.usepackage} when using a {\KOMAScript} class, nor
should the option be given as an optional argument thereto. If the option is
changed via \DescRef{\LabelBase.cmd.KOMAoptions} or
\DescRef{\LabelBase.cmd.KOMAoption} after loading the package, the textblock
and margins are automatically recalculated anew.%
%
\EndIndexGroup


\begin{Declaration}
  \OptionValue{DIV}{current}%
  \OptionValue{DIV}{last}
\end{Declaration}%
Readers\ChangedAt{v3.00}{\Package{typearea}} who have followed the
examples with acuity actually already know how to calculate a
\Var{DIV} value dependent on the chosen font, when a {\KOMAScript}
class is used together with a font package.

\begin{Explain}
  The problem is that the {\KOMAScript} class already loads the
  \Package{typearea} package itself. Thus, it is not possible to pass options
  as optional arguments to \DescRef{\LabelBase.cmd.usepackage}. It would also
  be pointless to pass the \OptionValueRef{\LabelBase}{DIV}{calc} option as an
  optional argument to \DescRef{\LabelBase.cmd.documentclass}. This option
  would be evaluated immediately on loading the \Package{typearea} package and
  as a result the text block and margin would be chosen according to the
  {\LaTeX} standard font and not for the later loaded font. However, it is
  quite possible to recalculate the text block and margins anew after loading
  the font, with the aid of \DescRef{\LabelBase.cmd.KOMAoptions}%
  \PParameter{\OptionValueRef{\LabelBase}{DIV}{calc}} or
  \DescRef{\LabelBase.cmd.KOMAoption}%
  \PParameter{\DescRef{\LabelBase.option.DIV}}\PParameter{calc}. Via
  \PValue{calc} an appropriate \Var{DIV} value for a good line length is then
  chosen.

  As it is often more practical to set the \Option{DIV} option not
  after loading the font, but at a more visible point, such as when
  loading the class, the \Package{typearea} package offers two further
  symbolic values for this option.
\end{Explain}

With \OptionVName{DIV}{current}\ChangedAt{v3.00}{\Package{typearea}}
a renewed calculation of text block and margin is requested, in which
the currently set \Var{DIV} will be used. This is less of interest for
renewed type area calculations after loading a different font; it is
rather more useful for determining, for example, after changing the
leading, while keeping \Var{DIV} the same, that the marginal condition
is fulfilled that \Length{textheight} less
\Length{topskip} is a multiple of \Length{baselineskip}.

With \OptionVName{DIV}{last}\ChangedAt{v3.00}{\Package{typearea}} a
renewed calculation of text block and margin is requested, where
exactly the same setting is used as in the last calculation.

\begin{Example}
  Let us take up the previous example again, in which a good line
  length is required for a type area using the Bookman font. At the
  same time, a {\KOMAScript} class is to be used. This is easily
  possible using the symbolic value \PValue{last} and the command
  \DescRef{\LabelBase.cmd.KOMAoptions}:
\begin{lstcode}
  \documentclass[BCOR=12mm,DIV=calc,twoside]{scrartcl}
  \usepackage{bookman}
  \KOMAoptions{DIV=last}
\end{lstcode}
If it should later be decided that a different \Var{DIV} value is
required, then only the setting of the optional argument to
\DescRef{\LabelBase.cmd.documentclass} need be changed.
\end{Example}

A summary of all possible symbolic values for the \Option{DIV} option
can be found in \autoref{tab:symbolicDIV}. At this point it is noted
that the use of the \Package{fontenc}\IndexPackage{fontenc} package
can also lead to \LaTeX{} loading a different font.

\begin{table}
  \caption[{Symbolic values for the \Option{DIV} option and the
    \PName{DIV} argument to \DescRef{\LabelBase.cmd.typearea}}]{Possible symbolic values for the \Option{DIV} option or the \PName{DIV} argument to
    \DescRef{\LabelBase.cmd.typearea}\OParameter{BCOR}\Parameter{DIV}}
  \label{tab:symbolicDIV}
  \begin{desctabular}
    \pventry{areaset}{Recalculate page
      layout.\IndexOption{DIV~=areaset}}%
    \pventry{calc}{Recalculate type area including choice of
      appropriate \Var{DIV} value.\IndexOption{DIV~=calc}}%
    \pventry{classic}{Recalculate type area using Middle Age book
      design canon (circle-based
      calculation).\IndexOption{DIV~=classic}}%
    \pventry{current}{Recalculate type area using current \Var{DIV}
      value.\IndexOption{DIV~=current}}%
    \pventry{default}{Recalculate type area using the standard value
      for the current page format and current font size. If no
      standard value exists, \PValue{calc} is
      used.\IndexOption{DIV~=default}}%
    \pventry{last}{Recalculate type area using the same \PName{DIV}
      argument as was used in the last call.\IndexOption{DIV~=last}}%
  \end{desctabular}
\end{table}

Often the renewed type area calculation is required in combination
with a change in the line spacing
(\emph{leading})\Index{leading}. Since the type area should be
calculated such that an integer number of lines fit in the text block,
a change in the leading normally requires a recalculation of the page
layout.
 
\begin{Example}
  For a thesis document, a font of size 10\Unit{pt} and a spacing of
  1.5 lines is required. By default, {\LaTeX} sets the leading for
  10\Unit{pt} at 2\Unit{pt}, in other words 1.2 lines. Therefore, an
  additional stretch factor of 1.25 is needed. Additionally, a binding
  correction of \(12\Unit{mm}\) is stipulated. Then the solution could be
  written as follows:
\begin{lstcode}
  \documentclass[10pt,twoside,BCOR=12mm,DIV=calc]{scrreprt}
  \linespread{1.25}
  \KOMAoptions{DIV=last}
\end{lstcode}\IndexCmd{linespread}
Since \Package{typearea} always executes the command
\Macro{normalsize} itself upon calculation of a new type area, it is
not necessary to activate the chosen leading with \Macro{selectfont}
after \Macro{linespread}, since this will be used already in the
recalculation.

When using the \Package{setspace} package (see
\cite{package:setspace}), the same example would appear as follows:
\begin{lstcode}
  \documentclass[10pt,twoside,BCOR=12mm,DIV=calc]{scrreprt}
  \usepackage{setspace}
  \onehalfspacing
  \KOMAoptions{DIV=last}
\end{lstcode}
As can be seen, with the use of the \Package{setspace} package one no
longer neesds to know the correct stretch value.

At this point it should be noted that the line spacing for the title
page should be reset to the normal value.
\iffalse% Umbruchkorrektur
  A complete example would be:
\fi
\begin{lstcode}
  \documentclass[10pt,twoside,BCOR=12mm,DIV=calc]
                {scrreprt}
  \usepackage{setspace}
  \onehalfspacing
  \AfterTOCHead{\singlespacing}
  \KOMAoptions{DIV=last}
  \begin{document}
  \title{Title}
  \author{Markus Kohm}
  \begin{spacing}{1}
    \maketitle
  \end{spacing}
  \tableofcontents
  \chapter{Ok}
  \end{document}
\end{lstcode}
  See further also the notes in \autoref{sec:typearea.tips}. The command
  \Macro{AfterTOCHead}\IndexCmd{AfterTOCHead} will be described in
  \autoref{cha:tocbasic} of \autoref{part:forExperts} on
  \DescPageRef{tocbasic.cmd.AfterTOCHead}.
\end{Example}

Please note that when using this option with one of the {\KOMAScript} classes
as in the example above, it must be used either as a class option, or passed
via \DescRef{\LabelBase.cmd.KOMAoptions} or
\DescRef{\LabelBase.cmd.KOMAoption} after loading the class. The
\Package{typearea} package should neither be loaded explicitly with
\DescRef{\LabelBase.cmd.usepackage} when using a {\KOMAScript} class, nor
should the option be given as an optional argument thereto. If the option is
changed via \DescRef{\LabelBase.cmd.KOMAoptions} or
\DescRef{\LabelBase.cmd.KOMAoption} after loading the package, the textblock
and margins are automatically recalculated anew.%
%
\EndIndexGroup


\begin{Declaration}
  \Macro{typearea}\OParameter{BCOR}\Parameter{DIV}%
  \Macro{recalctypearea}
\end{Declaration}%
\begin{Explain}
  If the \DescRef{\LabelBase.option.DIV} option or the
  \DescRef{\LabelBase.option.BCOR} option is set after loading of the
  \Package{typearea} package, then internally the command \Macro{typearea} is
  called. When setting the \DescRef{\LabelBase.option.DIV} option the symbolic
  value \PValue{current} is used internally for \PName{BCOR}, which for
  reasons of completeness is found also in \autoref{tab:symbolicBCOR}. When
  setting the \DescRef{\LabelBase.option.BCOR} option, the symbolic value
  \PValue{last} is used internally for \PName{DIV}. If it is instead desired
  that the text block and margins should be recalculated using the symbolic
  value \PValue{current} for \PName{DIV}, then
  \Macro{typearea}\POParameter{current}\PParameter{current} can be used
  directly.
\end{Explain}

\begin{table}
  \caption[{Symoblic \PName{BCOR} arguments for
    \Macro{typearea}}]{Possible symbolic \PName{BCOR} arguments for
    \Macro{typearea}\OParameter{BCOR}\Parameter{DIV}}
  \label{tab:symbolicBCOR}
  \begin{desctabular}
    \pventry{current}{Recalculate type area with the currently valid
      \Var{BCOR} value.\IndexOption{BCOR~=current}}
  \end{desctabular}
\end{table}

If both \PName{BCOR} and \PName{DIV} need changing, then it is recommended to
use \Macro{typearea}, since then the text block and margins are recalculated
only once. With
\DescRef{\LabelBase.cmd.KOMAoptions}%
\PParameter{\OptionVName{\DescRef{\LabelBase.option.DIV}}{DIV},%
  \OptionVName{\DescRef{\LabelBase.option.BCOR}}{BCOR}} the text block and
margins are recalculated once for the change to \PName{DIV} and again for the
change to \PName{BCOR}.

\begin{Explain}
  The command \Macro{typearea} is currently defined so as to make it
  possible to change the type area anywhere within a
  document. Several assumptions about the structure of the {\LaTeX}
  kernel are however made and internal definitions and sizes of the
  kernel changed. There is a definite possibility, but no guarantee,
  that this will continue to function in future versions of
  \LaTeXe{}. When used within the document, a page break will result.
\end{Explain}

Since \Macro{typearea}\POParameter{current}\PParameter{last} or
\DescRef{\LabelBase.cmd.KOMAoptions}%
\PParameter{\OptionValueRef{\LabelBase}{DIV}{last}} are often needed for
recalculation of the type area, there exists specially the abbreviated command
\Macro{recalctypearea}\ChangedAt{v3.00}{\Package{typearea}}.

\begin{Example}
  If one finds the notation
\begin{lstcode}
  \KOMAoptions{DIV=last}
\end{lstcode}
  or
\begin{lstcode}
  \typearea[current]{last}
\end{lstcode}
  for the recalculation of text block and margins too complicated for
  reasons of the many special characters, then one may use more simply
  the following.
\begin{lstcode}
  \recalctypearea
\end{lstcode}
\end{Example}%
\EndIndexGroup


\begin{Declaration}
  \OptionVName{twoside}{simple switch}%
  \OptionValue{twoside}{semi}
\end{Declaration}%
As already explained in \autoref{sec:typearea.basics}, the margin
configuration is dependent on whether the document is to be typeset
single- or double-sided. For single-sided typesetting, the left and
right margins are equally wide, whereas for double-sided printing the
inner margin of one page is only half as wide as the corresponding
outer margin. In order to implement this distinction, the
\Package{typearea} package must be given the \Option{twoside} option,
if the document is to be typeset double-sided. Being a \PName{simple switch},
any of the standard values for simple switches in
\autoref{tab:truefalseswitch} are valid. If the option is passed
without a value, the value \PValue{true} is assumed, so double-sided
typesetting is carried out. Deactivation of the option leads to
single-sided typesetting.

\begin{table}
  \centering
  \caption{Standard values for simple switches in \KOMAScript}
  \begin{tabular}{ll}
    \toprule
    Value & Description \\
    \midrule
    \PValue{true} & activates the option \\
    \PValue{on}   & activates the option \\
    \PValue{yes}  & activates the option \\
    \PValue{false}& deactivates the option \\
    \PValue{off}  & deactivates the option \\
    \PValue{no}   & deactivates the option \\
    \bottomrule
  \end{tabular}
  \label{tab:truefalseswitch}
\end{table}

Apart from the values in \autoref{tab:truefalseswitch} the value
\PValue{semi}\ChangedAt{v3.00}{\Package{typearea}} can also be given. The
value \PValue{semi} results in a double-sided typesetting with single-sided
margins and single-sided, i.\ e., not alternating, margin
notes. Nevertheless\ChangedAt{v3.12}{\Package{typearea}}, since \KOMAScript{}
version 3.12 binding corrections (see \DescRef{\LabelBase.option.BCOR},
\DescPageRef{typearea.option.BCOR}) will be part of the left margin on
odd pages but part of the right margin on even pages. But if you use
compatibility with prior versions of \KOMAScript (see
\autoref{sec:typearea.compatibilityOptions},
\autopageref{sec:typearea.compatibilityOptions}), binding correction will be
part of the left margin on both pages while using \OptionValue{twoside}{semi}.

The option can also be passed as class option in
\DescRef{\LabelBase.cmd.documentclass}, as package option to
\DescRef{\LabelBase.cmd.usepackage}, or even after loading of the
\Package{typearea} package with the use of
\DescRef{\LabelBase.cmd.KOMAoptions} or
\DescRef{\LabelBase.cmd.KOMAoption}. Use of the option after loading the
\Package{typearea} package results automatically in recalculation of the type
area using \DescRef{\LabelBase.cmd.recalctypearea} (see
\DescPageRef{typearea.cmd.recalctypearea}). If double-sided typesetting
was active before the option was set, then before the recalculation a page
break is made to the next odd page.%
%
\EndIndexGroup


\begin{Declaration}
  \OptionVName{twocolumn}{simple switch}
\end{Declaration}
For the calculation of a good type area with the help of
\OptionValueRef{\LabelBase}{DIV}{calc} it is useful to know in advance if the
document is to be typeset one-column or two-column. Since the observations
about line length in \autoref{sec:typearea.basics} then apply to each column,
the width of a type area in a two-column document can be up to double that in
a one-column document.

To implement this difference, the \Package{typearea} package must be
told via the \Option{twocolumn} option whether the document is to be
two-column. Since this is a \PName{simple switch}, any of the standard values
for simple switches from \autoref{tab:truefalseswitch} is valid. If
the option is passed without a value, the value \PValue{true} is
assumed, i.\,e., two-column typesetting. Deactivation of the option
results in one-column typesetting.

The option can also be passed as class option in
\DescRef{\LabelBase.cmd.documentclass}, as package option to
\DescRef{\LabelBase.cmd.usepackage}, or even after loading of the
\Package{typearea} package with the use of
\DescRef{\LabelBase.cmd.KOMAoptions} or
\DescRef{\LabelBase.cmd.KOMAoption}. Use of the option after loading the
\Package{typearea} package results automatically in recalculation of the type
area using \DescRef{\LabelBase.cmd.recalctypearea} (see
\DescPageRef{typearea.cmd.recalctypearea}).%
%
\EndIndexGroup


\begin{Declaration}
  \OptionVName{headinclude}{simple switch}%
  \OptionVName{footinclude}{simple switch}
\end{Declaration}%
\begin{Explain}%
  So far we have discussed how the type area is
  calculated\Index{type area} and the relationship of the
  margins\Index{margins} to one another and between margins and text
  block. However, one important question has not been answered: What
  constitutes the margins?

  At first glance the question appears trivial: Margins are those
  parts on the right, left, top and bottom which remain empty. But
  this is only half the story. Margins are not always empty. There may
  be margin notes, for example (see \Macro{marginpar} command in
  \cite{lshort} or \autoref{sec:maincls.marginNotes}).

  One could also ask whether headers\Index{page header} and
  footers\Index{page footer} belong to the upper and lower margins or
  to the text. This can not be answered unambiguously. Of course an
  empty footer or header belongs to the margins, since they can not be
  distinguished from the rest of the margin. A header or footer that
  contains only a page
  number\Index[indexmain]{pagination}\footnote{Pagination refers to
    the indication of the page number.} will optically appear more
  like a margin. For the optical appearance it is not important
  whether headers or footers are easily recognized as such during
  reading.  Of importance is only how a well-filled page appears when
  viewed \emph{out of focus}. One could use the glasses of one's
  far-sighted grandparents, or, lacking those, adjust one's vision to
  infinity and look at the page with one eye only. Those wearing
  spectacles will find this much easier, of course.  If the footer
  contains not only the page number, but other material like a
  copyright notice, it will optically appear more like a part of the
  text body.  This needs to be taken into account when calculating
  text layout.

  For the header this is even more complicated. The header frequently
  contains running headings\Index[indexmain]{running
    headings}.\footnote{Running headings refer to the repetition of a
    title in titling font, which is more often typeset in the page
    header, less often in the page footer.}  In the case of running
  headings with long chapter and section titles, the header lines will
  be very long and appear to be part of the text body.  This effect
  becomes even more significant when the header contains not only the
  chapter or section title but also the page number. With material on
  the right and left side, the header will no longer appear as an
  empty margin. It is more difficult if the pagination is in the
  footer, and the length of the titles varies so that the header may
  appear as a margin on one page and as text on another.  However,
  these pages should not be treated differently under any
  circumstances, as this would lead to vertically jumping headers. In
  this case it is probably best to count the header as part of the
  text.

  The decision is easy when text and header or footer are separated
  from the text body by a line. This will give a ``closed'' appearance
  and header or footer become part of the text body.  Remember: It is
  irrelevant that the line improves the optical separation of text and
  header or footer; only the appearance when viewed out
  of focus is important.
\end{Explain}

The \Package{typearea} package cannot make the decision whether or
not to count headers and footers as part of the text body or the
margin. Options \Option{headinclude} and \Option{footinclude} cause
the header or footer to be counted as part of the text.  These
options, being a \PName{simple switch}\ChangedAt{v3.00}{\Package{typearea}},
understand the standard values for simple switches in
\autoref{tab:truefalseswitch}. One may use the options without
specifying a value, in which case the value \PValue{true} is used for
the \PName{simple switch}, i.\,e., the header or footer is counted as part of
the text.

Readers who are unsure about the the correct setting should re-read
the above explanations. Default is usually
\OptionValue{headinclude}{false} and \OptionValue{footinclude}{false},
but this can change depending on {\KOMAScript} class and {\KOMAScript}
packages used (see \autoref{sec:maincls.options} and
\autoref{cha:scrlayer-scrpage}).

Please note that when using these options with one of the
{\KOMAScript} classes as in the example above, they must be used
either as a class option, or passed via \DescRef{\LabelBase.cmd.KOMAoptions} or
\DescRef{\LabelBase.cmd.KOMAoption} after loading the class. Changing of these options
after loading the \Package{typearea} package does not result in an
automatic recalculation of the type area. Instead, the changes only
take effect at the next recalculation of the type area. For
recalculation of the type area, refer to the \DescRef{\LabelBase.option.DIV} option with
the values \PValue{last} or \PValue{current} (see
\DescPageRef{typearea.option.DIV.last}) or the
\DescRef{\LabelBase.cmd.recalctypearea} command (see
\DescPageRef{typearea.cmd.recalctypearea}).%
%
\EndIndexGroup


\begin{Declaration}
  \OptionVName{mpinclude}{simple switch}
\end{Declaration}
Besides\ChangedAt{v2.8q}{\Class{scrbook}\and \Class{scrreprt}\and
  \Class{scrartcl}} documents where the head and foot are part of the
text area, there are also documents where the margin-note area must be
counted as part of the text body as well.  The option \Option{mpinclude} does
exactly this.  The option, as a
\PName{simple switch}\ChangedAt{v3.00}{\Package{typearea}}, understands the
standard values for simple switches in
\autoref{tab:truefalseswitch}. One may also pass this option without
specifying a value, in which case the value \PValue{true} for
\PName{simple switch} is assumed.

The effect of \OptionValue{mpinclude}{true} is that one width-unit of
the text body is taken for the margin-note area.  Using option
\OptionValue{mpinclude}{false}, the default setting, the normal
margin is used for the margin-note area.  The width of that area is
one or one and a half width-unit, depending on whether one-sided or
double-sided page layout has been chosen.  The option
\OptionValue{mpinclude}{true} is mainly for experts and so is not recommended.
  
\begin{Explain}
  In the cases where the option \Option{mpinclude} is used, often a
  wider margin-note area is required.  In many cases not the whole
  margin-note width should be part of the text area, for example if
  the margin is used for quotations.  Such quotations are typeset as
  ragged text with the flushed side where the text body is.  Since
  ragged text gives no homogeneous optical impression, the long lines
  can reach right into the normal margin.  This can be done using
  option \Option{mpinclude} and by an enlargement of length
  \Length{marginparwidth} after the type area has been set up.  The
  length can be easily enlarged with the command \Macro{addtolength}.
  How much the length has to be enlarged depends on the special
  situation and it requires some flair.  This is another reason the
  \Option{mpinclude} option is primarily left for experts.  Of course
  one can set up the margin-width to reach a third right into the
  normal margin; for example, using
\begin{lstcode}
  \setlength{\marginparwidth}{1.5\marginparwidth}
\end{lstcode}
  gives the desired result.

  Currently there is no option to enlarge the margin by a given
  amount.  The only solution is to either not use the option
  \Option{mpinclude} or to set \Option{mpinclude} to \PValue{false},
  and after the type area has been calculated, one reduces the
  width of the text body \Length{textwidth} and enlarges the margin
  width \Length{marginparwidth} by the same amount.  Unfortunately,
  this cannot be combined with automatic calculation of the
  \PName{DIV} value.  In contrast
  \OptionValueRef{\LabelBase}{DIV}{calc}\IndexOption{DIV~=calc} (see
  \DescPageRef{typearea.option.DIV.calc}) heeds
  \Option{mpinclude}.
\end{Explain}

Please note that when using this option with one of the {\KOMAScript}
classes as in the example above, it must be used either as a class
option, or passed via \DescRef{\LabelBase.cmd.KOMAoptions} or \DescRef{\LabelBase.cmd.KOMAoption} after
loading the class. Changing of this option after loading the
\Package{typearea} package does not result in an automatic
recalculation of the type area. Instead, the changes only take effect
at the next recalculation of the type area. For recalculation of the
type area, refer to the \DescRef{\LabelBase.option.DIV} option with the values
\PValue{last} or \PValue{current} (see
\DescPageRef{typearea.option.DIV.last}) or the
\DescRef{\LabelBase.cmd.recalctypearea} command (see
\DescPageRef{typearea.cmd.recalctypearea}).%
%
\EndIndexGroup


\begin{Declaration}
  \OptionVName{headlines}{number of lines}\\
  \OptionVName{headheight}{height}
\end{Declaration}%
We have seen how to calculate the type area using the
\Package{typearea} package and how to specify whether header and
footer are part of the text or the margins. However, in particular for
the header, we still have to specify the height. This is achieved with
the options \Option{headlines} and
\Option{headheight}\ChangedAt{v3.00}{\Package{typearea}}.

The option \Option{headlines} is set to the number of header
lines. The \Package{typearea} package uses a default of 1.25. This is
a compromise, large enough for underlined headers (see
\autoref{sec:maincls.options}) and small enough that the relative
weight of the top margin is not affected too much when the header is
not underlined. Thus in most cases you may leave \Option{headlines} at
its default value and adapt it only in special cases.

\begin{Example}
  Assume that you want to use a header with two lines. Normally this would
  result in an ``\texttt{overfull} \Macro{vbox}'' warning for each page. To
  prevent this from happening, the \Package{typearea} package is told to
  calculate an appropriate type area:
\begin{lstcode}
  \documentclass[a4paper]{article}
  \usepackage[headlines=2.1]{typearea}
\end{lstcode}
If you use a {\KOMAScript} class, it is recommended to pass this option
directly as a class option:
\begin{lstcode}
  \documentclass[a4paper,headlines=2.1]{scrartcl}
\end{lstcode}
Commands that can be used to define the contents of a header with two lines
are described in \autoref{cha:scrlayer-scrpage}.
\end{Example}

In some cases it is useful to be able to specify the header height not
in lines but directly as a length measurement. This is accomplished
with the aid of the alternative option \Option{headheight}. For
\PName{height} any lengths and sizes that \LaTeX{} understands are
valid. It should be noted though that when using a \LaTeX{} length
such as \Length{baselineskip} its value at the time of the calculation
of the type area and margins, not at the time of setting of the
option, is decisive.

Please note that when using these options with one of the
{\KOMAScript} classes as in the example above, they must be used
either as a class option, or passed via \DescRef{\LabelBase.cmd.KOMAoptions} or
\DescRef{\LabelBase.cmd.KOMAoption} after loading the class. Changing of these options
after loading the \Package{typearea} package does not result in an
automatic recalculation of the type area. Instead, the changes only
take effect at the next recalculation of the type area. For
recalculation of the type area, refer to the \DescRef{\LabelBase.option.DIV} option with
the values \PValue{last} or \PValue{current} (see
\DescPageRef{typearea.option.DIV.last}) or the
\DescRef{\LabelBase.cmd.recalctypearea} command (see
\DescPageRef{typearea.cmd.recalctypearea}).%
%
\EndIndexGroup


\begin{Declaration}
  \OptionVName{footlines}{number of lines}%
  \OptionVName{footheight}{height}%
  \Length{footheight}%
\end{Declaration}
As\ChangedAt{v3.12}{\Package{typearea}} well as we needed a height value for
the head, we need a height value for the page footer. But in difference to the
height of the head, \LaTeX itself do not provide a length for the height of
the page footer. So \Package{typearea} defines the new length
\Length{footheight}\IndexLength[indexmain]{footheight}, if
it does not exist. Wether or not this length will be used by classes or
packages depends on the classes and packages, that will be used. The
\KOMAScript{} package \Package{scrlayer-scrpage} incorporates
\Length{footheight} and actively cooperates with
\Package{typearea}. The \KOMAScript{} classes do not recognize
\Length{footheight}, because without any package assistance
they provide only page styles with single-line page footers.

You can use \Option{footlines} to setup the \PName{number of lines} in the
page footer, similar to \DescRef{\LabelBase.option.headlines} for the number
of lines in the page header. By default \Package{typearea} uses 1.25
footlines. This is a compromise, large enough for overlining or underlining
footers and small enough that the relative weight of the bottom margin is not
affected too mich when the footer is neither over- nor underlined. Thus in
most cases you may leave \PName{number of lines} at its default value and
adapt it only in special cases.

\begin{Example}
  Assume a two-lined copyright note should be placed in the page
  footer. Indeed, \LaTeX{} itself does not test, whether or not the footer has
  room enough for that, exceeding of the available height would probably could
  result in unbalanced margins. Moreover, for example package
  \Package{scrlayer-scrpage}, that may be used to define such a page footer,
  would definitely do such a test and would notify a recognised oversize. So
  it makes sense, to declare the needed footheight already for the calculation
  of the text area and the margins:
\begin{lstcode}
  \documentclass[a4paper]{article}
  \usepackage[footlines=2.1]{typearea}
\end{lstcode}
  Again, if you use a \KOMAScript{} class, it is recommended to pass this
  option directly a a class option:
\begin{lstcode}
  \documentclass[footlines=2.1]{scrartcl}
\end{lstcode}
  Commands that can be used to define the contents of a footer with two lines
  are described in \autoref{cha:scrlayer-scrpage}.
\end{Example}

In some cases it is useful to be able to specify the footer height not
in lines but directly as a length measurement. This is accomplished
with the aid of the alternative option \Option{footheight}. For
\PName{height} any lengths and sizes that \LaTeX{} understands are
valid. It should be noted though that when using a \LaTeX{} length
such as \Length{baselineskip} its value at the time of the calculation
of the type area and margins, not at the time of setting of the
option, is decisive.

Please note that when using these options with one of the
{\KOMAScript} classes as in the example above, they must be used
either as a class option, or passed via \DescRef{\LabelBase.cmd.KOMAoptions} or
\DescRef{\LabelBase.cmd.KOMAoption} after loading the class. Changing of these options
after loading the \Package{typearea} package does not result in an
automatic recalculation of the type area. Instead, the changes only
take effect at the next recalculation of the type area. For
recalculation of the type area, refer to the \DescRef{\LabelBase.option.DIV} option with
the values \PValue{last} or \PValue{current} (see
\DescPageRef{typearea.option.DIV.last}) or the
\DescRef{\LabelBase.cmd.recalctypearea} command (see
\DescPageRef{typearea.cmd.recalctypearea}).%
\EndIndexGroup


\begin{Declaration}
  \Macro{areaset}\OParameter{BCOR}\Parameter{width}\Parameter{height}
\end{Declaration}%
So far we have seen how a good or even very good type area\Index{type area} is
calculated and how the \Package{typearea} package can support these
calculations, giving you at the same time the freedom to adapt the layout to
your needs.  However, there are cases where the text body has to fit exactly
some specified dimensions. At the same time the margins should be well spaced
and a binding correction should be possible. The \Package{typearea} package
offers the command \Macro{areaset} for this purpose. As parameters this
command accepts the binding correction and the width and height of the text
body.  Width and position of the margins will then be calculated
automatically, taking account of the options
\DescRef{\LabelBase.option.headinclude},
\OptionValueRef{\LabelBase}{headinclude}{false},
\DescRef{\LabelBase.option.footinclude} and
\OptionValueRef{\LabelBase}{footinclude}{false} where needed.  On the other
hand, the options \DescRef{\LabelBase.option.headlines}\IndexOption{headlines}
and \DescRef{\LabelBase.option.headheight}\IndexOption{headheight} are
ignored!

The default of \PName{BCOR} is 0\Unit{pt}. If you want to re-use the current
binding correction, e.\,g. the value set by option
\DescRef{\LabelBase.option.BCOR}\IndexOption{BCOR}, you can use the symbolic
value \PValue{current} at the optional argument.

\begin{Example}
  Assume a text, printed on A4 paper, should have a width of exactly 60
  characters of typewriter font and a height of exactly 30 lines. This could
  be achieved as follows:
\begin{lstcode}
  \documentclass[a4paper,11pt]{article}
  \usepackage{typearea}
  \newlength{\CharsLX}% Width of 60 characters
  \newlength{\LinesXXX}% Height of 30 lines
  \settowidth{\CharsLX}{\texttt{1234567890}}
  \setlength{\CharsLX}{6\CharsLX}
  \setlength{\LinesXXX}{\topskip}
  \addtolength{\LinesXXX}{29\baselineskip}
  \areaset{\CharsLX}{\LinesXXX}
\end{lstcode}
You need only 29 instead of 30, because the base line of the topmost
text line is \Macro{topskip} below the top margin of the type area, as
long as the height of the topmost line is less than
\Macro{topskip}. Thus, the uppermost line does not require any
height. The descenders of characters on the lowermost line, on the
other hand, hang below the dimensions of the type area.

\item A poetry book with a square text body with a page length of
  15\Unit{cm} and a binding correction of 1\Unit{cm} could be
  achieved like this:
\begin{lstcode}
  \documentclass{poetry}
  \usepackage{typearea}
  \areaset[1cm]{15cm}{15cm}
\end{lstcode}
\end{Example}
\EndIndexGroup


\begin{Declaration}
  \OptionValue{DIV}{areaset}
\end{Declaration}%
In\ChangedAt{v3.00}{\Package{typearea}} rare cases it is useful to be able to
reconstruct the current type area anew. This is possible via the option
\OptionValue{DIV}{areaset}, where
\DescRef{\LabelBase.cmd.KOMAoptions}\PParameter{\OptionValue{DIV}{areaset}}
corresponds to the
\begin{lstcode}
  \areaset[current]{\textwidth}{\textheight}
\end{lstcode}
command. The same result is obtained if one uses
\OptionValueRef{\LabelBase}{DIV}{last} and the typearea was last set with
\DescRef{\LabelBase.cmd.areaset}.%
%
\EndIndexGroup


The \Package{typearea} package was not made to set up predefined
margin values. If you have to do so you may use package
\Package{geometry}\IndexPackage{geometry} (see
\cite{package:geometry}).


\section{Paper Format Selection}
\seclabel{paperTypes}%
\BeginIndexGroup
\BeginIndex{}{paper>format}%

The paper format is a definitive characteristic of any document. As
already mentioned in the description of the supported page layout
constructions (see \autoref{sec:typearea.basics} to
\autoref{sec:typearea.circleConstruction} from
\autopageref{sec:typearea.basics} onwards), the entire page division
and document layout depends on the paper format. Whereas the {\LaTeX}
standard classes are restricted to a few formats, {\KOMAScript}
supports in conjunction with the \Package{typearea} package even
exotic paper sizes.


\begin{Declaration}
  \OptionVName{paper}{format}
\end{Declaration}%
The option \Option{paper}\ChangedAt{v3.00}{\Package{typearea}} is the
central element for format selection in \KOMAScript.  \PName{Format}
supports first of all the American formats \Option{letter},
\Option{legal}, and \Option{executive}. In addition, it supports the
ISO formats of the series A, B, C, and D, for example \PValue{A4}
or\,---\,written in lowercase\,---\,\PValue{a4}. 

Landscape formats are supported by specifying the option again, this time with
value \PValue{landscape}\Index{paper>orientation} or
\PValue{seascape}\ChangedAt{v3.02c}{\Package{typearea}}. The difference is
that application \File{dvips} rotates at \PValue{landscape} by
-90\Unit{\textdegree}, while it rotates by +90\Unit{\textdegree} at
\PValue{seascape}. So you may use \PValue{seascape} whenever a PostScript
viewer application shows landscape pages upside-down. But you may see the
difference only if you do not deactivate option
\DescRef{\LabelBase.option.pagesize}, which will be described next.

Additionally\ChangedAt{v3.01b}{\Package{typearea}}, the \PName{format} can
also be specified in the form \PName{height}\texttt{:}\PName{width} or
\PName{width}\texttt{:}\PName{height}.
Whether\ChangedAt{v3.22}{\Package{typearea}} the first or the second value is
the \PName{height} or \PName{width} depends on the paper orientation. With
\OptionValue{paper}{landscape} or \OptionValue{paper}{seascape} the smaller
value is the \PName{height} and the larger one is the \PName{width}. With
\OptionValue{paper}{portrait} the smaller value is the \PName{width} and the
larger one is the \PName{height}.

Note\textnote{Attention!} that until version~3.01a the first value was always
the \PName{height} and the second one the \PName{width}. From version~3.01b
until version~3.21, the first value was always the \PName{width} and the
second one the \PName{height}. This is important if you use compatibility
settings (see option
\DescRef{\LabelBase.option.version}%
\IndexOption{version}\important{\DescRef{\LabelBase.option.version}},
\autoref{sec:typearea.compatibilityOptions},
\DescPageRef{typearea.option.version}).

\begin{Example}
 Assume one wishes to print on ISO A8 file cards in landscape
 orientation. Margins should be very small, no header or footer
 will be used.
\begin{lstcode}
  \documentclass{article}
  \usepackage[headinclude=false,footinclude=false,
              paper=A8,landscape]{typearea}
  \areaset{7cm}{5cm}
  \pagestyle{empty}
  \begin{document}
  \section*{Supported Paper Sizes}
  letter, legal, executive, a0, a1 \dots\ %
  b0, b1 \dots\ c0, c1 \dots\ d0, d1 \dots
  \end{document}
\end{lstcode}
If the file cards have the special format (height:width)
5\Unit{cm}\,:\,3\Unit{cm}, this can be achieved using the following
code.
\begin{lstcode}
  \documentclass{article}
  \usepackage[headinclude=false,footinclude=false,%
              paper=landscape,paper=5cm:3cm]{typearea}
  \areaset{4cm}{2.4cm}
  \pagestyle{empty}
  \begin{document}
  \section*{Supported Paper Sizes}
  letter, legal, executive, a0, a1 \dots\ %
  b0, b1 \dots\ c0, c1 \dots\ d0, d1 \dots
  \end{document}
\end{lstcode}
\end{Example}

As part of the predefined defaults, {\KOMAScript} uses A4 paper in
portrait orientation. This is in contrast to the standard classes,
which by default use the American letter paper format.

Please note that when using these options with one of the {\KOMAScript}
classes, it must be used either as a class option, or passed via
\DescRef{\LabelBase.cmd.KOMAoptions} or \DescRef{\LabelBase.cmd.KOMAoption}
after loading the class. Changing of this option after loading the
\Package{typearea} package does not result in an automatic recalculation of
the type area. Instead, the changes only take effect at the next recalculation
of the type area. For recalculation of the type area, refer to the
\DescRef{\LabelBase.option.DIV} option with the values \PValue{last} or
\PValue{current} (see \DescPageRef{typearea.option.DIV.last}) or the
\DescRef{\LabelBase.cmd.recalctypearea} command (see
\DescPageRef{typearea.cmd.recalctypearea}).%
%
\EndIndexGroup


\begin{Declaration}
  \OptionVName{pagesize}{output driver}
\end{Declaration}%
\begin{Explain}%
  The above-mentioned mechanisms for choice of paper format only
  affect the output insofar as internal {\LaTeX} lengths are set. The
  \Package{typearea} package then uses them in the division of the
  page into type area and margins. 
  The specification of the DVI
  formats\Index{DVI}, however, does not include any indications of paper
  format. If printing is done directly from DVI format to a low-level
  printer language such as PCL%
  \iffalse% Umbruchkorrektur
  \footnote{PCL is the printer language used by HP for its inkjet and
    laser printers.}%
  \fi \ or ESC/P2%
  \iffalse% Umbruchkorrektur
  \footnote{ESC/P2 is the printer language used by EPSON for its
    dot-matrix, inkjet and laser printers.}%
  \fi , this is usually not an issue since with this output also the
  zero-position is at the top left, identical to DVI. If, however,
  translation is made into a language such as
  PostScript\Index{PostScript} or PDF\Index{PDF}, in which the
  zero-position is at a different point, and in which also the paper
  format should be specified in the output data, then this information
  is missing. To solve this problem, the respective drivers use a
  predefined paper size, which the user can change either by means of
  an option or via a corresponding command in the {\TeX} source
  file. When using the DVI driver \File{dvips} the information can be
  given in the form of a \Macro{special} command. With {pdf\TeX} or
  {V\TeX} one sets instead two lengths.
\end{Explain}
With option \Option{pagesize} you may select an output driver for writing the
paper size into the destination document. Supported output drivers are listed
at \autoref{tab:typearea.outputdriver}. The
default\ChangedAt{v3.17}{\Package{typearea}} is \Option{pagesize}. This usage
without value is same like \OptionValue{pagesize}{auto}.
%
\begin{table}
  \caption{Output driver for option \OptionVName{pagesize}{output driver}}
  \begin{desctabular}
    \pventry{auto}{Uses output driver \PValue{pdftex} if pdf\TeX-specific
      registers \Macro{pdfpagewidth}\IndexLength{pdfpagewidth} and
      \Macro{pdfpageheight}\IndexLength{pdfpageheight} are defined. In
      addition, output driver \PValue{dvips} will be
      used.\IndexOption{pagesize~=\PValue{auto}}}%
    \pventry{automedia}{Almost the same as \PValue{auto} but if the
      \mbox{V\TeX}-specific registers
      \Macro{mediawidth}\IndexLength{mediawidth} and
      \Macro{mediaheight}\IndexLength{mediaheight} are defined, they will be
      set additionally.\IndexOption{pagesize~=\PValue{automedia}}}%
    \entry{\PValue{false}, \PValue{no}, \PValue{off}}{%
      Does not set any output driver and does not send page size information to
      the output driver.\IndexOption{pagesize~=\PValue{false}}}%
    \pventry{dvipdfmx}{\ChangedAt{v3.05a}{\Package{typearea}}Writes paper size
      into DVI files using
      \Macro{special}\PParameter{pagesize=\PName{width},\PName{height}}. The
      name of the output driver is \PValue{dvipdfmx} because application
      \File{dvipdfmx} handles such specials not only at document preamble but
      at the document body too.\IndexOption{pagesize~=\PValue{dvipdfmx}}}%
    \pventry{dvips}{Using this option at the document preamble sets paper size
      using
      \Macro{special}\PParameter{pagesize=\PName{width},\PName{height}}. While
      application \File{dvips} cannot handle changes of paper size at the
      inner document pages a hack is needed to achieve such changes. Use
      changes of paper size after \Macro{begin}\PParameter{document} on your
      own risk, if you are using
      \PValue{dvips}!\IndexOption{pagesize~=\PValue{dvips}}}%
    \pventry{pdftex}{Sets paper size using the pdf\TeX-specific registers
      \Macro{pdfpagewidth}\IndexLength{pdfpagewidth} and
      \Macro{pdfpageheight}\IndexLength{pdfpageheight}. You may do this at any
      time in your document.\IndexOption{pagesize~=\PValue{pdftex}}}%
  \end{desctabular}
  \label{tab:typearea.outputdriver}
\end{table}

\begin{Example}
  Assume that a document should be available both as a DVI data file
  and in PDF format for online viewing. Then the preamble might begin
  as follows:
\begin{lstcode}
  \documentclass{article}
  \usepackage[paper=A4,pagesize]{typearea}
\end{lstcode}
If the {pdf\TeX} engine is used \emph{and} PDF output is
activated, then the two lengths \Macro{pdfpagewidth} and
\Macro{pdfpageheight} are set appropriately.  If, however, a DVI data
file is created\,---\,regardless of whether by {\LaTeX} or by
{pdf\LaTeX}\,---\,then a \Macro{special} is written at the start of
this data file.
\end{Example}%
It is recommended always to specify this option. Generally the method
without \PName{output driver}, or with \PValue{auto} or
\PValue{automedia}, is useful.%
\EndIndexGroup
%
\EndIndexGroup


\section{Tips}
\seclabel{tips}

For theses  many rules exist that violate even the most
elementary rules of typography. The reasons for such rules include
typographical incompetence of those making them, but also the fact
that they were originally meant for mechanical typewriters. With a
typewriter or a primitive text processor dating back to the early
1980s, it was not possible to produce typographically correct output
without extreme effort. Thus rules were created that appeared to be
achievable and still allowed easy correction. To avoid short lines
made worse by ragged margins, the margins were kept narrow and the
line spacing was increased to 1.5 for corrections. Before the advent
of modern text processing systems, single-spaced would have been the
only alternative\,---\,other than with \TeX. In such a single-spaced
document even correction signs would have been difficult to add. When
computers became more widely available for text processing, some
students tried to use a particularly ``nice'' font to make their work
look better than it really was. They forgot however that such fonts
are often more difficult to read and therefore unsuitable for this
purpose. Thus two bread-and-butter fonts became widely used which
neither fit together nor are particularly suitable for the job. In
particular Times is a relatively narrow font which was developed at
the beginning of the 20$^{th}$ century for the narrow columns of
British newspapers. Modern versions usually are somewhat improved. But
still the Times font required in many rules does not really fit to the
margin sizes prescribed.

{\LaTeX} already uses sufficient line spacing, and the margins are
wide enough for corrections. Thus a page will look generous, even when
quite full of text.

To some extent, the questionable rules are difficult to implement in
{\LaTeX}. A fixed number of characters per line can be kept only when
a non-proportional font is used. There are very few good
non-proportional fonts available. Hardly a text typeset in this way
looks really good. In many cases font designers try to increase the
serifs on the `i' or `l' to compensate for the different character
width. This cannot work and results in a fragmented and
agitated-looking text. If one uses {\LaTeX} for one's paper, some of
these rules have to be either ignored or at least interpreted
generously. For example one may interpret ``60 characters per line''
not as a fixed, but as an average or maximal value.%

As executed, record regulations are usually intended to obtain a usable result
even if the author does not know what needs to be considered. \emph{Usable}
frequently means readable and correctable.  In the author's opinion the type
area of a text set with {\LaTeX} and the \Package{typearea} package meets
these criteria well right from the start.  Thus if one is confronted with
regulations which deviate obviously substantially from it, then the author
recommends submitting an extract from the text to the responsible person and
inquiring whether it is permitted to submit the work despite deviations in the
format.  If necessary the type area can be moderately adapted by modification
of option \DescRef{\LabelBase.option.DIV}.  The author advises against the use
of \DescRef{\LabelBase.cmd.areaset} for this purpose however.  In the worst
case one may make use of the geometry package (see \cite{package:geometry}),
which is not part of \KOMAScript, or change the type area parameters of
\LaTeX.  One may find the values determined by \Package{typearea} in the
\File{log} file of one's document.  Thus moderate adjustments should be
possible.  However, one should make absolutely sure that the proportions of
the text area correspond approximately to those of the page including
consideration of the binding correction.

If it should prove absolutely necessary to set the text with a line
spacing of 1.5, then one should not under any circumstances redefine
\Macro{baselinestretch}.  Although this procedure is recommended all
too frequently, it has been obsolete since the introduction of
{\LaTeXe} in 1994.  In the worst case one may use the instruction
\Macro{linespread}.  The author recommends the package
\Package{setspace}\IndexPackage{setspace} (see
\cite{package:setspace}), which is not part of \KOMAScript.  Also one
should let \Package{typearea} recalculate a new type area after the
conversion of the line spacing.  However, one should switch back to
the normal line spacing for the title, preferably also for the table
contents and various listings\,---\,as well as the bibliography and
the index.  The \Package{setspace} package offers for this a special
environment and its own instructions.

The \Package{typearea} package, even with option
\OptionValueRef{\LabelBase}{DIV}{calc}, calculates a very generous text area.
Many conservative typographers will state that the resulting line length is
still excessive. The calculated \Var{DIV} value may be found in the \File{log}
file for the respective document.  Thus one can select a smaller value easily
after the first {\LaTeX} run.

The question is not infrequently put to the author, why he spends an
entire chapter discussing type area calculations, when it would be
very much simpler to merely give the world a package with which anyone
can adjust the margins like in a word processor.  Often it is added
that such a package would in any case be the better solution, since
everyone can judge for themselves how good margins are to be chosen,
and that the margins calculated by {\KOMAScript} are anyway not that
great.  The author takes the liberty of translating a suitable
quotation from \cite{TYPO:ErsteHilfe}. One may find the original
German words in the German scrguide.

\begin{quote}
  \phantomsection\seclabel{tips.cite}%
  \textit{The practice of doing things oneself is long-since
    widespread, but the results are often dubious because layman
    typographers do not see what is incorrect and cannot know what is
    important. Thus one becomes accustomed to incorrect and poor
    typography.} [\dots] \textit{Now the objection could be made that
    typography is dependent on taste. If it concerned decoration,
    perhaps one could let that argument slip by; however, since
    typography is primarily concerned with information, errors cannot
    only irritate, but may even cause damage.}
\end{quote}
%
\EndIndexGroup

%%% Local Variables:
%%% mode: latex
%%% coding: us-ascii
%%% TeX-master: "../guide"
%%% End:
