% ======================================================================
% common-5.tex
% Copyright (c) Markus Kohm, 2001-2010
%
% This file is part of the LaTeX2e KOMA-Script bundle.
%
% This work may be distributed and/or modified under the conditions of
% the LaTeX Project Public License, version 1.3c of the license.
% The latest version of this license is in
%   http://www.latex-project.org/lppl.txt
% and version 1.3c or later is part of all distributions of LaTeX 
% version 2005/12/01 or later and of this work.
%
% This work has the LPPL maintenance status "author-maintained".
%
% The Current Maintainer and author of this work is Markus Kohm.
%
% This work consists of all files listed in manifest.txt.
% ----------------------------------------------------------------------
% common-5.tex
% Copyright (c) Markus Kohm, 2001-2010
%
% Dieses Werk darf nach den Bedingungen der LaTeX Project Public Lizenz,
% Version 1.3c, verteilt und/oder veraendert werden.
% Die neuste Version dieser Lizenz ist
%   http://www.latex-project.org/lppl.txt
% und Version 1.3c ist Teil aller Verteilungen von LaTeX
% Version 2005/12/01 oder spaeter und dieses Werks.
%
% Dieses Werk hat den LPPL-Verwaltungs-Status "author-maintained"
% (allein durch den Autor verwaltet).
%
% Der Aktuelle Verwalter und Autor dieses Werkes ist Markus Kohm.
% 
% Dieses Werk besteht aus den in manifest.txt aufgefuehrten Dateien.
% ======================================================================
%
% Paragraphs that are common for several chapters of the KOMA-Script guide
% Maintained by Markus Kohm
%
% ----------------------------------------------------------------------
%
% Abs�tze, die mehreren Kapiteln der KOMA-Script-Anleitung gemeinsam sind
% Verwaltet von Markus Kohm
%
% ======================================================================

\ProvidesFile{common-5.tex}[2009/01/04 KOMA-Script guide (common paragraphs)]
\translator{Gernot Hassenpflug}

\makeatletter
\@ifundefined{ifCommonmaincls}{\newif\ifCommonmaincls}{}%
\@ifundefined{ifCommonscrextend}{\newif\ifCommonscrextend}{}%
\@ifundefined{ifCommonscrlttr}{\newif\ifCommonscrlttr}{}%
\@ifundefined{ifIgnoreThis}{\newif\ifIgnoreThis}{}%
\makeatother


\section{Textauszeichnungen}
\label{sec:\csname label@base\endcsname.textmarkup}%
\ifshortversion\IgnoreThisfalse\IfCommon{scrlttr2}{\IgnoreThistrue}\fi%
\ifIgnoreThis%
Es gilt sinngem��, was in \autoref{sec:maincls.textmarkup} geschrieben
wurde. Dabei sind Namen und Bedeutung der einzelnen Elemente in 
\else
\BeginIndex{}{Text>Auszeichnung}%
\BeginIndex{}{Schriftart}%

\LaTeX{} verf�gt �ber eine ganze Reihe von Anweisungen zur
Textauszeichnung. Neben der Wahl der Schriftart geh�ren dazu auch Befehle zur
Wahl einer Textgr��e oder der Textausrichtung. N�heres zu den normalerweise
definierten M�glichkeiten ist \cite{l2kurz}, \cite{latex:usrguide} und
\cite{latex:fntguide} zu entnehmen.


\begin{Declaration}
  \Macro{textsuperscript}\Parameter{Text}\\
  \Macro{textsubscript}\Parameter{Text}
\end{Declaration}
\BeginIndex{Cmd}{textsubscript}%
\BeginIndex{Cmd}{textsuperscript}%
Im \LaTeX-Kern ist bereits die Anweisung
\Macro{textsuperscript}\IndexCmd{textsuperscript} definiert, mit der
\PName{Text} h�her gestellt werden kann.  Leider bietet \LaTeX{} selbst keine
entsprechende Anweisung, um Text tief\Index{Tiefstellung} statt
hoch\Index{Hochstellung} zu stellen. \KOMAScript{} definiert daf�r
\Macro{textsubscript}. %
\IfCommon{scrlttr2}{Ein Anwendungsbeispiel finden Sie in
  \autoref{sec:maincls.textmarkup},
  \autopageref{desc:maincls.textsubscript.example}.}
\IfCommon{scrextend}{Ein Anwendungsbeispiel finden Sie in
  \autoref{sec:maincls.textmarkup},
  \autopageref{desc:maincls.textsubscript.example}.}
%
\ifCommonmaincls
\begin{Example}
  \phantomsection\label{desc:maincls.textsubscript.example}%
  Sie schreiben einen Text �ber den menschlichen Stoffwechsel. Darin kommen
  hin und wieder einfache chemische Summenformeln vor. Dabei sind einzelne
  Ziffern tief zu stellen. Im Sinne des logischen Markups definieren Sie
  zun�chst in der Dokumentpr�ambel oder einem eigenen Paket:
  \begin{lstcode}
  \newcommand*{\Molek}[2]{#1\textsubscript{#2}}
  \end{lstcode}
  \newcommand*{\Molek}[2]{#1\textsubscript{#2}}%
  Damit schreiben Sie dann:
  \begin{lstcode}
  Die Zelle bezieht ihre Energie unter anderem aus der
  Reaktion von \Molek C6\Molek H{12}\Molek O6 und
  \Molek O2 zu \Molek H2\Molek O{} und 
  \Molek C{}\Molek O2. Arsen (\Molek{As}{}) wirkt sich
  allerdings auf den Stoffwechsel sehr nachteilig aus.
  \end{lstcode}
  Das Ergebnis sieht daraufhin so aus:
  \begin{ShowOutput}
    Die Zelle bezieht ihre Energie unter anderem aus der Reaktion von
    \Molek C6\Molek H{12}\Molek O6 und \Molek O2 zu \Molek H2\Molek
    O{} und \Molek C{}\Molek O2.  Arsen (\Molek{As}{}) wirkt sich
    allerdings auf den Stoffwechsel sehr nachteilig aus.
  \end{ShowOutput}
  Etwas sp�ter entscheiden Sie, dass Summenformeln grunds�tzlich serifenlos
  geschrieben werden sollen. Nun zeigt sich, wie gut die Entscheidung f�r
  konsequentes logisches Markup war. Sie m�ssen nur die
  \Macro{Molek}-Anweisung umdefinieren:
  \begin{lstcode}
  \newcommand*{\Molek}[2]{%
    \textsf{#1\textsubscript{#2}}%
  }
  \end{lstcode}
  \renewcommand*{\Molek}[2]{\textsf{#1\textsubscript{#2}}}%
  Schon �ndert sich die Ausgabe im gesamten Dokument:
  \begin{ShowOutput}
    Die Zelle bezieht ihr Energie unter anderem aus der Reaktion von
    \Molek C6\Molek H{12}\Molek O6 und \Molek O2 zu \Molek H2\Molek
    O{} und \Molek C{}\Molek O2.  Arsen (\Molek{As}{}) wirkt sich
    allerdings auf den Stoffwechsel sehr nachteilig aus.
  \end{ShowOutput}
\end{Example}
\iffalse % vielleicht in einer sp�teren Auf-lage
F�r Experten ist in \autoref{sec:experts.knowhow},
\autopageref{desc:experts.macroargs} dokumentiert, warum das Beispiel
funktioniert, obwohl teilweise die Argumente von \Macro{Molek} nicht in
geschweifte Klammern gesetzt wurden.%
\fi
\fi% \ifCommonmaincls
%
\EndIndex{Cmd}{textsuperscript}%
\EndIndex{Cmd}{textsubscript}%


\begin{Declaration}
  \Macro{setkomafont}\Parameter{Element}\Parameter{Befehle}\\
  \Macro{addtokomafont}\Parameter{Element}\Parameter{Befehle}\\
  \Macro{usekomafont}\Parameter{Element}
\end{Declaration}%
\BeginIndex{Cmd}{setkomafont}%
\BeginIndex{Cmd}{addtokomafont}%
\BeginIndex{Cmd}{usekomafont}%
Mit\ChangedAt{v2.8p}{%
  \Class{scrbook}\and \Class{scrreprt}\and \Class{scrartcl}} Hilfe der beiden
Anweisungen \Macro{setkomafont} und \Macro{addtokomafont} ist es m�glich, die
\PName{Befehle} festzulegen, mit denen die Schrift eines bestimmten
\PName{Element}s umgeschaltet wird. Theoretisch k�nnten als \PName{Befehle}
alle m�glichen Anweisungen einschlie�lich Textausgaben verwendet werden.  Sie
sollten sich jedoch unbedingt auf solche Anweisungen beschr�nken, mit denen
wirklich nur ein Schriftattribut umgeschaltet wird. In der Regel werden dies
die Befehle \Macro{normalfont}, \Macro{rmfamily}, \Macro{sffamily},
\Macro{ttfamily}, \Macro{mdseries}, \Macro{bfseries}, \Macro{upshape},
\Macro{itshape}, \Macro{slshape}, \Macro{scshape} sowie die Gr��enbefehle
\Macro{Huge}, \Macro{huge}, \Macro{LARGE}, \Macro{Large}, \Macro{large},
\Macro{normalsize}, \Macro{small}, \Macro{footnotesize} und \Macro{tiny}
sein. Die Erkl�rung zu diesen Befehlen entnehmen Sie bitte \cite{l2kurz},
\cite{latex:usrguide} oder \cite{latex:fntguide}. Auch Farbumschaltungen wie
\Macro{normalcolor} sind m�glich (siehe \cite{package:graphics} und
\cite{package:xcolor}). 
\iffalse % Umbruchkorrekturtext
Das Verhalten bei Verwendung anderer Anweisungen,
inbesondere solcher, die zu Umdefinierungen f�hren oder Ausgaben t�tigen, ist
nicht definiert. Seltsames Verhalten ist m�glich und stellt keinen Fehler dar.
\else
Die Verwendung anderer Anweisungen, inbesondere
solcher, die Umdefinierungen vornehmen oder zu Ausgaben f�hren, ist nicht
vorgesehen.  Seltsames Verhalten ist in diesen F�llen m�glich und stellt
keinen Fehler dar.
\fi

Mit \Macro{setkomafont} wird die Schriftumschaltung eines Elements mit einer
v�llig neuen Definition versehen. Demgegen�ber wird mit \Macro{addtokomafont}
die existierende Definition lediglich erweitert.  Es wird empfohlen, beide
Anweisungen nicht innerhalb des Dokuments, sondern nur in der Dokumentpr�ambel
zu verwenden. Beispiele f�r die Verwendung entnehmen Sie bitte den Abschnitten
zu den jeweiligen Elementen. Namen und Bedeutung der einzelnen Elemente sind
in %
\fi % IgnoreThis
\IfNotCommon{scrextend}{\autoref{tab:\csname
    label@base\endcsname.elementsWithoutText} }%
\IfCommon{scrextend}{\autoref{tab:maincls.elementsWithoutText},
  \autopageref{tab:maincls.elementsWithoutText} }%
aufgelistet. %
\IfCommon{scrextend}{Allerdings werden davon in \Package{scrextend} nur die
  Elemente f�r den Dokumenttitel, den schlauen Spruch und die Fu�noten
  behandelt.  Das Element \FontElement{disposition} ist zwar auch verf�gbar,
  wird jedoch von \Package{scrextend} ebenfalls nur f�r den Dokumenttitel
  verwendet. }%
Die Voreinstellungen sind den jeweiligen Abschnitten zu entnehmen.


%%% Local Variables:
%%% mode: latex
%%% coding: iso-latin-1
%%% TeX-master: "../guide"
%%% End:
