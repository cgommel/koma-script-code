% ======================================================================
% typearea.tex
% Copyright (c) Markus Kohm, 2001-2012
%
% This file is part of the LaTeX2e KOMA-Script bundle.
%
% This work may be distributed and/or modified under the conditions of
% the LaTeX Project Public License, version 1.3c of the license.
% The latest version of this license is in
%   http://www.latex-project.org/lppl.txt
% and version 1.3c or later is part of all distributions of LaTeX 
% version 2005/12/01 or later and of this work.
%
% This work has the LPPL maintenance status "author-maintained".
%
% The Current Maintainer and author of this work is Markus Kohm.
%
% This work consists of all files listed in manifest.txt.
% ----------------------------------------------------------------------
% typearea.tex
% Copyright (c) Markus Kohm, 2001-2012
%
% Dieses Werk darf nach den Bedingungen der LaTeX Project Public Lizenz,
% Version 1.3c, verteilt und/oder veraendert werden.
% Die neuste Version dieser Lizenz ist
%   http://www.latex-project.org/lppl.txt
% und Version 1.3c ist Teil aller Verteilungen von LaTeX
% Version 2005/12/01 oder spaeter und dieses Werks.
%
% Dieses Werk hat den LPPL-Verwaltungs-Status "author-maintained"
% (allein durch den Autor verwaltet).
%
% Der Aktuelle Verwalter und Autor dieses Werkes ist Markus Kohm.
% 
% Dieses Werk besteht aus den in manifest.txt aufgefuehrten Dateien.
% ======================================================================
%
% Chapter about typearea for experts of the KOMA-Script guide
% Maintained by Markus Kohm
%
% ----------------------------------------------------------------------
%
% Kapitel ueber typearea fuer Experten in der KOMA-Script-Anleitung
% Verwaltet von Markus Kohm
%
% ======================================================================

\ProvidesFile{typearea-experts.tex}[2012/01/31 KOMA-Script guide (chapter: typearea)]
\translator{Markus Kohm\and Gernot Hassenpflug}

% Date of the translated German file: 2008/09/11

\chapter{Additional Information about package \Package{typearea}}
\labelbase{typearea-experts}

This chapter gives additional information about package
\Package{typearea}. \iffree{Some parts of the chapter are subject to the
  \KOMAScript{} book \cite{book:komascript} only. This shouldn't be a problem,
  because the}{The} average user, who only want to use the package, won't need
the information, that is addressed to users with non-standard requirements or
who want to write their own packages using \Package{typearea}. Another part of
the information describes features of \Package{typearea} that exist only
because of compatibility to former releases of \KOMAScript{} or the standard
classes. The features, that exist only because of compatibility to former
\KOMAScript{} releases, are printed with a sans serif font. You shouldn't use
them any longer.


\section{Expert Commands}
\label{sec:typearea-experts.experts}

This section describes commands, that aren't of interest for average
users. Nevertheless these commands provide additional features for
experts. Because the information is addressed to experts it's condensed.

\begin{Declaration}
  \Macro{activateareas}
\end{Declaration}%
\BeginIndex{Cmd}{activateareas}%
Package \Package{typearea} uses this command to assign settings of typing area
and margins to internal \LaTeX{} lengths, whenever the type area is newly
calculated inside of the document, i\.e., after
\Macro{begin}\PParameter{document}. If option \Option{pagesize} has been used,
it will be executed again afterward. With this, e.\,g., the page size may vary
inside of a PDF document.

Experts may use this command, if they change lengths like \Length{textwidth}
or \Length{textheight} inside a document manually for any reason. Nevertheless
the expert himself is responsible for eventually needed page breaks before or
after usage of \Macro{activateareas}. Moreover all changes of
\Macro{activateareas} are local!%
% 
\EndIndex{Cmd}{activateareas}

\begin{Declaration}
  \Macro{storeareas}\Parameter{\Macro{command}}
\end{Declaration}
\BeginIndex{Cmd}{storearea}%
With \Macro{storeareas} a \PName{\Macro{command}} will be defined, that may be
used to restore all current settings of typing area and margins. This provides
to store the current settings, change them, do anything with valid changed
settings and restore the previous settings afterwards.

\begin{Example}
  You want a landscape page inside a document with portrait format. No problem
  using \Macro{storeareas}:
\begin{lstcode}
  \documentclass[pagesize]{scrartcl}

  \begin{document}
  \rule{\textwidth}{\textheight}

  \storeareas\MySavedValues
  \KOMAoptions{paper=landscape,DIV=current}
  \rule{\textwidth}{\textheight}

  \clearpage
  \MySavedValues
  \rule{\textwidth}{\textheight}
  \end{document}
\end{lstcode}
  Command \Macro{clearpage}\textnote{Attention} before calling
  \Macro{MySavedValues} is important to restore the saved values at the next
  page.
\end{Example}
Please note\textnote{Attention!} that neither \Macro{storeareas} nor the
defined \PName{\Macro{command}} should be used inside a group. Internally
\Macro{newcommand}\IndexCmd{newcommand*}\important{\Macro{newcommand*}} is
used to define the \PName{\Macro{command}}. So re-usage of a
\PName{\Macro{command}} to store settings again would result in a
corresponding error message.
%
\EndIndex{Cmd}{storeareas}

\begin{Declaration}
  \Macro{AfterCalculatingTypearea}\Parameter{instructions}
\end{Declaration}%
\BeginIndex{Cmd}{AfterCalculatingTypearea}%
This command manages a \emph{hook} that provides experts to execute
\PName{instructions} every time \Package{typearea} has recalculated the typing
area and margins either implicitly or because of an explicit usage of
\Macro{typearea}. The \PName{instructions} thereby will be executed instantly
before execution of \Macro{activateareas}.\iffree{}{ In
  \autoref{cha:modernletters} the command will be used in letter class option
  \File{asymTypB.lco} to change the division of the margins.}%
% 
\EndIndex{Cmd}{AfterCalculatingTypearea}


\section{Local Settings with File \File{typearea.cfg}}
\label{sec:typearea-experts.cfg}
\BeginIndex{File}{typearea.cfg}

\LoadNonFree{typearea-experts}{0}

\EndIndex{File}{typearea.cfg}

\section{More or Less Obsolete Options and Commands}
\label{sec:typearea-experts.obsolete}

\LoadNonFree{typearea-experts}{1}

\endinput

%%% Local Variables: 
%%% mode: latex
%%% mode: flyspell
%%% ispell-local-dictionary: "english"
%%% coding: us-ascii
%%% TeX-master: "../guide"
%%% End: 
