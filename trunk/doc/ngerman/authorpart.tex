% ======================================================================
% authorpart.tex
% Copyright (c) Markus Kohm, 2002-2017
%
% This file is part of the LaTeX2e KOMA-Script bundle.
%
% This work may be distributed and/or modified under the conditions of
% the LaTeX Project Public License, version 1.3c of the license.
% The latest version of this license is in
%   http://www.latex-project.org/lppl.txt
% and version 1.3c or later is part of all distributions of LaTeX 
% version 2005/12/01 or later and of this work.
%
% This work has the LPPL maintenance status "author-maintained".
%
% The Current Maintainer and author of this work is Markus Kohm.
%
% This work consists of all files listed in manifest.txt.
% ----------------------------------------------------------------------
% authorpart.tex
% Copyright (c) Markus Kohm, 2002-2017
%
% Dieses Werk darf nach den Bedingungen der LaTeX Project Public Lizenz,
% Version 1.3c, verteilt und/oder veraendert werden.
% Die neuste Version dieser Lizenz ist
%   http://www.latex-project.org/lppl.txt
% und Version 1.3c ist Teil aller Verteilungen von LaTeX
% Version 2005/12/01 oder spaeter und dieses Werks.
%
% Dieses Werk hat den LPPL-Verwaltungs-Status "author-maintained"
% (allein durch den Autor verwaltet).
%
% Der Aktuelle Verwalter und Autor dieses Werkes ist Markus Kohm.
% 
% Dieses Werk besteht aus den in manifest.txt aufgefuehrten Dateien.
% ======================================================================
%
% First part: KOMA-Script for Authors
% Maintained by Markus Kohm
%
% ----------------------------------------------------------------------
%
% Erster Teil: KOMA-Script f�r Autoren
% Verwaltet von Markus Kohm
%
% ======================================================================

\ProvidesFile{authorpart.tex}[2008/05/06 Part KOMA-Script for Authors]

\setpartpreamble{%
  \vspace*{2\baselineskip}%
  \setlength{\parindent}{1em}%

  \noindent In diesem Teil sind die Informationen f�r die Autoren von
  Artikeln, Berichten, B�chern und Briefen zu finden. Dabei wird davon
  ausgegangen, dass der normale Anwender sich weniger daf�r interessiert, wie
  in \KOMAScript{} die Dinge implementiert sind und wo die Schwierigkeiten
  dabei liegen. Auch ist es f�r den normalen Anwender wenig interessant,
  welche obsoleten Optionen und Anweisungen noch enthalten sind. Er will
  wissen, wie er aktuell etwas erreichen kann. Eventuell ist er noch an
  typografischen Hintergrundinformationen interessiert.

  Die wenigen Passagen, die weiterf�hrende Informationen und Begr�ndungen
  enthalten und deshalb f�r ungeduldige Leser weniger von Interesse sind,
  wurden in diesem Teil in serifenloser Schrift gesetzt und k�nnen bei Bedarf
  �bersprungen werden. Wer hingegen noch mehr Informationen zu Hintergr�nden
  der Implementierung, Nebenwirkungen bei Verwendung anderer Pakete und
  zu obsoleten Optionen oder Anweisungen sucht, sei auf
  \autoref{part:forExperts} ab \autopageref{part:forExperts}
  verwiesen. Dar�ber hinaus besch�ftigt sich jener Teil von \KOMAScript{} auch
  mit all den M�glichkeiten, die speziell f�r Autoren von Paketen und Klassen
  geschaffen wurden.  
}

\part{\KOMAScript{} f�r Autoren}
\label{part:forAuthors}

\endinput

%%% Local Variables:
%%% mode: latex
%%% coding: iso-latin-1
%%% TeX-master: "guide.tex"
%%% End:
