% ======================================================================
% common-3.tex
% Copyright (c) Markus Kohm, 2001-2009
%
% This file is part of the LaTeX2e KOMA-Script bundle.
%
% This work may be distributed and/or modified under the conditions of
% the LaTeX Project Public License, version 1.3c of the license.
% The latest version of this license is in
%   http://www.latex-project.org/lppl.txt
% and version 1.3c or later is part of all distributions of LaTeX 
% version 2005/12/01 or later and of this work.
%
% This work has the LPPL maintenance status "author-maintained".
%
% The Current Maintainer and author of this work is Markus Kohm.
%
% This work consists of all files listed in manifest.txt.
% ----------------------------------------------------------------------
% common-3.tex
% Copyright (c) Markus Kohm, 2001-2009
%
% Dieses Werk darf nach den Bedingungen der LaTeX Project Public Lizenz,
% Version 1.3c, verteilt und/oder veraendert werden.
% Die neuste Version dieser Lizenz ist
%   http://www.latex-project.org/lppl.txt
% und Version 1.3c ist Teil aller Verteilungen von LaTeX
% Version 2005/12/01 oder spaeter und dieses Werks.
%
% Dieses Werk hat den LPPL-Verwaltungs-Status "author-maintained"
% (allein durch den Autor verwaltet).
%
% Der Aktuelle Verwalter und Autor dieses Werkes ist Markus Kohm.
% 
% Dieses Werk besteht aus den in manifest.txt aufgefuehrten Dateien.
% ======================================================================
%
% Paragraphs that are common for several chapters of the KOMA-Script guide
% Maintained by Markus Kohm
%
% ----------------------------------------------------------------------
%
% Abs�tze, die mehreren Kapiteln der KOMA-Script-Anleitung gemeinsam sind
% Verwaltet von Markus Kohm
%
% ======================================================================

\ProvidesFile{common-3.tex}[2009/02/27 KOMA-Script guide (common paragraphs)]

\makeatletter
\@ifundefined{ifCommonmaincls}{\newif\ifCommonmaincls}{}%
\@ifundefined{ifCommonscrextend}{\newif\ifCommonscrextend}{}%
\@ifundefined{ifCommonscrlttr}{\newif\ifCommonscrlttr}{}%
\@ifundefined{ifIgnoreThis}{\newif\ifIgnoreThis}{}%
\makeatother


\section{\texorpdfstring{Seitenauf"|teilung}{Seitenaufteilung}}
\label{sec:\csname label@base\endcsname.typearea}
\BeginIndex{}{Seitenaufteilung=Seitenauf\/teilung}

Eine Dokumentseite besteht aus unterschiedlichen Teilen, wie den R�ndern, dem
Kopf, dem Fu�, dem Textbereich, einer Marginalienspalte und den Abst�nden
zwischen diesen Elementen. \KOMAScript{} unterscheidet dabei auch noch
zwischen der Gesamtseite oder dem Papier und der sichtbaren Seite. Ohne
Zweifel geh�rt die Auf"|teilung der Seite in diese unterschiedlichen Teile zu
den Grundf�higkeiten einer Klasse. Bei \KOMAScript{} wird diese Arbeit an das
Paket \Package{typearea} delegiert. Dieses Paket kann auch zusammen mit
anderen Klassen verwendet werden. Die \KOMAScript-Klassen laden
\Package{typearea} jedoch selbst�ndig. Es ist daher weder notwendig noch
sinnvoll, das Paket bei Verwendung einer \KOMAScript-Klasse auch noch explizit
per \Macro{usepackage} zu laden. Siehe hierzu auch
\autoref{sec:\csname label@base\endcsname.options}.

Einige Einstellungen der \KOMAScript{}-Klassen haben auch Auswirkungen auf die
Seitenauf"|teilung und umgekehrt. Diese Auswirkungen werden bei den
entsprechenden Einstellungen dokumentiert.

F�r die weitere Erkl�rung zur Wahl des Papierformats, der Auf"|teilung der Seite
in R�nder und Satzspiegel und die Wahl von ein- oder zweispaltigem Satz sei
auf die Anleitung des Pakets \Package{typearea} verwiesen. Diese ist in
\autoref{cha:typearea} ab \autopageref{cha:typearea} zu finden.



%%% Local Variables:
%%% mode: latex
%%% coding: iso-latin-1
%%% TeX-master: "../guide"
%%% End:
