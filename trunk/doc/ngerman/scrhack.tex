% ======================================================================
% scrhack.tex
% Copyright (c) Markus Kohm, 2001-2012
%
% This file is part of the LaTeX2e KOMA-Script bundle.
%
% This work may be distributed and/or modified under the conditions of
% the LaTeX Project Public License, version 1.3c of the license.
% The latest version of this license is in
%   http://www.latex-project.org/lppl.txt
% and version 1.3c or later is part of all distributions of LaTeX 
% version 2005/12/01 or later and of this work.
%
% This work has the LPPL maintenance status "author-maintained".
%
% The Current Maintainer and author of this work is Markus Kohm.
%
% This work consists of all files listed in manifest.txt.
% ----------------------------------------------------------------------
% scrhack.tex
% Copyright (c) Markus Kohm, 2001-2012
%
% Dieses Werk darf nach den Bedingungen der LaTeX Project Public Lizenz,
% Version 1.3c, verteilt und/oder veraendert werden.
% Die neuste Version dieser Lizenz ist
%   http://www.latex-project.org/lppl.txt
% und Version 1.3c ist Teil aller Verteilungen von LaTeX
% Version 2005/12/01 oder spaeter und dieses Werks.
%
% Dieses Werk hat den LPPL-Verwaltungs-Status "author-maintained"
% (allein durch den Autor verwaltet).
%
% Der Aktuelle Verwalter und Autor dieses Werkes ist Markus Kohm.
% 
% Dieses Werk besteht aus den in manifest.txt aufgefuehrten Dateien.
% ======================================================================
%
% Chapter about scrhack of the KOMA-Script guide
% Maintained by Markus Kohm
%
% ----------------------------------------------------------------------------
%
% Kapitel �ber scrhack in der KOMA-Script-Anleitung
% Verwaltet von Markus Kohm
%
% ============================================================================

\ProvidesFile{scrhack.tex}[2011/10/07 KOMA-Script guide (chapter: scrhack)]

\chapter{Fremdpakete verbessern mit \Package{scrhack}}
\labelbase{scrhack}

\BeginIndex{Package}{scrhack}
Einige Pakete au�erhalb von \KOMAScript{} arbeiten nicht sehr gut mit
\KOMAScript{} zusammen. F�r den \KOMAScript-Autor ist es dabei oftmals sehr
m�hsam, die Autoren der jeweiligen Pakete von einer Verbesserung zu
�berzeugen. Das betrifft auch Pakete, deren Entwicklung eingestellt
wurde. Deshalb wurde das Paket \Package{scrhack} begonnen. Dieses Paket �ndert
Anweisungen und Definitionen anderer Pakete, damit sie besser mit
\KOMAScript{} zusammenarbeiten.

\section{Entwicklungsstand}
\label{sec:hack.draft}

Obwohl das Paket bereits seit l�ngerer Zeit Teil von \KOMAScript{} ist und
auch bereits von vielen Anwendern genutzt wird, hat es auch ein Problem. Bei
der Umdefinierung von Makros fremder Pakete ist es von der genauen Definition
und Verwendung dieser Makros abh�ngig. Damit ist es gleichzeitig auch von
bestimmten Versionen dieser Pakete abh�ngig. Wird eine unbekannte Version
eines der entsprechenden Pakete verwendet, kann \Package{scrhack} den
notwendigen Patch eventuell nicht ausf�hren. Im Extremfall kann aber umgekehrt
der Patch einer unbekannten Version auch zu einem Fehler f�hren.

Da also \Package{scrhack} immer wieder an neue Versionen fremder Pakete
angepasst werden muss, kann es niemals als fertig angesehen werden. Daher
existiert von \Package{scrhack} dauerhaft nur eine Beta-Version. Obwohl die
Benutzung in der Regel einige Vorteile mit sich bringt, kann die Funktion
nicht dauerhaft garantiert werden.

\LoadCommon{0}

\section{Verwendung von \Package{tocbasic}}
\label{sec:hack.improvement}

In den Anf�ngen von \KOMAScript{} gab es von Anwenderseite den Wunsch, dass
Verzeichnisse von Gleitumgebungen, die mit dem Paket
\Package{float}\IndexPackage{float}\important{\Package{float}} erzeugt werden,
genauso behandelt werden, wie das Abbildungsverzeichnis oder das
Tabellenverzeichnis, das von den \KOMAScript-Klassen selbst angelegt
wird. Damals setzte sich der \KOMAScript-Autor mit dem Autor von
\Package{float} in Verbindung, um diesem eine Schnittstelle f�r entsprechende
Erweiterungen zu unterbreiten. In etwas abgewandelter Form wurde diese in
Gestalt der beiden Anweisungen
\Macro{float@listhead}\IndexCmd[indexmain]{float@listhead} und
\Macro{float@addtolists}\IndexCmd[indexmain]{float@addtolists} realisiert.

Sp�ter zeigte sich, dass diese beiden Anweisungen nicht genug Flexibilit�t f�r
eine umfangreiche Unterst�tzung aller \KOMAScript-M�glichkeiten boten. Leider
hatte der Autor von \Package{float} die Entwicklung aber bereits eingestellt,
so dass hier keine �nderungen mehr zu erwarten sind.

Andere Paketautoren haben die beiden Anweisungen ebenfalls �bernommen. Dabei
zeigte sich, dass die Implementierung in einigen Paketen, darunter auch
\Package{float}, dazu f�hrt, dass all diese Pakete nur in einer bestimmten
Reihenfolge geladen werden k�nnen, obwohl sie ansonsten in keinerlei Beziehung
zu einander stehen.

Um all diese Nachteile und Probleme zu beseitigen, unterst�tzt \KOMAScript{}
diese alte Schnittstelle offiziell nicht mehr. Stattdessen wird bei Verwendung
dieser Schnittstelle von \KOMAScript{} gewarnt. Gleichzeitig wurde in
\KOMAScript{} das Paket
\Package{tocbasic}\IndexPackage{tocbasic}\important{\Package{tocbasic}} (siehe
\autoref{cha:tocbasic}) als zentrale Schnittstelle f�r die Verwaltung von
Verzeichnissen entworfen und realisiert. Die Verwendung dieses Pakets bietet
weit mehr Vorteile und M�glichkeiten als die beiden alten Anweisungen.

Obwohl der Aufwand zur Verwendung dieses Pakets sehr gering ist, haben bisher
die Autoren der Pakete, die auf die beiden alten Anweisungen gesetzt haben,
keine Anpassung vorgenommen. Daher enth�lt \Package{scrhack} selbst
entsprechende Anpassungen f�r die Pakete
\Package{float}\IndexPackage{float}\important{\Package{float},
  \Package{floatrow}, \Package{listings}},
\Package{floatrow}\IndexPackage{floatrow} und
\Package{listings}\IndexPackage{listings}. Allein durch das Laden von
\Package{scrhack} reagieren diese Pakete dann nicht nur auf die Einstellungen
von Option \Option{listof}\IndexOption{listof~=\PName{Einstellung}}, sondern
beachtet auch Sprachumschaltungen durch das
\Package{babel}-Paket\IndexPackage{babel}. N�heres zu den M�glichkeiten, die
durch die Umstellung der Pakete auf \Package{tocbasic} nun zur Verf�gung
stehen, ist \autoref{sec:tocbasic.toc} zu entnehmen.

Sollte diese �nderung f�r eines der Pakete nicht erw�nscht sein oder zu
Problemen f�hren, so kann sie selektiv mit den Einstellungen
\OptionValue{float}{false}\IndexOption[indexmain]{float~=\PValue{false}},
\OptionValue{floatrow}{false}\IndexOption[indexmain]{floatrow~=\PValue{false}}
und
\OptionValue{listings}{false}\IndexOption[indexmain]{listings~=\PValue{false}}
abgeschaltet werden. Wichtig\textnote{Achtung!} dabei ist, dass eine �nderung
der Optionen nach dem Laden des zugeh�rigen Pakets keinen Einfluss mehr hat!


\section{Sonderfall \Package{hyperref}}
\label{sec:hack.hyperref}

�ltere Versionen von
\Package{hyperref}\IndexPackage{hyperref}\important{\Package{hyperref}} vor
6.79h haben bei den Sternformen der Gliederungsbefehle hinter, statt vor oder
auf die Gliederungs�berschriften verlinkt. Inzwischen ist dieses Problem auf
Vorschlag des \KOMAScript-Autors beseitigt. Da die entsprechende �nderung aber
�ber ein Jahr auf sich warten lies, wurde in \Package{scrhack} ein
entsprechender Patch aufgenommen. Zwar kann dieser ebenfalls durch
\OptionValue{hyperref}{false} deaktiviert werden, empfohlen wird jedoch
stattdessen die aktuelle Version von \Package{hyperref} zu verwenden. In
diesem Fall wird die �nderung durch \Package{scrhack} automatisch verhindert.


\EndIndex{Package}{scrwfile}

%%% Local Variables: 
%%% mode: latex
%%% coding: iso-latin-1
%%% TeX-master: "../guide"
%%% End: 

% LocalWords:  Eindateiensystem Schreibdatei Zieldatei Zielendung Quellendung
% LocalWords:  Verzeichnis�berschrift Dateiendung Zielendungen Verzeichnisdatei
% LocalWords:  Benutzeranweisungen Dokumentpr�ambel Kapitelebene Sternformen
% LocalWords:  Sprachumschaltungen Gliederungs�berschriften Gliederungsbefehle
% LocalWords:  Paketautoren
