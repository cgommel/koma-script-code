% ======================================================================
% typearea-experts.tex
% Copyright (c) Markus Kohm, 2008-2011
%
% This file is part of the LaTeX2e KOMA-Script bundle.
%
% This work may be distributed and/or modified under the conditions of
% the LaTeX Project Public License, version 1.3c of the license.
% The latest version of this license is in
%   http://www.latex-project.org/lppl.txt
% and version 1.3c or later is part of all distributions of LaTeX 
% version 2005/12/01 or later and of this work.
%
% This work has the LPPL maintenance status "author-maintained".
%
% The Current Maintainer and author of this work is Markus Kohm.
%
% This work consists of all files listed in manifest.txt.
% ----------------------------------------------------------------------
% typearea-experts.tex
% Copyright (c) Markus Kohm, 2008-2011
%
% Dieses Werk darf nach den Bedingungen der LaTeX Project Public Lizenz,
% Version 1.3c, verteilt und/oder veraendert werden.
% Die neuste Version dieser Lizenz ist
%   http://www.latex-project.org/lppl.txt
% und Version 1.3c ist Teil aller Verteilungen von LaTeX
% Version 2005/12/01 oder spaeter und dieses Werks.
%
% Dieses Werk hat den LPPL-Verwaltungs-Status "author-maintained"
% (allein durch den Autor verwaltet).
%
% Der Aktuelle Verwalter und Autor dieses Werkes ist Markus Kohm.
% 
% Dieses Werk besteht aus den in manifest.txt aufgefuehrten Dateien.
% ======================================================================
%
% Chapter about typearea of the KOMA-Script guide part II
% Maintained by Markus Kohm
%
% ----------------------------------------------------------------------
%
% Kapitel �ber typearea in der KOMA-Script-Anleitung Teil II
% Verwaltet von Markus Kohm
%
% ======================================================================


\ProvidesFile{typearea-experts.tex}[2008/09/11 KOMA-Script guide (chapter:
typearea for experts)]

\chapter{Zus�tzliche Informationen zum Paket \Package{typearea.sty}}
\labelbase{typearea-experts}

In diesem Kapitel finden Sie zus�tzliche Informationen zum Paket
\Package{typearea}. \iffree{Einige Teile des Kapitels sind dabei dem
  \KOMAScript-Buch \cite{book:komascript} vorbehalten. Dies sollte kein
  Problem sein, denn der}{Der} normale Anwender, der das Paket einfach nur
verwenden will, wird diese Informationen normalerweise nicht ben�tigen. Ein
Teil der Informationen richtet sich an Anwender, die ausgefallene Aufgaben
l�sen oder eigene Pakete schreiben wollen, die auf \Package{typearea}
basieren. Ein weiterer Teil der Informationen behandelt M�glichkeiten von
\Package{typearea}, die aus Gr�nden der Kompatibilit�t zu den Standardklassen
oder fr�heren Versionen von \KOMAScript{} existieren. Die Teile, die nur aus
Gr�nden der Kompatibilit�t zu fr�heren Versionen von \KOMAScript{} existieren,
sind in serifenloser Schrift gesetzt und sollten nicht mehr verwendet werden.


\section{Anweisungen f�r Experten}
\label{sec:typearea-experts.experts}

In diesem Abschnitt werden Anweisungen beschrieben, die f�r den normalen
Anwender kaum oder gar nicht von Interesse sind. Experten bieten diese
Anweisungen zus�tzliche M�glichkeiten. Da die Beschreibungen f�r Experten
gedacht sind, sind sie k�rzer gefasst.

\begin{Declaration}
  \Macro{activateareas}
\end{Declaration}%
\BeginIndex{Cmd}{activateareas}%
Diese Anweisung wird von \Package{typearea} genutzt, um die Einstellungen
f�r Satzspiegel und R�nder in die internen \LaTeX-L�ngen zu �bertragen, wenn
der Satzspiegel innerhalb des Dokuments, also nach
\Macro{begin}\PParameter{document} neu berechnet wurde. Wurde die Option
\Option{pagesize} verwendet, so wird diese anschlie�end mit demselben Wert
neu aufgerufen. Damit kann beispielsweise innerhalb von PDF-Dokumenten die
Seitengr��e tats�chlich variieren.

Experten k�nnen diese Anweisung auch verwenden, wenn Sie aus irgendwelchen
Gr�nden, L�ngen wie \Length{textwidth} oder \Length{textheight} innerhalb des
Dokuments manuell ge�ndert haben. Der Experte ist dabei f�r eventuell
notwendige Seitenumbr�che vor oder nach der Verwendung jedoch selbst
verantwortlich! Dar�ber hinaus sind alle von \Macro{activateareas}
durchgef�hrten �nderungen lokal!%
% 
\EndIndex{Cmd}{activateareas}

\begin{Declaration}
  \Macro{storeareas}\Parameter{Anweisung}
\end{Declaration}
\BeginIndex{Cmd}{storearea}%
Mit Hilfe von \Macro{storeareas} wird eine \PName{Anweisung} definiert, �ber
die alle aktuellen Seitenspiegeleinstellungen wieder hergestellt werden
k�nnen. So ist es m�glich, die aktuellen Einstellungen zu speichern,
anschlie�end die Einstellungen zu �ndern und dann die gespeicherten
Einstellungen wieder zu reaktivieren.

\begin{Example}
  Sie wollen in einem Dokument im Hochformat eine Seite im Querformat
  unterbringen. Mit \Macro{storeareas} kein Problem:
\begin{lstcode}
  \documentclass[pagesize]{scrartcl}

  \begin{document}
  \rule{\textwidth}{\textheight}

  \storeareas\meinegespeichertenWerte
  \KOMAoptions{paper=landscape,DIV=current}
  \rule{\textwidth}{\textheight}

  \clearpage
  \meinegespeichertenWerte
  \rule{\textwidth}{\textheight}
  \end{document}
\end{lstcode}
  Wichtig\textnote{Achtung!} ist dabei die Anweisung \Macro{clearpage} vor dem 
  Aufruf von \Macro{meinegespeichertenWerte}, damit die Wiederherstellung erst
  auf der n�chsten Seite erfolgt.
\end{Example}
Bei\textnote{Achtung!} der Verwendung von \Macro{storeareas} ist zu beachten,
dass sowohl \Macro{storeareas} als auch die damit definierte \PName{Anweisung}
nicht innerhalb einer Gruppe aufgerufen werden sollten. Die Definition der
\PName{Anweisung} erfolgt intern mit
\Macro{newcommand}\IndexCmd{newcommand*}\important{\Macro{newcommand*}}. Bei
erneuter Verwendung einer bereits definierte \PName{Anweisung} wird eine
entsprechende Fehlermeldung ausgegeben.
%
\EndIndex{Cmd}{storeareas}

\begin{Declaration}
  \Macro{AfterCalculatingTypearea}\Parameter{Anweisungen}
\end{Declaration}%
\BeginIndex{Cmd}{AfterCalculatingTypearea}%
Diese Anweisung dient der Verwaltung eines \emph{Hooks} und erm�glicht es dem
Experten jedes Mal, nachdem \Package{typearea} eine neue Aufteilung in
Satzspiegel und R�nder berechnet hat, also nach jeder impliziten oder
expliziten Ausf�hrung von \Macro{typearea}, \PName{Anweisungen} ausf�hren zu
lassen. Diese \PName{Anweisungen} werden dabei unmittelbar vor
\Macro{activateareas} ausgef�hrt. \iffree{}{In \autoref{cha:modernletters}
  wird von dieser Anweisung f�r die \emph{Letter-Class-Option}
  \File{asymTypB.lco} Gebrauch gemacht, um die Randverteilung zu �ndern.}%
% 
\EndIndex{Cmd}{AfterCalculatingTypearea}


\section{Lokale Einstellungen durch die Datei \File{typearea.cfg}}
\label{sec:typearea-experts.cfg}
\BeginIndex{File}{typearea.cfg}

\LoadNonFree{typearea-experts}{0}

\section{Mehr oder weniger obsolete Optionen und Anweisungen}
\label{sec:typearea-experts.obsolete}

\LoadNonFree{typearea-experts}{1}

%%% Local Variables:
%%% mode: latex
%%% coding: iso-latin-1
%%% TeX-master: "../guide"
%%% End:

% LocalWords:  Hochformat Seitenspiegels
