% \CheckSum{792}
% \iffalse^^A meta-comment
% ======================================================================
% scrlfile.dtx
% Copyright (c) Markus Kohm, 2002-2019
%
% This file is part of the work `scrlfile' which is part of the LaTeX2e
% KOMA-Script bundle.
%
% This work may be distributed and/or modified under the conditions of
% the LaTeX Project Public License, version 1.3c of the license.
% The latest version of this license is in
%   http://www.latex-project.org/lppl.txt
% and version 1.3c or later is part of all distributions of LaTeX 
% version 2005/12/01 or later and of this work.
%
% This work has the LPPL maintenance status "author-maintained".
%
% The Current Maintainer and author of this work is Markus Kohm.
%
% This work consists of all files listed in manifest.txt.
% ----------------------------------------------------------------------
% scrlfile.dtx
% Copyright (c) Markus Kohm, 2002-2019
%
% Diese Datei ist Teil des Werks `scrlfile', das wiederum Teil des
% LaTeX2e KOMA-Script Pakets ist.
%
% Dieses Werk darf nach den Bedingungen der LaTeX Project Public Lizenz,
% Version 1.3c, verteilt und/oder veraendert werden.
% Die neuste Version dieser Lizenz ist
%   http://www.latex-project.org/lppl.txt
% und Version 1.3c ist Teil aller Verteilungen von LaTeX
% Version 2005/12/01 oder spaeter und dieses Werks.
%
% Dieses Werk hat den LPPL-Verwaltungs-Status "author-maintained"
% (allein durch den Autor verwaltet).
%
% Der Aktuelle Verwalter und Autor dieses Werkes ist Markus Kohm.
% 
% Dieses Werk besteht aus den in manifest.txt aufgefuehrten Dateien.
% ======================================================================
% \fi^^A meta-comment
%
% \CharacterTable
%  {Upper-case    \A\B\C\D\E\F\G\H\I\J\K\L\M\N\O\P\Q\R\S\T\U\V\W\X\Y\Z
%   Lower-case    \a\b\c\d\e\f\g\h\i\j\k\l\m\n\o\p\q\r\s\t\u\v\w\x\y\z
%   Digits        \0\1\2\3\4\5\6\7\8\9
%   Exclamation   \!     Double quote  \"     Hash (number) \#
%   Dollar        \$     Percent       \%     Ampersand     \&
%   Acute accent  \'     Left paren    \(     Right paren   \)
%   Asterisk      \*     Plus          \+     Comma         \,
%   Minus         \-     Point         \.     Solidus       \/
%   Colon         \:     Semicolon     \;     Less than     \<
%   Equals        \=     Greater than  \>     Question mark \?
%   Commercial at \@     Left bracket  \[     Backslash     \\
%   Right bracket \]     Circumflex    \^     Underscore    \_
%   Grave accent  \`     Left brace    \{     Vertical bar  \|
%   Right brace   \}     Tilde         \~}
%
% \iffalse
%%% From File: $Id$
%<*dtx>
\ifx\ProvidesFile\undefined\def\ProvidesFile#1[#2]{}\fi
\begingroup
  \def\filedate$#1: #2-#3-#4 #5${\gdef\filedate{#2/#3/#4}}
  \filedate$Date$
  \def\filerevision$#1: #2 ${\gdef\filerevision{r#2}}
  \filerevision$Revision: 2631 $
  \edef\reserved@a{%
    \noexpand\endgroup
    \noexpand\ProvidesFile{scrlfile.dtx}%
                          [\filedate\space\filerevision\space
                           KOMA-Script package source
  }%
\reserved@a
%</dtx>
%<*dtx|scrlfile>
%<scrlfile>\ProvidesPackage{scrlfile}[%
%!KOMAScriptVersion
%<scrlfile>  package 
  (loading files)]
%</dtx|scrlfile>
%<*dtx>
\ifx\documentclass\undefined
  \input scrdocstrip.tex
  \@@input scrkernel-version.dtx
  \@@input scrstrip.inc
  \KOMAdefVariable{COPYRIGHTFROM}{2002}
  \generate{\usepreamble\defaultpreamble
    \file{scrlfile.sty}{%
      \from{scrlfile.dtx}{scrlfile}%
      \from{scrlogo.dtx}{logo}%
    }%
  }
  \@@input scrstrop.inc
\else
  \let\endbatchfile\relax
\fi
\endbatchfile
\documentclass[parskip=half-]{scrdoc}
\usepackage[german,english]{babel}
\CodelineIndex
\RecordChanges
\GetFileInfo{scrlfile.dtx}
\title{\KOMAScript{} \partname\ \texttt{\filename}%
  \thanks{This file is version \fileversion\ of file \texttt{\filename}.}}  
\date{\filedate}
\author{Markus Kohm\thanks{mailto:komascript@gmx.info}~\thanks{Parts of
    description sections of the documentation where written by Jens-Uwe
    Morawski.}}

\begin{document}
  \maketitle
  \begin{abstract}
    This package provides hooks before and after loading files, packages or
    classes. It also provides a hook after the last \cs{clearpage} of the
    document. It allowes to replace files, packages and classes by other
    files, packages and classes. It is inteded to be used by package and class
    authors but may also be used by \LaTeX{} users.
  \end{abstract}

  \DocInput{\filename}
\end{document}
%</dtx>
% \fi
%
% \selectlanguage{english}
%
% \changes{v2.95}{2002/06/11}{First version split from hugh scrclass.dtx}%^^A
% \changes{v3.24}{2017/05/06}{standalone manual removed}%^^A
%
% \StopEventually{\PrintIndex\PrintChanges}
%
% \section{Implementation of \textsf{scrlfile}}
%
% \iffalse
%<*scrlfile>
% \fi
%
% \subsection{Preamble}
%
% \changes{v3.24}{2017/05/06}{Preamble sub-section removed}%^^A
% Since \KOMAScript~3.24 we do not longer output version information into the
% \texttt{log}-file.
%
% \subsection{Options}
%
% Currently there are no options.
%    \begin{macrocode}
\ExecuteOptions{}
\ProcessOptions\relax
%    \end{macrocode}
%
%
% \subsection{The macros}
%
% \begin{macro}{\InputIfFileExists}
% \changes{v2.96}{2006/07/19}{replacement code added}%^^A
% \LaTeX{} uses this macro for loading files with
% \cs{input}\marg{filename}, \cs{include}, \cs{LoadClass},
% \cs{LoadClassWithOptions}, \cs{documentclass}, \cs{documentstyle},
% \cs{RequirePackage}, \cs{usepackage}. Only
% \cs{input}~\meta{filename} loads \meta{filename} without using
% \cs{InputIfFileExists}.
%
% Because of this, it's very easy to add stuff before and after
% loading a file. We simply change this macro and add to
% macros\,---\,one before loading the file and one after loading the
% file. But before we redefine the macro, we test, if the macro was
% defined like we expected. \LaTeX{} will warn, if not.
% \changes{v3.27}{2019/03/28}{\LaTeX{} 2019-04-01 changes the definition}%^^A
%    \begin{macrocode}
\long\def\reserved@a#1#2{%
  \IfFileExists{#1}%
    {#2\@addtofilelist{#1}\@@input \@filef@und}}%
\ifx\reserved@a\InputIfFileExists\else
  \CheckCommand{\InputIfFileExists}[2]{%
    \IfFileExists{#1}%
    {%
      \expandafter\swap@args\expandafter
      {\@filef@und}{#2\@addtofilelist{#1}\@@input}}}%
\fi
%    \end{macrocode}
% Now we can redefine the macro like described above:
%    \begin{macrocode}
\renewcommand{\InputIfFileExists}[2]{%
  \begingroup\expandafter\expandafter\expandafter\endgroup
  \expandafter\ifx\csname #1-@alias\endcsname\relax
    \expandafter\@secondoftwo
  \else
    \scr@replacefile@msg{\csname #1-@alias\endcsname}{#1}%
    \expandafter\@firstoftwo
  \fi
  {%
    \expandafter\InputIfFileExists\expandafter{\csname
      #1-@alias\endcsname}{#2}%
  }%
  {\IfFileExists{#1}{%
      \expandafter\scr@input@withhook\expandafter{\@filef@und}{#1}{#2}%
    }}%
}
%    \end{macrocode}
% \begin{macro}{\scr@input@withhook}
% \changes{v3.27}{2019/03/28}{new (internal)}%^^A
% We need a new internal, because of expanding \cs{@file@und} before inputting
% the file.
%    \begin{macrocode}
\newcommand{\scr@input@withhook}[3]{%
  \scr@load@hook{before}{#2}%
  #3\@addtofilelist{#2}%
  \@@input #1%
  \scr@load@hook{after}{#2}%
}%
%    \end{macrocode}
% \end{macro}
% \end{macro}
%
%  \begin{macro}{\@loadwithoptions}
% \changes{v2.96}{2006/07/19}{new redefinition}
%  \begin{macro}{\scr@loadwithoptions}
% \changes{v2.96}{2006/07/19}{new helper macro}
%  \begin{macro}{\scr@@loadwithoptions}
% \changes{v2.97a}{2007/03/07}{new helper macro}
% \LaTeX{} uses this macro for loading classes and packages using
% \cs{LoadClassWithOptions} or \cs{RequirePackageWithOptions}. We simply say
% now: if the class or package should be replaced by another, try it again
% with the replacement name. The stack is used. This will result in an
% error if the replacements are nested too deep. This should be, because it
% will also result in an error using ring replacements.
%    \begin{macrocode}
\newcommand*{\scr@@loadwithoptions}{}
\newcommand*{\scr@loadwithoptions}{}
\let\scr@loadwithoptions\@loadwithoptions
\renewcommand*{\@loadwithoptions}[3]{%
  \begingroup\expandafter\expandafter\expandafter\endgroup
  \expandafter\ifx\csname #3.#1-@aliasname\endcsname\relax
    \def\scr@@loadwithoptions{\scr@loadwithoptions{#1}{#2}{#3}}%
  \else
    \scr@replacefile@msg{\csname #3.#1-@aliasname\endcsname.#1}{#3.#1}% 
    \def\scr@@loadwithoptions{%
      \@loadwithoptions{#1}{#2}{\csname #3.#1-@aliasname\endcsname}%
    }%
  \fi
  \scr@@loadwithoptions
}
%    \end{macrocode}
%  \end{macro}
%  \end{macro}
%  \end{macro}
%
%  \begin{macro}{\@onefilewithoption}
% \changes{v2.96}{2006/07/19}{new redefinition}
%  \begin{macro}{\scr@onefilewithoption}
% \changes{v2.96}{2006/07/19}{new helper macro}
%  \begin{macro}{\scr@@onefilewithoption}
% \changes{v2.98}{2007/12/18}{new helper macro}
% \LaTeX{} uses this macro for loading classes and packages using
% \cs{LoadClass}, \cs{documentclass}, \cs{RequirePackage}, and
% \cs{usepackage}. We simply say now: if the class or package should be
% replaced by another, try it again with the replacement name. The stack is
% used. This will result in an error if the replacements are nested too
% deep. This should be, because it will also result in an error using ring
% replacements.
%    \begin{macrocode}
\newcommand*{\scr@onefilewithoptions}{}
\let\scr@onefilewithoptions\@onefilewithoptions
\def\@onefilewithoptions#1[#2][#3]#4{%
  \begingroup\expandafter\expandafter\expandafter\endgroup
  \expandafter\ifx\csname #1.#4-@aliasname\endcsname\relax
%    \end{macrocode}
% Do this ridiculous definition and call outside of \cs{ifx}\dots\cs{fi}
% because of some errornous packages like some versions of
% \textsf{breakurl}. This does not avoid an error, but the message about
% unclosed \cs{if} will be outside of \textsf{scrlfile}.
%    \begin{macrocode}
    \def\scr@@onefilewithoptions{\scr@onefilewithoptions{#1}}%
  \else
    \scr@replacefile@msg{\csname #1.#4-@aliasname\endcsname.#4}{#1.#4}%
    \edef\scr@@onefilewithoptions{%
      \noexpand\@onefilewithoptions{\csname #1.#4-@aliasname\endcsname}}%
  \fi
  \scr@@onefilewithoptions[{#2}][{#3}]{#4}%
  \scr@load@hook{lateafter}{#1.#4}%
}
%    \end{macrocode}
%  \end{macro}
%  \end{macro}
%  \end{macro}
%
% \begin{macro}{\ReplaceInput}
% \changes{v2.96}{2006/07/19}{new command}
% \changes{v3.12}{2013/11/15}{implemented}
% Defining the replacement by definition of \cs{\#1-@alias}.
%    \begin{macrocode}
\newcommand*{\ReplaceInput}[2]{%
  \expandafter\edef\csname #1-@alias\endcsname{#2}%
}
%    \end{macrocode}
% \begin{macro}{\ReplaceFile}
% \changes{v2.96}{2006/07/19}{new command}
% \changes{v3.12}{2013/11/15}{throw error}
% Sorry, but I've implemented the wrong command some time ago.
%    \begin{macrocode}
\newcommand*{\ReplaceFile}{%
  \PackageError{scrlfile}{%
    usage of not documented command `\string\ReplaceFile'%
  }{%
    Command `\string\ReplaceFile' has been implemented accidentally in
    KOMA-Script 2.96\MessageBreak
    and replaced by `\string\ReplaceInput' in KOMA-Script 3.12.\MessageBreak
    You should replace `\string\ReplaceFile' by `\string\ReplaceInput' to
    avoid this error\MessageBreak
    message%
  }%
  \ReplaceInput
}
%    \end{macrocode}
% \end{macro}
% \end{macro}
%
% \begin{macro}{\UnReplaceInput}
% \changes{v3.12}{2013/11/15}{new command}
% Do not longer replace a file.
%    \begin{macrocode}
\newcommand*{\UnReplaceInput}[1]{%
  \ifcsname #1-@alias\endcsname
    \expandafter\let\csname #1-@alias\endcsname\relax
  \fi
}
%    \end{macrocode}
% \end{macro}
%
% \begin{macro}{\ReplacePackage}
% \changes{v2.96}{2006/07/19}{new command}
% \begin{macro}{\ReplaceClass}
% \changes{v2.96}{2006/07/19}{new command}
% \begin{macro}{\@replacefilewithext}
% \changes{v2.96}{2006/07/19}{new helper command}%^^A
% Note that replacing a package or class will not replace the input of the
% file of the package or class but only the input of package or class.
%    \begin{macrocode}
\newcommand*{\@replacefilewithext}[3]{%
  \expandafter\edef\csname #1.#3-@aliasname\endcsname{#2}%
 }
\newcommand*{\ReplacePackage}[2]{%
  \@replacefilewithext{#1}{#2}\@pkgextension
}
\newcommand*{\ReplaceClass}[2]{%
  \@replacefilewithext{#1}{#2}\@clsextension
}
%    \end{macrocode}
% \end{macro}
% \end{macro}
% \end{macro}
%
% \begin{macro}{\UnReplacePackage}
% \changes{v3.12}{2013/11/15}{new command}
% \begin{macro}{\UnReplaceClass}
% \changes{v3.12}{2013/11/15}{new command}
% \begin{macro}{\@unreplacefilewithext}
% \changes{v3.12}{2013/11/15}{new helper command}%^^A
% Do not nolonger replace a class or package.
%    \begin{macrocode}
\newcommand*{\@unreplacefilewithext}[2]{%
  \ifcsname #1.#2-@aliasname\endcsname
    \expandafter\let\csname #1.#2-@aliasname\endcsname\relax
  \fi
}
\newcommand*{\UnReplacePackage}[1]{%
  \@unreplacefilewithext{#1}\@pkgextension
}
\newcommand*{\UnReplaceClass}[1]{%
  \@unreplacefilewithext{#1}\@clsextension
}
%    \end{macrocode}
% \end{macro}
% \end{macro}
% \end{macro}
% 
%
%  \begin{macro}{\scr@replacefile@msg}
% \changes{v2.96}{2006/07/19}{new helper command}
% \changes{v2.98}{2007/12/18}{spellfix}
% Last but not least \textsf{scrlfile} will write a message to the log file at
% every replacement.
%    \begin{macrocode}
\newcommand*{\scr@replacefile@msg}[2]{%
  \PackageInfo{scrlfile}{inputing `#1' instead of `#2'}%
}
%    \end{macrocode}
%  \end{macro}
%
%  \begin{macro}{\scr@load@hook}
% This macro simply calls a load-file-hook, either the \emph{before}
% or the \emph{after} one, if the hook exists. After calling the hook
% it will be deleted. First argument is the type of hook
% (\meta{before} or \meta{after}), second argument is a filename. 
%    \begin{macrocode}
\newcommand*{\scr@load@hook}[2]{%
  \@ifundefined{#2-@#1}{}{%
    \@nameuse{#2-@#1}%
    \expandafter\let\csname #2-@#1\endcsname=\relax}}
%    \end{macrocode}
%  \end{macro}
%
%  \begin{macro}{\scr@set@load@hook}
% This command adds stuff to a load-file-hook, either the
% \emph{before} or the \emph{after} one. First argument is the type of
% hook (\meta{before} or \meta{after}), second is a filename. There's
% a third argument not parsed by the macro itself but
% \cs{g@addto@macro}. This third argument is the stuff, which will be
% added to the hook. Because of using \cs{g@addto@macro} the hook must
% be defined. So we define it, if it isn't.
%    \begin{macrocode}
\newcommand*{\scr@set@load@hook}[2]{%
  \@ifundefined{#2-@#1}{\@namedef{#2-@#1}{}}{}%
  \expandafter\g@addto@macro\csname #2-@#1\endcsname}
%    \end{macrocode}
%  \end{macro}
%
%  \begin{macro}{\BeforeFile}
%  \begin{macro}{\AfterFile}
%  \begin{macro}{\BeforeClass}
%  \begin{macro}{\AfterClass}
% \changes{v3.20}{2016/04/12}{\cs{@ifnextchar} replaced by
%     \cs{kernel@ifnextchar}}%^^A
% \changes{v3.20}{2016/04/12}{\cs{@ifstar} durch \cs{kernel@ifstar}
%     ersetzt}%^^A
%  \begin{macro}{\AfterAtEndOfClass}
%  \begin{macro}{\@AfterClass}
%  \begin{macro}{\@@AfterClass}
% \changes{v3.09}{2011/03/08}{new (internal)}
%  \begin{macro}{\@@@AfterClass}
% \changes{v3.09}{2011/03/09}{new (internal)}
%  \begin{macro}{\@sAfterClass}
%  \begin{macro}{\BeforePackage}
%  \begin{macro}{\AfterPackage}
% \changes{v3.20}{2016/04/12}{\cs{@ifnextchar} replaced by
%     \cs{kernel@ifnextchar}}%^^A
% \changes{v3.20}{2016/04/12}{\cs{@ifstar} durch \cs{kernel@ifstar}
%     ersetzt}%^^A
%  \begin{macro}{\AfterAtEndOfPackage}
%  \begin{macro}{\@AfterPackage}
%  \begin{macro}{\@@AfterPackage}
% \changes{v3.09}{2011/03/08}{new (internal)}
%  \begin{macro}{\@@@AfterPackage}
% \changes{v3.09}{2011/03/09}{new (internal)}
%  \begin{macro}{\@sAfterPackage}
% These macros add stuff to a load-file-hook of a file, class or
% package. First argument is the name of the file (with extension),
% class (without extension) or package (without extension). There's a
% second argument parsed by wether the macros themself nor
% \cs{scr@set@load@hook} but \cs{g@addto@macro}. See
% \cs{scr@set@load@hook} for more information. The star-forms
% \cs{AfterClass*} and \cs{AfterPackage*} does the action immediate,
% if the class or package was already loaded.
%    \begin{macrocode}
\newcommand*{\BeforeFile}{\scr@set@load@hook{before}}
\newcommand*{\AfterFile}{\scr@set@load@hook{after}}
\newcommand*{\BeforeClass}[1]{%
  \scr@set@load@hook{before}{#1.\@clsextension}}
\newcommand*{\AfterClass}{%
  \kernel@ifstar {\@sAfterClass}{%
    \kernel@ifnextchar +\@@AfterClass{%
      \kernel@ifnextchar !\@@@AfterClass\@AfterClass
    }%
  }%
}
\newcommand*{\@@@AfterClass}[2]{%
  \begingroup\ifx\csname #2.\@clsextension-h@@k\endcsname\relax
    \aftergroup\@sAfterClass
  \else
    \aftergroup\AfterAtEndOfClass
  \fi
  \endgroup{#2}%
}
\newcommand*{\AfterAtEndOfClass}[1]{%
  \scr@set@load@hook{lateafter}{#1.\@clsextension}}
\newcommand*{\@@AfterClass}[2]{%
  \begingroup
    \expandafter\ifx\csname #2.\@clsextension-h@@k\endcsname\relax
      \aftergroup\@sAfterClass
    \else
      \aftergroup\@AfterClass
    \fi
  \endgroup{#2}%
}
\newcommand*{\@AfterClass}[1]{%
  \scr@set@load@hook{after}{#1.\@clsextension}}
\newcommand*{\@sAfterClass}[1]{%
  \begingroup
    \@ifclassloaded{#1}{%
      \aftergroup\@secondoftwo
    }{%
      \aftergroup\@AfterClass
    }%
  \endgroup
  {#1}%
}
\newcommand*{\BeforePackage}[1]{%
  \scr@set@load@hook{before}{#1.\@pkgextension}}
\newcommand*{\AfterPackage}{%
  \kernel@ifstar {\@sAfterPackage}{%
    \kernel@ifnextchar +\@@AfterPackage{%
      \kernel@ifnextchar !\@@@AfterPackage\@AfterPackage
    }%
  }%
}
\newcommand*{\@@@AfterPackage}[2]{%
  \begingroup\ifx\csname #2.\@pkgextension-h@@k\endcsname\relax
    \aftergroup\@sAfterPackage
  \else
    \aftergroup\AfterAtEndOfPackage
  \fi
  \endgroup{#2}%
}
\newcommand*{\AfterAtEndOfPackage}[1]{%
  \scr@set@load@hook{lateafter}{#1.\@pkgextension}}
\newcommand*{\@@AfterPackage}[2]{%
  \begingroup
    \expandafter\ifx\csname #2.\@pkgextension-h@@k\endcsname\relax
      \aftergroup\@sAfterPackage
    \else
      \aftergroup\@AfterPackage
    \fi
  \endgroup{#2}%
}
\newcommand*{\@AfterPackage}[1]{%
  \scr@set@load@hook{after}{#1.\@pkgextension}}
\newcommand*{\@sAfterPackage}[1]{%
  \begingroup
    \@ifpackageloaded{#1}{%
      \aftergroup\@secondoftwo
    }{%
      \aftergroup\@AfterPackage
    }%
  \endgroup
  {#1}%
}
%    \end{macrocode}
%  \end{macro}
%  \end{macro}
%  \end{macro}
%  \end{macro}
%  \end{macro}
%  \end{macro}
%  \end{macro}
%  \end{macro}
%  \end{macro}
%  \end{macro}
%  \end{macro}
%  \end{macro}
%  \end{macro}
%  \end{macro}
%  \end{macro}
%  \end{macro}
%
% \begin{macro}{\PreventPackageFromLoading}
% \changes{v3.08}{2010/11/04}{New}
% \changes{v3.11a}{2012/06/13}{write fault error in \cs{MessageBreak} fixed}
% \changes{v3.12}{2012/06/15}{new star version allowes empty argument with
%     info instead of warning}
% \changes{v3.20}{2016/04/12}{\cs{@ifstar} durch \cs{kernel@ifstar}
%     ersetzt}%^^A
% \begin{macro}{\@PreventPackageFromLoading}
% \changes{v3.12}{2012/06/15}{New (internal)}%^^A
% \changes{v3.20}{2016/04/12}{\cs{@ifnextchar} replaced by
%     \cs{kernel@ifnextchar}}%^^A
% \begin{macro}{\@@PreventPackageFromLoading}
% \changes{v3.12}{2013/11/22}{New (internal)}%^^A
% \changes{v3.12}{2013/11/22}{\cs{usepackage} before \cs{documentclass}
%     fixed}%^^A
% \begin{macro}{\scrlfile@excludedpackages}
% \changes{v3.08}{2010/11/04}{New (internal)}%^^A
% \begin{macro}{\scrlfile@saved@RequirePackage}
% \changes{v3.08}{2010/11/04}{New (internal)}%^^A
%    \begin{macrocode}
\newcommand*{\scrlfile@excludedpackages}{}
\let\scrlfile@excludedpackages\@empty
\newcommand*{\scrlfile@saved@RequirePackage}{}
\newcommand*{\PreventPackageFromLoading}{%
  \kernel@ifstar{\@PreventPackageFromLoading\PackageInfo}%
          {\@PreventPackageFromLoading\PackageWarning}%
}
\newcommand*{\@PreventPackageFromLoading}[1]{%
  \kernel@ifnextchar [%]
    {\@@PreventPackageFromLoading#1}{\@@PreventPackageFromLoading#1[]}%
}
\newcommand*{\@@PreventPackageFromLoading}{}
\def\@@PreventPackageFromLoading#1[#2]#3{%
%    \end{macrocode}
% We do nothing, if the argument was empty:
%    \begin{macrocode}
  \edef\reserved@b{\zap@space#3 \@empty}%
  \ifx \reserved@b\@empty
    #1{scrlfile}{%
      \string\PreventPackageFromLoading\space with empty packages\MessageBreak
      argument ignored%
    }%
  \else
%    \end{macrocode}
% If the argument was not empty we first need to patch \cs{RequirePackage} and
% maybe \cs{usepackage} too.
%    \begin{macrocode}
    \ifx\scrlfile@excludedpackages\@empty
      \let\scrlfile@saved@RequirePackage\RequirePackage
      \let\RequirePackage\scrlfile@RequirePackage
      \def\reserved@a##1##{%  
        \@latex@error
        {\noexpand \usepackage before \string\documentclass}%
        {\noexpand \usepackage may only appear in the document
          preamble, i.e.,\MessageBreak
          between \noexpand\documentclass and
          \string\begin{document}.}%
        \@gobble}%
      \ifx\reserved@a\usepackage\else
        \ifx\usepackage\scrlfile@saved@RequirePackage\else
          \PackageWarning{scrlfile}{%
            Maybe serious problem: unexpected definition of\MessageBreak
            \string\usepackage
          }%
        \fi
        \let\usepackage\RequirePackage
      \fi
    \fi
%    \end{macrocode}
% We have to test, if a package was already loaded. Only not yet loaded
% packages will be added to the exclusion list. Empty arguments will be
% ignored.
% \changes{v3.11a}{2012/06/11}{missing \cs{scrlfile@excludedpackages} at
%   redefinition of \cs{scrlfile@excludedpackages} added}%^^A
% \changes{v3.26}{2018/08/29}{spurious \cs{expandafter} removed}%^^A
%    \begin{macrocode}
    \@for\reserved@a:=\reserved@b\do {%
      \ifx\reserved@a\@empty
        #1{scrlfile}{%
          empty package argument for
          \string\PreventPackageFromLoading\MessageBreak
          ignored%
        }%
      \else
        \expandafter\@ifpackageloaded\expandafter{\reserved@a}{%
          #1{scrlfile}{%
            package `\reserved@a' already loaded.\MessageBreak
            Cannot prevent it from beeing loaded%
          }%
        }{%
          \edef\scrlfile@excludedpackages{\scrlfile@excludedpackages,%
            \reserved@a}%
          \if\relax\detokenize{#2}\relax\else
            \@ifundefined{scrlfile@exclude@package@\reserved@a @do}{%
              \@namedef{scrlfile@exclude@package@\reserved@a @do}{}%
            }{}%
            \expandafter\def
            \csname scrlfile@exclude@package@\reserved@a 
                    @do\expandafter\expandafter\expandafter\endcsname
            \expandafter\expandafter\expandafter{%
              \csname scrlfile@exclude@package@\reserved@a @do\endcsname 
              #2%
            }%
          \fi
        }%
      \fi
    }%
%    \end{macrocode}
% If the exclusion list is still empty (because all packages has been loaded
% already), remove the patch:
%    \begin{macrocode}
    \ifx\scrlfile@excludedpackages\@empty
      \ResetPreventPackageFromLoading
    \fi
  \fi
}
\@onlypreamble\PreventPackageFromLoading
%    \end{macrocode}
% \begin{macro}{\ResetPreventPackageFromLoading}
% \changes{v3.08}{2010/11/04}{New}
% \begin{macro}{\StorePreventPackageFromLoading}
% \changes{v3.08}{2010/11/04}{New}
% These two are very simple
%    \begin{macrocode}
\newcommand*{\ResetPreventPackageFromLoading}{%
  \let\scrlfile@excludedpackages\@empty
  \ifx\RequirePackage\scrlfile@RequirePackage
    \ifx\usepackage\RequirePackage
      \let\usepackage\scrlfile@saved@RequirePackage
    \fi
    \let\RequirePackage\scrlfile@saved@RequirePackage
  \fi
}
\@onlypreamble\ResetPreventPackageFromLoading
\newcommand*{\StorePreventPackageFromLoading}[1]{%
  \let#1\scrlfile@excludedpackages
}
\@onlypreamble\StorePreventPackageFromLoading
%    \end{macrocode}
% \begin{macro}{\UnPreventPackageFromLoading}
% \changes{v3.12}{2013/11/22}{New}%^^A
% \changes{v3.20}{2016/04/12}{\cs{@ifstar} durch \cs{kernel@ifstar}
%     ersetzt}%^^A
% \begin{macro}{\@UnPreventPackageFromLoading}
% \changes{v3.12}{2013/11/22}{New (internal)}%^^A
% This removes packages from the prevent exclude list. The star variant also
% undefines the instead code.
%    \begin{macrocode}
\newcommand*{\UnPreventPackageFromLoading}{%
  \kernel@ifstar {\@tempswatrue\@UnPreventPackageFromLoading}%
           {\@tempswafalse\@UnPreventPackageFromLoading}%
}
\newcommand*{\@UnPreventPackageFromLoading}[1]{%
  \edef\reserved@b{\zap@space#1 \@empty}%
  \if@tempswa
    \@for\reserved@a:=\reserved@b\do {%
      \ifx\reserved@a\@empty \else
        \ifcsname scrlfile@exclude@package@\reserved@a @do\endcsname
          \expandafter\let
          \csname scrlfile@exclude@package@\reserved@a @do\endcsname
          \undefined
        \fi
      \fi
    }%
  \fi
  \ifcsname scrlfile@excludedpackages\endcsname
    \let\reserved@a\scrlfile@excludedpackages
    \let\scrlfile@excludedpackages\@empty
    \expandafter\@for\expandafter\reserved@c
    \expandafter:\expandafter=\reserved@a
    \do{%
      \ifx\reserved@c\@empty\else
        \@expandtwoargs\in@{,\reserved@c,}{,\reserved@b,}%
        \ifin@
        \else
          \edef\scrlfile@excludedpackages{%
            \scrlfile@excludedpackages,\reserved@c
          }%
        \fi
      \fi
    }%
    \ifx\scrlfile@excludedpackages\@empty\ResetPreventPackageFromLoading\fi
  \fi
}
%    \end{macrocode}
% \end{macro}
% \end{macro}
% \begin{macro}{\scrlfile@RequirePackage}
% \changes{v3.08}{2010/11/04}{New (internal)}
% \changes{v3.12}{2013/11/22}{tolerate spaces}
% \changes{v3.12}{2013/11/22}{execute the optional argument of
%     \cs{PreventPackageFromLoading}}
% This macro will replace \cs{RequirePackage} whenever the exlude list is not
% empty
%    \begin{macrocode}
\newcommand*{\scrlfile@RequirePackage}[2][]{%
  \let\reserved@c\@empty
  \let\reserved@d\@empty
  \edef\reserved@b{\zap@space#2 \@empty}%
  \expandafter\@for\expandafter\reserved@b\expandafter:\expandafter=\reserved@b
  \do {%
    \begingroup
      \@tempswatrue
      \@for\reserved@a:=\scrlfile@excludedpackages\do {%
        \ifx\reserved@a\reserved@b
          \@tempswafalse
        \fi
      }%
      \if@tempswa
        \xdef\reserved@c{\reserved@c,\reserved@b}%
      \else
        \ifx\reserved@b\@empty\else
          \PackageInfo{scrlfile}{Excluding package `\reserved@b'\MessageBreak
            from loading}%
          \@ifundefined{scrlfile@exclude@package@\reserved@b @do}{%
          }{%
            \expandafter\g@addto@macro\expandafter\reserved@d
            \csname scrlfile@exclude@package@\reserved@b @do\endcsname
          }%
        \fi
      \fi
    \endgroup
  }%
  \ifx\reserved@c\@empty
    \let\reserved@a\scr@gobbleopt
  \else
    \edef\reserved@a{\noexpand\scrlfile@saved@RequirePackage[#1]{%
        \expandafter\@gobble\reserved@c\@empty}}%
  \fi
  \expandafter\reserved@d
  \reserved@a
}
%    \end{macrocode}
% \end{macro}
% \end{macro}
% \end{macro}
% \end{macro}
% \end{macro}
% \end{macro}
% \end{macro}
% \end{macro}
%
% \begin{macro}{\scr@gobbleopt}
% \changes{v3.08}{2010/11/04}{New (internal)}
% \changes{v3.08a}{2011/02/11}{\cs{@gobbleopt} renamed due to compaibility
%     problems with \texttt{ntheorem}}
% Eat the optional argument if it exists.
%    \begin{macrocode}
\newcommand{\scr@gobbleopt}[1][]{}
%    \end{macrocode}
% \end{macro}
%
% \begin{macro}{\kernel@ifstar}
% \changes{v3.20}{2016/04/12}{New (internal)}
% I hope, the \LaTeX{} kernel will define this command in future. But
% nevertheless, there are \LaTeX{} kernels in real live without this
% command. So I do it again if needed:
%    \begin{macrocode}
\providecommand*\kernel@ifstar[1]{\kernel@ifnextchar*{\@firstoftwo{#1}}}
%    \end{macrocode}
% \end{macro}%^^A \kernel@ifstar
%
% \begin{macro}{\BeforeClosingMainAux}
% \changes{v2.95}{2002/09/03}{new}
% \begin{macro}{\AfterReadingMainAux}
% \changes{v3.03}{2009/03/25}{new}
% These are completly different. First we have to establish a new hook
% at \cs{end\{document\}}, just after closing the group and after the
% final \cs{clearpage}.
%    \begin{macrocode}
\AtEndDocument{%
  \let\scr@saved@checkend=\@checkend
  \renewcommand*{\@checkend}[1]{%
    \def\reserved@a{#1}\def\reserved@b{document}%
    \scr@saved@checkend{#1}%
    \ifx\reserved@a\reserved@b
      \let\scr@saved@clearpage=\clearpage
      \renewcommand*{\clearpage}{\scr@saved@clearpage
        \scr@hook@bcma
        \scr@restore@newpage
        \let\clearpage=\scr@saved@clearpage}%
    \fi%
    \let\scr@saved@dofilelist\@dofilelist
    \def\@dofilelist{%
      \scr@hook@acma
      \let\@dofilelist\scr@saved@dofilelist
      \scr@saved@dofilelist
    }%
  }%
}
%    \end{macrocode}
% \begin{macro}{\scr@saved@checkend}
% \changes{v2.95}{2002/09/03}{new (internal)}
% \begin{macro}{\scr@saved@dofilelist}
% \changes{v3.03}{2009/03/25}{new}
% You may see above, that we need save-to macros.
%    \begin{macrocode}
\newcommand*{\scr@saved@checkend}{}
\newcommand*{\scr@saved@dofilelist}{}
%    \end{macrocode}
% \end{macro}
% \end{macro}
% \begin{macro}{\scr@redefine@newpage}
% \changes{v3.00}{2008/07/09}{new (internal)}
%   Some classes (e.g. \textsf{letter}) write output to the aux file.
%   Because of this, redefinition of \cs{newpage} will only be done, if
%   \cs{BeforeClosingMainAux} is used.
% \begin{macro}{\scr@saved@newpage}
% \changes{v3.00}{2008/07/09}{new (internal)}
%   And \cs{newpage} will be restored after the hook.
% \begin{macro}{\scr@restore@newpage}
%    \begin{macrocode}
\newcommand*{\scr@redefine@newpage}{%
  \let\scr@saved@newpage\newpage
  \def\scr@restore@newpage{%
    \renewcommand*{\newpage}{%
      \PackageWarningNoLine{%
        scrlfile%
      }{%
        \string\newpage\space at main aux file!\MessageBreak
        You have output material at the main aux file.\MessageBreak
        Note, that the \string\BeforeClosingMainAux\space commands\MessageBreak
        are processed before the output from aux file%
      }%
      \let\newpage\scr@saved@newpage
      \newpage
    }%
  }%
  \renewcommand*{\newpage}{%
    \PackageError{%
      scrlfile%
    }{%
      \string\newpage\space at \string\BeforeClosingMainAux
    }{%
      You should never use \string\newpage\space or
      \string\clear(double)page or other output material at the\MessageBreak
      argument of \string\BeforeClosingMainAux!\MessageBreak
      If you'll continue, the \string\newpage\space will be ignored to
      avoid corruption of\MessageBreak
      every prior use of \string\BeforeClosingMainAux.
    }%
  }%
}
\newcommand*{\scr@restore@newpage}{}
%    \end{macrocode}
% \end{macro}
% \end{macro}
% \end{macro}
% \begin{macro}{\scr@hook@bcma}
% \changes{v2.95}{2002/09/03}{new (internal)}
% \begin{macro}{\scr@hook@acma}
% \changes{v3.03}{2009/03/25}{new (internal)}
% We also need macros, that saves the commands at the hooks and are
% extendable using \cs{g@addto@macro}.
%    \begin{macrocode}
\newcommand*{\scr@hook@bcma}{}
\newcommand*{\scr@hook@acma}{}
%    \end{macrocode}
%  \end{macro}
%  \end{macro}
% Now we are able to write the user command. Here we also warn, if we detect
% one more \cs{newpage}\,---\,maybe at a \cs{clearpage} or
% \cs{cleardoublepage}.
%    \begin{macrocode}
\newcommand*{\BeforeClosingMainAux}{%
  \ifx\scr@hook@bcma\@empty
    \gdef\scr@hook@bcma{%
      \renewcommand*{\BeforeClosingMainAux}[1]{####1}%
      \let\protected@write\protected@immediate@write
      \scr@redefine@newpage
    }%
  \fi
  \g@addto@macro\scr@hook@bcma%
}
\newcommand*{\AfterReadingMainAux}{%
  \ifx\scr@hook@acma\@empty
    \gdef\scr@hook@acma{%
      \renewcommand*{\AfterReadingMainAux}[1]{####1}%
      \let\protected@write\protected@immediate@write
    }%
  \fi
  \g@addto@macro\scr@hook@acma%
}
%    \end{macrocode}
% \end{macro}
% \end{macro}
%
% \begin{macro}{\protected@immediate@write}
% Like \LaTeX{} kernel's |\protected@write| but using |\immediate\write|. In
% this case it is even not a good idea to protect |\thepage|!
%    \begin{macrocode}
\long\def\protected@immediate@write#1#2#3{%
  \begingroup
    #2%
    \let\protect\@unexpandable@protect
    \edef\reserved@a{\immediate\write#1{#3}}%
    \reserved@a
  \endgroup
  \if@nobreak\ifvmode\nobreak\fi\fi
}
%    \end{macrocode}
% \end{macro}
%
% \Finale
%
% \iffalse
%</scrlfile>
%<*load>
% Used at \KOMAScript{} classes.
%    \begin{macrocode}
\RequirePackage{scrlfile}[%
%!KOMAScriptVersion
]
%    \end{macrocode}
%</load>
% \fi
%
\endinput
%
% end of file `scrlfile.dtx'

%%% Local Variables:
%%% mode: doctex
%%% TeX-master: t
%%% End:
