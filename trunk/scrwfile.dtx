% \CheckSum{306}
% \iffalse meta-comment
% ======================================================================
% scrwfile.dtx
% Copyright (c) Markus Kohm, 2010-2019
%
% This file is part of the LaTeX2e KOMA-Script bundle.
%
% This work may be distributed and/or modified under the conditions of
% the LaTeX Project Public License, version 1.3c of the license.
% The latest version of this license is in
%   http://www.latex-project.org/lppl.txt
% and version 1.3c or later is part of all distributions of LaTeX
% version 2005/12/01 and of this work.
%
% This work has the LPPL maintenance status "author-maintained".
%
% The Current Maintainer and author of this work is Markus Kohm.
%
% The KOMA-Script bundle consists of all files listed in manifest.txt.
% The work `scrwfile' consists of the files `scrwfile.dtx' and 
% `scrlogo.dtx'.
%
% If you're missing the installation batch file `scrwfile.ins', try
%    tex scrwfile.dtx
% to unpack all files. 
% The package documentation may be produced repeating
%    latex scrwfile.dtx
% at least three times.
% ----------------------------------------------------------------------
% scrwfile.dtx
% Copyright (c) Markus Kohm, 2010-2019
%
% Diese Datei ist Teil des LaTeX2e KOMA-Script-Pakets.
%
% Dieses Werk darf nach den Bedingungen der LaTeX Project Public Lizenz,
% Version 1.3c.
% Die neuste Version dieser Lizenz ist
%   http://www.latex-project.org/lppl.txt
% und Version 1.3c ist Teil aller Verteilungen von LaTeX
% Version 2005/12/01 und dieses Werks.
%
% Dieses Werk hat den LPPL-Verwaltungs-Status "author-maintained"
% (allein durch den Autor verwaltet).
%
% Der Aktuelle Verwalter und Autor dieses Werkes ist Markus Kohm.
%
% Das KOMA-Script-Paket besteht aus allen Dateien, die in manifest.txt
% genannt sind.
% Das Werk `scrwfile' besteht aus den Dateien `scrwfile.dtx' und
% `scrlogo.dtx'.
%
% Falls Sie die Installations-Batch-Datei `scrwfile.ins' vermissen,
% probieren Sie einfach einmal
%    tex scrwfile.dtx
% um alle Dateien auszupacken. Zur englischen Anleitung siehe den
% englischen Kommentar oben.
% ======================================================================
% \fi
%
% \CharacterTable
%  {Upper-case    \A\B\C\D\E\F\G\H\I\J\K\L\M\N\O\P\Q\R\S\T\U\V\W\X\Y\Z
%   Lower-case    \a\b\c\d\e\f\g\h\i\j\k\l\m\n\o\p\q\r\s\t\u\v\w\x\y\z
%   Digits        \0\1\2\3\4\5\6\7\8\9
%   Exclamation   \!     Double quote  \"     Hash (number) \#
%   Dollar        \$     Percent       \%     Ampersand     \&
%   Acute accent  \'     Left paren    \(     Right paren   \)
%   Asterisk      \*     Plus          \+     Comma         \,
%   Minus         \-     Point         \.     Solidus       \/
%   Colon         \:     Semicolon     \;     Less than     \<
%   Equals        \=     Greater than  \>     Question mark \?
%   Commercial at \@     Left bracket  \[     Backslash     \\
%   Right bracket \]     Circumflex    \^     Underscore    \_
%   Grave accent  \`     Left brace    \{     Vertical bar  \|
%   Right brace   \}     Tilde         \~}
%
% \iffalse
%%% From File: $Id$
%<*dtx>
\ifx\ProvidesFile\undefined\def\ProvidesFile#1[#2]{}\fi
\ProvidesFile{scrwfile.dtx}
%</dtx>
%<(package&identify)|driver>\NeedsTeXFormat{LaTeX2e}[1995/06/01]
%<package&identify>\ProvidesPackage{scrwfile}
%<driver>\ProvidesFile{scrwfile.drv}
%<*dtx|(package&identify)|driver>
  [2013/08/05 v0.1f-alpha KOMA-Script package (write and clone files)]
%</dtx|(package&identify)|driver>
%<*dtx>
\ifx\documentclass\undefined
  \input scrdocstrip.tex
  \@@input scrstrip.inc
  \KOMAdefVariable{COPYRIGHTFROM}{2010}

  \generate{\usepreamble\defaultpreamble
    \file{scrwfile.sty}{%
      \from{scrwfile.dtx}{package,trace,scrwfile,identify,option,body}%
      \from{scrlogo.dtx}{trace,logo}%
    }%
  }

  \@@input scrstrop.inc
\else
  \let\endbatchfile\relax
\fi
\endbatchfile
%</dtx>
%<*driver>
  \documentclass{scrdoc}
  \usepackage[ngerman,english]{babel}
  \usepackage{listings}
  \lstnewenvironment{lstcode}[1][]{%
    \lstset{language=[LaTeX]TeX,basicstyle=\ttfamily\small,#1}%
  }{}

  \CodelineIndex
  \RecordChanges
  \GetFileInfo{scrwfile.dtx}
  \title{The \KOMAScript{} package \texttt{scrwfile}%
    \footnote{This is version \fileversion\ of file \texttt{\filename}.}}
  \date{\filedate}
  \author{Markus Kohm}
  
  \newenvironment{Explain}{\par}{\par}
  \let\Macro\cs
  \let\Package\textsf
  \let\File\texttt
  \let\Option\texttt
  \let\Counter\texttt
  \let\ShowOutput\quote
  \let\endShowOutput\endquote
  \let\Parameter\marg
  \let\OParameter\oarg
  \providecommand\PParameter[1]{\mbox{\texttt{\{#1\}}}}
  \let\PName\meta
  \let\PValue\texttt
  \providecommand*{\important}[1]{}
  \providecommand*{\autoref}[1]{\expandafter\AUTOREF#1:}
  \providecommand*{\AUTOREF}{}
  \makeatletter
  \def\AUTOREF#1:#2:{%
    \edef\@tempa{#1}%
    \edef\@tempb{tab}\ifx\@tempa\@tempb table~\fi
    \edef\@tempb{sec}\ifx\@tempa\@tempb section~\fi
    \ref{#1:#2}%
  }
  \providecommand*{\IndexCmd}[2][]{}
  \providecommand*{\textnote}[2][]{}
  \makeatother

  \begin{document}
  \maketitle
  \DocInput{\filename}
  \end{document}
%</driver>
% \fi
%
% \selectlanguage{english}
%
% \changes{v0.1}{2010/10/01}{start of new package}
% \changes{v0.1g}{2014/10/13}{manual moved to \KOMAScript{} manual}
%
% \StopEventually{%
%   \clearpage
%   \PrintIndex\PrintChanges}
%
%
% \section{Implementation of \Package{scrwfile}}
%
% \iffalse
%<*package>
%<*identify>
% \fi
%
%    \begin{macrocode}
\PackageWarningNoLine{scrwfile}{%
  THIS IS AN ALPHA VERSION!\MessageBreak
  USAGE OF THIS VERSION IS ON YOUR OWN RISK!\MessageBreak
  EVERYTHING MAY HAPPEN!\MessageBreak
  EVERYTHING MAY CHANGE IN FUTURE!\MessageBreak
  THERE IS NO SUPPORT, IF YOU USE THIS PACKAGE!\MessageBreak
  Maybe, it would be better not to load this package.%
}
%    \end{macrocode}
%\iffalse
%</identify>
%\fi
%
% \subsection{Options}
% \iffalse
%<*option>
% \fi
%
% Currently we don't need options.
%
% \iffalse
%</option>
% \fi
%
% \subsection{Body}
% \iffalse
%<*body>
% \fi
%
% \subsubsection{Needed Packages}
%
% Package \textsf{scrbase} is needed, because of using several \KOMAScript{}
% basic commands.
%    \begin{macrocode}
\RequirePackage{scrbase}[2010/09/17]
%    \end{macrocode}
%
% Package \textsf{tocbasic} is needed for the lists of cloned TOCs.
%    \begin{macrocode}
\RequirePackage{tocbasic}[2010/10/01]
%    \end{macrocode}
%
% Package \textsf{scrlfile} is needed because of the \texttt{aux} file
% handling and \verb|\protected@immediate@write|.
%    \begin{macrocode}
\RequirePackage{scrlfile}[2010/09/30]
%    \end{macrocode}
%
% \subsubsection{\LaTeX{} Kernel Patchs}
%
% For some features we need to patch \LaTeX{} kernel macros. Those features
% and macros are:
% \begin{description}
% \item[Single handle feature] means, that \LaTeX{} will no longer need a file
%   handle for every help file, but only one for all files. We will patch
%  |\@starttoc| and |\@writefile| to do so.
% \item[Clone file feature] means, that every write to one file may be done to
%   another file, too. We will patch |\@writefile| to do so.
% \end{description}
% Every patch should be minimum invasive, so that files, that are not under
% \textsf{scrwfile}'s control are changed as little as possible.
%
% \begin{macro}{\scrwfile@if@only}
% First of all we check, if the file should be handled by \textsf{scrwfile}.
%    \begin{macrocode}
\newcommand*{\scrwfile@if@only}[1]{%
  \begingroup
    \scr@ifundefinedorrelax{scrwfile@only}{\@tempswatrue}{%
      \@tempswafalse
      \edef\reserved@b{#1}%
      \@for\reserved@a:=\scrwfile@only\do
        {\ifx\reserved@a\reserved@b\@tempswatrue\fi}%
    }%
    \if@tempswa
      \scr@ifundefinedorrelax{scrwfile@never}{}{%
        \edef\reserved@b{#1}%
        \@for\reserved@a:=\scrwfile@only\do
          {\ifx\reserved@a\reserved@b\@tempswafalse\fi}%
      }%
    \fi
  \expandafter\endgroup
  \if@tempswa
    \expandafter\@firstoftwo
  \else
    \expandafter\@secondoftwo
  \fi
}
%    \end{macrocode}
% \end{macro}
%
% \begin{macro}{\scrwfile@starttoc}
% \begin{macro}{\scrwfile@@starttoc}
% This is the internal redefinition of |\@starttoc|. First of all test, if it
% should be used, then use it or not.
%    \begin{macrocode}
\newcommand*{\scrwfile@starttoc}[1]{%
  \scrwfile@if@only{#1}{\scrwfile@@starttoc}{\scrwfile@saved@starttoc}{#1}%
}
\newcommand*{\scrwfile@@starttoc}[1]{%
%<trace>  \typeout{Use my own \string\@starttoc\space for #1}%
  \begingroup
    \if@filesw
      \xdef\scrwfile@writefilelist{\scrwfile@writefilelist,#1}%
    \fi
    \@fileswfalse
    \scrwfile@saved@starttoc{#1}%
  \endgroup
}
%    \end{macrocode}
% \end{macro}
% \end{macro}
%
% \begin{macro}{\scrwfile@writefile}
% \begin{macro}{\scrwfile@@writefile}
% \begin{macro}{\scrwfile@wrtout}
% \begin{macro}{\scrwfile@writefilelist}
% This is the internal redefinition of |\@writefile|. First of all test, if it
% should be used, then use it or not.
%    \begin{macrocode}
\newcommand*{\scrwfile@writefile}[1]{%
  \scrwfile@if@only{#1}{\scrwfile@@writefile}{\scrwfile@saved@writefile}{#1}%
}
\newcommand{\scrwfile@@writefile}[2]{%
%<trace>  \typeout{Use my own \string\@writefile\space for #1}%
  \ifnum\scrwfile@wrtout>0
    \begingroup
      \@temptokena{#2}%
      \immediate\write\scrwfile@wrtout{%
        \string\@writefile{#1}{\the\@temptokena}%
      }%
%    \end{macrocode}
% This was the entry for the real file. But we also may have clone files:
%    \begin{macrocode}
      \scrwfile@process@clones{#1}%
    \endgroup
  \fi
}
\chardef\scrwfile@wrtout\z@
\newcommand*{\scrwfile@writefilelist}{}
%    \end{macrocode}
% \end{macro}
% \end{macro}
% \end{macro}
% \end{macro}
%
% \begin{macro}{\@writefile}
% We have to add the single handle feature and the clone file feature to this.
% \begin{macro}{\scrwfile@saved@writefile}
% This is the original definition, that will be used, if the file is not under
% \texttt{scrwfile}'s control.
%    \begin{macrocode}
\newcommand*{\scrwfile@saved@writefile}{}
\BeforeClosingMainAux{%
  \ifx\scrwfile@writefilelist\@empty\else
    \let\scrwfile@saved@writefile\@writefile
    \let\scrwfile@wrtout\@partaux
    \immediate\openout\scrwfile@wrtout \jobname.wrt
    \let\@writefile\scrwfile@writefile
  \fi
}
\AfterReadingMainAux{%
  \ifx\scrwfile@writefilelist\@empty\else
    \immediate\closeout\scrwfile@wrtout
    \chardef\scrwfile@wrtout\z@
    \begingroup
      \let\@writefile\scrwfile@saved@writefile
      \@for\@currext:=\scrwfile@writefilelist\do{%
        \begingroup
          \ifx\@currext\@empty\else
            \scr@ifundefinedorrelax{tf@\@currext}{%
%<trace>              \typeout{Process extension: `\@currext'}
              \immediate\openout\@partaux \jobname.\@currext
              \expandafter\let\csname tf@\@currext\endcsname\@partaux
              \@input@{\jobname.wrt}%
              \immediate\closeout\@partaux
            }{}%
          \fi
        \endgroup
      }%
    \endgroup
  \fi
}
%    \end{macrocode}
% \end{macro}
% \end{macro}
%
% \begin{macro}{\@starttoc}
% We have to add the single handle feature to this.
% \begin{macro}{\scrwfile@saved@starttoc}
% \changes{v0.1e}{2013/07/24}{take care for unfriendly \textsl{hyperref}}
% This is the original definition, that will be used, if the file is not under
% \texttt{scrwfile}'s control. Because of package \textsl{hyperref}, that does
% a hard redefinition without any care for changed definitions, we have to
% take care for that unfriendly package and cannot do a simple
% \begin{verbatim}
% \let\scrwfile@saved@starttoc\@starttoc
% \let\@starttoc\scrwfile@starttoc
% \end{verbatim}
%    \begin{macrocode}
\newcommand*{\scrwfile@saved@starttoc}{}
\AtBeginDocument{%
  \begingroup
    \@ifpackageloaded{hyperref}{%
      \scr@ifundefinedorrelax{Hy@AtBeginDocument}{%
        \PackageWarning{scrwfile}{%
          Incompatible `hyperref` package detected!\MessageBreak
          You are using a `hyperref` version, that\MessageBreak
          doesn't provide `\string\Hy@AtBeginDocument`.\MessageBreak
          This may or may not work.\MessageBreak
          Using immediate redefinition%
        }%
        \aftergroup\@firstofone
      }{%
        \def\reserved@a{\AtBeginDocument}%
        \ifx\Hy@AtBeginDocumentHook\@undefined
          \PackageInfo{scrwfile}{%
            Using immediate redefinition%
          }%
          \aftergroup\@firstofone
        \else
          \PackageInfo{scrwfile}{%
            Using `\string\Hy@AtBeginDocument`%
          }%
          \aftergroup\Hy@AtBeginDocument
        \fi
      }%
    }{%
      \aftergroup\@firstofone
    }%
  \endgroup
  {%
    \PackageInfo{scrwfile}{%
      Extending `\string\@starttoc`
    }%
    \let\scrwfile@saved@starttoc\@starttoc
    \let\@starttoc\scrwfile@starttoc
  }%
}
%    \end{macrocode}
% \end{macro}
% \end{macro}
%
%
% \subsubsection{Clone TOC Feature}
%
% \textsf{scrwfile} may clone a TOC, that means, every entry to one file will
% be copied to other files, too.  You must not clone recursively!
%
% \begin{macro}{\scrwfile@process@clones}
%    \begin{macrocode}
\newcommand*{\scrwfile@process@clones}[1]{%
  \scr@ifundefinedorrelax{scrwfile@clone@#1}{}{%
    \begingroup
      \let\@@protect\protect\let\protect\@empty\afterassignment\restore@protect
      \edef\reserved@b{\csname scrwfile@clone@#1\endcsname}%
      \edef\reserved@c{,#1}%
      \@for \reserved@a:=\reserved@b\do {%
        \@tempswatrue
        \@for \reserved@d:=\reserved@c\do {%
          \ifx\reserved@d\reserved@a\@tempswafalse\fi
        }%
        \if@tempswa
%<trace>          \typeout{clone entry from `#1' to `\reserved@a'}%
          \immediate\write\scrwfile@wrtout{%
            \string\@writefile{\reserved@a}{\the\@temptokena}%
          }%
          \edef\reserved@c{\reserved@c,\reserved@a}%
        \fi
      }%
    \endgroup
  }%
}%
%    \end{macrocode}
% \end{macro}
%
% \begin{macro}{\TOCclone}
% Clone the entries from the second (first mandatory) argument TOC to the
% third (second mandatory) argument TOC.  If the first (optional) argument was
% given, define |\listof#3name| to this and also define |\listof#3| and clone
% the toc features \texttt{leveldown}, \texttt{numbered}, \texttt{onecolumn}
% and \texttt{totoc} of |#2| to |#3|. The toc feature \texttt{nobabel} will
% always be set, because the babel entries at TOC |#3| will be cloned from TOC
% |#2|. \textbf{Note:} We use owner \texttt{TOCclone} for all cloned
% extensions.
%    \begin{macrocode}
\newcommand*{\TOCclone}[3][]{%
  \scr@ifundefinedorrelax{scrwfile@clone@#2}{%
    \expandafter\protected@edef\csname scrwfile@clone@#2\endcsname{%
      #3,\protect\csname scrwfile@clone@#3\endcsname
    }%
  }{%
    \edef\reserved@b{\csname scrwfile@clone@#2\endcsname}%
    \expandafter\protected@edef\csname scrwfile@clone@#2\endcsname{%
      \csname scrwfile@clone@#2\endcsname,%
      #3,%
      \protect\csname scrwfile@clone@#3\endcsname
    }%
  }%
  \scr@ifundefinedorrelax{scrwfile@clone@#3}{%
    \expandafter\let\csname scrwfile@clone@#3\endcsname\@empty
  }{}%
  \ifattoclist{#3}{%
    \PackageWarning{scrwfile}{`#3' already under control of
      tocbasic.\MessageBreak
      Nevertheless features will be set
    }%
  }{%
    \addtotoclist[TOCclone]{#3}%
  }%
  \setuptoc{#3}{nobabel}%
  \ifstr{#1}{}{%
  }{%
    \@namedef{listof#3name}{#1}%
    \@namedef{listof#3}{\listoftoc{#3}}%
    \iftocfeature{#2}{leveldown}{\setuptoc{#3}{leveldown}}{}%
    \iftocfeature{#2}{numbered}{\setuptoc{#3}{numbered}}{}%
    \iftocfeature{#2}{onecolumn}{\setuptoc{#3}{leveldownonecolumn}}{}%  
    \iftocfeature{#2}{totoc}{\setuptoc{#3}{totoc}}{}%
  }%
}
%    \end{macrocode}
% \end{macro}
%
% \iffalse
%</body>
%</package>
% \fi
%
% \Finale
%
\endinput
%
% end of file `scrwfile.dtx'

%%% Local Variables:
%%% mode: doctex
%%% mode: flyspell
%%% ispell-local-dictionary: "english"
%%% TeX-master: t
%%% End:
