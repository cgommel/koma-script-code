% \CheckSum{118}
% \iffalse meta-comment
% ======================================================================
% scrkernel-tocstyle.dtx
% Copyright (c) Markus Kohm, 2002-2015
%
% This file is part of the LaTeX2e KOMA-Script bundle.
%
% This work may be distributed and/or modified under the conditions of
% the LaTeX Project Public License, version 1.3c of the license.
% The latest version of this license is in
%   http://www.latex-project.org/lppl.txt
% and version 1.3c or later is part of all distributions of LaTeX 
% version 2005/12/01 or later and of this work.
%
% This work has the LPPL maintenance status "author-maintained".
%
% The Current Maintainer and author of this work is Markus Kohm.
%
% This work consists of all files listed in manifest.txt.
% ----------------------------------------------------------------------
% scrkernel-tocstyle.dtx
% Copyright (c) Markus Kohm, 2002-2015
%
% Dieses Werk darf nach den Bedingungen der LaTeX Project Public Lizenz,
% Version 1.3c, verteilt und/oder veraendert werden.
% Die neuste Version dieser Lizenz ist
%   http://www.latex-project.org/lppl.txt
% und Version 1.3c ist Teil aller Verteilungen von LaTeX
% Version 2005/12/01 oder spaeter und dieses Werks.
%
% Dieses Werk hat den LPPL-Verwaltungs-Status "author-maintained"
% (allein durch den Autor verwaltet).
%
% Der Aktuelle Verwalter und Autor dieses Werkes ist Markus Kohm.
% 
% Dieses Werk besteht aus den in manifest.txt aufgefuehrten Dateien.
% ======================================================================
% \fi
%
% \CharacterTable
%  {Upper-case    \A\B\C\D\E\F\G\H\I\J\K\L\M\N\O\P\Q\R\S\T\U\V\W\X\Y\Z
%   Lower-case    \a\b\c\d\e\f\g\h\i\j\k\l\m\n\o\p\q\r\s\t\u\v\w\x\y\z
%   Digits        \0\1\2\3\4\5\6\7\8\9
%   Exclamation   \!     Double quote  \"     Hash (number) \#
%   Dollar        \$     Percent       \%     Ampersand     \&
%   Acute accent  \'     Left paren    \(     Right paren   \)
%   Asterisk      \*     Plus          \+     Comma         \,
%   Minus         \-     Point         \.     Solidus       \/
%   Colon         \:     Semicolon     \;     Less than     \<
%   Equals        \=     Greater than  \>     Question mark \?
%   Commercial at \@     Left bracket  \[     Backslash     \\
%   Right bracket \]     Circumflex    \^     Underscore    \_
%   Grave accent  \`     Left brace    \{     Vertical bar  \|
%   Right brace   \}     Tilde         \~}
%
% \iffalse
%%% From File: $Id: scrkernel-tocstyle.dtx 2254 2015-11-12 09:31:08Z kohm $
% The runs of run-time file generation:
%<preidentify>%%%            (run: preidentify)
% - everything that is needed before selfidentification of resulting files
%   (unused)
%<identify>%%%            (run: identify)
% - everything that is needed for selfidentification of resulting files
%   (unused)
%<prepare>%%%            (run: prepare)
% - everything that is needed to define options
%<option>%%%            (run: option)
% - definition of options
%<execoption>%%%            (run: execoption)
% - initial option executions up to \KOMAProcessOptions
%<body>%%%            (run: body)
% - everything, that should be done after \KOMAProcessOptions
%<exit>%%%            (run: exit)
% - everything, that should be done immediately before leaving the file
%<*driver>
% \fi
\ifx\ProvidesFile\undefined\def\ProvidesFile#1[#2]{}\fi
\begingroup
  \def\filedate$#1: #2-#3-#4 #5${\gdef\filedate{#2/#3/#4}}
  \filedate$Date: 2015-11-12 10:31:08 +0100 (Do, 12. Nov 2015) $
  \def\filerevision$#1: #2 ${\gdef\filerevision{#2}}
  \filerevision$Revision: 2254 $
  \edef\reserved@a{%
    \noexpand\endgroup
    \noexpand\ProvidesFile{scrkernel-tocstyle.dtx}%
                          [\filedate\space\filerevision\space
                           KOMA-Script (TOC styles)]
  }%
\reserved@a
% \iffalse
\documentclass[halfparskip-]{scrdoc}
\usepackage[english,ngerman]{babel}
\usepackage[latin1]{inputenc}
\CodelineIndex
\RecordChanges
\GetFileInfo{scrkernel-tocstyle.dtx}
\title{\KOMAScript{} \partname\ \texttt{\filename}%
  \footnote{Dies ist Version \fileversion\ von Datei
    \texttt{\filename}.}}
\date{\filedate}
\author{Markus Kohm}

\begin{document}
  \maketitle
  \tableofcontents
  \DocInput{\filename}
\end{document}
%</driver>
% \fi
%
% \selectlanguage{ngerman}
%
% \section{Verzeichnisstile}
%
% Verzeichnisstile werden derzeit nur in Kooperation mit Gliederungsstilen
% unters�tzt. In Zukunft soll das aber mal anders werden. Dann sollen
% Verzeichnisstile Teil der Grundfunktionalit�t f�r Verzeichnisse (Paket
% \textsf{tocbasic}) werden und auch unabh�ngig von Gliederungsstilen
% einsetzbar sein.
%
% \StopEventually{\PrintIndex\PrintChanges}
%
% \changes{v3.20}{2015/11/13}{neu und nur bei Vorhandensein verwendet}
%
%
% \subsection{Verzeichniseintragsstile}
%
% \emph{Hinweis: Dieser Code wird m�glicherweise sp�ter in \textsf{tocbasic}
% wandern. Das ist aber erst m�glich, wenn eine von \cs{DeclareSectionCommand}
% unabh�ngige M�glichkeit zur Deklaration von Verzeichniseintr�gen
% existiert. Das ist bisher nicht der Fall.}
%
% Ab \KOMAScript~3.20 halten Stile Einzug in die Verzeichnisse. Der erste
% Schritt dabei sind die Stile der Verzeichniseintr�ge. Jeder Stil besteht aus
% einer Anweisung, mit der ein Verzeichnisbefehl definiert wird und
% zugeh�rigen Optionen. Die Optionen dienen dazu, Einstellungen f�r die
% einzelnen Verzeichnisbefehle vorzunehmen.
%
% \begin{macro}{\DefineTOCEntryOption}
%   \changes{v3.20}{2015/11/10}{neu}
% Die Definition der Optionen geschieht mit einer einheitlichen
% Schnittstelle. Wichtig dabei ist, dass hier nicht tats�chlich die Optionen
% definiert werden, sondern eine Anweisung, die ihrerseits die Optionen bei
% Bedarf definiert. Ein solcher Bedarf besteht beispielsweise bei der
% Definition des Eintragsbefehls selbst. Er besteht aber auch innerhalb von
% \cs{DeclareSectionCommand}, weil diese Anweisung ebenfalls indirekt
% Verzeichniseintr�ge definiert und deshalb die Optionen auswerten k�nnen
% soll. Es gibt folgende Parameter:
% \begin{description}
% \item[\marg{Stil}: \meta{String},] Name des Eintragsstils
% \item[\marg{Option}: \meta{String},] Name der Option
% \item[\oarg{Wert}:] der S�umniswert der Option
% \item[\marg{Definition}:] Definition der Option
% \end{description}
% \begin{macro}{\@DefineTOCEntryOptionWithoutDefault}
%   \changes{v3.20}{2015/11/10}{neu (intern)}
% \begin{macro}{\@DefineTOCEntryOptionWithDefault}
%   \changes{v3.20}{2015/11/10}{neu (intern)}
% Da mit einem optionalen Argument an Position 3 gearbeitet wird, werden zwei
% Hilfsmakros ben�tigt.
%    \begin{macrocode}
%<*body>
\newcommand*{\DefineTOCEntryOption}[2]{%
  \scr@ifundefinedorrelax{scr@dte@#1@opts}{\@namedef{scr@dte@#1@opts}{}}{}%
  \@ifnextchar [%]
    {\@DefineTOCEntryOptionWithDefault{#1}{#2}}%
    {\@DefineTOCEntryOptionWithoutDefault{#1}{#2}}%
}
\newcommand*{\@DefineTOCEntryOptionWithoutDefault}[3]{%
  \DeclareSectionCommandDummyOption{#2}\relax
  \expandafter\l@addto@macro\csname scr@dte@#1@opts\endcsname{%
    \DefineFamilyKey[.dsc]{KOMAarg}{#2}{#3}%
  }%
}
\newcommand*{\@DefineTOCEntryOptionWithDefault}{}
\def\@DefineTOCEntryOptionWithDefault#1#2[#3]#4{%
  \DeclareSectionCommandDummyOption{#2}*%
  \expandafter\l@addto@macro\csname scr@dte@#1@opts\endcsname{%
    \DefineFamilyKey[.dsc]{KOMAarg}{#2}[{#3}]{#4}%
  }%
}
%</body>
%    \end{macrocode}
% \end{macro}^^A \@DefineTOCEntryOptionWithoutDefault
% \end{macro}^^A \@DefineTOCEntryOptionWithDefault
% \end{macro}^^A \DefineTOCEntryOption
%
% \begin{macro}{\DefineTOCEntryLengthOption}
%   \changes{v3.20}{2015/11/10}{neu}
% Verwendet \cs{DefineTOCEntryOption}, um eine Option zu definieren, die ein
% L�ngenmacro definiert. Es gibt folgende Parameter:
% \begin{description}
% \item[\marg{Stil}: \meta{String},] Name des Eintragsstils
% \item[\marg{Option}: \meta{String},] Name der Option
% \item[\oarg{Wert}:] der S�umniswert der Option
% \item[\marg{Prefix}: \meta{String},] Pr�fix des zu definierenden Makros
% \item[\marg{Postfix}: \meta{String},] Postfix des zu definierenden Makros
% \item[\marg{Erkl�rung}: \meta{String},] Erkl�rung der Option
% \end{description}
% \begin{macro}{\@DefineTOCEntryLengthOptionWithoutDefault}
%   \changes{v3.20}{2015/11/10}{neu (intern)}^^A
% \begin{macro}{\@DefineTOCEntryLengthOptionWithDefault}
%   \changes{v3.20}{2015/11/10}{neu (intern)}^^A
% \begin{macro}{\@DefineTOCEntryLengthOptionMacro}
%   \changes{v3.20}{2015/11/10}{neu (intern)}^^A
% Da mit einem optionalen Argument an Position 3 gearbeitet wird, werden drei
% Hilfsmakros ben�tigt.
%    \begin{macrocode}
%<*body>
\newcommand*{\DefineTOCEntryLengthOption}[2]{%
  \@ifnextchar [%]
    {\@DefineTOCEntryLengthOptionWithDefault{#1}{#2}}
    {\@DefineTOCEntryLengthOptionWithoutDefault{#1}{#2}}%
}
\newcommand*{\@DefineTOCEntryLengthOptionWithoutDefault}[5]{%
  \DefineTOCEntryOption{#1}{#2}{%
    \@DefineTOCEntryLengthOptionMacro{#2}{#3}{#4}{##1}%
  }%
  \TOCStyleNeedsCommandByOption{#1}{#3}{#4}{#2}{#5}%
}
\newcommand*{\@DefineTOCEntryLengthOptionWithDefault}{}
\def\@DefineTOCEntryLengthOptionWithDefault#1#2[#3]#4#5#6{%
  \DefineTOCEntryOption{#1}{#2}[{#3}]{%
    \@DefineTOCEntryLengthOptionMacro{#2}{#4}{#5}{##1}%
  }%
  \TOCStyleNeedsCommandByOption{#1}{#4}{#5}{#2}{#6}%
}
\newcommand*{\@DefineTOCEntryLengthOptionMacro}[4]{%
  \FamilySetLength{KOMAarg}{#1}{\@tempskipa}{#4}%
  \ifx\FamilyKeyState\FamilyKeyStateProcessed
    \ifscr@dsc@expandtopt
      \expandafter\edef\csname #2\scr@dsc@current#3\endcsname{%
        \the\@tempskipa}%
    \else
      \expandafter\edef\csname #2\scr@dsc@current#3\endcsname{#4}%
    \fi
  \fi
}
%</body>
%    \end{macrocode}
% \end{macro}^^A \@DefineTOCEntryLengthOptionMacro
% \end{macro}^^A \@DefineTOCEntryLengthOptionWithDefault
% \end{macro}^^A \@DefineTOCEntryLengthOptionWithoutDefault
% \end{macro}^^A \DefineTOCEntryLengthOption
%
% \begin{macro}{\DefineTOCEntryNumberOption}
%   \changes{v3.20}{2015/11/12}{neu}
% Verwendet \cs{DefineTOCEntryOption}, um eine Option zu definieren, die ein
% L�ngenmacro definiert. Es gibt folgende Parameter:
% \begin{description}
% \item[\marg{Stil}: \meta{String},] Name des Eintragsstils
% \item[\marg{Option}: \meta{String},] Name der Option
% \item[\oarg{Wert}:] der S�umniswert der Option
% \item[\marg{Prefix}: \meta{String},] Pr�fix des zu definierenden Makros
% \item[\marg{Postfix}: \meta{String},] Postfix des zu definierenden Makros
% \item[\marg{Erkl�rung}: \meta{String},] Erkl�rung der Option
% \end{description}
% \begin{macro}{\@DefineTOCEntryNumberOptionWithoutDefault}
%   \changes{v3.20}{2015/11/12}{neu (intern)}
% \begin{macro}{\@DefineTOCEntryNumberOptionWithDefault}
%   \changes{v3.20}{2015/11/12}{neu (intern)}
% \begin{macro}{\@DefineTOCEntryNumberOptionMacro}
%   \changes{v3.20}{2015/11/12}{neu (intern)}
% Da mit einem optionalen Argument an Position 3 gearbeitet wird, werden drei
% Hilfsmakros ben�tigt.
%    \begin{macrocode}
%<*body>
\newcommand*{\DefineTOCEntryNumberOption}[2]{%
  \@ifnextchar [%]
    {\@DefineTOCEntryNumberOptionWithDefault{#1}{#2}}
    {\@DefineTOCEntryNumberOptionWithoutDefault{#1}{#2}}%
}
\newcommand*{\@DefineTOCEntryNumberOptionWithoutDefault}[5]{%
  \DefineTOCEntryOption{#1}{#2}{%
    \@DefineTOCEntryNumberOptionMacro{#2}{#3}{#4}{##1}%
  }%
  \TOCStyleNeedsCommandByOption{#1}{#3}{#4}{#2}{#5}%
}
\newcommand*{\@DefineTOCEntryNumberOptionWithDefault}{}
\def\@DefineTOCEntryNumberOptionWithDefault#1#2[#3]#4#5#6{%
  \DefineTOCEntryOption{#1}{#2}[{#3}]{%
    \@DefineTOCEntryNumberOptionMacro{#2}{#4}{#5}{##1}%
  }%
  \TOCStyleNeedsCommandByOption{#1}{#4}{#5}{#2}{#6}%
}
\newcommand*{\@DefineTOCEntryNumberOptionMacro}[4]{%
  \FamilySetCounter{KOMAarg}{#1}{\@tempcnta}{#4}%
  \ifx\FamilyKeyState\FamilyKeyStateProcessed
    \ifscr@dsc@expandtopt
      \expandafter\edef\csname #2\scr@dsc@current#3\endcsname{%
        \the\@tempcnta}%
    \else
      \expandafter\edef\csname #2\scr@dsc@current#3\endcsname{#4}%
    \fi
  \fi
}
%</body>
%    \end{macrocode}
% \end{macro}^^A \@DefineTOCEntryNumberOptionMacro
% \end{macro}^^A \@DefineTOCEntryNumberOptionWithDefault
% \end{macro}^^A \@DefineTOCEntryNumberOptionWithoutDefault
% \end{macro}^^A \DefineTOCEntryNumberOption
%
% \begin{macro}{\TOCStyleNeedsCommandByOption}
%   \changes{v3.20}{2015/11/12}{neu}
% F�gt einen oder mehrere Befehle der Liste der ben�tigten Befehle hinzu.
% \begin{description}
% \item[\marg{Stil}:] Der Stil \meta{string}, der die Befehle ben�tigt.
% \item[\marg{Pr�fix}:] Pr�fix \meta{string} des ben�tigten Befehls.
% \item[\marg{Postfix}:] Postfix \meta{string} des ben�tigten Befehls.
% \item[\marg{Option}:] Die Option \meta{string}, �ber die dieser Befehl
%   definiert wird.
% \item[\marg{Erkl�rung}:] Bedeutung \meta{string} des Befehls.
% \end{description}
% Zwischen Pr�fix und Postfix wird dann automatisch der Name der
% Gliederungsebene eingef�gt.
%    \begin{macrocode}
%<*body>
\newcommand*{\TOCStyleNeedsCommandByOption}[5]{%
  \expandafter\providecommand\expandafter*\csname scr@dte@#1@needs\endcsname{}%
  \expandafter\l@addto@macro\csname scr@dte@#1@needs\endcsname{%
    \do{#2}{#3}{#5}{#4}%
  }%
}
%</body>
%    \end{macrocode}
% \end{macro}^^A \TOCStyleNeedsCommandByOption
%
% TODO: Bisher k�nnen nur einige Optionen deklariert werden, die jedoch f�r
% noch gar nichts verwendet werden.
%
% \Finale
%
\endinput
%
% end of file `scrkernel-tocstyle.dtx'
%%% Local Variables:
%%% mode: doctex
%%% coding: iso-latin-1
%%% TeX-master: t
%%% End:
