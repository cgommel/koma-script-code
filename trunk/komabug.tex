% ======================================================================
% komabug.tex
% Copyright (c) Markus Kohm, 1995-2009
%
% This file is part of the LaTeX2e KOMA-Script bundle.
%
% This work may be distributed and/or modified under the conditions of
% the LaTeX Project Public License, version 1.3c of the license.
% The latest version of this license is in
%   http://www.latex-project.org/lppl.txt
% and version 1.3c or later is part of all distributions of LaTeX
% version 2005/12/01 or later and of this work.
%
% This work has the LPPL maintenance status "author-maintained".
%
% The Current Maintainer and author of this work is Markus Kohm.
%
% This work consists of all files listed in manifest.txt.
% ----------------------------------------------------------------------
% komabug.tex
% Copyright (c) Markus Kohm, 1995-2009
%
% Dieses Werk darf nach den Bedingungen der LaTeX Project Public Lizenz,
% Version 1.3c, verteilt und/oder veraendert werden.
% Die neuste Version dieser Lizenz ist
%   http://www.latex-project.org/lppl.txt
% und Version 1.3c ist Teil aller Verteilungen von LaTeX
% Version 2005/12/01 oder spaeter und dieses Werks.
%
%
% Dieses Werk hat den LPPL-Verwaltungs-Status "author-maintained"
% (allein durch den Autor verwaltet).
%
% Der Aktuelle Verwalter und Autor dieses Werkes ist Markus Kohm.
%
% Dieses Werk besteht aus den in manifest.txt aufgefuehrten Dateien.
% ======================================================================
%%
%% @ is a letter
%%
\catcode`\@=11

%%
%% Grab the initex file list
%%
%% If this file is called via
%%     latex "\input{latexbug}" or some
%% similar command sequence rather than
%%     latex latexbug
%% then the debugging info in \reserved@a will already have been lost.
%% This might not matter, but if it does we may ask the user to resubmit
%% the report.
\ifx\reserved@b\@undefined
  \ifx\reserved@a\@gobble
    \def\@inputfiles{NONE}
  \else
    \let\@inputfiles\reserved@a
  \fi
\else
  \def\@inputfiles{LOST}
\fi

%%
%% Output stream to produce the bug report template.
%%
\newif\ifinteractive\interactivetrue
\newwrite\@msg
\immediate\openout\@msg=\jobname.msg
\immediate\write\@msg{%
An fatal error occured before writing anything to the message file.^^J%
You should have a look at \jobname.log to see the reason of the error.^^J%
^^J%
Maybe you didn't use an interactive terminal to run ``latex komabug''.^^J%
At this case you should try an interactive terminal or add^^J%
\space\space\space\space\string\interactivefalse^^J%
to your local ``komabug.cnf'' to create an empty message.^^J%
You have to edit that message file after creating the empty message.^^J%
}
\immediate\closeout\@msg\let\msg\m@ne

%%
%% We have no end line char (so we have no paragraphs)
%%
\endlinechar=-1

%%
%% Check that LaTeX2e is being used.
%%
\ifx\undefined\newcommand
 \newlinechar`\^^J%
 \immediate\write17{^^J%
    You must use LaTeX2e to generate the bug report!^^J^^J%
    Sie muessen LaTeX2e verwenden, um die Fehlermeldung zu erzeugen!}%

 \let\relax\end
\else
 \def\@tempa{LaTeX2e}\ifx\@tempa\fmtname\else
  \immediate\write17{^^J%
   You must use LaTeX2e to generate the bug report!^^J^^J%
   Aeltere Versionen von LaTeX werden nicht unterstuetzt.^^J%
   Sie muessen LaTeX2e verwenden, um die Fehlermeldung zu erzeugen!}%
  \let\relax\@@end
\fi\fi

%%
%% \wmsg writes to the terminal, and the .msg file
%% \wmsg* just writes to the .msg file
%% \typeout just writes to the terminal
%%

\def\wmsg{%
  \ifnum\msg<0\relax\let\msg\@msg\immediate\openout\msg=\jobname.msg\fi
  \begingroup
    \@ifstar {\interactivefalse\@wmsg}{\@wmsg}
}

\def\@wmsg#1{%
    \ifinteractive\immediate\write17{#1}\fi%
    \immediate\write\msg{#1}%
  \endgroup
}

%%
%% Prompt for an answer from the user, if the answer is not
%% provided by the cfg file.
%%

\def\readifnotknown#1{%
 \@ifundefined{#1}%
    {{\message{#1> }%
     \catcode`\^^I=12 \let\do\@makeother\dospecials
     \global\read\m@ne t\expandafter o\csname#1\endcsname}}%
    {\message{\csname#1\endcsname}}}

%%
%% Get number
%% #1 = message
%% #2 = max value
%% --> \answer
%%
\def\scannumber#1{\@tempswatrue
  \typeout{`#1'}%
  \def\@tempa{\expandafter\@scannumber\expandafter0#1\@nil}%
  \edef\@tempa{\@tempa}%
  \count@\@tempa\relax
  \if@tempswa\else\count@\z@\relax\fi
}%
\def\@scannumber#1{%
  \ifx\@nil#1\else
    \if 0#1#1\else
      \ifnum 0<0#1 #1\else
        \noexpand\@tempswafalse%
      \fi
    \fi
    \expandafter\@scannumber
  \fi
}%
\def\ReadNumber#1#2{%
  \typeout{#1}%
  \ifenglish
    \message{Input the corresponding number between 1 and #2:  }%
  \else
    \message{Geben Sie die betreffende Zahl zwischen 1 und #2 ein:  }%
  \fi
  \count@=\z@\relax
  \read\m@ne to \answer
  \scannumber{\answer}%
  \ifnum\count@>\number#2\relax
    \count@=\z@\relax
  \fi
%  \typeout{\the\count@ > 0?}%
  \ifnum\count@>\z@\relax
%    \typeout{YES}%
    \advance\count@ by \m@ne
    \edef\answer{\the\count@}%
    \message{^^J}%
  \else
%    \typeout{NO}%
    \ifenglish
      \typeout{Value not valid!}%
    \else
      \typeout{Wert nicht im erlaubten Bereich!}%
    \fi
    \pause
    \def\@tempa{\ReadNumber{#1}{#2}}%
    \expandafter\@tempa
  \fi
}

%%
%% Get Yes or No
%% #1 Question
%% --> \if@tempswa (yes is true, no is false)
%%
\def\ReadYesNo#1{%
  \typeout{#1}%
  \ifenglish
    \message{Answer `yes' or `no':  }%
    \read\m@ne to \answer
  \else
    \message{Antworten Sie bitte mit `ja' oder `nein': }%
    \read\m@ne to \answer
    \edef\answer{\uccode`\expandafter\@car\answer\@nil}
    \ifnum \answer=`J \def\answer{\uccode`Y}\fi
  \fi
  \def\@tempa{\message{^^J}}%
  \ifnum \answer=`Y \@tempswatrue
  \else
    \ifnum \answer=`N \@tempswafalse
    \else
      \ifenglish
        \typeout{Answer not valid!}%
      \else
        \typeout{Antwort unverstaendlich!}%
      \fi
      \def\@tempa{\ReadYesNo{#1}}%
    \fi
  \fi
  \@tempa
}

%%
%% Pause so messages do not scroll off screen.
%%
\def\pause{%
  \ifinteractive
    \ifenglish
      \message{Press <return> to continue. }%
    \else
      \message{Mit der <Return>-Taste geht es weiter. }%
    \fi
    \read\m@ne to \@tempa
    \message{^^J}%
  \fi
}

%%
%% german or english report generator
%%
\newif\ifenglish\englishfalse

%%
%% Opening Banner.
%%

\InputIfFileExists{komabug.cfg}{%
  \ifenglish
    \typeout{** using komabug.cfg **}%
  \else
    \typeout{** komabug.cfg wird verwendet **}%
  \fi
}{}

\ifenglish
  \typeout{^^J%
    ============================================================^^J%
    ^^J%
    KOMA bug report generator^^J%
    =========================^^J%
    Running this file through LaTeX generates a formular ``\jobname.msg''^^J%
    containing a bug report to KOMA-Script bundle.^^J^^J%
    * Please use german, if possible.^^J \space
      If you're not able to use german, write the report in english.^^J%
    * Please write a short report not a large one.^^J%
    * Please tell me everything, which may be important.^^J}
  \pause
\else
  \typeout{^^J%
    ============================================================^^J%
    ^^J%
    KOMA-Script Fehlermeldungsgenerator^^J%
    ===================================^^J%
    Die Bearbeitung dieser Datei mit LaTeX erzeugt das Formular \jobname.msg,^^J%
    um Fehlermeldungen zum KOMA-Script-Paket zu melden.^^J^^J%
    * Schreiben Sie Ihre Meldung nach Moeglichkeit in Deutsch.^^J \space
      Notfalls ist auch Englisch moeglich.^^J%
    * Bitte fassen Sie sich kurz.^^J%
    * Bitte halten Sie keine Information zurueck, die moeglicherweise^^J \space
    wichtig sein koennte.^^J}%
  \typeout{%
    If you prefere english, you may write a file ``komabug.cfg'' with\space 
    contents:^^J\space
    \string\englishtrue^^J%
    at same folder as ``komabug.tex'' and restart ``latex komabug.tex''.^^J}%
  \englishtrue\pause\englishfalse
\fi

%%
%% if \interactivefalse just make a blank template.
%%

\ifinteractive
  \ifenglish
    \ReadYesNo{%
      There are two kinds of using this generator.^^J%
      At the interactive mode, you have to answer questions. At the other^^J%
      mode an empty formular will be generated, you have to fill using^^J%
      an editor.^^J^^J%
      Do you want an interactive session?
    }%
  \else
    \ReadYesNo{%
      Dieser Generator kann auf zwei Arten verwendet werden.^^J%
      Im interaktiven Betrieb, werden Ihnen Fragen zur direkten
      Beantwortung^^J%
      gestellt. Ansonsten wird ein leeres Formular erzeugt, das Sie dann
      mit^^J% 
      einem Editor ausfuellen muessen.^^J^^J%
      Wollen Sie eine interaktive Sitzung?
    }%
  \fi
  \if@tempswa\interactivetrue\else\interactivefalse\fi
\fi

%%
%% Fatal error
%%
\def\fatalerror#1{%
  \ifenglish
    \errhelp{This error is fatal! You cannot continue.^^J%
      You should terminate using `x' and restart ``latex komabug''.}%
    \errmessage{#1}%
  \else
    \errhelp{Dieser Fehler erlaubt keine Fortsetzung der Bearbeitung.^^J%
      Sie sollten mit `x' abbrechen und ``latex komabug'' neu starten.}%
    \errmessage{#1}%
  \fi
  \batchmode\csname @@end\endcsname
  \fatalerror{#1}%
}

%%
%% Try to get Version
%%
\def\GetFileVersionWithExtend#1#2{%
  \def\ProvidesFile##1{\@ifnextchar [{\@P@F}{\@P@F[1996/10/31 ]}}%
  \def\@P@F[##1 ##2]{\xdef\komaversion{##1}\csname endinput\endcsname}%
  \let\ProvidesClass\ProvidesFile
  \let\ProvidesPackage\ProvidesFile
  \InputIfFileExists{#1#2}{}{%
    \ifenglish
      \def\noscrclass{%
        ! File ``#1#2'' not found!^^J%
        ! The file must be at the same folder like ``komabug.tex''^^J%
        ! or must be readable by LaTeX to get the version information!}%
    \else
      \def\noscrclass{%
        ! Die Datei ``#1#2'' konnte nicht gefunden werden!^^J%
        ! Zur Bestimmung der aktuellen Version ist es unbedingt^^J%
        ! erforderlich, dass diese Datei sich im selben Verzeichnis^^J%
        ! wie ``komabug.tex'' befindet oder zumindest von LaTeX^^J%
        ! gefunden werden kann!}%
    \fi
    \errmessage{\noscrclass}
    \ifenglish
      \errhelp{Terminate TeX using `x' and restart komabug}
    \else
      \errhelp{Beenden Sie TeX mit `x' und starten Sie komabug neu}
    \fi
  }%
}
\def\GetVersion#1{%
  \GetFileVersionWithExtend{#1}\@empty
}

\ifinteractive
  \ifenglish
    \ReadNumber{%
      There are several categories, related to several parts and files^^J%
      of the KOMA-Script bundle:^^J^^J
      1) Installation of KOMA-Script^^J
      2) KOMA-Script manual^^J
      3) Basics (scrbook, scrreprt, scrartcl, scrlttr2, typearea, scrlfile)^^J
      4) Pagestyle definition (scrpage2)^^J
      5) Time or date (scrtime, scrdate)^^J
      6) Address file handling (scraddr)^^J
      7) Obsolete (scrlettr, scrpage)^^J%
    }{7}%
  \else
    \ReadNumber{%
      Verschiedene Bereiche werden von diesem Generator unterstuetzt, die^^J%
      sich auf verschiedene Dateien im KOMA-Script-Paket beziehen:^^J^^J
      1) Auspacken und Installieren von KOMA-Script^^J
      2) KOMA-Script-Anleitung^^J
      3) Grundfunktion (scrbook, scrreprt, scrartcl, scrlttr2, typearea,
      scrlfile)^^J
      4) Definition von Seitenstilen (scrpage2)^^J
      5) Zeit oder Datum (scrtime, scrdate)^^J
      6) Umgang mit Adressdateien (scraddr)^^J
      7) Obsoletes (scrlettr, scrpage)^^J%
    }{7}%
  \fi

  \ifcase\expandafter\number\answer
    % 1
    \def\category{Installation}\GetVersion{scrkernel-version.dtx}
    \let\categoryversion\komaversion
  \or
    % 2
    \def\category{Manual}\GetVersion{scrartcl.cls}
    \let\categoryversion\komaversion
  \or
    % 3
    \def\category{Basics}\GetVersion{scrartcl.cls}
    \let\categoryversion\komaversion
  \or
    % 4
    \def\category{Addons, scrpage2}\GetVersion{scrpage2.sty}%
    \let\categoryversion\komaversion\GetVersion{scrartcl.cls}
  \or
    % 5
    \def\category{Addons, scrtime/date}\GetVersion{scrtime.sty}%
    \let\categoryversion\komaversion\GetVersion{scrartcl.cls}
  \or
    % 6
    \def\category{Addons, scraddr}\GetVersion{scraddr.sty}%
    \let\categoryversion\komaversion\GetVersion{scrartcl.cls}
  \or
    % 7
    \ifenglish
      \typeout{These obsolete parts of KOMA-Script are unsupported!^^J%
        You should use scrlttr2 instead of scrlettr and scrpage2 instead of
        scrpage.}
    \else
      \typeout{Fuer diese obsoleten Teile von KOMA-Script gibt es keinen
        Support mehr!^^J%
        Sie sollten scrlttr2 an Stelle von scrlettr bzw. scrpage2 an Stelle
        von^^J%
        scrpage verwenden.}
    \fi
    \batchmode\csname @@end\endcsname
  \fi
\else% \ifinteractive
  \ifenglish
    \def\category{<PLEASE REPLACE BY ONE OF `Installation', `Manual',
      `Basics', `Addons, scrpage2', `Addons, scrtime/date', `Addons,
      scraddr'.>}
    \def\categoryversion{<PLEASE REPLACE BY VERSION DATE FROM USED CLASS OR
      PACKAGE.>}
  \else
    \def\category{<BITTE DURCH EINE DER ANGABEN `Installation', `Manual',
      `Basics', `Addons, scrpage2', `Addons, scrtime/date', `Addons,
      scraddr' ERSETZEN.>}
    \def\categoryversion{<BITTE DURCH DAS VERSIONSDATUM AUS DER ENTSPRECHENDEN
      KLASSE BZW. DEM ENTSPRECHENDEN PAKET ERSETZEN.>}
  \fi
  \GetVersion{scrartcl.cls}
\fi

\ifinteractive
  \ifenglish
    \typeout{^^J%
      ===========================================================^^J%
      ^^J%
      Please give a one line description (< 50 chars) of your problem.%
      ^^J^^J%
      If your using email for sending the report, please use this^^J%
      description as subject, too:%
      ^^J \@spaces\@spaces\space
      |<------------------------------------------------>|}
  \else
    \typeout{^^J%
      ============================================================^^J%
      ^^J%
      Bitte eine einzeilige (< 50 Zeichen) Beschreibung des Problems.%
      ^^J^^J%
      Wenn Sie fuer die Meldung eMail verwenden, setzen Sie diese
      Beschreibung^^J%
      bitte auch als Betreff (`Subject') der eMail ein:%
      ^^J \@spaces\@spaces\space
      |<------------------------------------------------>|}
  \fi
  \loop
    \let\synopsis\relax
    \readifnotknown{synopsis}
    \ifx\synopsis\@empty
  \repeat
\else%\ifinteractive
  \ifenglish
    \def\synopsis{<PLEASE REPLACE BY SHORT ONE-LINE DESCRIPTION.>}
  \else
    \def\synopsis{<BITTE DURCH EINE KURZE, EINZEILIGE BESCHREIBUNG ERSETZEN.>}
  \fi
\fi

%%
%% Header in the msg file.
%%
\ifenglish
  \wmsg*{^^J%
    KOMA bug report.^^J%
    \ifinteractive Interactive \else Formular \fi
    generated using komabug.tex at
    \space\number\year-\two@digits\month-\two@digits\day.^^J%
    ^^J
    You may send this message to komascript@gmx.info.^^J%
    If you do so, please use subject:^^J%
    \space KOMA-BUG:\space\synopsis^^J%
    ============================================================^^J
  }
\else
  \wmsg*{^^J%
    KOMA-Fehlermeldung.^^J%
    \ifinteractive Interaktiv \else Formular \fi
    erzeugt mit komabug.tex am
    \space\number\year-\two@digits\month-\two@digits\day.^^J%
    ^^J%
    Die Meldung kann per E-Mail an komascript@gmx.info^^J%
    verschickt werden.^^J%
    Bitte verwenden Sie dabei als Betreff:^^J%
    \space KOMA-BUG:\space\synopsis^^J%
    ============================================================^^J
  }
\fi

%%
%% Category of bug, obtained earlier but put out now, after the header.
%%
\wmsg{>Bereich: \category}
\wmsg{>Version: \categoryversion}

%%
%% synopsis of bug, obtained earlier but put out now, after the header.
%%
\wmsg{>Betreff: \synopsis}

\begingroup
 \global\let\format\@empty
 \gdef\hyphenation{standard}
 \def\immediate#1#{\xdef\hyphenation}
 \def\typeout#1{\xdef\format{\format#1}}
 \the\everyjob
\endgroup

\wmsg{>Format: \format}

\wmsg{>KOMA-Script: \komaversion}

\ifinteractive
%%
%% if interactive, \wread reads a line (verbatim) and write it to the
%% .msg file, until a blank line is entered.
%%
  \def\wread{{%
      \catcode`\^^I=12
      \let\do\@makeother\dospecials
      \let\lastanswer\answer
      \message{=> }\read\m@ne to \answer
      \ifx\lastanswer\@empty
        \let\lastanswer\answer
      \fi
      \ifx\lastanswer\@empty
      \else
        \immediate\write\msg{\answer}
        \expandafter\wread
      \fi
    }%
  }
\else
%%
%% If non-interactive, \wread just writes a blank line to the .msg file,
%% and \wmsg does not write to the terminal.
%%
  \def\wread{\wmsg{}}
\fi

%%
%% \copytomsg copies the contents of a file into the .msg file.
%% (at least it does it as well as TeX can, so there may be
%% transcription problems with 8-bit characters).
%%
%% It does a line count, and complains if the test file is
%% too large.

\chardef\inputfile=15

\newcount\linecount

\def\copytomsg#1{{%
    \endlinechar=-1
    \def\do##1{\catcode`##1=11}%
    \dospecials
    \global\linecount\z@
    \openin\inputfile#1\relax
    \def\thefile{#1}%
    \@copytomsg
    \closein\inputfile}}

\def\@copytomsg{%
   \ifeof\inputfile
      \typeout{*** \thefile\space Zeilen = \the\linecount}
   \else
      \global\advance\linecount\@ne
      \read\inputfile to \inputline
      \ifx\inputline\@empty
         \wmsg*{}
      \else
         \wmsg*{\inputline}
      \fi
      \expandafter\@copytomsg
   \fi}


%%
%% Test the age of the current format.
%%
\def\getage#1/#2/#3\@nil{%
  \count@\year
  \advance\count@-#1\relax
  \multiply\count@ by 12\relax
  \advance\count@\month
  \advance\count@-#2\relax}
%
\expandafter\getage\fmtversion\@nil
%%
%% \count@ should now be the age of the format in months.
%%
\ifnum\count@>24
  \ifenglish
    \def\oldformat{^^J%
      ! Your LaTeX installation is older than two years.^^J%
      ! Updating woul dbe a good idea before sending this report.^^J%
      ! You should compare the date of the package with the date of^^J
      ! the used LaTeX version. If LaTeX is more than two years older^^J
      ! than KOMA-Script, this could be the reason of the problem.}
    \errhelp{If you want to continue, press <return>.}
  \else
    \def\oldformat{^^J%
      ! Ihre LaTeX-Installation ist ueber zwei Jahre alt.^^J%
      ! Bitte denken Sie ueber ein Update nach, ehe Sie diese Meldung^^J%
      ! abschicken.^^J%
      ! Vergleichen Sie wenigstens das Datum des Paketes mit dem Datum^^J%
      ! der LaTeX-Version. LaTeX sollte nicht mehr als zwei Jahre aelter^^J%
      ! als KOMA-Script sein. Ansonsten koennte der Fehler in einer^^J%
      ! Unvertraeglichkeit zwischen Format- und Paketversion liegen.}
%%
%% Put the message in a macro to improve the look of the error mesage.
%%
%
    \errhelp{Wenn Sie dennoch fortfahren wollen, druecken Sie einfach <Return>.}
  \fi
  \errmessage{\oldformat}
\fi
%
\expandafter\getage\komaversion\@nil
\ifnum\count@>18
  \ifenglish
    \def\oldversion{^^J%
      ! Your KOMA-Script installation is older than one year.^^J%
      ! You should check for an update, bevor you're sending this message.^^J
      ! You should compare the date of the package with the date of^^J
      ! the used LaTeX version. If LaTeX is years younger than KOMA-Script,^^J
      ! this could be the reason of the problem.}
    \errhelp{If you want to continue, press <return>.}
  \else
    \def\oldversion{^^J%
      ! Ihre KOMA-Script-Version ist ueber ein Jahr alt.^^J%
      ! Bitte denken Sie ueber ein Update nach, ehe Sie diese Meldung^^J%
      ! abschicken.^^J%
      ! Vergleichen Sie wenigstens das Datum des Paketes mit dem Datum^^J%
      ! der LaTeX-Version. LaTeX sollte nicht Jahre juenger als KOMA-Script sein.^^J%
      ! Ansonsten koennte der Fehler in einer Unvertraeglichkeit zwischen^^J%
      ! Format- und Paketversion liegen.}
    \errhelp{Wenn Sie dennoch fortfahren wollen, druecken Sie einfach <Return>.}
  \fi
  \errmessage{\oldversion} 
\fi


%%
%% Now use \wmsg and \wread for each of the multi-line fields
%% in the form. Currently one-line fields use \read directly.
%%
\ifinteractive
  \ifenglish
    \typeout{^^JYour name:}
  \else
    \typeout{^^JIhr Name:}
  \fi
  \readifnotknown{name}
\else
  \ifx\name\undefined
    \ifenglish
      \def\name{<REPLACE THIS BY YOUR NAME>}
    \else
      \def\name{<GEBEN SIE IHREN NAMEN EIN>}
    \fi
  \fi
\fi


\ifinteractive
  \ifenglish
    \typeout{^^JYour address (email if possible):}
  \else
    \typeout{^^JIhre Adresse (eMail bevorzugt):}
  \fi
  \readifnotknown{address}
\else
  \ifx\address\undefined
    \ifenglish
      \def\address{<PEPLACE THIS BE YOUR (EMAIL-)ADDRESS>}
    \else
      \def\address{<GEBEN SIE IHRE (EMAIL-)ADRESSE EIN>}
    \fi
  \fi
\fi


\ifinteractive
  \ifenglish
    \typeout{^^JYour computer system (z. B. Atari, Linux, Mac, Win98SE):}
  \else
    \typeout{^^JDas verwendete Computersystem (z. B. Atari, Linux, Mac, Win98SE):}
  \fi
  \readifnotknown{computersystem}
\else
  \ifx\computersystem\undefined
    \ifenglish
      \def\computersystem{<REPLACE THIS BY YOUR COMPUTERSYSTEM>}
    \else
      \def\computersystem{<GEBEN SIE IHR VERWENDETES COMPUTERSYSTEM EIN>}
    \fi
  \fi
\fi


\wmsg*{>Adresse: \name\space<\address>}

%%
%% >Organisation: is really a GNATS multiline field
%% but we treat it as a one-line field.
%%
\wmsg*{>Organisation: \ifx\organisation\undefined
                        \ifx\organization\undefined\else
                           \organization
                        \fi
                       \else
                         \organisation
                       \fi}


%%
%% Test which format is being used. These fields are completed
%% automatically even if the blank template is being produced.
%%

\wmsg*{>Voraussetzungen:}
\wmsg*{ \string\@TeXversion: \meaning\@TeXversion
        \ifx\@TeXversion\@@undefined
         \space (Standardeinstellung fuer TeX3.141 und spaeter)\fi}
\wmsg*{ \string\@currdir: \meaning\@currdir}
\wmsg*{ \string\input@path: \meaning\input@path
        \ifx\input@path\@@undefined
         \space (Standardeinstellung)\fi}
\wmsg*{ System: \computersystem}

\wmsg*{>Beschreibung:}
\ifinteractive
  \ifenglish
    \typeout{%
      Description of your problem:^^J^^J%
      \@spaces You can use multiple lines (each is askes by the prompt^^J%
      \@spaces ``=>'').^^J%
      \@spaces Use two blank lines to finish your answer.}
  \else
    \typeout{%
      Beschreibung des Problems:^^J^^J%
      \@spaces Die Beantwortung dieser Frage kann mehrere Zeilen^^J%
      \@spaces einnehmen (jede wird durch die Eingabeaufforderung^^J%
      \@spaces ``=>'' eingeleitet).^^J%
      \@spaces Durch zwei aufeinanderfolgende Leerzeilen wird die^^J%
      \@spaces Antwort beenden.}
  \fi
\else
  \ifenglish
    \wmsg{<REPLACE THIS BY YOUR DESCRIPTION OF THE PROBLEM>}
  \else
    \wmsg{<GEBEN SIE HIER IHRE PROBLEMBESCHREIBUNG EIN>}
  \fi
\fi
\wread


%%
%% insertion of the test file
%%
\ifinteractive
  \ifenglish
    \typeout{^^J%
      Name of a short self describing file, which shows the problem^^J%
      (file should be as short as possible, not more than 60 lines):^^J^^J%
      If the file is not at current directory please enter the hole^^J%
      name (directory inclusive), so LaTeX may load it.^^J^^J%
      If no testfile exists, because your not reporting a bug at a class^^J%
      or package, simply press <return>.}
  \else
    \typeout{^^J%
      Name einer KURZEN, SELBSTERKLAERENDEN Datei, bei der das Problem^^J%
      auftritt (die Datei sollte wirklich so kurz wie moeglich sein,^^J%
      nicht mehr als 60 Zeilen):^^J^^J%
      Damit LaTeX diese Datei einlesen kann, geben Sie bitte den kompletten^^J%
      Namen einschliesslich des Verzeichnisses an, falls die Datei nicht im^^J%
      aktuellen Verzeichnis zu finden ist.^^J^^J%
      Falls Sie keinen Fehler in einer der Klassen oder Pakete melden und^^J%
      daher keine Testdatei existiert, druecken Sie einfach <Return>.}
  \fi
  \message{filename> }\read\m@ne to \filename
\else
   \let\filename\@empty
\fi

%%
%% Try to find the .tex file and .log file
%%


\ifx\filename\@empty
  \ifinteractive
    \ifenglish
      \ReadNumber{^^J^^JNo Testfile.^^J^^J%
        Three kinds of reports are possible:^^J^^J%
        1) SW bug:^^J\@spaces
        Software bug, you have to add a test file!^^J
        2) DOC bug:^^J\@spaces
        Bug at manual or you do not understand the manual.^^J
        3) Ask for change:^^J\@spaces
        Not a bug, but you'd like a change or simply help.^^J}{3}
    \else
      \ReadNumber{^^J^^JKeine Testdatei.^^J^^J%
        Drei Arten von Meldungen werden unterstuetzt:^^J^^J%
        1) SW-Fehler:^^J\@spaces
        Software-Fehler, unbedingt eine Testdatei beilegen!^^J
        2) DOC-Fehler:^^J\@spaces
        Fehler in oder unverstaendliche Anleitung.^^J
        3) Aenderungswunsch:^^J\@spaces
        Kein Fehler sondern eine Frage nach Aenderung oder Hilfe.^^J}{3}
    \fi
  \else
    \def\answer{0}%
  \fi
\else
  \def\answer{0}%
  \filename@parse\filename

  \IfFileExists{\filename}{\edef\samplefile{\filename}%
    \IfFileExists{\filename@area\filename@base.log}{%
      \edef\logfile{\filename@area\filename@base.log}%
    }{%
      \IfFileExists{\filename@area\filename@base.lis}{%
        \edef\logfile{\filename@area\filename@base.lis}%
      }{%
        \ifenglish
          \typeout{^^J%
            Log file ``\filename@area\filename@base.log'' not found.^^J%
            Please add the log file at ``\jobname.msg''.}%
        \else
          \typeout{^^J%
            Logdatei ``\filename@area\filename@base.log'' nicht gefunden.^^J%
            Bitte ergaenzen Sie das Beispiel in ``\jobname.msg''.}%
        \fi
        \pause
      }%
    }%
  }{%
    \ifenglish
      \typeout{^^J%
        Test file ``\filename'' not found.^^J%
        Please add the test file at ``\jobname.msg''.}
    \else
      \typeout{^^J%
        Beispieldatei ``\filename'' nicht gefunden.^^J%
        Bitte ergaenzen Sie das Beispiel in ``\jobname.msg''.}
    \fi
    \pause
  }%
\fi

\ifcase\expandafter\number\answer
  \ifinteractive\wmsg{>Unterbereich: SW-Fehler}\fi
  \wmsg*{%
    Beispieldatei, die das Problem verdeutlicht:^^J%
    ============================================}
  \ifx\samplefile\undefinedcommand
    \ifenglish
      \typeout{^^J! 
        Please add test file and log file at your report message!}% 
    \else
      \typeout{^^J! 
        Bitte ergaenzen Sie Beispiel- und LOG-Datei in der Meldung!}%
    \fi
    \pause
    \ifenglish
      \wmsg*{<REPLACE THIS BY YOUR TEST FILE>}
    \else
      \wmsg*{<HIER TESTDATEI EINFUEGEN>}
    \fi
  \else
    %%
    %% The example file goes here:
    %%
    \copytomsg{\samplefile}
    \ifnum\linecount>60
      \ifenglish
        \typeout{%
          ^^J%
          !!! Your test file has \the\linecount\space lines.^^J%
          !!! Large test files make it difficult to find the cause of the
          problem:^^J% 
          !!! ^^J%
          !!! Please decrease your test file as much as possible.^^J}
      \else
        \typeout{%
          ^^J%
          !!! Ihre Testdatei ist \the\linecount\space Zeilen lang.^^J%
          !!! So grosse Testdateien erschweren die Ursachenfindung:^^J%
          !!! ^^J%
          !!! Bitte verkleinern Sie Ihre Testdatei, soweit das ueberhaupt^^J%
          !!! moeglich ist, so dass das Problem gerade noch auftritt.^^J}
      \fi
      \pause
    \fi
  \fi
  \wmsg*{^^J%
    LOG-Datei vom LaTeX-Lauf der Beispieldatei:^^J%
    ===========================================}
  \ifx\logfile\undefinedcommand
    \ifenglish
      \typeout{^^J%
        Log file of your test file not found.^^J%
        Please add the log file of your test file at ``\jobname.msg''.}
     \wmsg*{<REPLACE THIS BY THE LOG OF YOUR TEST FILE>}
    \else
      \typeout{^^J%
        Log-Datei zur Testdatei nicht gefunden.^^J%
        Bitte ergaenzen Sie die LOG-Datei in ``\jobname.msg''.}
      \wmsg*{<HIER LOG ZUR TESTDATEI EINFUEGEN >}
    \fi
    \pause
  \else
    \copytomsg{\logfile}
  \fi
\or
  \wmsg{>Unterbereich: DOC-Fehler}
 \or
  \wmsg{>Unterbereich: Aenderungswunsch}
\fi

%%
%% Closing Banner.
%%
\typeout{^^J%
============================================================}

\ifinteractive
  \ifenglish
    \typeout{^^J%
      You may edit file ``\jobname.msg'' for additional changes.}
  \else
    \typeout{^^J%
      Weiteren Aenderungen koennen sie direkt in der Datei^^J%
      ``\jobname.msg'' mit Hilfe eines Editors vornehmen.}
  \fi
\else
  \ifenglish
    \typeout{^^J%
      The formular of the report will be saved to ``\jobname.msg''.^^J%
      Please use an editor to replace all information fields, before^^J%
      sending it.}
  \else
    \typeout{^^J%
      Das Formular fuer die Erstellung der Meldung wurde in die^^J%
      Datei ``\jobname.msg'' geschrieben, die Sie bitte mit Hilfe^^J%
      eines Editors ergaenzen, bevor Sie sie abschicken.}
  \fi
\fi

\let\ifinteractivetrue\iftrue
\ifenglish
  \typeout{^^J%
    Please send ``\jobname.msg'' to komascript@gmx.info using subject:^^J%
    \@spaces ``KOMA-BUG:\space\synopsis''^^J%
    ^^J%
    Thank you for spending time for a bug report.}
\else
  \typeout{^^J%
    Wenn Sie ueber eMail verfuegen, so senden Sie ``\jobname.msg''^^J%
    bitte an komascript@gmx.info.^^J%
    Verwenden Sie dabei bitte als Betreff (Subject):^^J%
    \@spaces ``KOMA-BUG:\space\synopsis''^^J%
    ^^J%
    Danke, dass Sie sich die Zeit genommen haben.}
\fi
\wmsg*{^^J%
  ============================================================^^J
  Ende der KOMA-Fehlermeldung.^^J%
  ============================================================}

%%
%% Close the .msg output stream.
%%
\immediate\closeout\msg

%%
%% This is the TeX primitive \end command.
%%
\@@end

%%% Local Variables: 
%%% mode: latex
%%% TeX-master: t
%%% End: 
