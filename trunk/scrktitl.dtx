% \CheckSum{658}
% \iffalse meta-comment
% ======================================================================
% scrktitl.dtx
% Copyright (c) Markus Kohm, 2002-2008
%
% This file is part of the LaTeX2e KOMA-Script bundle.
%
% This work may be distributed and/or modified under the conditions of
% the LaTeX Project Public License, version 1.3b of the license.
% The latest version of this license is in
%   http://www.latex-project.org/lppl.txt
% and version 1.3b or later is part of all distributions of LaTeX 
% version 2005/12/01 or later and of this work.
%
% This work has the LPPL maintenance status "author-maintained".
%
% The Current Maintainer and author of this work is Markus Kohm.
%
% This work consists of all files listed in manifest.txt.
% ----------------------------------------------------------------------
% scrktitl.dtx
% Copyright (c) Markus Kohm, 2002-2008
%
% Dieses Werk darf nach den Bedingungen der LaTeX Project Public Lizenz,
% Version 1.3b, verteilt und/oder veraendert werden.
% Die neuste Version dieser Lizenz ist
%   http://www.latex-project.org/lppl.txt
% und Version 1.3b ist Teil aller Verteilungen von LaTeX
% Version 2005/12/01 oder spaeter und dieses Werks.
%
% Dieses Werk hat den LPPL-Verwaltungs-Status "author-maintained"
% (allein durch den Autor verwaltet).
%
% Der Aktuelle Verwalter und Autor dieses Werkes ist Markus Kohm.
% 
% Dieses Werk besteht aus den in manifest.txt aufgefuehrten Dateien.
% ======================================================================
% \fi
%
% \CharacterTable
%  {Upper-case    \A\B\C\D\E\F\G\H\I\J\K\L\M\N\O\P\Q\R\S\T\U\V\W\X\Y\Z
%   Lower-case    \a\b\c\d\e\f\g\h\i\j\k\l\m\n\o\p\q\r\s\t\u\v\w\x\y\z
%   Digits        \0\1\2\3\4\5\6\7\8\9
%   Exclamation   \!     Double quote  \"     Hash (number) \#
%   Dollar        \$     Percent       \%     Ampersand     \&
%   Acute accent  \'     Left paren    \(     Right paren   \)
%   Asterisk      \*     Plus          \+     Comma         \,
%   Minus         \-     Point         \.     Solidus       \/
%   Colon         \:     Semicolon     \;     Less than     \<
%   Equals        \=     Greater than  \>     Question mark \?
%   Commercial at \@     Left bracket  \[     Backslash     \\
%   Right bracket \]     Circumflex    \^     Underscore    \_
%   Grave accent  \`     Left brace    \{     Vertical bar  \|
%   Right brace   \}     Tilde         \~}
%
% \iffalse
%%% From File: scrktitl.dtx
%<*driver>
% \fi
\ProvidesFile{scrktitl.dtx}[2008/11/21 v3.01a KOMA-Script (title pages)]
% \iffalse
\documentclass[halfparskip-]{scrdoc}
\usepackage[english,ngerman]{babel}
\usepackage[latin1]{inputenc}
\CodelineIndex
\RecordChanges
\GetFileInfo{scrktitl.dtx}
\title{\KOMAScript{} \partname\ \texttt{\filename}%
  \footnote{Dies ist Version \fileversion\ von Datei
    \texttt{\filename}.}}
\date{\filedate}
\author{Markus Kohm}

\begin{document}
  \maketitle
  \tableofcontents
  \DocInput{\filename}
\end{document}
%</driver>
% \fi
%
% \selectlanguage{ngerman}
%
% \changes{v2.95}{2002/06/26}{%
%   erste Version aus der Aufteilung von \texttt{scrclass.dtx}}
%
% \section{Die Titelei}
%
% Die Titelei ist gegen�ber den Standardklassen erheblich
% erweitert. Trotzdem kann ein Titel einer Standardklasse unmittelbar
% mit einer \KOMAScript-Klasse gesetzt werden. Briefe besitzen allerdings
% keine Titelei im eigentlichen Sinne.
%
% \StopEventually{\PrintIndex\PrintChanges}
%
% \iffalse
%<*!letter>
% \fi
%
% \iffalse
%<*option>
% \fi
%
%
% \subsection{Optionen der Titelei}
%
%  \begin{option}{titlepage}
%    \changes{v2.95c}{2006/08/21}{als \textsf{keyval}-Option}
%  \begin{option}{notitlepage}
%    \changes{v2.95c}{2006/08/21}{Option ist obsolet}
%    \changes{v3.01a}{2008/11/21}{standard statt obsolet}
% Es werden zei Arten von Titeln unterst�tzt. Da sind zum einen eigene
% Titelseiten, zum anderen sogenannte Titelk�pfe, also Titel, die nicht eine
% eigene Seite erhalten, sondern am Anfang einer Seite stehen. Zwischen den
% beiden Arten wird mit einer Option umgeschaltet.
%  \begin{macro}{\if@titlepage}
%  \begin{macro}{\@titlepagetrue}
%  \begin{macro}{\@titlepagefalse}
% Die gew�hlte Einstellung wird in einem Schalter gespeichert.
%    \begin{macrocode}
%<*class>
\newif\if@titlepage
%<article>\@titlepagefalse
%<report|book>\@titlepagetrue
%</class>
%    \end{macrocode}
% Im Falle von \textsf{scrextend} ist die Option Teil der Erweiterung.
%    \begin{macrocode}
%<*package&extend>
\scr@ext@activateable{title}{%
  \scr@ifundefinedorrelax{if@titlepage}{%
    \expandafter\newif\csname if@titlepage\endcsname
    \@titlepagefalse
  }{}
%</package&extend>
%    \end{macrocode}
%  \end{macro}
%  \end{macro}
%  \end{macro}
%    \begin{macrocode}
%<*class|package>
\KOMA@ifkey{titlepage}{@titlepage}%
%<*extend>
%    \end{macrocode}
% Diese Erweiterung soll dann auch tats�chlich sofort aktiviert werden:
%    \begin{macrocode}
  \def\scr@ext@immediate@title{%
    \scr@ext@activate{title}%
    \let\scr@ext@immediate@title\relax
  }%
}
%</extend>
%</class|package>
\KOMA@DeclareStandardOption%
%<package>  [scrextend]%
  {notitlepage}{titlepage=false}
%    \end{macrocode}
%  \end{option}
%  \end{option}
%
%  \begin{option}{abstract}
%    \changes{v2.95c}{2006/08/21}{neue \textsf{keyval}-Option}
%  \begin{option}{abstracton}
%    \changes{v2.95c}{2006/08/21}{Option ist obsolet}
%    \changes{v3.01a}{2008/11/21}{standard statt obsolet}
%  \begin{option}{abstractoff}
%    \changes{v2.95c}{2006/08/21}{Option ist obsolet}
%    \changes{v3.01a}{2008/11/21}{standard statt obsolet}
% Obwohl die Zusammenfassung nicht unbedingt auf einer Titelseite steht,
% geh�rt sie doch zur Titelei und wird daher hier mit behandelt. Die
% Zusammenfassung kann mit einem standardm��igen Titel versehen werden. Der
% Titel kann aber auch unterdr�ckt werden.
%
% \iffalse
%<*class>
%<*article|report>
% \fi
% Bei B�chern gibt es keine Zusammenfassung als gesonderte Umgebung. Als
% Ersatz kann bei B�chern auf \cs{addchap*} zur�ckgegriffen werden.
%
%  \begin{macro}{\if@abstrt}
%  \begin{macro}{\@abstrttrue}
%  \begin{macro}{\@abstrtfalse}
% Die Enscheidung, ob der Titel der Zusammenfassung gesetzt werden soll, wird
% in einem Schalter gespeichert.
%    \begin{macrocode}
\newif\if@abstrt\@abstrtfalse
%    \end{macrocode}
%  \end{macro}
%  \end{macro}
%  \end{macro}
%
%    \begin{macrocode}
\KOMA@ifkey{abstract}{@abstrt}
\KOMA@DeclareStandardOption{abstracton}{abstract=true}
\KOMA@DeclareStandardOption{abstractoff}{abstract=false}
%    \end{macrocode}
%  \end{option}
%  \end{option}
%  \end{option}
% \iffalse
%</article|report>
%</class>
% \fi
%
%
% \iffalse
%</option>
%<*body>
% \fi
%
%
% \subsection{Definitionen der Titelei}
%
% \begin{macro}{\extratitle}
% \begin{macro}{\@extratitle}
% \begin{macro}{\titlehead}
% \begin{macro}{\@titlehead}
% \begin{macro}{\subject}
% \begin{macro}{\@subject}
% \begin{macro}{\subtitle}
%   \changes{v2.97c}{2007/06/20}{neue M�glichkeit}
% \begin{macro}{\@subtitle}
%   \changes{v2.97c}{2007/06/20}{neu (intern)}
% \begin{macro}{\publishers}
% \begin{macro}{\@publishers}
% \begin{macro}{\uppertitleback}
% \begin{macro}{\@uppertitleback}
% \begin{macro}{\lowertitleback}
% \begin{macro}{\@lowertitleback}
% \begin{macro}{\dedication}
% \begin{macro}{\@dedication}
% Da der Titel im \KOMAScript{} Paket wesentlich mehr Angaben erlaubt
% als bei den Standardklassen, gibt es nat�rlich auch einige Befehle,
% mit denen diese gesetzt werden k�nnen. Alle zus�tzlichen Angaben
% sind optional und k�nnen auch weggelassen werden. In diesem Fall
% werden Leerfelder verwendet. Bis auf \cs{subject} sind alle
% Zusatzbefehle \cs{long} deklariert, k�nnen also ganze Abs�tze
% enthalten.
%    \begin{macrocode}
%<package&extend>\scr@ext@addto@activateable{title}{%
\newcommand*{\@extratitle}{}%
\newcommand{\extratitle}[1]{\gdef\@extratitle{%
%<extend>    ##1%
%<!extend>    #1%
}}%
\newcommand*{\@titlehead}{}%
\newcommand{\titlehead}[1]{\gdef\@titlehead{%
%<extend>    ##1%
%<!extend>    #1%
}}%
\newcommand*{\@subject}{}%
\newcommand*{\subject}[1]{\gdef\@subject{%
%<extend>    ##1%
%<!extend>    #1%
}}%
\newcommand*{\subtitle}[1]{\gdef\@subtitle{%
%<extend>    ##1%
%<!extend>    #1%
}}%
\newcommand*{\@subtitle}{}%
\newcommand*{\@publishers}{}%
\newcommand{\publishers}[1]{\gdef\@publishers{%
%<extend>    ##1%
%<!extend>    #1%
}}%
\newcommand*{\@uppertitleback}{}%
\newcommand{\uppertitleback}[1]{\gdef\@uppertitleback{%
%<extend>    ##1%
%<!extend>    #1%
}}%
\newcommand*{\@lowertitleback}{}%
\newcommand{\lowertitleback}[1]{\gdef\@lowertitleback{%
%<extend>    ##1%
%<!extend>    #1%
}}%
\newcommand*{\@dedication}{}%
\newcommand{\dedication}[1]{\gdef\@dedication{%
%<extend>    ##1%
%<!extend>    #1%
}}%
%    \end{macrocode}
%  \end{macro}
%  \end{macro}
%  \end{macro}
%  \end{macro}
%  \end{macro}
%  \end{macro}
%  \end{macro}
%  \end{macro}
%  \end{macro}
%  \end{macro}
%  \end{macro}
%  \end{macro}
%  \end{macro}
%  \end{macro}
%  \end{macro}
%  \end{macro}
%
%  \begin{macro}{\next@tpage}
%    \changes{v2.3b}{1995/07/24}{\cs{null} entfernt.}
%    \changes{v2.3g}{1996/01/14}{\cs{newpage} durch \cs{clearpage}
%      ersetzt}
% Innerhalb eines Titels auf die n�chste Seite umschalten.
%    \begin{macrocode}
\newcommand*{\next@tpage}{\clearpage\thispagestyle{empty}}%
%    \end{macrocode}
%  \end{macro}
%  \begin{macro}{\maketitle}
% Mit dieser Anweisung wird der Titel generiert, dessen Inhalt zuvor
% gesetzt wurde. Es muss unterschieden werden, ob der Titel auf einer
% eigenen Titelseite oder als Seitenkopf erstellt werden soll. Der
% \KOMAScript-Titel ist u.\,U. sehr gro�. Im Fall dass diverse
% Zusatzelemente verwendet werden, sollte eigentlich immer eine
% Titelseite verwendet werden.
%    \changes{v2.1a}{1994/10/29}{das Hilfskonstrukt \cs{@maketitle}
%      wurde aus der Unterscheidung herausgenommen (f�r den Fall, dass
%      \cs{maketitle} von einem Paket �berladen wird)}
%    \changes{v2.3a}{1995/07/08}{\texttt{plus} durch \cs{@plus}
%      ersetzt}
%    \changes{v2.3a}{1995/07/08}{\texttt{fill} durch \cs{fill}
%      ersetzt}
%    \changes{v2.3d}{1995/08/19}{\cs{fill} durch \texttt{fill}
%      ersetzt}
%    \changes{v2.3a}{1995/07/08}{\cs{vfil} durch \cs{vfill} ersetzt}
%    \changes{v2.3b}{1995/07/24}{Umbruch bei \cs{@extratitle} in der
%      beidseitigen Titelseite korrigiert}
%    \changes{v2.3g}{1996/01/14}{\cs{footnote} funktioniert nun auch
%      im Titel} 
%    \changes{v2.4}{1996/02/25}{\cs{footnote} funktioniert nun
%      wirklich im Titel}
%    \changes{v2.4h}{1996/11/09}{egal ob der Titel auf eine Extraseite
%      kommt oder nicht, \cs{@title} wird mit \cs{sectfont} gesetzt,
%      wobei die Gr��e neuerdings danach eingestellt wird}
%    \changes{v2.4l}{1997/02/06}{symbolische Fu�noten und
%      Fu�notensymbolbreiten von 0\,pt zur besseren Zentrierung des
%      Autors sind auch bei Titelseiten angebracht}
%    \changes{v2.8p}{2001/09/22}{\cs{titlefont} an Stelle von
%      \cs{sectfont}}
%    \changes{v2.95c}{2006/08/21}{Absatzgrundeinstellungen}
%    \changes{v2.97c}{2007/08/31}{Vakatseiten in der Titelei immer im
%      Seitenstil \texttt{empty}}
%    \begin{macrocode}
%<package&extend>\let\maketitle\relax\let\@maketitle\relax
\if@titlepage
  \newcommand*\maketitle[1][1]{%
    \begin{titlepage}
      \setcounter{page}{%
%<extend>        ##1%
%<!extend>        #1%
      }%
      \let\footnotesize\small
      \let\footnoterule\relax
      \let\footnote\thanks
      \renewcommand*\thefootnote{\@fnsymbol\c@footnote}%
      \let\@oldmakefnmark\@makefnmark
      \renewcommand*{\@makefnmark}{\rlap\@oldmakefnmark}%
      \ifx\@extratitle\@empty \else
        \noindent\@extratitle\next@tpage\cleardoubleemptypage
        \thispagestyle{empty}%
      \fi
%<!extend>      \setparsizes{\z@}{\z@}{\z@\@plus 1fil}\par@updaterelative
%<extend>      \parskip\z@ \parindent\z@ \parfillskip\z@\@plus 1fil
      \ifx\@titlehead\@empty \else
        \begin{minipage}[t]{\textwidth}%
        \@titlehead
        \end{minipage}\par
      \fi
      \null\vfill
      \begin{center}
        \ifx\@subject\@empty \else
          {\subject@font \@subject \par}%
          \vskip 3em
        \fi
        {\titlefont\huge \@title\par}%
        \vskip 1em
        {\ifx\@subtitle\@empty\else\usekomafont{subtitle}\@subtitle\par\fi}%
        \vskip 2em
        {\Large \lineskip 0.75em
          \begin{tabular}[t]{c}
            \@author
          \end{tabular}\par
        }%
        \vskip 1.5em
        {\Large \@date \par}%
        \vskip \z@ \@plus3fill
        {\Large \@publishers \par}%
        \vskip 3em
      \end{center}\par
      \@thanks
      \vfill\null
      \if@twoside\next@tpage
        \begin{minipage}[t]{\textwidth}
          \@uppertitleback
        \end{minipage}\par
        \vfill
        \begin{minipage}[b]{\textwidth}
          \@lowertitleback
        \end{minipage}
      \fi
      \ifx\@dedication\@empty \else
        \next@tpage\null\vfill
        {\centering \Large \@dedication \par}%
        \vskip \z@ \@plus3fill
        \if@twoside \next@tpage\cleardoubleemptypage \fi
      \fi
    \end{titlepage}
    \setcounter{footnote}{0}%
%    \end{macrocode}
%    \changes{v2.3g}{1996/01/14}{Verwendung von \cs{global}\cs{let}
%      statt \cs{gdef}, um Speicher zu sparen}
%    \changes{v2.3g}{1996/01/14}{\cs{@date} und \cs{title}
%      u.\,�. ebenfalls l�schen, um Speicher zu sparen}
%    \begin{macrocode}
    \global\let\thanks\relax
    \global\let\maketitle\relax
    \global\let\@thanks\@empty
    \global\let\@author\@empty
    \global\let\@date\@empty
    \global\let\@title\@empty
    \global\let\@subtitle\@empty
    \global\let\@extratitle\@empty
    \global\let\@titlehead\@empty
    \global\let\@subject\@empty
    \global\let\@publishers\@empty
    \global\let\@uppertitleback\@empty
    \global\let\@lowertitleback\@empty
    \global\let\@dedication\@empty
    \global\let\author\relax
    \global\let\title\relax
    \global\let\extratitle\relax
    \global\let\titlehead\relax
    \global\let\subject\relax
    \global\let\publishers\relax
    \global\let\uppertitleback\relax
    \global\let\lowertitleback\relax
    \global\let\dedication\relax
    \global\let\date\relax
    \global\let\and\relax
  }%
\else
%    \end{macrocode}
%    \changes{v2.3a}{1995/07/08}{Verwendung von \cs{@makefnmark} in
%      \cs{@makefntext}}
%    \changes{v2.3a}{1995/07/08}{Definition von \cs{@makefnmark} und
%      \cs{@makefntext} unabh�ngig von math}
%    \changes{v2.3e}{1995/08/30}{optionales Argument bei der
%      einseitigen Version erlauben und ignorieren}
%    \changes{v2.3g}{1996/01/14}{mehrfach \cs{null} bei
%      \cs{next@tpage} erg�nzt}
%    \changes{v2.4l}{1997/02/06}{es ist nicht mehr notwendig
%      \cs{\@makefntext} umzudefinieren, stattdessen wird nur
%      \cs{\@makefnmark} umdefiniert}
%    \changes{v2.8d}{2001/07/05}{\cs{titlepagestyle} statt
%      \texttt{plain}}
%    \begin{macrocode}
  \newcommand*\maketitle[1][1]{\par
    \@tempcnta=%
%<extend>    ##1%
%<!extend>    #1%
    \relax\ifnum\@tempcnta=1\else
%<class>      \ClassWarning{\KOMAClassName}{%
%<package>      \PackageWarning{scrextend}{%
        Optional argument of \string\maketitle\space ignored
        at\MessageBreak
        notitlepage-mode%
      }%
    \fi
    \begingroup
      \renewcommand*\thefootnote{\@fnsymbol\c@footnote}%
      \let\@oldmakefnmark\@makefnmark
      \renewcommand*{\@makefnmark}{\rlap\@oldmakefnmark}
      \if@twocolumn
        \ifnum \col@number=\@ne
          \@maketitle
        \else
          \twocolumn[\@maketitle]%
        \fi
      \else
        \newpage
        \global\@topnum\z@
        \@maketitle
      \fi
      \thispagestyle{\titlepagestyle}\@thanks
    \endgroup
    \setcounter{footnote}{0}%
    \let\thanks\relax
    \let\maketitle\relax
    \let\@maketitle\relax
%    \end{macrocode}
%    \changes{v2.3g}{1996/01/14}{Verwendung von \cs{global}\cs{let}
%      statt \cs{gdef}, um Speicher zu sparen}
%    \changes{v2.3g}{1996/01/14}{\cs{@date} und \cs{title}
%      u.\,�. ebenfalls l�schen, um Speicher zu sparen}
%    \begin{macrocode}
    \global\let\@thanks\@empty
    \global\let\@author\@empty
    \global\let\@date\@empty
    \global\let\@title\@empty
    \global\let\@subtitle\@empty
    \global\let\@extratitle\@empty
    \global\let\@titlehead\@empty
    \global\let\@subject\@empty
    \global\let\@publishers\@empty
    \global\let\@uppertitleback\@empty
    \global\let\@lowertitleback\@empty
    \global\let\@dedication\@empty
    \global\let\author\relax
    \global\let\title\relax
    \global\let\extratitle\relax
    \global\let\titlehead\relax
    \global\let\subject\relax
    \global\let\publishers\relax
    \global\let\uppertitleback\relax
    \global\let\lowertitleback\relax
    \global\let\dedication\relax
    \global\let\date\relax
    \global\let\and\relax
  }%
\fi
%    \end{macrocode}
%  \begin{macro}{\@maketitle}
%    \changes{v2.95c}{2006/08/21}{Absatzgrundeinstellungen}
% Damit wird im Fall des Titelkopfes die eigentliche Arbeit geleistet.
%    \begin{macrocode}
\newcommand*{\@maketitle}{%
  \clearpage
  \let\footnote\thanks
  \ifx\@extratitle\@empty \else
    \noindent\@extratitle \next@tpage \if@twoside \null\next@tpage \fi
  \fi
%<!extend>  \setparsizes{\z@}{\z@}{\z@\@plus 1fil}\par@updaterelative
%<extend>  \parskip\z@ \parindent\z@ \parfillskip\z@\@plus 1fil
  \ifx\@titlehead\@empty \else
    \begin{minipage}[t]{\textwidth}
      \@titlehead
    \end{minipage}\par
  \fi
  \null
  \vskip 2em%
  \begin{center}%
    \ifx\@subject\@empty \else
      {\subject@font \@subject \par}
      \vskip 1.5em
    \fi
    {\titlefont\huge \@title \par}%
    \vskip .5em
    {\ifx\@subtitle\@empty\else\usekomafont{subtitle}\@subtitle\par\fi}%
    \vskip 1em
    {\Large
      \lineskip .5em%
      \begin{tabular}[t]{c}
        \@author
      \end{tabular}\par
    }%
    \vskip 1em%
    {\Large \@date \par}%
    \vskip \z@ \@plus 1em
    {\Large \@publishers \par}
    \ifx\@dedication\@empty \else
      \vskip 2em
      {\Large \@dedication \par}
    \fi
  \end{center}%
  \par
  \vskip 2em
}%
%    \end{macrocode}
%  \end{macro}
%  \end{macro}
%
%
% \subsection{Umgebung f�r die Titelseite}
%
%  \begin{environment}{titlepage}
% Die Titelseite bedarf ebenfalls einer gesonderter Umgebung. Beim
% zweispaltigen Layout soll die Titelseite z.\,B. einspaltig gedruckt
% werden.
%    \begin{macrocode}
%<package&extend>\scr@ifundefinedorrelax{titlepage}{%
\newenvironment{titlepage}{%
%<report|book>  \cleardoublepage
  \if@twocolumn
    \@restonecoltrue\onecolumn
  \else
    \@restonecolfalse\newpage
  \fi
  \thispagestyle{empty}%
  \if@compatibility
    \setcounter{page}{0}%
  \fi
}{%
  \if@restonecol\twocolumn \else \newpage \fi
}%
%<package&extend>}{}%
%    \end{macrocode}
%  \end{environment}
%
% Im Fall von \textsf{scrextend} muss jetzt das K�rzel \emph{title} beendet
% werden:
%    \begin{macrocode}
%<package&extend>}\csname scr@ext@immediate@title\endcsname
%    \end{macrocode}
%
% \subsection{Fonts f�r den Titel}
%
%  \begin{macro}{\titlefont}
%    \changes{v2.8p}{2001/09/22}{neu}
% Wie oben zu sehen ist, wird in der Titelei ein eigenes Schriftmakro
% verwendet. Dieses ist als internes Makro zu verstehen. Der Anwender
% sollte stattdessen auf das entsprechende Element zugreifen (siehe
% unten).
%  \begin{macro}{\subject@font}
%    \changes{v2.8q}{2002/01/14}{neu (intern)}
%    \changes{v2.95}{2002/06/26}{jetzt auch bei \textsf{scrbook},
%      \textsf{scrreprt} und \textsf{scrartcl}}
% Ebenso verh�lt es sich mit den \emph{Subject} im Titel. Hier ist
% aber von vornherein ein internes Makro definiert.
%    \begin{macrocode}
\newcommand*\titlefont{\sectfont}%
\newcommand*{\subject@font}{\normalfont\normalcolor\bfseries\Large}%
%    \end{macrocode}
%  \end{macro}
%  \end{macro}
%
% \begin{KOMAfont}{subtitle}
%   \changes{v2.97c}{2007/06/20}{neu (intern)}
%    \begin{macrocode}
\newkomafont{subtitle}{\usekomafont{title}\large}%
%    \end{macrocode}
% \end{KOMAfont}
%
%
%  \begin{macro}{\scr@fnt@title}
%    \changes{v2.8o}{2001/09/14}{neues Element \texttt{title}}
%  \begin{macro}{\scr@fnt@subject}
%    \changes{v2.8q}{2002/01/14}{neues Element \texttt{subject}}
%    \changes{v2.95}{2002/06/26}{jetzt auch bei \textsf{scrbook},
%      \textsf{scrreprt} und \textsf{scrartcl}}
% Die beiden Elemente, deren Schrift ge�ndert werden kann.
%    \begin{macrocode}
\newcommand*{\scr@fnt@title}{\titlefont}%
\newcommand*{\scr@fnt@subject}{\subject@font}%
%    \end{macrocode}
%  \end{macro}
%  \end{macro}
%
%
% \subsection{Umgebung f�r die Zusammenfassung}
%
%  \begin{environment}{abstract}
% \textsf{scrartcl} und \textsf{scrreprt} bieten die M�glichkeit einer
% Zusammenfassung, eines sogenannten \emph{Abstracts}. Wenn eine
% Titelseite verlangt wurde, wird auch die Zusammenfassung auf eine
% eigene Seite gesetzt. \textsf{scrbook} kennt keine Zusammenfassung
% dieser Art. In B�chern werden Zusammenfassungen �blicherweise mit
% Kapitelcharakter gesetzt.
% \iffalse
%<*class>
%<*article|report>
% \fi
%    \changes{v2.3a}{1995/07/08}{\cs{@endparpenalty} zur Verhinderung
%      eines Seitenumbruchs nach dem Abstract-Kopf eingef�gt}
%    \changes{v2.3g}{1996/01/14}{\cs{@beginparpenalty} zur
%      Verhinderung eines Seitenumbruchs vor dem Abstract-Kopf
%      eingef�gt}
%    \changes{v2.7a}{2001/01/04}{statt \cs{section*} wird nun
%      \cs{addsec*} verwendet, damit die Kolumnentitel korrekt
%      behandelt werden}
%    \begin{macrocode}
\if@titlepage
  \newenvironment{abstract}{%
    \titlepage
    \null\vfil
    \@beginparpenalty\@lowpenalty
    \if@abstrt
      \begin{center}
        \normalfont\sectfont\nobreak\abstractname
        \@endparpenalty\@M
      \end{center}
    \fi
  }{%
    \par\vfil\null\endtitlepage
  }
\else
  \newenvironment{abstract}{%
    \if@twocolumn\if@abstrt
        \addsec*{\abstractname}
      \fi
    \else
      \if@abstrt
        \small
        \begin{center}
          {\normalfont\sectfont\nobreak\abstractname
            \vspace{-.5em}\vspace{\z@}}%
        \end{center}
      \fi
      \quotation
    \fi
  }{%
    \if@twocolumn\else\endquotation\fi
  }
\fi
%    \end{macrocode}
% \iffalse
%</article|report>
%</class>
% \fi
%  \end{environment}
%
%
% \iffalse
%</body>
% \fi
%
% \iffalse
%</!letter>
% \fi
%
% \Finale
%
\endinput
%
% end of file `scrktitl.dtx'
%%% Local Variables:
%%% mode: doctex
%%% TeX-master: t
%%% End:
