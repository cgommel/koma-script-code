% \CheckSum{597}
% \iffalse meta-comment
% ======================================================================
% scrkpage.dtx
% Copyright (c) Markus Kohm, 2002-2006
%
% This file is part of the LaTeX2e KOMA-Script bundle.
%
% This work may be distributed and/or modified under the conditions of
% the LaTeX Project Public License, version 1.3b of the license.
% The latest version of this license is in
%   http://www.latex-project.org/lppl.txt
% and version 1.3b or later is part of all distributions of LaTeX 
% version 2005/12/01 and of this work.
%
% This work has the LPPL maintenance status "author-maintained".
%
% The Current Maintainer and author of this work is Markus Kohm.
%
% This work consists of all files listed in manifest.txt.
% ----------------------------------------------------------------------
% scrkpage.dtx
% Copyright (c) Markus Kohm, 2002-2006
%
% Dieses Werk darf nach den Bedingungen der LaTeX Project Public Lizenz,
% Version 1.3b.
% Die neuste Version dieser Lizenz ist
%   http://www.latex-project.org/lppl.txt
% und Version 1.3b ist Teil aller Verteilungen von LaTeX
% Version 2005/12/01 und dieses Werks.
%
% Dieses Werk hat den LPPL-Verwaltungs-Status "author-maintained"
% (allein durch den Autor verwaltet).
%
% Der Aktuelle Verwalter und Autor dieses Werkes ist Markus Kohm.
% 
% Dieses Werk besteht aus den in manifest.txt aufgefuehrten Dateien.
% ======================================================================
% \fi
% \iffalse
%%% From File: scrkpage.dtx
%<*driver>
% \fi
\ProvidesFile{scrkpage.dtx}[2006/03/15 v3.0 KOMA-Script (page style)]
% \iffalse
\documentclass[halfparskip-]{scrdoc}
\usepackage[english,german]{babel}
\usepackage[latin1]{inputenc}
\CodelineIndex
\RecordChanges
\GetFileInfo{scrkpage.dtx}
\title{\KOMAScript{} \partname\ \texttt{\filename}%
  \footnote{Dies ist Version \fileversion\ von Datei \texttt{\filename}.}}
\date{\filedate}
\author{Markus Kohm}

\begin{document}
  \maketitle
  \tableofcontents
  \DocInput{\filename}
\end{document}
%</driver>
% \fi
%
% \selectlanguage{german}
%
% \changes{v3.0}{2002/06/25}{%
%   erste Version aus der Aufteilung von \texttt{scrclass.dtx}}
%
% \section{Seitenstil}
%
% Neben den Definitionen zum Seitenstil in dieser Datei gibt es auch
% noch das Paket \textsf{scrpage2} (siehe \texttt{scrpage.dtx}). 
% Es\marginline{Geplant!} ist deshalb davon auszugehen, dass zuk�nftig die
% Definitionen hier komplett durch \textsf{scrpage2} ersetzt werden und
% \textsf{scrpage2} von den \KOMAScript-Klassen direkt geladen wird.
%
% \StopEventually{\PrintIndex\PrintChanges}
%
% \subsection{Optionen f�r die Trennlinien in Kopf und Fu�}
%
% \iffalse
%<*option>
% \fi
%
%  \begin{option}{headsepline}
%  \begin{option}{headnosepline}
% Der Kopfteil kann mit einer Linie vom Textbereich getrennt werden.
%  \begin{macro}{\if@hsl}
%  \begin{macro}{\@hsltrue}
%  \begin{macro}{\@hslfalse}
% Die Entscheidung wird in einem Schalter gespeichert.
%    \begin{macrocode}
\newif\if@hsl
%    \end{macrocode}
%  \end{macro}
%  \end{macro}
%  \end{macro}
% Gleichzeitig wird dem \textsf{typearea}-Paket mitgeteilt, dass der
% Kopf zum Textbereich zu rechnen ist. Dies geschieht allerdings in
% umgekehrter Richtung nicht. Die Optionen sind also nicht
% symmetrisch.
%    \begin{macrocode}
\KOMA@key{headsepline}[true]{%
  \KOMA@set@ifkey{headsepline}{@hsl}{#1}%
  \@ifpackageloaded{typearea}{%
    \KOMAoptions{headinclude=#1}%
  }{%
    \PassOptionsToPackage{typearea}{headinclude=#1}%
  }%
}
\KOMA@DeclareObsoleteOption[\PackageInfo]{headnosepline}{headsepline=false}
%    \end{macrocode}
%  \end{option}
%  \end{option}
%
%
%  \begin{option}{footsepline}
%    \changes{v2.0e}{1994/08/14}{bei \cs{PassOptionsToPackage} fehlte
%      die Paket-Angabe}
%  \begin{option}{footnosepline}
% Der Fu�teil kann mit einer Linie vom Textbereich getrennt werden.
%  \begin{macro}{\if@fsl}
%  \begin{macro}{\@fsltrue}
%  \begin{macro}{\@fslfalse}
% Die Entscheidung wird in einem Schalter gespeichert.
%    \begin{macrocode}
\newif\if@fsl
%    \end{macrocode}
%  \end{macro}
%  \end{macro}
%  \end{macro}
% Gleichzeitig wird dem \texttt{typearea}-Paket mitgeteilt, dass der
% Fu� zum Textbereich zu rechnen ist. Auch hier sind die Optionen
% nicht symmetrisch.
%    \begin{macrocode}
%    \begin{macrocode}
\KOMA@key{footsepline}[true]{%
  \KOMA@set@ifkey{footsepline}{@fsl}{#1}%
  \@ifpackageloaded{typearea}{%
    \KOMAoptions{footinclude=#1}%
  }{%
    \PassOptionsToPackage{typearea}{footinclude=#1}%
  }%
}
\KOMA@DeclareObsoleteOption[\PackageInfo]{footnosepline}{footsepline=false}
%    \end{macrocode}
%  \end{option}
%  \end{option}
%
% In fr�heren Versionen wurde noch die Option \texttt{mpinclude} von
% den Klassen ausgewertet. Dies geschieht jetzt nur noch in
% \textsf{typearea}. Von den Klassen wird dann ggf. auf den dort
% definierten Schalter zugegriffen. Dieses Vorgehen ist deshalb
% sinnvoll, weil die Klassen ausnahmslos das \textsf{typearea}-Paket
% laden.
%
%
% \subsection{Optionen und Anweisungen f�r leere Seiten}
%
%  \begin{option}{cleardoublepage}
%    \changes{v3.0}{2004/08/24}{neuer Schalter}
%  \begin{option}{cleardoubleempty}
%    \changes{v2.8a}{2001/06/18}{neue Option}
%  \begin{option}{cleardoubleplain}
%    \changes{v2.8a}{2001/06/18}{neue Option}
%  \begin{option}{cleardoublestandard}
%    \changes{v2.8a}{2001/06/18}{neue Option}
%    \changes{v2.8q}{2001/11/06}{\cs{cleardoublestandardpage} statt
%      \cs{cleardoublestandard}}
%  \begin{macro}{\cleardoublestandardpage}
%    \changes{v2.8a}{2001/06/18}{neu}
%  \begin{macro}{\cleardoubleusingstyle}
%    \changes{v3.0}{2004/08/24}{neue Anweisung}
%  \begin{macro}{\cleardoubleemptypage}
%    \changes{v2.8a}{2001/06/18}{neu}
%  \begin{macro}{\cleardoubleplainpage}
%    \changes{v2.8a}{2001/06/18}{neu}
% Bei Verwendung von \texttt{twoside} und \texttt{openright} bei den
% Klassen \textsf{scrbook} und \textsf{scrreprt} wird bei \cs{chapter}
% implizit \cs{cleardoublepage} ausgef�hrt. Das f�hrt ggf. 
% normalerweise zu einer linken Seite, auf der noch der Kolumnentitel
% des vorherigen Kapitels und eine Seitenzahl steht. H�ufig wird
% stattdessen nur eine Seitenzahl oder auch gar nichts gew�nscht. Es
% soll also der Seitenstil \texttt{plain} oder \texttt{empty}
% verwendet werden. Dies wird mit den Optionen erm�glicht. Dazu
% werden auch gleich neue Makros definiert:
%    \begin{macrocode}
\KOMA@key{cleardoublepage}{%
  \begingroup%
    \def\@tempc{%
      \endgroup%
      \KOMA@unkown@keyval{cleardoublepage}{#1}{%
        'current' or any defined pagestyle e.g. 'empty','plain'}%
    }%
    \ifstr{#1}{current}{%
      \let\cleardoublepage\cleardoublestandardpage
    }{%
      \ifnotundefined{ps@#1}{%
        \def\@tempc{\endgroup%
          \def\cleardoublepage{\cleardoublepageusingstyle{#1}}%
        }%
      }{}%
    }%
  \@tempc
}
\newcommand*{\cleardoublestandardpage}{}
\let\cleardoublestandardpage\cleardoublepage
\newcommand*{\cleardoublepageusingstyle}[1]{\clearpage
  {\thispagestyle{#1}\cleardoublestandardpage}}
\newcommand*{\cleardoubleemptypage}{\cleardoublepageusingstyle{empty}}
\newcommand*{\cleardoubleplainpage}{\cleardoublepageusingstyle{plain}}
\KOMA@DeclareObsoleteOption[\PackageInfo]{cleardoubleempty}{%
  cleardoublepage=empty%
}
\KOMA@DeclareObsoleteOption[\PackageInfo]{cleardoubleplain}{%
  cleardoublepage=plain%
}
\KOMA@DeclareObsoleteOption[\PackageInfo]{cleardoublestandard}{%
  cleardoublepage=standard%
}
%    \end{macrocode}
%  \end{macro}
%  \end{macro}
%  \end{macro}
%  \end{macro}
%  \end{option}
%  \end{option}
%  \end{option}
%  \end{option}
%
%
% \iffalse
%</option>
%<*body>
% \fi
%
%
% \subsection{Befehle f�r den Kompatibilit�tsmodus}
%
%  \begin{macro}{\headincludeon}
%    \changes{v3.0}{2004/07/21}{ersatzlos gestrichen}
%  \begin{macro}{\headincludeoff}
%    \changes{v3.0}{2004/07/21}{ersatzlos gestrichen}
%  \begin{macro}{\headseplineon}
%    \changes{v3.0}{2004/07/21}{ersatzlos gestrichen}
%  \begin{macro}{\headseplineoff}
%    \changes{v3.0}{2004/07/21}{ersatzlos gestrichen}
%  \begin{macro}{\footincludeon}
%    \changes{v3.0}{2004/07/21}{ersatzlos gestrichen}
%  \begin{macro}{\footincludeoff}
%    \changes{v3.0}{2004/07/21}{ersatzlos gestrichen}
%  \begin{macro}{\footseplineon}
%    \changes{v3.0}{2004/07/21}{ersatzlos gestrichen}
%  \begin{macro}{\footseplineoff}
%    \changes{v3.0}{2004/07/21}{ersatzlos gestrichen}
% Da die \KOMAScript-Klassen ohnehin nicht mehr auf Kompatibilit�t zu
% \textsf{Script~2.0} �berpr�ft werden, wurde der Kompatibilit�tsmodus
% ersatzlos gestrichen. Dazu geh�ren nach �ber zehn Jahren nun auch diese
% Anweisungen. Ehemals waren sie wie folgt definiert:
% \begin{verbatim}
% \if@compatibility
%   \newcommand*\headincludeon{\@hincltrue}
%   \newcommand*\headincludeoff{\@hinclfalse}
%   \newcommand*\headseplineon{\@hsltrue \@hincltrue}
%   \newcommand*\headseplineoff{\@hslfalse \@hinclfalse}
%   \newcommand*\footincludeon{\@fincltrue}
%   \newcommand*\footincludeoff{\@finclfalse}
%   \newcommand*\footseplineon{\@fsltrue \@fincltrue}
%   \newcommand*\footseplineoff{\@fslfalse \@finclfalse}
% \fi
% \end{verbatim}
%  \end{macro}
%  \end{macro}
%  \end{macro}
%  \end{macro}
%  \end{macro}
%  \end{macro}
%  \end{macro}
%  \end{macro}
%
%
% \subsection{Definition der Standardseitenstile}
%
% \changes{v2.0e}{1994/08/10}{bei \textsf{scrbook} gibt es kein
%   einseitiges Layout}
% \changes{v2.3a}{1995/07/08}{da bei \textsf{book} seit Version 1.2v
%   die Option oneside wieder ein einseitiges Layout erzeugt, dieses
%   wieder eingebaut}
%
%  \begin{macro}{\ps@plain}
%  \begin{macro}{\ps@myheadings}
%  \begin{macro}{\ps@headings}
% Es wird das Aussehen der Kopf- und Fu�zeilen f�r die
% Standardseitenstile \texttt{empty}, \texttt{plain},
% \texttt{headings} und \texttt{myheadings} festgelegt. Diese sind
% au�erdem abh�ngig davon, ob es sich um einseitiges oder beidseitiges
% Layout handelt.
%
% Im Gegensatz zu den Seitenstilen der Standardklassen steht die
% Seitennummer beim \KOMAScript{} Paket immer in der Fu�zeile.
% Au�erdem sind Trennlinien zwischen Kopf- und Textbereich sowie
% zwischen Text- und Fu�bereich m�glich.
%
% Die Kopfzeile wird auch nicht mehr in Gro�buchstaben gewandelt.
%
%    \changes{v2.2a}{1995/02/07}{bei \textsf{scrbook} und
%      \textsf{scrreprt} Punkt hinter der section-Nummer entfernt}
%    \changes{v2.2c}{1995/05/25}{Punkt hinter der Kapitelnummer in der
%      Kopfzeile entfernt}
%    \changes{v2.2c}{1995/05/25}{Nummern in der Kopfzeile auf CJK
%      umgestellt}
%    \changes{v2.4f}{1996/10/08}{\cs{strut} in der Kopfzeile
%      eingef�gt}
%  \begin{macro}{\set@tempdima@hw}
%    \changes{v2.8q}{2002/03/28}{neu (intern)}
% Um nicht st�ndig das Gleiche schreiben zu m�ssen wird hier ein
% internes Makro verwendet.
%    \begin{macrocode}
\newcommand*{\set@tempdima@hw}{%
  \setlength{\@tempdima}{\textwidth}%
  \if@mincl
    \addtolength{\@tempdima}{\marginparsep}%
    \addtolength{\@tempdima}{\marginparwidth}%
  \fi
}
%    \end{macrocode}
%  \end{macro}
%
%  \begin{macro}{\pnumfont}
%    \changes{v2.8c}{2001/06/29}{\cs{normalcolor} eingef�gt}
%  \begin{macro}{\headfont}
%    \changes{v2.8c}{2001/06/29}{\cs{normalcolor} eingef�gt}
% Kopf- und Fu�zeile sowie die Seitennummer werden in einer speziellen
% Schriftart gesetzt. Die beiden Makros hier sind jedoch als interne
% Makros zu betrachten.
%    \begin{macrocode}
\newcommand*\pnumfont{\normalfont\normalcolor}
\newcommand*\headfont{\normalfont\normalcolor\slshape}
%    \end{macrocode}
%  \end{macro}
%  \end{macro}
%  \begin{macro}{\scr@fnt@pagenumber}
%    \changes{v2.8o}{2001/09/14}{neues Element \texttt{pagenumber}}
%  \begin{macro}{\scr@fnt@pagehead}
%    \changes{v2.8o}{2001/09/14}{neues Element \texttt{pagehead}}
%  \begin{macro}{\scr@fnt@wrn@pagehead}
%    \changes{v2.8o}{2001/09/14}{neue Warnung f�r Element
%      \texttt{pagehead}}
%  \begin{macro}{\scr@fnt@pagefoot}
%    \changes{v2.8o}{2001/09/14}{neues Element \texttt{pagefoot}}
%  \begin{macro}{\scr@fnt@wrn@pagefoot}
%    \changes{v2.8o}{2001/09/14}{neue Warnung f�r Element
%      \texttt{pagefoot}}
% Hier werden die Elemente definiert, deren Schriftart dann ge�ndert
% werden kann.
%    \begin{macrocode}
\newcommand*{\scr@fnt@pagenumber}{\pnumfont}
\let\scr@fnt@pagination=\scr@fnt@pagenumber
\newcommand*{\scr@fnt@pagehead}{\headfont}
\newcommand*{\scr@fnt@wrn@pagehead}[1]{%
  font selection of elements `pagehead' and `pagefoot'\MessageBreak
  changed, because you wanted to change font selection\MessageBreak
  of element `#1'%
}
\let\scr@fnt@pagefoot=\scr@fnt@pagehead
\let\scr@fnt@wrn@pagefoot=\scr@fnt@wrn@pagehead
%    \end{macrocode}
%  \end{macro}
%  \end{macro}
%  \end{macro}
%  \end{macro}
%  \end{macro}
%
% \begin{macro}{\pagemark}
%   \changes{v3.0}{2006/03/15}{nun f�r alle Klassen}
%   Das ist die Seitenmarke, die nun f�r alle Klassen verwendet wird. Damit
%   wird es leichter, \textsf{fancyhdr} statt \textsf{scrpage2} zu verwenden.
%    \begin{macrocode}
\newcommand*{\pagemark}{{\usekomafont{pagenumber}%
%<letter>    \pagename\ %
    \thepage}}
%    \end{macrocode}
% \end{macro}
%
% Zur�ck zur Definition der Seitenstile. Hier gibt es erhebliche Unterschiede
% zwischen der Briefklasse und den anderen Klassen. Das beginnt schon damit,
% dass es spezielle Stile f�r doppelseitige Briefe nicht gibt.
%    \begin{macrocode}
%<*!letter>
\if@twoside
  \renewcommand*{\ps@plain}{%
    \renewcommand*{\@evenhead}{}%
    \renewcommand*{\@oddhead}{}%
    \renewcommand*{\@evenfoot}{%
      \set@tempdima@hw\hss\hb@xt@ \@tempdima{\vbox{%
          \if@fsl \hrule \vskip 3\p@ \fi
          \hb@xt@ \@tempdima{{\pagemark\hfil}}}}}%
    \renewcommand*{\@oddfoot}{%
      \set@tempdima@hw\hb@xt@ \@tempdima{\vbox{%
          \if@fsl \hrule \vskip 3\p@ \fi
          \hb@xt@ \@tempdima{{\hfil\pagemark}}}}\hss}%
  }%
  \newcommand*{\ps@headings}{\let\@mkboth\markboth
    \renewcommand*{\@evenhead}{%
      \set@tempdima@hw\hss\hb@xt@ \@tempdima{\vbox{%
          \hb@xt@ \@tempdima{{\headfont\strut\leftmark\hfil}}%
          \if@hsl \vskip 1.5\p@ \hrule \fi}}}%
    \renewcommand*{\@oddhead}{%
      \set@tempdima@hw\hb@xt@ \@tempdima{\vbox{%
          \hb@xt@ \@tempdima{{\headfont\hfil\strut\rightmark}}%
          \if@hsl \vskip 1.5\p@ \hrule \fi}}\hss}%
    \renewcommand*{\@evenfoot}{%
      \set@tempdima@hw\hss\hb@xt@ \@tempdima{\vbox{%
          \if@fsl \hrule \vskip 3\p@ \fi
          \hb@xt@ \@tempdima{{\pagemark\hfil}}}}}%
    \renewcommand*{\@oddfoot}{%
      \set@tempdima@hw\hb@xt@ \@tempdima{\vbox{%
          \if@fsl \hrule \vskip 3\p@ \fi
          \hb@xt@ \@tempdima{{\hfil\pagemark}}}}\hss}%
%<*article>
    \renewcommand*{\sectionmark}[1]{%
      \markboth{\ifnum \c@secnumdepth >\z@%
          \sectionmarkformat\fi ##1}{}}%
    \renewcommand*{\subsectionmark}[1]{%
      \markright{\ifnum \c@secnumdepth >\@ne%
          \subsectionmarkformat\fi ##1}}%
%</article>
%<*report|book>
    \renewcommand*{\chaptermark}[1]{%
      \markboth{\ifnum \c@secnumdepth >\m@ne
%<book>          \if@mainmatter
            \chaptermarkformat
%<book>          \fi
        \fi
        ##1}{}%
    }%
    \renewcommand*{\sectionmark}[1]{%
      \markright{\ifnum \c@secnumdepth >\z@
          \sectionmarkformat\fi
        ##1}}%
%</report|book>
  }%
  \newcommand*{\ps@myheadings}{\let\@mkboth\@gobbletwo
    \renewcommand*{\@evenhead}{%
      \set@tempdima@hw\hss\hb@xt@ \@tempdima{\vbox{%
          \hb@xt@ \@tempdima{{\headfont\strut\leftmark\hfil}}%
          \if@hsl \vskip 1.5\p@ \hrule \fi}}}%
    \renewcommand*{\@oddhead}{%
      \set@tempdima@hw\hb@xt@ \@tempdima{\vbox{%
          \hb@xt@ \@tempdima{{\headfont\hfil\strut\rightmark}}%
          \if@hsl \vskip 1.5\p@ \hrule \fi}}\hss}%
    \renewcommand*{\@evenfoot}{%
      \set@tempdima@hw\hss\hb@xt@ \@tempdima{\vbox{%
          \if@fsl \hrule \vskip 3\p@ \fi
          \hb@xt@ \@tempdima{{\pagemark\hfil}}}}}%
    \renewcommand*{\@oddfoot}{%
      \set@tempdima@hw\hb@xt@ \@tempdima{\vbox{%
          \if@fsl \hrule \vskip 3\p@ \fi
          \hb@xt@ \@tempdima{{\hfil\pagemark}}}}\hss}%
%<!article>    \renewcommand*{\chaptermark}[1]{}%
%<article>    \renewcommand*{\subsectionmark}[1]{}%
    \renewcommand*{\sectionmark}[1]{}%
  }
\else
%</!letter>
  \renewcommand*{\ps@plain}{%
    \renewcommand*{\@oddhead}{%
%<*letter>
      \vbox{\vbox{\hsize=\textwidth\hbox to\textwidth{%
            \parbox[b]{\textwidth}{\strut
              \ifnum\@pageat>-1
                \ifnum\@pageat<3
                  \ifcase\@pageat\raggedright\or\centering\or\raggedleft\fi
                  \pagemark
                \else
                  \hfill
                \fi
              \else
                \hfill
              \fi
            }%
          }%
          \if@hsl\kern1pt\rule{\textwidth}{.4pt}\fi
        }%
      }%
%</letter>
    }%
    \let\@evenhead\@oddhead%
    \renewcommand*{\@oddfoot}{%
%<*letter>
      \parbox[t]{\textwidth}{%
        \if@fsl
          {%
            \raggedright%
            \vskip-\baselineskip\vskip.4pt
            \hrulefill\\
          }%
        \fi
        \ifnum\@pageat>2
          \ifcase\@pageat\or\or\or\raggedright\or\centering\or\raggedleft\fi
          \strut\pagemark
        \else
          \hfill
        \fi
      }%
%</letter>
%<*!letter>
      \set@tempdima@hw\hb@xt@ \@tempdima{\vbox{%
          \if@fsl \hrule \vskip 3\p@ \fi
          \hb@xt@ \@tempdima{{\hfil\pagemark\hfil}}}}\hss}%
%</!letter>
    }%
    \let\@evenfoot\@oddfoot
  }
  \newcommand*{\ps@headings}{\let\@mkboth\markboth
    \renewcommand*{\@oddhead}{%
%<*letter>
      \vbox{%
        \vbox{\hsize=\textwidth\hbox to\textwidth{\headfont\@nexthead}}%
        \if@hsl\kern1pt\rule{\textwidth}{.4pt}\fi%
      }%
%</letter>
%<*!letter>
      \set@tempdima@hw\hb@xt@ \@tempdima{\vbox{%
          \hb@xt@ \@tempdima{{\headfont\hfil\strut\rightmark\hfil}}
          \if@hsl \vskip 1.5\p@ \hrule \fi}}\hss
%</!letter>
    }%
    \let\@evenhead\@oddhead
    \renewcommand*{\@oddfoot}{%
%<*letter>
      \parbox[t]{\textwidth}{%
        \if@fsl
          {%
            \raggedright%
            \vskip-\baselineskip\vskip.4pt
            \hrulefill\\
            }%
        \fi
        \vbox{\hsize=\textwidth\hbox to\textwidth{\headfont\@nextfoot}}%
      }%
%</letter>
%<*!letter>
      \set@tempdima@hw\hb@xt@ \@tempdima{\vbox{%
          \if@fsl \hrule \vskip 3\p@ \fi
          \hb@xt@ \@tempdima{{\hfil\pagemark\hfil}}}\hss}}%
%<*article>
%    \end{macrocode}
%    \changes{v2.1b}{1994/12/31}{im einseitigen Seitenstil
%      \cs{markboth} durch \cs{markright} ersetzt}
%     \changes{v2.2a}{1995/02/07}{im einseitigen Seitenstil
%       bei \textsf{scrartcl} und \texttt{oneside} \cs{subsectionmark}
%       eingef�gt}
%     \changes{v2.2a}{1995/02/07}{im einseitigen Seitenstil bei
%       \textsf{scrreprt} und \texttt{oneside} \cs{sectionmark}
%       eingef�gt}
%    \begin{macrocode}
    \renewcommand*{\subsectionmark}[1]{}%
    \renewcommand*{\sectionmark}[1]{%
      \markright{\ifnum \c@secnumdepth >\z@\sectionmarkformat\fi
        ##1}}%
%</article>
%    \end{macrocode}
%    \changes{v2.0e}{1994/08/17}{im einseitigen Seitenstil
%      \cs{markboth} durch \cs{markright} ersetzt} 
%    \begin{macrocode}
%<*report|book>
    \renewcommand*{\sectionmark}[1]{}%
    \renewcommand*{\chaptermark}[1]{%
      \markright{\ifnum \c@secnumdepth >\m@ne
%<book>          \if@mainmatter
            \chaptermarkformat
%<book>          \fi
        \fi
        ##1}}%
%</report|book>
%</!letter>
    \let\@evenfoot\@oddfoot
  }
  \newcommand*{\ps@myheadings}{%
%<letter>    \ps@headings
    \let\@mkboth\@gobbletwo
%<*!letter>
    \renewcommand*{\@evenhead}{}%
    \renewcommand*{\@oddhead}{%
      \set@tempdima@hw\hb@xt@ \@tempdima{\vbox{%
          \hb@xt@ \@tempdima{{\headfont\hfil\strut\rightmark\hfil}}%
          \if@hsl \vskip 1.5\p@ \hrule \fi}}\hss}%
    \renewcommand*{\@evenfoot}{}%
    \renewcommand*{\@oddfoot}{%
      \set@tempdima@hw\hb@xt@ \@tempdima{\vbox{%
          \if@fsl \hrule \vskip 3\p@ \fi
          \hb@xt@ \@tempdima{{\hfil\pagemark\hfil}}}}\hss}%
%<article>    \renewcommand*{\subsectionmark}[1]{}%
%<!article>    \renewcommand*{\chaptermark}[1]{}%
    \renewcommand*{\sectionmark}[1]{}%
%</!letter>
  }
%<!letter>\fi
%    \end{macrocode}
%  \end{macro}
%  \end{macro}
%  \end{macro}
%
%
% \subsection{Festlegung des Seitenstils auf besonderen Seiten}
%
% So etwas gibt es derzeit nicht f�r Briefe.
% \iffalse
%<*!letter>
% \fi
%
%  \begin{macro}{\titlepagestyle}
%    \changes{v2.8d}{2001/07/05}{neu}
%  \begin{macro}{\partpagestyle}
%    \changes{v2.8d}{2001/07/05}{neu}
%  \begin{macro}{\chapterpagestyle}
%    \changes{v2.8d}{2001/07/05}{neu}
%  \begin{macro}{\indexpagestyle}
%    \changes{v2.8d}{2001/07/05}{neu}
% Auf verschiedenen Seiten wird automatisch mit \cs{thispagestyle} auf
% einen anderen Seitenstil umgeschaltet. Welcher das ist, ist in
% diesem Makros abgelegt und kann bei Bedarf ge�ndert werden. 
% Voreingestellt ist der in fr�heren Versionen fest verdrahtete
% Seitenstil \texttt{plain}. 
%    \begin{macrocode}
\newcommand*{\titlepagestyle}{plain}
\newcommand*{\partpagestyle}{plain}
%<book|report>\newcommand*{\chapterpagestyle}{plain}
\newcommand*{\indexpagestyle}{plain}
%    \end{macrocode}
%  \end{macro}
%  \end{macro}
%  \end{macro}
%  \end{macro}
%
% \iffalse
%</!letter>
% \fi
%
% \subsection{Standardeinstellungen}
%
% Der voreingestellte Seitenstil h�ngt von der verwendeten Klasse ab:
%    \begin{macrocode}
%<report|article|letter>\pagestyle{plain}
%<book>\pagestyle{headings}
%    \end{macrocode}
% Die Nummerierung erfolgt hingegen immer mit arabischen Zahlen:
%    \begin{macrocode}
\pagenumbering{arabic}
%    \end{macrocode}
%
% Im zweiseitigen Satz wird die letzte Zeile b�ndig gesetzt, im
% einseitigen und bei Briefen generell jedoch nicht:
%    \begin{macrocode}
%<*!letter>
\if@twoside
  \flushbottom
\else
%</!letter>
  \raggedbottom
%<!letter>\fi
%    \end{macrocode}
% Im zweispaltigen Satz wird \cs{sloppy} verwendet und die letzte
% Zeile jeweils b�ndig gesetzt. Briefe sind einspaltig.
%    \begin{macrocode}
%<*!letter>
\if@twocolumn
  \twocolumn
  \sloppy
  \flushbottom
\else
%</!letter>
  \onecolumn
%<!letter>\fi
%    \end{macrocode}
%
%
% \iffalse
%</body>
% \fi
%
% \Finale
%
\endinput
%
% end of file `scrkpage.dtx'
%%% Local Variables:
%%% mode: doctex
%%% TeX-master: t
%%% End:
