% \CheckSum{252}
% \iffalse meta-comment
% ======================================================================
% scrklist.dtx
% Copyright (c) Markus Kohm, 2002-2006
%
% This file is part of the LaTeX2e KOMA-Script bundle.
%
% This work may be distributed and/or modified under the conditions of
% the LaTeX Project Public License, version 1.3b of the license.
% The latest version of this license is in
%   http://www.latex-project.org/lppl.txt
% and version 1.3b or later is part of all distributions of LaTeX 
% version 2005/12/01 and of this work.
%
% This work has the LPPL maintenance status "author-maintained".
%
% The Current Maintainer and author of this work is Markus Kohm.
%
% This work consists of all files listed in manifest.txt.
% ----------------------------------------------------------------------
% scrklist.dtx
% Copyright (c) Markus Kohm, 2002-2006
%
% Dieses Werk darf nach den Bedingungen der LaTeX Project Public Lizenz,
% Version 1.3b.
% Die neuste Version dieser Lizenz ist
%   http://www.latex-project.org/lppl.txt
% und Version 1.3b ist Teil aller Verteilungen von LaTeX
% Version 2005/12/01 und dieses Werks.
%
% Dieses Werk hat den LPPL-Verwaltungs-Status "author-maintained"
% (allein durch den Autor verwaltet).
%
% Der Aktuelle Verwalter und Autor dieses Werkes ist Markus Kohm.
% 
% Dieses Werk besteht aus den in manifest.txt aufgefuehrten Dateien.
% ======================================================================
% \fi
% \iffalse
%%% From File: scrklist.dtx
%<*driver>
% \fi
\ProvidesFile{scrklist.dtx}[%
  2002/07/01 v3.0 KOMA-Script (lists and tabulars)]
% \iffalse
\documentclass[halfparskip-]{scrdoc}
\usepackage[english,german]{babel}
\usepackage[latin1]{inputenc}
\CodelineIndex
\RecordChanges
\GetFileInfo{scrklist.dtx}
\title{\KOMAScript{} \partname\ \texttt{\filename}%
  \footnote{Dies ist Version \fileversion\ von Datei
    \texttt{\filename}.}}
\date{\filedate}
\author{Markus Kohm}

\begin{document}
  \maketitle
  \tableofcontents
  \DocInput{\filename}
\end{document}
%</driver>
% \fi
%
% \selectlanguage{german}
%
% \changes{v3.0}{2002/07/01}{%
%   erste Version aus der Aufteilung von \texttt{scrclass.dtx}}
%
% \section{Listen und Tabellen}
%
% \KOMAScript{} verf�gt �ber einige zus�tzliche Listenumgebungen. Ein
% Teil dieser Listenumgebungen ist nicht sofort auf den ersten Blick
% als solche zu erkennen, so wie dies schon bei den Standardklassen
% beispielsweise mit \texttt{quote} und \texttt{quotation} der Fall
% ist.
%
% \StopEventually{\PrintIndex\PrintChanges}
%
% \iffalse
%<*option>
% \fi
%
% \subsection{Optionen}
% Diese Umgebungen werden nicht durch Optionen beeinflusst.
%
%
% \iffalse
%</option>
%<*body>
% \fi
%
%
% \subsection{R�nder in Listen}
%
%  \begin{Length}{leftmargini}
%  \begin{Length}{leftmarginii}
%  \begin{Length}{leftmarginiii}
%  \begin{Length}{leftmarginiv}
%  \begin{Length}{leftmarginv}
%  \begin{Length}{leftmarginvi}
%  \begin{Length}{leftmargin}
%  \begin{Length}{labelsep}
%  \begin{Length}{labelwidth}
% F�r die Listenumgebungen in verschiedenen Stufen m�ssen R�nder
% definiert werden. Hinzu kommen noch die Breite eines Labels und der
% Abstand nach einem Label.
%    \begin{macrocode}
%<*!letter>
\if@twocolumn
  \setlength{\leftmargini}{2em}
\else
%</!letter>
  \setlength{\leftmargini}{2.5em}
%<!letter>\fi
\setlength{\leftmarginii}{2.2em}
\setlength{\leftmarginiii}{1.87em}
\setlength{\leftmarginiv}{1.7em}
%<*!letter>
\if@twocolumn
  \setlength{\leftmarginv}{.5em}
  \setlength{\leftmarginvi}{.5em}
\else
%</!letter>
  \setlength{\leftmarginv}{1em}
  \setlength{\leftmarginvi}{1em}
%<!letter>\fi
\setlength{\leftmargin}{\leftmargini}
\setlength{\labelsep}{.5em}
\setlength{\labelwidth}{\leftmargini}
\addtolength{\labelwidth}{-\labelsep}
%    \end{macrocode}
%  \end{Length}
%  \end{Length}
%  \end{Length}
%  \end{Length}
%  \end{Length}
%  \end{Length}
%  \end{Length}
%  \end{Length}
%  \end{Length}
%
% Weil die Gelegenheit so g�nstig ist, setzen wir hier auch gleich
% noch ein paar \emph{penalties} (Erlaubsnispunkte).
%    \begin{macrocode}
\@beginparpenalty=-\@lowpenalty
\@endparpenalty  =-\@lowpenalty
\@itempenalty    =-\@lowpenalty
%    \end{macrocode}
%
%
% \subsection{Verschiedene L�ngen f�r Tabellen, Arrays und Tabulatoren}
% 
% Vorbemerkung: Arrays sind im Prinzip Tabellen zur Verwendung im
% mathemathischen Modus.
%
%  \begin{Length}{arraycolsep}
%  \begin{Length}{tabcolsep}
%  \begin{Length}{arrayrulewidth}
%  \begin{Length}{doublerulewidth}
%  \begin{Length}{tabbingsep}
% Bei diesen L�ngen handelt es sich um die Abst�nde der Spalten in
% \texttt{array}- und \texttt{tabular}-Umgebungen, sowie um die Breite
% und den Abstand der Trennlinien. Hinzu kommt noch der \cs{'} Abstand
% in \texttt{tabbing}-Umgebungen.
%    \begin{macrocode}
\setlength\arraycolsep{5\p@}
\setlength\tabcolsep{6\p@}
\setlength\arrayrulewidth{.4\p@}
\setlength\doublerulesep{2\p@}
\setlength\tabbingsep{\labelsep}
%    \end{macrocode}
%  \end{Length}
%  \end{Length}
%  \end{Length}
%  \end{Length}
%  \end{Length}
%
%
% \subsection{Marken in Listen}
%
%  \begin{macro}{\theenumi}
%  \begin{macro}{\theenumii}
%  \begin{macro}{\theenumiii}
%  \begin{macro}{\theenumiv}
%  \begin{macro}{\labelenumi}
%  \begin{macro}{\labelenumii}
%    \changes{v2.4m}{1997/02/28}{�ffnende Klammer entsprechend Doku
%      entfernt}
%  \begin{macro}{\labelenumiii}
%  \begin{macro}{\labelenumiv}
%  \begin{macro}{\p@enumii}
%  \begin{macro}{\p@enumiii}
%  \begin{macro}{\p@enumiv}
%  \begin{macro}{\labelitemi}
%    \changes{v2.5g}{1999/10/09}{Mathemodus eliminiert}
%  \begin{macro}{\labelitemii}
%    \changes{v2.5g}{1999/10/09}{\cs{textendash} verwendet}
%  \begin{macro}{\labelitemiii}
%    \changes{v2.5g}{1999/10/09}{Mathemodus eliminiert}
%  \begin{macro}{\labelitemiv}
%    \changes{v2.5g}{1999/10/09}{Mathemodus eliminiert}
% F�r die numerierten Listenumgebungen m�ssen Marken bzw. Numerierungen
% festgelegt werden. Dazu wird eine Darstellung der Z�hler, ihr Prefix
% und ihre Labeldarstellung neu definiert.
%    \begin{macrocode}
\renewcommand*\theenumi{\@arabic\c@enumi}
\renewcommand*\theenumii{\@alph\c@enumii}
\renewcommand*\theenumiii{\@roman\c@enumiii}
\renewcommand*\theenumiv{\@Alph\c@enumiv}
\newcommand*\labelenumi{\theenumi.}
\newcommand*\labelenumii{\theenumii)}
\newcommand*\labelenumiii{\theenumiii.}
\newcommand*\labelenumiv{\theenumiv.}
\renewcommand*\p@enumii{\theenumi}
\renewcommand*\p@enumiii{\p@enumii(\theenumii)}
\renewcommand*\p@enumiv{\p@enumiii\theenumiii}
\newcommand*\labelitemi{\textbullet}
\newcommand*\labelitemii{\normalfont\bfseries\textendash}
\newcommand*\labelitemiii{\textasteriskcentered}
\newcommand*\labelitemiv{\textperiodcentered}
%    \end{macrocode}
%  \end{macro}
%  \end{macro}
%  \end{macro}
%  \end{macro}
%  \end{macro}
%  \end{macro}
%  \end{macro}
%  \end{macro}
%  \end{macro}
%  \end{macro}
%  \end{macro}
%  \end{macro}
%  \end{macro}
%  \end{macro}
%  \end{macro}
%
%
% \subsection{Definition der Umgebungen}
%
%  \begin{environment}{description}
% Die \texttt{description}-Umgebung dient der Beschreibung von
% einzelnen Begriffen. Der Begriff aus dem optionalen Argument des
% \cs{item}-Befehls wird in einem speziellen Font, dem \cs{descfont},
% gesetzte.
%    \begin{macrocode}
\newenvironment{description}{%
  \list{}{\labelwidth\z@ \itemindent-\leftmargin
    \let\makelabel\descriptionlabel}%
}{%
  \endlist
}
\newcommand*{\descriptionlabel}[1]{%
  \hspace{\labelsep}\descfont #1%
}
%    \end{macrocode}
%  \end{environment}
%
%  \begin{environment}{labeling}
%  \begin{macro}{labelinglabel}
% Die \cs{labeling}-Umgebung ist eine Erweiterung des \textsf{Script}
% Pakets. Sie erwartet ein optionales und ein normales Argument. Das
% optionale Argument beschreibt einen speziellen Trenntext zwischen
% \cs{item}-Marke und \cs{item}-Beschreibung. Das eigentliche Argument
% wird f�r die Ermittlung der Einr�ckung des Beschreibungstextes
% bzw. des Trenntextes ben�tigt.
%    \begin{macrocode}
\newenvironment{labeling}[2][]{%
  \def\sc@septext{#1}%
  \list{}{\settowidth{\labelwidth}{#2#1}%
    \leftmargin\labelwidth \advance\leftmargin by \labelsep
    \let\makelabel\labelinglabel}%
}{%
  \endlist
}
\newcommand\labelinglabel[1]{#1\hfil\sc@septext}
%    \end{macrocode}
%  \end{macro}
%  \end{environment}
%
%  \begin{environment}{verse}
%    \changes{v2.3g}{1996/01/14}{\cs{item} muss keine "`[]"'-Klammern
%      mehr verarbeiten}
% Die Verse-Umgebung ist f�r Zitate in Gedichtform und �hnliches
% gedacht.
%    \begin{macrocode}
\newenvironment{verse}{%
  \let\\=\@centercr
  \list{}{\itemsep=\z@
    \itemindent=-1.5em
    \listparindent=\itemindent
    \rightmargin=\leftmargin
    \advance\leftmargin by1.5em
  }%
  \item\relax
}{%
  \endlist
}
%    \end{macrocode}
%  \end{environment}
%
%  \begin{environment}{quotation}
%    \changes{v2.3g}{1996/01/14}{\cs{item} muss keine "`[]"'-Klammern
%      mehr verarbeiten}
%  \begin{environment}{quote}
%    \changes{v2.3g}{1996/01/14}{\cs{item} muss keine "`[]"'-Klammern
%      mehr verarbeiten}
% Die \texttt{quotation}- und die \texttt{quote}-Umgebung erlauben
% rechts und links einger�ckte Passagen. Abs�tze werden entweder in
% der ersten Zeile zus�tzlich einger�ckt oder durch vertikalen Abstand
% markiert.
%    \begin{macrocode}
\newenvironment{quotation}{%
  \list{}{\listparindent 1em%
    \itemindent    \listparindent
    \rightmargin   \leftmargin
    \parsep        \z@ \@plus\p@
  }%
  \item\relax
}{%
  \endlist
}
\newenvironment{quote}{%
  \list{}{\rightmargin\leftmargin}%
  \item\relax
}{%
  \endlist
}
%    \end{macrocode}
%  \end{environment}
%  \end{environment}
%
%  \begin{environment}{addmargin}
%    \changes{v2.8q}{2001/11/29}{neue Umgebung}
%  \begin{environment}{addmargin*}
%    \changes{v2.8q}{2001/11/29}{neue Umgebung}
% Diese beiden Umgebungen �hneln \texttt{quote} und
% \texttt{quotation}. Dabei werden Absatzeinzug und Absatzabstand
% nicht ver�ndert. Die einzige Ver�nderung besteht in den R�ndern. Wie
% stark die R�nder ver�ndert werden, h�ngt dabei von den Parametern
% ab. Ist nur der obligatorische Parameter angegeben, werden die
% R�nder auf beiden Seiten um diesen Wert vergr��ert. Ist ein
% optionaler Parameter angegeben, so ist dies bei \texttt{addmargin}
% der linke und bei \texttt{addmargin*} der innere Rand. Der
% obligatorische Parameter ist dann der andere Rand. Diese
% Entscheidung ist das einzige, was die Umgebungen selbst erledigen
% m�ssen.
%  \begin{macro}{\@addmargin}
%    \changes{v2.8q}{2001/11/29}{neu (intern)}
%    \changes{v3.0}{2002/07/01}{\cs{item} muss keine "`[]"'-Klammern
%      mehr verarbeiten}
%    \changes{v2.9q}{2003/03/24}{\cs{labelsep} bleibt unver"andert}
% Der Rest wird von diesem Makro erledigt.
%    \begin{macrocode}
\newenvironment{addmargin}{%
  \@tempswafalse\@addmargin
}{%
  \endlist
}
\newenvironment{addmargin*}{%
  \@tempswafalse
  \if@twoside\ifthispageodd{}{\@tempswatrue}\fi
  \@addmargin
}{%
  \endlist
}
\newcommand*{\@addmargin}[2][\@tempa]{%
  \@tempcnta=\@listdepth
  \list{}{%
    \if@tempswa
      \def\@tempa{\leftmargin}%
      \setlength{\leftmargin}{#2}%
      \setlength{\rightmargin}{#1}%
    \else
      \def\@tempa{\rightmargin}%
      \setlength{\rightmargin}{#2}%
      \setlength{\leftmargin}{#1}%
    \fi
    \setlength{\listparindent}{\parindent}%
    \setlength{\itemsep}{\parskip}%
    \setlength{\itemindent}{\z@}%
    \setlength{\topsep}{\z@}%
    \setlength{\parsep}{\parskip}%
    \setlength{\partopsep}{\parskip}%
    \let\makelabel\@gobble
    \setlength{\labelwidth}{\z@}%
    \@listdepth=\@tempcnta
  }%
  \item\relax%
}
%    \end{macrocode}
%  \end{macro}
%  \end{environment}
%  \end{environment}
%
%
% \subsection{Schriftarten f�r Listen}
%
%  \begin{macro}{\descfont}
% Dies ist die Schriftart, in der das Label der Eintr�ge in eine
% \texttt{description}-Umgebung gesetzt wird. Das Makro ist als intern
% zu betrachten. Anwender habe stattdessen das entsprechende Element
% zu verwenden.
%    \begin{macrocode}
\newcommand*\descfont{\sffamily\bfseries}
%    \end{macrocode}
%  \end{macro}
%
%  \begin{macro}{\scr@fnt@descriptionlabel}
%    \changes{v2.8o}{2001/09/14}{neues Element
%      \texttt{descriptionlabel}}
% Das Element auf das \cs{descfont} angewandt wird:
%    \begin{macrocode}
\newcommand*{\scr@fnt@descriptionlabel}{\descfont}
%    \end{macrocode}
%  \end{macro}
%
%
% \iffalse
%</body>
% \fi
%
% \Finale
%
\endinput
%
% end of file `scrklist.dtx'
%%% Local Variables:
%%% mode: doctex
%%% TeX-master: t
%%% End:
