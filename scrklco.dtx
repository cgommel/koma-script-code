% \CheckSum{503}
% \iffalse meta-comment
% ======================================================================
% scrklco.dtx
% Copyright (c) Markus Kohm, 2006-2009
%
% This file is part of the LaTeX2e KOMA-Script bundle.
%
% This work may be distributed and/or modified under the conditions of
% the LaTeX Project Public License, version 1.3b of the license.
% The latest version of this license is in
%   http://www.latex-project.org/lppl.txt
% and version 1.3b or later is part of all distributions of LaTeX 
% version 2005/12/01 or later and of this work.
%
% This work has the LPPL maintenance status "author-maintained".
%
% The Current Maintainer and author of this work is Markus Kohm.
%
% This work consists of all files listed in manifest.txt.
% ----------------------------------------------------------------------
% scrklco.dtx
% Copyright (c) Markus Kohm, 2006-2009
%
% Dieses Werk darf nach den Bedingungen der LaTeX Project Public Lizenz,
% Version 1.3b, verteilt und/oder veraendert werden.
% Die neuste Version dieser Lizenz ist
%   http://www.latex-project.org/lppl.txt
% und Version 1.3b ist Teil aller Verteilungen von LaTeX
% Version 2005/12/01 oder spaeter und dieses Werks.
%
% Dieses Werk hat den LPPL-Verwaltungs-Status "author-maintained"
% (allein durch den Autor verwaltet).
%
% Der Aktuelle Verwalter und Autor dieses Werkes ist Markus Kohm.
% 
% Dieses Werk besteht aus den in manifest.txt aufgefuehrten Dateien.
% ======================================================================
% \fi
%
% \CharacterTable
%  {Upper-case    \A\B\C\D\E\F\G\H\I\J\K\L\M\N\O\P\Q\R\S\T\U\V\W\X\Y\Z
%   Lower-case    \a\b\c\d\e\f\g\h\i\j\k\l\m\n\o\p\q\r\s\t\u\v\w\x\y\z
%   Digits        \0\1\2\3\4\5\6\7\8\9
%   Exclamation   \!     Double quote  \"     Hash (number) \#
%   Dollar        \$     Percent       \%     Ampersand     \&
%   Acute accent  \'     Left paren    \(     Right paren   \)
%   Asterisk      \*     Plus          \+     Comma         \,
%   Minus         \-     Point         \.     Solidus       \/
%   Colon         \:     Semicolon     \;     Less than     \<
%   Equals        \=     Greater than  \>     Question mark \?
%   Commercial at \@     Left bracket  \[     Backslash     \\
%   Right bracket \]     Circumflex    \^     Underscore    \_
%   Grave accent  \`     Left brace    \{     Vertical bar  \|
%   Right brace   \}     Tilde         \~}
%
% \iffalse
%%% From File: scrklco.dtx
%<*driver>
% \fi
\ProvidesFile{scrklco.dtx}[2007/11/20 v2.97e KOMA-Script (letter class options)]
% \iffalse
\documentclass{scrdoc}
\usepackage[english,ngerman]{babel}
\usepackage[latin1]{inputenc}
\CodelineIndex
\RecordChanges
\GetFileInfo{scrklco.dtx}
\title{\KOMAScript{} \partname\ \texttt{\filename}%
  \footnote{Dies ist Version \fileversion\ von Datei \texttt{\filename}.}}
\date{\filedate}
\author{Markus Kohm}

\begin{document}
  \maketitle
  \tableofcontents
  \DocInput{\filename}
\end{document}
%</driver>
% \fi
%
% \selectlanguage{ngerman}
%
% \changes{v2.95}{2006/03/22}{%
%   erste Version aus der Aufteilung von \textsf{scrclass.dtx}}
%
% \section{Optionendateien f�r Briefe}
%
% Bei der Briefklasse tritt eine Besonderheit ein. Hier ist es
% m�glich, spezielle \texttt{lco}-Dateien zu definieren, die �ber
% Optionen geladen werden k�nnen. Die unterschiedlichen Brieftypen
% sind ein Fall f�r solche \emph{\textbf{l}etter \textbf{c}lass
% \textbf{o}ption} Dateien.
%
% \StopEventually{\PrintIndex\PrintChanges}
%
% \iffalse
%<*letter>
% \fi
%
% \iffalse
%<*option>
% \fi
%
% \subsection{Optionen f�r Letter-Class-Options}
%
% \begin{macro}{\lco@test}
%   \changes{v2.95}{2006/03/22}{Neu}
%   \changes{v2.97c}{2007/09/12}{an neue Optionenabarbeitung angepasst}
% Letter-Class-Optionen k�nnen auch f�r jede per |\documentclass| angegebene
% keyval-Option oder sonstige scan-Option (das sind Optionen, die nicht per
% |\DeclareOption| definiert wurden) geladen werden. Dabei wird die
% Abarbeitung der Option nicht unterbrochen. Dazu wird dieses Makro in den
% scan-Vorgang eingebaut
%    \begin{macrocode}
\newcommand*{\lco@test}{%
  \IfFileExists{\CurrentOption.lco}{%
    \expandafter\AtEndOfClass\expandafter{%
      \expandafter\LoadLetterOption\expandafter{\CurrentOption}%
    }%
    \expandafter\def\expandafter\scr@pti@nerr@r\expandafter{%
      \expandafter\def\expandafter\scr@pti@nerr@r\expandafter{%
        \scr@pti@nerr@r
      }%
    }%
  }{}%
}
%    \end{macrocode}
% \end{macro}
%
% Als Voreinstellung wird am Ende schon einmal die Datei \texttt{DIN.lco}
% geladen.
%    \begin{macrocode}
\AtEndOfClass{\LoadLetterOption{DIN}}
%    \end{macrocode}
%
% \iffalse
%</option>
%<*body>
% \fi
%
% \subsection{Makros f�r Letter-Class-Options}
%
% \begin{macro}{\LoadLetterOption}
%   \changes{v2.8q}{2001/10/08}{Neu}
% Bei der neuen Briefklasse existiert eine Anweisung, mit der eine
% \emph{\textbf{l}etter-\texttt{c}lass-\texttt{o}ption}-Datei geladen
% werden kann. Das Laden wird mit entsprechender Information in der
% \texttt{log}-Datei vermerkt. Damit in der Datei interne Makros
% verwendet werden k�nnen, muss der catcode von "`@"' entsprechend
% gesetzt sein. Damit der Befehl sowohl in der Pr�ambel als auch in
% weiteren Optionen-Dateien oder in der Klasse selbst verwendet werden
% kann, muss der catcode am Ende wieder restauriert werden. Dabei muss
% etwas trickreich gearbeitet werden, damit tats�chlich eine
% Schachtelung m�glich ist. So erfolgt die Umschaltung des catcodes
% immer nur beim ersten Aufruf innerhalb einer Schachtel.
%    \begin{macrocode}
\newcommand*{\LoadLetterOption}[1]{%
  \@ifundefined{@restore@catcode@level}%
    {\let\@restore@catcode@level=\z@}{}%
  \ifnum\@restore@catcode@level =\z@
    \ifnum\catcode`\@=11
      \let\@restore@catcode\relax
    \else
      \@tempcnta=\catcode`\@
      \edef\@restore@catcode{%
        \noexpand\catcode`\noexpand\@=\the\@tempcnta}%
      \makeatletter
    \fi
  \fi
  \@tempcnta=\@restore@catcode@level\relax
  \advance\@tempcnta by \@ne\relax
  \edef\@restore@catcode@level{\the\@tempcnta}%
  \InputIfFileExists{#1.lco}{%
    \ClassInfo{%
      scrlttr2%
      }{Letter-Class-Option `#1' loaded}%
    }{%
    \ClassError{%
      scrlttr2%
      }{%
      Letter-Class-Option file `#1.lco' not found%
      }{%
      You've told me to load the Letter-Class-Option `#1'. So I have
      to load\MessageBreak
      the file `#1.lco'. But the file isn't available.}%
    }%
  \@tempcnta=\@restore@catcode@level\relax
  \advance\@tempcnta by \m@ne\relax
  \edef\@restore@catcode@level{\the\@tempcnta}%
  \ifnum\@tempcnta =\z@
    \@restore@catcode
  \fi
  }
%    \end{macrocode} 
% \end{macro}
%
% \begin{macro}{\LetterOptionNeedsPapersize}
%   \changes{v2.8q}{2001/10/17}{Neu}
% Es besteht die M�glichkeit, in \texttt{lco}-Dateien eine bestimmte
% Papiergr��e zu fordern. Dazu wird der Name der \texttt{lco}-Datei
% und die geforderte Papiergr�se als Parameter �bergeben. Stimmt die
% Papiergr��e nicht �berein, so wird bei \cs{opening} eine Warnung
% ausgegeben.
% \begin{macro}{\@PapersizeWarning}
%   \changes{v2.8q}{2002/03/25}{Neu (intern)}
% \begin{macro}{\@@PapersizeWarning}
%   \changes{v2.8q}{2002/03/25}{Neu (intern)}
% \begin{macro}{\LOPNP@size}
%   \changes{v2.8q}{2002/03/25}{Neu (intern)}
% \begin{macro}{\LOPNP@option}
%   \changes{v2.8q}{2002/03/25}{Neu (intern)}
% Zur einfacheren Realisierung wird mit vier internen Hilfsmakros
% gearbeitet.
%    \begin{macrocode}
\newcommand*{\LOPNP@option}{}
\newcommand*{\LOPNP@size}{}
\newcommand*{\@PapersizeWarning}{%
  \begingroup%
    \edef\@tempc{\noexpand\@tempswatrue
      \noexpand\ifdim\paperwidth=\the\paperwidth\noexpand\else
      \noexpand\@tempswafalse\noexpand\fi
      \noexpand\ifdim\paperheight=\the\paperheight\noexpand\else
      \noexpand\@tempswafalse\noexpand\fi
      \noexpand\@@PapersizeWarning
      }  
    \def\@tempb{letter}\ifx\LOPNP@size\@tempb%
      \setlength{\paperheight}{11in}\setlength{\paperwidth}{8.5in}%
    \else\def\@tempb{legal}\ifx\LOPNP@size\@tempb%
      \setlength{\paperheight}{14in}\setlength{\paperwidth}{8.5in}%
    \else\def\@tempb{executive}\ifx\LOPNP@size\@tempb%
      \setlength{\paperheight}{10.5in}\setlength{\paperwidth}{7.25in}%
    \else%
      \def\ta@opt@err{%
        \ClassError{%
          scrlttr2%
          }{papersize `\LOPNP@size' undefined}{%
          You've told me to check for paper size
          `\LOPNP@size'\MessageBreak
          at letter option file `\LOPNP@option.lco',\MessageBreak
          but this paper size is not supported.\MessageBreak
          See the KOMA-Script manual for informations about known
          paper sizes.}%
        }%
      \let\@headlines\ta@opt@err%
      \expandafter\x@@paper\LOPNP@size paperpaper\x@@paper%
    \fi\fi\fi
    \@tempc
  \endgroup%
}
\newcommand*{\@@PapersizeWarning}{%
  \if@tempswa\else%
  \ClassWarningNoLine{%
    scrlttr2%
    }{%
    Letter option file `\LOPNP@option.lco'\MessageBreak
    needs paper size `\LOPNP@size'.\MessageBreak
    Current paper size is not `\LOPNP@size'!\MessageBreak
    You should add `paper=\LOPNP@size' at the\MessageBreak
    option list of \string\documentclass!\MessageBreak
    Maybe you know what you are doing,\MessageBreak
    so I do not change this myself}%
  \fi
  }
%    \end{macrocode}
% \end{macro}
% \end{macro}
% \end{macro}
% \end{macro}
%    \begin{macrocode}
\newcommand*{\LetterOptionNeedsPapersize}[2]{%
  \edef\LOPNP@size{#2}\edef\LOPNP@option{#1}%
  }
%    \end{macrocode}
% \end{macro}
%
% \iffalse
%</body>
%</letter>
% \fi
%
% \subsection{Die Letter-Class-Optionen}
%
% Die Letter-Class-Optionen selbst werden wie bereits erw�hnt �ber Dateien
% realisiert. In den vorgefertigen Dateien f�r unterschiedliche Arten von
% Briefen werden in erster Linie Pseudo-L�ngen passend gesetzt. In einigen
% F�llen werden aber auch zus�tzlich Befehle umdefiniert.
%
% \iffalse
%<*lco>
% \fi
%
% \subsubsection{Der Datei Kopf}
%
% \iffalse
%<*head>
% \fi
%
% Im Kopf ist nat�rlich die Datei zu deklarieren:
%    \begin{macrocode}
\ProvidesFile{%
%<DIN>  DIN%
%<DINmoretext> DINmtext%
%<SN>  SN%
%<SNold>  SNold%
%<SwissLeft>  SNleft%
%<KOMAold>  KOMAold%
  .lco}[\KOMAScriptVersion\space letter-class-option]
%    \end{macrocode}
%
% \begin{macro}{\LCOWarningNoLine}
%   \changes{v2.95}{2006/03/24}{Neu}
% \begin{macro}{\LCOWarning}
%   \changes{v2.95}{2006/03/24}{Neu}
% Au�erdem wird ein Makro ben�tigt, das analog zu |\ClassWarningNoLine| oder
% |\PackageWarningNoLine| funktioniert.
\providecommand*\LCOWarningNoLine[2]{%
  \LCOWarning{#1}{#2\@gobble}%
}
\providecommand*\LCOWarning[2]{%
  \GenericWarning{%
    (#1)\@spaces\@spaces\@spaces\@spaces\@spaces\@spaces\@spaces
  }{%
    Letter class option  #1 Warning: #2%
  }%
}
% \end{macro}
% \end{macro}
%
% \iffalse
%</head>
%<*body>
% \fi
%
% \subsubsection{Der Hauptteil}
%
% Die Ma�e f�r einige Optionen wurden nicht von mir selbst recherchiert:
% \begin{center}
%   \begin{tabular}{ll}
%   \texttt{lco}-Datei & Recherchiert von \\\hline\\[-1.6ex]
%   DIN                & mir \\
%   DINmtext           & mir \\
%   KOMAold            & mir \\
%   SNleft             & Roger Luethi \\
%   SN                 & Roger Luethi \\
%   \end{tabular}{ll}
% \end{center}
%
% Zun�chst wird gepr�ft, ob �berhaupt die richtige Klasse geladen ist.
%    \begin{macrocode}
\@ifundefined{scr@fromname@var}{%
  \LCOWarningNoLine{%
%<DIN>    DIN%
%<DINmoretext>   DINmtext%
%<SN>    SN%
%<SNold>    SNold%
%<SwissLeft>    SNleft%
%<KOMAold>    KOMAold%
  }{%
    This letter class option file was made only\MessageBreak
    to be used with KOMA-Script letter class\MessageBreak
    `scrlttr2'. Use with other classes may result\MessageBreak
    in a lot of errors%
  }%
}{}
%    \end{macrocode}
%
% Dann wird gepr�ft, ob das richtige Papierformat vorhanden ist.
%    \begin{macrocode}
%<*DIN|DINmoretext|SN|SNold|SwissLeft|KOMAold>
\LetterOptionNeedsPapersize{%
%<DIN>  DIN%
%<DINmoretext>  DINmtext%
%<SN>  SN%
%<SNold>  SNold%
%<SwissLeft>  SNleft%
%<KOMAold>  KOMAold%
}{a4}
%</DIN|DINmoretext|SN|SNold|SwissLeft|KOMAold>
%    \end{macrocode}
%
% Dann werden die Werte gesetzt.
% \changes{v2.9u}{2005/03/05}{Der Fu� hat optional eine feste Position 16\,mm
%   �ber der unteren Papierkante.}
% \changes{v2.97e}{2007/11/20}{Die neuen Faltmarken werden ber�cksichtigt.}
%    \begin{macrocode}
\@setplength{foldmarkhpos}{3.5mm}
%<DIN|SwissLeft|SN>\@setplength{tfoldmarkvpos}{105mm}
%<DINmoretext>\@setplength{tfoldmarkvpos}{87mm}
%<KOMAold>\@setplength{tfoldmarkvpos}{103.5mm}
\@setplength{mfoldmarkvpos}{\z@}
%<DIN|SwissLeft|SN>\@setplength{bfoldmarkvpos}{210mm}
%<DINmoretext>\@setplength{bfoldmarkvpos}{192mm}
%<KOMAold>\@setplength{bfoldmarkvpos}{202.5mm}
\@setplength{lfoldmarkhpos}{\z@}
%<DIN|SN>\@setplength{toaddrvpos}{45mm}
%<DINmoretext>\@setplength{toaddrvpos}{27mm}
%<SwissLeft>\@setplength{toaddrvpos}{35.5mm}
%<KOMAold>\@setplength{toaddrvpos}{49mm}
%<DIN|SN>\@setplength{refvpos}{98.5mm}
%<DINmoretext>\@setplength{refvpos}{80.5mm}
%<SwissLeft>\@setplength{refvpos}{89mm}
%<KOMAold>\@setplength{refvpos}{89.5mm}
%<!KOMAold>\@setplength{refaftervskip}{\baselineskip}
%<KOMAold>\@setplength[2]{refaftervskip}{\baselineskip}
%<DIN|DINmoretext|SwissLeft>\@setplength{toaddrhpos}{20mm}
%<SN>\@setplength{toaddrhpos}{-8mm}
%<KOMAold>\@setplength{toaddrhpos}{1in}
%<KOMAold>\@addtoplength{toaddrhpos}{\oddsidemargin}
%<DIN|DINmoretext>\@setplength{toaddrwidth}{85mm}
%<SN>\@setplength{toaddrwidth}{90mm}
%<SwissLeft>\@setplength{toaddrwidth}{100mm}
%<KOMAold>\@setplength{toaddrwidth}{70mm}
%<DIN|DINmoretext|SN|SwissLeft|KOMAold>\@setplength{toaddrheight}{40mm}
\@setplength{toaddrindent}{\z@}
\@setplength{specialmailindent}{\fill}
\@setplength{specialmailrightindent}{1em}
\@setplength{locwidth}{\z@}
%<DIN|DINmoretext|SN|SwissLeft|KOMAold>\@setplength{backaddrheight}{5mm}
%<DIN|DINmoretext|SN|SwissLeft>\@setplength{firstheadvpos}{8mm}
%<KOMAold>\@setplength{firstheadvpos}{6mm}
%<KOMAold>\@setplength{firstheadwidth}{\textwidth}
%<*!KOMAold>
\@setplength{firstheadwidth}{\paperwidth}
\ifdim \useplength{toaddrhpos}>\z@
  \@addtoplength[-2]{firstheadwidth}{\useplength{toaddrhpos}}
\else
  \@addtoplength[2]{firstheadwidth}{\useplength{toaddrhpos}}
\fi
%</!KOMAold>
\@setplength{firstfootwidth}{\useplength{firstheadwidth}}
%<KOMAold>\@setplength{firstfootvpos}{\paperheight}
%<KOMAold>\@addtoplength{firstfootvpos}{-2cm}
%<!KOMAold>\ifnum \scr@compatibility >\@nameuse{scr@v@2.9t}\relax
%<!KOMAold>  \@setplength{firstfootvpos}{\paperheight}
%<!KOMAold>  \@addtoplength{firstfootvpos}{-16mm}
%<!KOMAold>\else
%<!KOMAold>  \@setplength{firstfootvpos}{1in}
%<!KOMAold>  \@addtoplength{firstfootvpos}{\topmargin}
%<!KOMAold>  \@addtoplength{firstfootvpos}{\headheight}
%<!KOMAold>  \@addtoplength{firstfootvpos}{\headsep}
%<!KOMAold>  \@addtoplength{firstfootvpos}{\textheight}
%<!KOMAold>  \@addtoplength{firstfootvpos}{\footskip}
%<!KOMAold>\fi
%    \end{macrocode}
% Bei der Breite der Gesch�ftszeile brauchen wir wieder eine Sonderl�sung,
% damit die Option "`wide"' auch sp�ter noch gesetzt werden kann. Also
% verwenden wir hier den Wert 0 als mit der Bedeutung "`ist bei \cs{opening}
% zu berechnen.
%    \begin{macrocode}
\@setplength{refwidth}{0pt}
\@setplength{sigindent}{0mm}
\@setplength{sigbeforevskip}{2\baselineskip}
%<!KOMAold>\let\raggedsignature=\centering
%<KOMAold>\let\raggedsignature=\raggedright
%<*DINmoretext>
\renewcommand*{\@firsthead}{\noindent
  \parbox[b]{\useplength{firstheadwidth}}{\centering%
    \usekomafont{fromaddress}{%
      \ifkomavarempty{fromname}{}{%
        \usekomafont{fromname}%
        {\ignorespaces\usekomavar{fromname}\\[-8pt]
          \rule{\useplength{firstheadwidth}}{.4pt}\\
          }%
        }%
      \ifkomavarempty{fromaddress}{}{%
        \def\\{,\ \@ogobble}%
        \ignorespaces\usekomavar{fromaddress}%
        }%
      }%
    }%
  }
%</DINmoretext>
%    \end{macrocode}
%
% Au�erdem gibt es noch die Kompatibilit�tseinstellungen, die ebenfalls von
% \texttt{KOMAold} aktiviert werden.
%    \begin{macrocode}
%<*KOMAold>
\def\yourref{\setkomavar{yourref}}
\def\yourmail{\setkomavar{yourmail}}
\def\myref{\setkomavar{myref}}
\def\customer{\setkomavar{customer}}
\def\invoice{\setkomavar{invoice}}
\@ifundefined{scr@refitemi@var}{\newkomavar{refitemi}}{}
\def\refitemi{\setkomavar{refitemi}}
\def\refitemnamei{\setkomavar*{refitemi}}
\@ifundefined{scr@refitemii@var}{\newkomavar{refitemii}}{}
\def\refitemii{\setkomavar{refitemii}}
\def\refitemnameii{\setkomavar*{refitemii}}
\@ifundefined{scr@refitemiii@var}{\newkomavar{refitemiii}}{}
\def\refitemiii{\setkomavar{refitemiii}}
\def\refitemnameiii{\setkomavar*{refitemiii}}
\l@addto@macro{\defaultreffields}{%
  \addtoreffields{refitemi}%
  \addtoreffields{refitemii}%
  \addtoreffields{refitemiii}%
  }
\addtoreffields{refitemi}%
\addtoreffields{refitemii}%
\addtoreffields{refitemiii}%
\def\toname{\usekomavar{toname}}
\def\toaddress{\usekomavar{toaddress}}
\@ifundefined{scr@branch@var}{\newkomavar{branch}}{}
\def\branch{\setkomavar{branch}}
\def\frombranch{\usekomavar{branch}}
\def\signature{\setkomavar{signature}}
\def\fromsig{\usekomavar{signature}}
\def\name{\setkomavar{fromname}}
\def\fromname{\usekomavar{fromname}}
\def\address{\setkomavar{fromaddress}}
\def\fromaddress{\usekomavar{fromaddress}}
\def\location{\setkomavar{location}}
\def\fromlocation{\usekomavar{location}}
\def\backaddress{\setkomavar{backaddress}}
\def\@tempa{%
  \ifkomavarempty{fromname}{}{%
    \strut\ignorespaces\usekomavar{fromname}%
    \ifkomavarempty{fromaddress}{}{\\}}%
  \ifkomavarempty{fromaddress}{}{%
    \strut\ignorespaces\usekomavar{fromaddress}}%
  }
\ifx\@tempa\scr@backaddress@var%
  \setkomavar{backaddress}{}
\fi
\def\frombackaddress{\usekomavar{backaddress}}
\def\telephone{\setkomavar{fromphone}}
\def\telephonenum{\usekomavar{fromphone}}
\def\specialmail{\setkomavar{specialmail}}
\def\@specialmail{\usekomavar{specialmail}}
\def\title{\setkomavar{title}}
\def\@title{\usekomavar{title}}
\def\subject{\setkomavar{subject}}
\def\@subject{\usekomavar{subject}}
\def\place{\setkomavar{place}}
\def\fromplace{\usekomavar{place}}
\let\ccnameseparator=\scr@ccseparator@var
\setkomavar{ccseparator}{\ccnameseparator}
\let\enclnameseparator=\scr@enclseparator@var
\setkomavar{enclseparator}{\enclnameseparator}
\setkomafont{fromname}{\scshape}
\def\foldmarkson{\@ObsoleteCommand{\foldmarkson}{foldmarks=on}}
\def\foldmarksoff{\@ObsoleteCommand{\foldmarksoff}{foldmarks=off}}
\def\addrfieldon{\@ObsoleteCommand{\addrfieldon}{addrfield=on}}
\def\addrfieldoff{\@ObsoleteCommand{\addrfieldoff}{addrfield=off}}
\def\subjecton{\@ObsoleteCommand{\subjecton}{subject=titled}}
\def\subjectoff{\@ObsoleteCommand{\subjectoff}{subject=untitled}}
\def\subjectafteron{%
  \@ObsoleteCommand{\subjectafteron}{subject=afteropening}}
\def\subjectafteroff{%
  \@ObsoleteCommand{\subjectafteroff}{subject=beforeopening}}
%</KOMAold>
%    \end{macrocode}
%
% \iffalse
%</body>
%</lco>
% \fi
%
% \Finale
%
\endinput
%
% end of file `scrklco.dtx'
%%% Local Variables:
%%% mode: doctex
%%% TeX-master: t
%%% End:
