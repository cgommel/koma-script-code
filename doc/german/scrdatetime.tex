% ============================================================================
% scrdatetime.tex
% Copyright (c) Markus Kohm, 2001-2005
%
% This file is part of the LaTeX2e KOMA-Script-Bundle
%
% This file can be redistributed and/or modified under the terms
% of the LaTeX Project Public License Version 1.0 distributed
% together with this file. See LEGAL.TXT or LEGALDE.TXT.
% ----------------------------------------------------------------------------
% scrdatetime.tex
% Copyright (c) Markus Kohm, 2001-2005
%
% Diese Datei ist Teil des LaTeX2e KOMA-Script-Pakets.
%
% Diese Datei kann nach den Regeln der LaTeX Project Public
% Licence Version 1.0, wie sie zusammen mit dieser Datei verteilt
% wird, weiterverbreitet und/oder modifiziert werden. Siehe dazu
% auch LEGAL.TXT oder LEGALDE.TXT.
% ============================================================================
%
% Chapter about scrpage2 of the KOMA-Script guide
% Maintained by Markus Kohm
%
% ----------------------------------------------------------------------
%
% Kapitel �ber scrpage2 in der KOMA-Script-Anleitung
% Verwaltet von Markus Kohm
%
% ============================================================================

\ProvidesFile{scrdatetime.tex}[{2005/05/18 KOMA-Script guide (chapter:
  scrdate, scrtime)}]

\chapter{Wochentag und Uhrzeit mit \Package{scrdate} und \Package{scrtime}}
\label{cha:datetime}

Zu \KOMAScript{} geh�ren auch zwei Pakete, um den Umgang mit
Datum\Index{Datum} und Zeit\Index{Zeit} �ber die beiden
Standardbefehle \Macro{today} und \Macro{date} hinaus zu
erweitern. Ebenso wie die anderen Pakete aus {\KOMAScript} k�nnen
diese Pakete auch mit den Standardklassen verwendet werden.

\section{Der aktuelle Wochentag mit \Package{scrdate}}
\label{sec:datetime.scrdate}
\BeginIndex{Package}{scrdate}

Das erste Problem ist die Frage nach dem aktuellen Wochentag. Dieses kann mit
Hilfe des Pakets \Package{scrdate} gel�st werden.

\begin{Declaration}
  \Macro{todaysname}
\end{Declaration}%
\BeginIndex{Cmd}{todaysname}%
Bekanntlich erh�lt man mit \Macro{today}\IndexCmd{today} das aktuelle
Datum in der landestypischen Schreibweise. \Package{scrdate} bietet
mit \Macro{todaysname} eine Anweisung, um den aktuellen Wochentag zu
erhalten.

\begin{Example}
  Sie wollen in Ihrem Dokument ausgeben, an welchem Tag es mit
  \LaTeX{} in eine \File{DVI}-Datei �bersetzt wurde. Sie schreiben
  dazu
\begin{lstlisting}
  Dieses Dokument entstand an einem {\todaysname}.
\end{lstlisting}
  und erhalten beispielsweise:
  \begin{quote}
    Dieses Dokument entstand an einem {\todaysname}.
  \end{quote}
\end{Example}

Es sei darauf hingewiesen, dass das Paket nat�rlich nicht deklinieren
kann. Gespeichert sind die Wochentage im Nominativ Singular, wie er
beispielsweise f�r eine Datumsangabe in Briefen ben�tigt wird. Obiges Beispiel
funktioniert daher nur sprachabh�ngig und eher zuf�llig.

\begin{Explain}
  \textbf{Tipp:} Wenn Sie den Namen des Tages in Kleinbuchstaben
  ben�tigen, weil das in der entsprechenden Sprache innerhalb des
  Satzes so �blich ist, k�nnen Sie das erreichen, obwohl die Namen der
  Wochentage in \Package{scrdate} alle gro� geschrieben sind. In
  diesem Fall k�nnen Sie einfach auf die \LaTeX-Anweisung
  \Macro{MakeLowercase}\IndexCmd{MakeLowercase} zur�ckgreifen und
  \Macro{MakeLowercase}\PParameter{\Macro{todaysname}} schreiben.
\end{Explain}
\EndIndex{Cmd}{todaysname}

\begin{Declaration}
  \Macro{nameday}\Parameter{Name}
\end{Declaration}%
\BeginIndex{Cmd}{nameday}%
So wie mit \Macro{date}\IndexCmd{date} die Ausgabe von \Macro{today}
direkt ge�ndert werden kann, setzt \Macro{nameday} die Ausgabe von
\Macro{todaysname} auf den Wert \PName{Name}.
\begin{Example}
  Sie setzen mit \Macro{date} das aktuelle Datum auf einen festen
  Wert. F�r die Ausgabe des zugeh�rigen Wochentags interessiert es nur,
  dass dieser Tag ein Werktag war. Daher schreiben Sie
\begin{lstlisting}
  \nameday{Werktag}
\end{lstlisting}
  und erhalten so mit dem Satz aus dem vorherigen Beispiel:
  \begin{quote}\nameday{Werktag}
    Dieses Dokument entstand an einem {\todaysname}.
  \end{quote}
\end{Example}
\EndIndex{Cmd}{nameday}

Das \Package{scrdate}-Paket beherrscht derzeit die Sprachen Englisch
(english and USenglish), Deutsch (german, ngerman und austrian),
Franz�sisch (french), Italienisch (italian), Spanisch (spanish) und
Kroatisch (croatian), kann aber auch f�r andere Sprachen konfiguriert
werden. N�heres dazu entnehme man \File{scrdate.dtx}.

Bei der aktuellen Version ist es egal, ob \Package{scrdate} vor oder
nach \Package{german}\IndexPackage{german},
\Package{babel}\IndexPackage{babel} oder �hnlichen Paketen geladen
wird, in jedem Falle wird die korrekte Sprache
gew�hlt.

\begin{Explain}
  Etwas genauer ausgedr�ckt: Solange die Sprachauswahl in einer zu
  \Package{babel}\IndexPackage{babel} bzw.
  \Package{german}\IndexPackage{german} kompatiblen Form erfolgt und
  die Sprache \Package{scrdate} bekannt ist, wird die Sprache korrekt
  gew�hlt. Ist dies nicht der Fall, werden (US-)englische Ausdr�cke
  verwendet.
\end{Explain}
\EndIndex{Package}{scrdate}

\section{Die aktuelle Zeit mit \Package{scrtime}}
\label{sec:datetime.scrtime}
\BeginIndex{Package}{scrtime}

Das zweite Problem ist die Frage nach der aktuellen Zeit. Dieses kann mit
Hilfe des Pakets \Package{scrtime} gel�st werden.

\begin{Declaration}%
  \Macro{thistime}\OParameter{Trennung}\\
  \Macro{thistime*}\OParameter{Trennung}
\end{Declaration}%
\BeginIndex{Cmd}{thistime}\BeginIndex{Cmd}{thistime*}%
\Macro{thistime} liefert die aktuelle Zeit\Index{Zeit}. Als
Trennbuchstabe zwischen den Werten Stunden, Minuten und Sekunden
wird das optionale Argument \PName{Trennung} verwendet. Die
Voreinstellung ist hierbei das Zeichen "`\PValue{:}"'.

\Macro{thistime*} funktioniert fast genau wie \Macro{thistime}. Der
einzige Unterschied besteht darin, dass im Gegensatz zu
\Macro{thistime} bei \Macro{thistime*} die Minutenangaben bei Werten
kleiner 10 nicht durch eine vorangestellte Null auf zwei Stellen
erweitert wird.
\begin{Example}
  Die Zeile
\begin{lstlisting}
  Ihr Zug geht um \thistime\ Uhr.
\end{lstlisting}
  liefert als Ergebnisse beispielsweise eine Zeile wie
  \begin{quote}
    Ihr Zug geht um \thistime\ Uhr.
  \end{quote}
  oder
  \begin{quote}
    Ihr Zug geht um 23:09 Uhr.
  \end{quote}
  \bigskip
  Demgegen�ber liefert die Zeile
\begin{lstlisting}
  Beim n�chsten Ton ist es \thistime*[\ Uhr,\ ] 
  Minuten und 42 Sekunden.
\end{lstlisting}
  als m�gliches Ergebniss etwas wie:
  \begin{quote}
    Beim n�chsten Ton ist es \thistime*[\ Uhr,\ ] Minuten und 42 Sekunden.
  \end{quote}
  oder
  \begin{quote}
    Beim n�chsten Ton ist es 23\ Uhr,\ 9 Minuten und 42 Sekunden.
  \end{quote}
\end{Example}
\EndIndex{Cmd}{thistime}\EndIndex{Cmd}{thistime*}

\begin{Declaration}%
 \Macro{settime}\Parameter{Wert}
\end{Declaration}%
\BeginIndex{Cmd}{settime}%
\Macro{settime} setzt die Ausgabe von \Macro{thistime} und
\Macro{thistime*} auf einen festen \PName{Wert}%
%\footnote{Allerdings darf man nicht erwarten, dass nun die Zeit
%  stillsteht!}
. %
Anschlie�end wird das optionale Argument von \Macro{thistime} bzw.
\Macro{thistime*} ignoriert, da ja die komplette Zeichenkette, die
\Macro{thistime} bzw. \Macro{thistime*} nun liefert, hiermit explizit
festgelegt wurde.%
\EndIndex{Cmd}{settime}

\begin{Declaration}
  \Option{12h}\\
  \Option{24h}
\end{Declaration}%
\BeginIndex{Option}{12h}\BeginIndex{Option}{24h}%
Mit den Optionen \Option{12h} und \Option{24h} kann ausgew�hlt
werden, ob die Zeit bei \Macro{thistime} und \Macro{thistime*} im
12-Stunden- oder 24-Stunden-Format ausgegeben werden soll.
Voreingestellt ist \Option{24h}%
%\footnote{Leider beherrscht das \Package{scrtime}-Paket noch nicht die
%  Sternzeit nach \textsc{StarTrek}\Index{StarTrek}, ein echter
%  Mangel!}
. %
Die Option verliert bei einem Aufruf von \Macro{settime} ebenfalls ihre
G�ltigkeit.%
\EndIndex{Option}{12h}\EndIndex{Option}{24h}
\EndIndex{Package}{scrtime}

%%% Local Variables:
%%% mode: latex
%%% TeX-master: "../guide"
%%% End:
