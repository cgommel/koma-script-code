% ======================================================================
% common-15.tex
% Copyright (c) Markus Kohm, 2001-2011
%
% This file is part of the LaTeX2e KOMA-Script bundle.
%
% This work may be distributed and/or modified under the conditions of
% the LaTeX Project Public License, version 1.3c of the license.
% The latest version of this license is in
%   http://www.latex-project.org/lppl.txt
% and version 1.3c or later is part of all distributions of LaTeX 
% version 2005/12/01 or later and of this work.
%
% This work has the LPPL maintenance status "author-maintained".
%
% The Current Maintainer and author of this work is Markus Kohm.
%
% This work consists of all files listed in manifest.txt.
% ----------------------------------------------------------------------
% common-15.tex
% Copyright (c) Markus Kohm, 2001-2011
%
% Dieses Werk darf nach den Bedingungen der LaTeX Project Public Lizenz,
% Version 1.3c, verteilt und/oder veraendert werden.
% Die neuste Version dieser Lizenz ist
%   http://www.latex-project.org/lppl.txt
% und Version 1.3c ist Teil aller Verteilungen von LaTeX
% Version 2005/12/01 oder spaeter und dieses Werks.
%
% Dieses Werk hat den LPPL-Verwaltungs-Status "author-maintained"
% (allein durch den Autor verwaltet).
%
% Der Aktuelle Verwalter und Autor dieses Werkes ist Markus Kohm.
% 
% Dieses Werk besteht aus den in manifest.txt aufgefuehrten Dateien.
% ======================================================================
%
% Paragraphs that are common for several chapters of the KOMA-Script guide
% Maintained by Markus Kohm
%
% ----------------------------------------------------------------------
%
% Abs�tze, die mehreren Kapiteln der KOMA-Script-Anleitung gemeinsam sind
% Verwaltet von Markus Kohm
%
% ======================================================================

\ProvidesFile{common-15.tex}[2011/09/21 KOMA-Script guide (common paragraphs)]

\makeatletter
\@ifundefined{ifCommonmaincls}{\newif\ifCommonmaincls}{}%
\@ifundefined{ifCommonscrextend}{\newif\ifCommonscrextend}{}%
\@ifundefined{ifCommonscrlttr}{\newif\ifCommonscrlttr}{}%
\@ifundefined{ifIgnoreThis}{\newif\ifIgnoreThis}{}%
\makeatother


\section{Schlauer Spruch}
\label{sec:\csname label@base\endcsname.dictum}%
\ifshortversion\IgnoreThisfalse\IfNotCommon{maincls}{\IgnoreThistrue}\fi%
\ifIgnoreThis %+++++++++++++++++++++++++++++++++++++++++++++ nicht maincls +
Es gilt sinngem��, was in \autoref{sec:maincls.dictum} geschrieben
wurde. \IfCommon{scrextend}{Allerdings werden von \Package{scrextend} die
  Anweisungen \Macro{setchapterpreamble} und \Macro{setpartpreamble} nicht
  definiert. Ob die verwendete Klasse eine entsprechende Anweisung bietet, ist
  der Anleitung zur jeweiligen Klasse zu entnehmen.}
\else %------------------------------------------------------- nur maincls -
\BeginIndex{}{Spruch}%
\BeginIndex{}{Zitat}%
\BeginIndex{}{Redewendung}%

Ein h�ufiger anzutreffendes Element ist ein Zitat oder eine Redewendung, die
rechtsb�ndig unter oder �ber einer �berschrift gesetzt wird. Dabei werden der
Spruch selbst und der Quellennachweis in der Regel speziell formatiert.


\begin{Declaration}
  \Macro{dictum}\OParameter{Urheber}\Parameter{Spruch}\\
  \Macro{dictumwidth}\\
  \Macro{dictumauthorformat}\Parameter{Urheber}\\
  \Macro{dictumrule}\\
  \Macro{raggeddictum}\\
  \Macro{raggeddictumtext}\\
  \Macro{raggeddictumauthor}
\end{Declaration}%
\BeginIndex{Cmd}{dictum}%
\BeginIndex{Cmd}{dictumwidth}%
\BeginIndex{Cmd}{dictumauthorformat}%
\BeginIndex{Cmd}{dictumrule}%
\BeginIndex{Cmd}{raggeddictum}%
\BeginIndex{Cmd}{raggeddictumtext}%
\BeginIndex{Cmd}{raggeddictumauthor}%
Ein solcher Spruch kann mit Hilfe der Anweisung \Macro{dictum} gesetzt werden.
\IfCommon{maincls}{Bei\textnote{Tipp!} \KOMAScript-Klassen wird f�r Kapitel
  oder Teile empfohlen, \Macro{dictum} als obligatorisches Argument der
  Anweisung \Macro{setchapterpreamble} beziehungsweise \Macro{setpartpreamble}
  (siehe \autoref{sec:maincls.structure},
  \autopageref{desc:maincls.cmd.setchapterpreamble}) zu verwenden. Dies ist
  jedoch nicht zwingend.\par}%
Der Spruch wird\IfCommon{scrextend}{ hierzu} zusammen mit einem optional anzugebenden \PName{Urheber} in
einer \Macro{parbox}\IndexCmd{parbox} (siehe \cite{latex:usrguide}) der Breite
\Macro{dictumwidth} gesetzt. Dabei ist \Macro{dictumwidth} keine L�nge, die
mit \Macro{setlength} gesetzt wird. Es handelt sich um ein Makro, das mit
\Macro{renewcommand} umdefiniert werden kann. Vordefiniert ist
\lstinline;0.3333\textwidth;, also ein Drittel der jeweiligen Textbreite. Die
Box selbst wird mit der Anweisung \Macro{raggeddictum}
ausgerichtet. Voreingestellt ist dabei
\Macro{raggedleft}\IndexCmd{raggedleft}, also rechtsb�ndig.
\Macro{raggeddictum} kann mit Hilfe von \Macro{renewcommand} umdefiniert
werden.

Innerhalb der Box wird der \PName{Spruch} mit \Macro{raggeddictumtext}
angeordnet. Voreingestellt ist hier \Macro{raggedright}\IndexCmd{raggedright},
also linksb�ndig. Eine Umdefinierung ist auch hier mit \Macro{renewcommand}
m�glich. Die Ausgabe erfolgt in der f�r Element
\FontElement{dictumtext}\IndexFontElement[indexmain]{dictumtext}%
\important{\FontElement{dictumtext}} eingestellten Schriftart, die mit den
Anweisungen \Macro{setkomafont} und \Macro{addtokomafont} (siehe
\autoref{sec:\csname label@base\endcsname.textmarkup},
\autopageref{desc:\csname label@base\endcsname.cmd.setkomafont}) ge�ndert
werden kann. Die Voreinstellung entnehmen Sie bitte
\autoref{tab:maincls.dictumfont}%
\IfNotCommon{maincls}{, \autopageref{tab:maincls.dictumfont}}%
.

\ifCommonmaincls
\begin{table}
%  \centering%
%  \caption
  \KOMAoptions{captions=topbeside}%
  \setcapindent{0pt}%
  \begin{captionbeside}
    [{Schriftvoreinstellungen f�r die Elemente des
      Spruchs}]
    {\label{tab:maincls.dictumfont}\hspace{0pt plus 1ex}%
      Voreinstel\-lungen der Schrift f�r die Elemente des Spruchs}
    [l]
  \begin{tabular}[t]{ll}
    \toprule
    Element & Voreinstellung \\
    \midrule
    \FontElement{dictumtext} & 
    \Macro{normalfont}\Macro{normalcolor}\Macro{sffamily}\Macro{small}\\
    \FontElement{dictumauthor} &
    \Macro{itshape}\\
    \bottomrule
  \end{tabular}
  \end{captionbeside}
\end{table}
\fi

Ist ein \PName{Urheber} angegeben, so wird dieser mit einer Linie �ber die
gesamte Breite der \Macro{parbox} vom \PName{Spruch} abgetrennt.
Diese\IfCommon{maincls}{\ChangedAt{v3.10}{\Class{scrbook}\and
    \Class{scrreprt}\and \Class{scrartcl}}}%
\IfCommon{scrextend}{\ChangedAt{v3.10}{\Package{scrextend}}} Linie ist in
\Macro{dictumrule} definiert. Es handelt sich dabei um ein vertikales Objekt,
das mit
% Satzkorrektur listings
\begin{lstcode}[belowskip=\dp\strutbox]
  \newcommand*{\dictumrule}{\vskip-1ex\hrulefill\par}
\end{lstcode}
vordefiniert ist.

Mit \Macro{raggeddictumauthor} wird die Ausrichtung f�r die Linie und den
Urheber vorgenommen.  Voreingestellt ist \Macro{raggedleft}. Auch diese
Anweisung kann mit \Macro{renewcommand} umdefiniert werden. Die Ausgabe
erfolgt in der Form, die mit \Macro{dictumauthorformat} festgelegt ist. Das
Makro erwartet schlicht den \Macro{Urheber} als Argument. In der
Voreinstellung ist \Macro{dictumauthorformat} als
% Satzkorrektur listings
\begin{lstcode}[belowskip=\dp\strutbox]
  \newcommand*{\dictumauthorformat}[1]{(#1)}
\end{lstcode}
definiert. Der \PName{Urheber} wird also in runde Klammern gesetzt. F�r das
Element \FontElement{dictumauthor}\IndexFontElement[indexmain]{dictumauthor}%
\important{\FontElement{dictumauthor}} kann dabei eine Abweichung der Schrift
von der des Elementes
\FontElement{dictumtext}\IndexFontElement[indexmain]{dictumtext}%
\important{\FontElement{dictumtext}} definiert werden. Die Voreinstellung
entnehmen Sie bitte \autoref{tab:maincls.dictumfont}. Eine �nderung ist mit
Hilfe der Anweisungen \Macro{setkomafont} und \Macro{addtokomafont} (siehe
\autoref{sec:\csname label@base\endcsname.textmarkup},
\autopageref{desc:\csname label@base\endcsname.cmd.setkomafont}) m�glich.%
%
\ifCommonmaincls

Wird \Macro{dictum} innerhalb der Anweisung \Macro{setchapterpreamble} oder
\Macro{setpartpreamble} (siehe \autoref{sec:maincls.structure},
\autopageref{desc:maincls.cmd.setchapterpreamble}) verwendet, so ist Folgendes
zu beachten:\textnote{Achtung!} Die horizontale Anordnung erfolgt immer mit
\Macro{raggeddictum}. Das optionale Argument zur horizontalen Anordnung, das
die beiden Anweisungen vorsehen, bleibt daher ohne Wirkung. \Macro{textwidth}
ist nicht die Breite des gesamten Textk�rpers, sondern wie bei
\Environment{minipage} die aktuelle Textbreite. Ist also die Breite
\Macro{dictumwidth} als \lstinline;.5\textwidth; definiert und bei
\Macro{setchapterpreamble} wird als optionales Argument f�r die Breite
ebenfalls \lstinline;.5\textwidth; angegeben, so erfolgt die Ausgabe in einer
Box, deren Breite ein Viertel der Breite des Textk�rpers ist. Es wird
empfohlen\textnote{Tipp!}, bei Verwendung von \Macro{dictum} auf die
optionale Angabe einer Breite bei \Macro{setchapterpreamble} oder
\Macro{setpartpreamble} zu verzichten.

Sollen\textnote{Tipp!} mehrere schlaue Spr�che untereinander gesetzt werden,
so sollten diese durch einen zus�tzlichen Abstand vertikal voneinander
abgesetzt werden. Ein solcher kann leicht mit der Anweisung
\Macro{bigskip}\IndexCmd{bigskip} gesetzt werden.

\begin{Example}
  Sie schreiben ein Kapitel �ber die moderne Ehe. Dabei wollen sie in der
  Pr�ambel zur Kapitel�berschrift einen schlauen Spruch setzen. Dieser soll
  unter der �berschrift erscheinen. Also schreiben Sie:
\begin{lstcode}
  \setchapterpreamble[u]{%
    \dictum[Schiller]{Drum pr�fe, 
      wer sich ewig bindet \dots}}
  \chapter{Die moderne Ehe}
\end{lstcode}
  Die Ausgabe erfolgt dann in der Form:
  \begin{ShowOutput}
    {\usekomafont{disposition}\usekomafont{chapter}\Large
      17\enskip Die moderne Ehe\raggedright\par}
    \vspace{\baselineskip}
    \dictum[Schiller]{Drum pr�fe, wer sich ewig bindet~\dots}
  \end{ShowOutput}

  Wenn Sie wollen, dass nicht ein Drittel, sondern nur ein Viertel der
  verf�gbaren Textbreite f�r den Spruch verwendet wird, so definieren Sie
  \Macro{dictumwidth} wie folgt um:
\begin{lstcode}
  \renewcommand*{\dictumwidth}{.25\textwidth}
\end{lstcode}
\end{Example}

An dieser Stelle sei noch auf das Paket~\cite{package:ragged2e} hingewiesen,
mit dem man Flattersatz mit Trennung erreichen kann.%
\fi
%
\EndIndex{Cmd}{dictum}%
\EndIndex{Cmd}{dictumwidth}%
\EndIndex{Cmd}{dictumauthorformat}%
\EndIndex{Cmd}{dictumrule}%
\EndIndex{Cmd}{raggeddictum}%
\EndIndex{Cmd}{raggeddictumtext}%
\EndIndex{Cmd}{raggeddictumauthor}%
%
\EndIndex{}{Redewendung}%
\EndIndex{}{Zitat}%
\EndIndex{}{Spruch}%
\fi  %**************************************************** Ende nur maincls *


%%% Local Variables:
%%% mode: latex
%%% coding: iso-latin-1
%%% TeX-master: "../guide"
%%% End:
