\documentclass[fontsize=12pt,pagesize,parskip=half]
              {scrartcl}

\usepackage[ngerman]{babel}

\usepackage[T1]{fontenc}
\usepackage{lmodern}
\usepackage{charter,helvet}
\usepackage{textcomp}

\usepackage{selinput}
\SelectInputMappings{
  adieresis={�},
  germandbls={�},
}

\usepackage{enumerate}

\usepackage[clausemark=forceboth,
            juratotoc,
            juratocnumberwidth=2.5em]
           {scrjura}
\useshorthands{'}
\defineshorthand{'S}{\Sentence\ignorespaces}
\defineshorthand{'.}{. \Sentence\ignorespaces}

\pagestyle{myheadings}

\begin{document}

\subject{Satzung}
\title{VfVmai}
\subtitle{Verein f�r Vereinsmaierei mit ai n.e.V.}
\date{11.\,11.\,2011}
\maketitle

\tableofcontents

\addsec{Pr�ambel}

Die Vereinslandschaft in Deutschland ist vielf�ltig.
Doch leider mussten wir feststellen, dass es dabei oft
am ernsthaften Umgang mit der Ernsthaftigkeit krankt.

\appendix

\section{Allgemeines}

\begin{contract}
\Clause{title={Name, Rechtsform, Sitz des Vereins}}

Der Verein f�hrt den Namen �Verein f�r Vereinsmaierei mit 
ai n.e.V.� und ist in keinem Vereinsregister eingetragen.

'S Der Verein ist ein nichtwirtschaftlicher, unn�tzer
Verein'. Er hat keinen Sitz und muss daher stehen.

Gesch�ftsjahr ist vom 31.~M�rz bis zum 1.~April.

\Clause{title={Zweck des Vereins}}

'S Der Verein ist zwar sinnlos, aber nicht zwecklos'.
Vielmehr soll er den ernsthaften Umgang mit der
Ernsthaftigkeit auf eine gesunde Basis stellen.

Zu diesem Zweck kann der Verein
\begin{enumerate}[\qquad a)]
\item in der Nase bohren,
\item N�sse knacken,
\item am Daumen lutschen.
\end{enumerate}

Der Verein ist selbsts�chtig und steht dazu.

Der Verein verf�gt �ber keinerlei Mittel.\label{a:mittel}

\Clause{title={Vereins�mter}}

Die Vereins�mter sind Ehren�mter.

'S W�rde der Verein �ber Mittel verf�gen 
(siehe \ref{a:mittel}), so k�nnte er einen
hauptamtlichen Gesch�ftsf�hrer bestellen'. Ohne
die notwendigen Mittel ist dies nicht m�glich.

\Clause{title={Vereinsmaier},dummy}
\label{p.maier}
\end{contract}

\section{Mitgliedschaft}

\begin{contract}
\Clause{title={Mitgliedsarten},dummy}

\Clause{title={Erwerb der Mitgliedschaft}}

Die Mitgliedschaft kann jeder zu einem angemessenen 
Preis von einem der in \refClause{p.maier}
genannten Vereinsmaier erwerben.\label{a.preis}

'S Zum Erwerb der Mitgliedschaft ist ein formloser
Antrag erforderlich'. Dieser Antrag ist in gr�ner
Tinte auf rosa Papier einzureichen.

Die Mitgliedschaft kann nicht abgelehnt werden.

\SubClause{title={Erg�nzung zu vorstehendem 
    Paragraphen}}

'S Mit Abschaffung von \refClause{p.maier} verliert
\ref{a.preis} seine Umsetzbarkeit'. Mitgliedschaften
k�nnen ersatzweise vererbt werden.

\Clause{title={Ende der Mitgliedschaft}}

'S Die Mitgliedschaft endet mit dem Leben'. Bei nicht
lebenden Mitgliedern endet die Mitgliedschaft nicht.

\Clause{title={Mitgliederversammlung}}

Zweimal j�hrlich findet eine Mitgliederversammlung statt.

Der Abstand zwischen zwei Mitgliederversammlungen 
betr�gt h�chstens 6~Monate, 1~Woche und 2~Tage.

Fr�hestens 6~Monate nach der letzten Mitgliederversammlung
hat die Einladung zur n�chsten Mitgliederversammlung zu 
erfolgen.

\SubClause{title={Erg�nzung zur Mitgliederversammlung}}

Die Mitgliederversammlung darf fr�hstens 2~Wochen nach
letztem Eingang der Einladung abgehalten werden.
\end{contract}

\section{G�ltigkeit}

\begin{contract}
\Clause{title={In Kraft treten}}

Diese Satzung tritt am 11.\,11.\,2011 um 11:11~Uhr 
in Kraft.

'S Sollten irgendwelche Bestimmungen dieser Satzung im
Widerspruch zueinander stehen, tritt die Satzung am
11.\,11.\,2011 um 11:11~Uhr und 11~Sekunden wieder 
au�er Kraft'. Der Verein ist in diesem Fall als 
aufgel�st zu betrachten.

\end{contract}

\end{document}
