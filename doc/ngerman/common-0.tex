% ======================================================================
% common-0.tex
% Copyright (c) Markus Kohm, 2001-2009
%
% This file is part of the LaTeX2e KOMA-Script bundle.
%
% This work may be distributed and/or modified under the conditions of
% the LaTeX Project Public License, version 1.3c of the license.
% The latest version of this license is in
%   http://www.latex-project.org/lppl.txt
% and version 1.3c or later is part of all distributions of LaTeX 
% version 2005/12/01 or later and of this work.
%
% This work has the LPPL maintenance status "author-maintained".
%
% The Current Maintainer and author of this work is Markus Kohm.
%
% This work consists of all files listed in manifest.txt.
% ----------------------------------------------------------------------
% common-0.tex
% Copyright (c) Markus Kohm, 2001-2009
%
% Dieses Werk darf nach den Bedingungen der LaTeX Project Public Lizenz,
% Version 1.3c, verteilt und/oder veraendert werden.
% Die neuste Version dieser Lizenz ist
%   http://www.latex-project.org/lppl.txt
% und Version 1.3c ist Teil aller Verteilungen von LaTeX
% Version 2005/12/01 oder spaeter und dieses Werks.
%
% Dieses Werk hat den LPPL-Verwaltungs-Status "author-maintained"
% (allein durch den Autor verwaltet).
%
% Der Aktuelle Verwalter und Autor dieses Werkes ist Markus Kohm.
% 
% Dieses Werk besteht aus den in manifest.txt aufgefuehrten Dateien.
% ======================================================================
%
% Paragraphs that are common for several chapters of the KOMA-Script guide
% Maintained by Markus Kohm
%
% ----------------------------------------------------------------------
%
% Abs�tze, die mehreren Kapiteln der KOMA-Script-Anleitung gemeinsam sind
% Verwaltet von Markus Kohm
%
% ======================================================================

\ProvidesFile{common-0.tex}[2009/02/27 KOMA-Script guide (common paragraphs)]

\makeatletter
\@ifundefined{ifCommonmaincls}{\newif\ifCommonmaincls}{}%
\@ifundefined{ifCommonscrextend}{\newif\ifCommonscrextend}{}%
\@ifundefined{ifCommonscrlttr}{\newif\ifCommonscrlttr}{}%
\@ifundefined{ifIgnoreThis}{\newif\ifIgnoreThis}{}%
\makeatother


\section{Fr�he oder sp�te Optionenwahl}
\label{sec:\csname label@base\endcsname.options}
\ifshortversion\IgnoreThisfalse\IfNotCommon{typearea}{\IgnoreThistrue}\fi
\ifIgnoreThis %++++++++++++++++++++++++++++++++++++++++++++ nicht typearea +
Es gilt sinngem��, was in \autoref{sec:typearea.options} geschrieben wurde.
\else %------------------------------------------------------ nur typearea -

\BeginIndex{}{Optionen}%

In diesem Abschnitt wird eine Besonderheit von \KOMAScript{} vorgestellt, die
neben %
\IfCommon{typearea}{\Package{typearea} auch andere \KOMAScript-Pakete und
  "~Klassen }%
\IfCommon{maincls}{den Klassen \Class{scrbook}, \Class{scrreprt} und
  \Class{scrartcl} auch andere \KOMAScript-Klassen und "~Pakete }%
\IfCommon{scrlttr2}{der Klasse \Class{scrlttr2} auch andere
  \KOMAScript-Klassen und "~Pakete }%
\IfCommon{scrextend}{den Klassen und \Package{scrextend} auch einige andere
  \KOMAScript-Pakete }%
betrifft. Damit die Anwender alle Informationen zu einem Paket oder einer
Klasse im jeweiligen Kapitel finden, ist dieser Abschnitt nahezu gleichlautend
in mehreren Kapiteln zu finden. Anwender, die nicht nur an der Anleitung zu
einem Paket oder einer Klasse interessiert sind, sondern sich einen
Gesamt�berblick �ber \KOMAScript{} verschaffen wollen, brauchen diesen
Abschnitt nur in einem der Kapitel zu lesen und k�nnen ihn beim weiteren
Studium der Anleitung dann �berspringen.

\begin{Declaration}
  \Macro{documentclass}\OParameter{Optionenliste}%
  \Parameter{\KOMAScript-Klasse}\\
  \Macro{usepackage}\OParameter{Optionenliste}%
  \Parameter{Paket-Liste}
\end{Declaration}
\BeginIndex{Cmd}{documentclass}%
\BeginIndex{Cmd}{usepackage}%
Bei \LaTeX{} ist vorgesehen, dass Anwender Klassenoptionen in Form einer durch
Komma getrennten Liste einfacher Schl�sselw�rter als optionales Argument von
\Macro{documentclass} angeben. Au�er an die Klasse werden diese Optionen auch
an alle Pakete weitergereicht, die diese Optionen verstehen. Ebenso ist
vorgesehen, dass Anwender Paketoptionen in Form einer durch Komma getrennten
Liste einfacher Schl�sselw�rter als optionales Argument von \Macro{usepackage}
angeben.  \KOMAScript{} erweitert\ChangedAt{v3.00}{\Class{scrbook}\and
  \Class{scrreprt}\and \Class{scrartcl}\and \Package{typearea}} den
Mechanismus der Optionen f�r die \KOMAScript-Klassen und einige Pakete um
weitere M�glichkeiten. So haben die meisten Optionen bei \KOMAScript{}
zus�tzlich einen Wert. Eine Option hat also nicht unbedingt nur die Form
\PName{Option}, sondern kann auch die Form
\PName{Option}\texttt{=}\PName{Wert} haben. Bis auf diesen Unterschied
arbeiten \Macro{documentclass} und \Macro{usepackage} bei \KOMAScript{} wie in
\cite{latex:usrguide} oder jeder \LaTeX-Einf�hrung, beispielsweise
\cite{l2kurz}, beschrieben.%
%
\IfNotCommon{scrextend}{\par%
  Bei Verwendung einer \KOMAScript-Klasse sollten im �brigen beim
% Die Alternativen an dieser Stelle dienen der Verbesserung des Umbruchs
\IfCommon{typearea}{�berfl�ssigen, expliziten }%
Laden des Pakets \Package{typearea} oder \Package{scrbase} keine Optionen
angegeben werden. Das ist darin begr�ndet, dass die Klasse diese Pakete
bereits ohne Optionen l�dt und \LaTeX{} das mehrmalige Laden eines Pakets mit
unterschiedlicher Angabe von Optionen verweigert.%
\IfCommon{maincls}{\ �berhaupt ist es bei Verwendung einer \KOMAScript-Klasse
  nicht notwendig, eines dieser Pakete auch noch explizit zu laden.}%
\IfCommon{scrlttr2}{\ �berhaupt ist es bei Verwendung einer \KOMAScript-Klasse
  nicht notwendig, eines dieser Pakete auch noch explizit zu laden.}}%
%
\EndIndex{Cmd}{usepackage}%
\EndIndex{Cmd}{documentclass}%


\BeginIndex{Cmd}{KOMAoptions}%
\BeginIndex{Cmd}{KOMAoption}%
\begin{Declaration}
  \Macro{KOMAoptions}\Parameter{Optionenliste}\\
  \Macro{KOMAoption}\Parameter{Option}\Parameter{Werteliste}
\end{Declaration}
\KOMAScript{}\ChangedAt{v3.00}{\Class{scrbook}\and \Class{scrreprt}\and
  \Class{scrartcl}\and \Package{typearea}} bietet bei
den meisten Klassen- und Paketoptionen auch die M�glichkeit, den Wert der
Optionen noch nach dem Laden der Klasse beziehungsweise des Pakets zu
�ndern. Man kann dann wahlweise mit der Anweisung \Macro{KOMAoptions} die
Werte einer Reihe von Optionen �ndern. Jede Option der \PName{Optionenliste}
hat dabei die Form \PName{Option}\texttt{=}\PName{Wert}. 

Einige Optionen besitzen auch einen S�umniswert (engl. \emph{default
  value}). Vers�umt man die Angabe eines Wertes, verwendet man die Option also
einfach in der Form \PName{Option}, so wird automatisch dieser S�umniswert
angenommen.

Manche Optionen k�nnen %
\IfCommon{maincls}{auch }% Umbruchkorrektur
\IfCommon{scrlttr2}{auch }% Umbruchkorrektur
gleichzeitig mehrere Werte besitzen. F�r solche Optionen besteht die
M�glichkeit, mit Hilfe von \Macro{KOMAoption} der einen \PName{Option}
nacheinander eine Reihe von Werten zuzuweisen. Die einzelnen Werte sind dabei
in der \PName{Werteliste} durch Komma voneinander getrennt.

\begin{Explain}
  Falls man in der \PName{Optionenliste} eine Option auf einen unzul�ssigen
  Wert setzt oder die \PName{Werteliste} einen unzul�ssigen Wert enth�lt, so
  wird ein Fehler gemeldet. Wird \LaTeX{} in einem Modus verwendet, in dem im
  Fehlerfall Interaktionen m�glich sind, so stoppt \LaTeX{} in diesem
  Fall. Durch Eingabe von �\texttt{h}� erh�lt man dann eine Hilfe, in der
  auch die m�glichen Werte f�r die entsprechende Option angegeben sind.

  Soll in einem \PName{Wert} ein Gleichheitszeichen oder ein Komma vorkommen,
  so ist der \PName{Wert} in geschweifte Klammern zu setzen.

  \KOMAScript{} bedient sich f�r die Realisierung dieser M�glichkeit der
  Anweisungen \Macro{FamilyOptions} und \Macro{FamilyOption} mit der Familie
  �\PValue{KOMA}�. N�heres zu diesen Anweisungen ist %
  \IfCommon{maincls}{f�r Experten }%
  \IfCommon{scrlttr2}{f�r Experten }%
  in \autoref{sec:scrbase.keyvalue},
  \autopageref{desc:scrbase.cmd.FamilyOptions} zu finden.
\end{Explain}
%
\EndIndex{Cmd}{KOMAoption}%
\EndIndex{Cmd}{KOMAoptions}%
%
\EndIndex{}{Optionen}%

\fi % ************************************************** Ende nur typearea *


%%% Local Variables:
%%% mode: latex
%%% coding: iso-latin-1
%%% TeX-master: "../guide"
%%% End:
