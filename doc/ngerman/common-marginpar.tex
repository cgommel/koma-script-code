% ======================================================================
% common-marginpar.tex
% Copyright (c) Markus Kohm, 2001-2018
%
% This file is part of the LaTeX2e KOMA-Script bundle.
%
% This work may be distributed and/or modified under the conditions of
% the LaTeX Project Public License, version 1.3c of the license.
% The latest version of this license is in
%   http://www.latex-project.org/lppl.txt
% and version 1.3c or later is part of all distributions of LaTeX 
% version 2005/12/01 or later and of this work.
%
% This work has the LPPL maintenance status "author-maintained".
%
% The Current Maintainer and author of this work is Markus Kohm.
%
% This work consists of all files listed in manifest.txt.
% ----------------------------------------------------------------------
% common-marginpar.tex
% Copyright (c) Markus Kohm, 2001-2018
%
% Dieses Werk darf nach den Bedingungen der LaTeX Project Public Lizenz,
% Version 1.3c, verteilt und/oder veraendert werden.
% Die neuste Version dieser Lizenz ist
%   http://www.latex-project.org/lppl.txt
% und Version 1.3c ist Teil aller Verteilungen von LaTeX
% Version 2005/12/01 oder spaeter und dieses Werks.
%
% Dieses Werk hat den LPPL-Verwaltungs-Status "author-maintained"
% (allein durch den Autor verwaltet).
%
% Der Aktuelle Verwalter und Autor dieses Werkes ist Markus Kohm.
% 
% Dieses Werk besteht aus den in manifest.txt aufgefuehrten Dateien.
% ======================================================================
%
% Paragraphs that are common for several chapters of the KOMA-Script guide
% Maintained by Markus Kohm
%
% ----------------------------------------------------------------------
%
% Abs�tze, die mehreren Kapiteln der KOMA-Script-Anleitung gemeinsam sind
% Verwaltet von Markus Kohm
%
% ======================================================================

\KOMAProvidesFile{common-marginpar.tex}
                 [$Date$
                  KOMA-Script guide (common paragraphs)]




\section{Randnotizen}
\seclabel{marginNotes}%
\BeginIndexGroup
\BeginIndex{}{Randnotizen}%

\IfThisCommonFirstRun{}{%
  Es gilt sinngem��, was in
  \autoref{sec:\ThisCommonFirstLabelBase.marginNotes} geschrieben wurde. Falls
  Sie also \autoref{sec:\ThisCommonFirstLabelBase.marginNotes} bereits gelesen
  und verstanden haben, k�nnen Sie auf
  \autopageref{sec:\ThisCommonLabelBase.marginNotes.next} mit
  \autoref{sec:\ThisCommonLabelBase.marginNotes.next} fortfahren.%
}

Au�er dem eigentlichen Textbereich, der normalerweise den Satzspiegel
ausf�llt, existiert in Dokumenten noch die sogenannte Marginalienspalte. In
dieser k�nnen Randnotizen gesetzt werden. %
\IfThisCommonLabelBase{scrlttr2}{%
  Bei Briefen sind Randnotizen allerdings eher un�blich und sollten �u�erst
  sparsam eingesetzt werden.%
}{%
  In diesem \iffree{Dokument}{Buch} wird davon ebenfalls Gebrauch gemacht.%
}%


\begin{Declaration}
  \Macro{marginpar}\OParameter{Randnotiz links}\Parameter{Randnotiz}
  \Macro{marginline}\Parameter{Randnotiz}
\end{Declaration}%
F�r Randnotizen\Index[indexmain]{Randnotizen} ist bei {\LaTeX} normalerweise
Anweisung \Macro{marginpar} vorgesehen. Die \PName{Randnotiz} wird dabei im
�u�eren Rand gesetzt. Bei einseitigen Dokumenten wird der rechte Rand
verwendet. Zwar kann bei \Macro{marginpar} optional eine abweichende Randnotiz
angegeben werden, falls die Randnotiz im linken Rand landet, jedoch werden
Randnotizen immer im Blocksatz ausgegeben. Die Erfahrung zeigt, dass bei
Randnotizen statt des Blocksatzes oft je nach Rand linksb�ndiger oder
rechtsb�ndiger Flattersatz zu bevorzugen ist. {\KOMAScript} bietet hierf�r
die Anweisung \Macro{marginline}.

\IfThisCommonFirstRun{%
  \iftrue%
}{%
  Ein ausf�hrliches Beispiel hierzu finden Sie in
  \autoref{sec:\ThisCommonFirstLabelBase.marginNotes},
  \PageRefxmpl{\ThisCommonFirstLabelBase.cmd.marginline}.%
  \csname iffalse\endcsname%
}%
  \begin{Example}
    \phantomsection\xmpllabel{cmd.marginline}%
    In diesem \iffree{Dokument}{Buch} ist an einigen Stellen die
    Klassenangabe \Class{scrartcl}
    im Rand zu finden. Diese kann mit:
\begin{lstcode}
  \marginline{\texttt{scrartcl}}
\end{lstcode}
    erreicht werden.%
    \iftrue % Umbruchkorrekturtext
      \footnote{Tats�chlich wurde nicht \Macro{texttt},
        sondern eine semantische Auszeichnung verwendet. Um nicht unn�tig
        zu verwirren, wurde diese im Beispiel ersetzt.}%
    \fi

    Statt der Anweisung \Macro{marginline} w�re auch die Verwendung von
    \Macro{marginpar} m�glich gewesen. Tats�chlich wird bei obiger Verwendung
    von \Macro{marginline} intern nichts anders gemacht als: 
\begin{lstcode}
  \marginpar[\raggedleft\texttt{scrartcl}]
    {\raggedright\texttt{scrartcl}}
\end{lstcode}
    Damit ist \Macro{marginline} also nur eine abk�rzende Schreibweise.%
  \end{Example}%
\fi

F�r\textnote{Achtung!} Experten sind in
\autoref{sec:maincls-experts.addInfos},
\DescPageRef{maincls-experts.cmd.marginpar} Probleme bei der Verwendung von
\Macro{marginpar} dokumentiert. Diese gelten ebenso f�r \Macro{marginline}.
Dar�ber hinaus wird in \autoref{cha:scrlayer-notecolumn} ein Paket
vorgestellt, mit dem sich auch Notizspalten mit eigenem Seitenumbruch
realisieren lassen.%
\iftrue% Umbruchkorrektur
  \ Allerdings ist das Paket
  \hyperref[cha:scrlayer-notecolumn]{\Package{scrlayer-notecolumn}}%
  \IndexPackage{scrlayer-notecolumn} eher als
  eine Konzeptstudie und weniger als fertiges Paket zu verstehen.%
\fi%
%
\EndIndexGroup
%
\EndIndexGroup


%%% Local Variables:
%%% mode: latex
%%% mode: flyspell
%%% coding: iso-latin-1
%%% ispell-local-dictionary: "de_DE"
%%% TeX-master: "../guide"
%%% End:
