% ======================================================================
% common-5.tex
% Copyright (c) Markus Kohm, 2001-2012
%
% This file is part of the LaTeX2e KOMA-Script bundle.
%
% This work may be distributed and/or modified under the conditions of
% the LaTeX Project Public License, version 1.3c of the license.
% The latest version of this license is in
%   http://www.latex-project.org/lppl.txt
% and version 1.3c or later is part of all distributions of LaTeX 
% version 2005/12/01 or later and of this work.
%
% This work has the LPPL maintenance status "author-maintained".
%
% The Current Maintainer and author of this work is Markus Kohm.
%
% This work consists of all files listed in manifest.txt.
% ----------------------------------------------------------------------
% common-5.tex
% Copyright (c) Markus Kohm, 2001-2012
%
% Dieses Werk darf nach den Bedingungen der LaTeX Project Public Lizenz,
% Version 1.3c, verteilt und/oder veraendert werden.
% Die neuste Version dieser Lizenz ist
%   http://www.latex-project.org/lppl.txt
% und Version 1.3c ist Teil aller Verteilungen von LaTeX
% Version 2005/12/01 oder spaeter und dieses Werks.
%
% Dieses Werk hat den LPPL-Verwaltungs-Status "author-maintained"
% (allein durch den Autor verwaltet).
%
% Der Aktuelle Verwalter und Autor dieses Werkes ist Markus Kohm.
% 
% Dieses Werk besteht aus den in manifest.txt aufgefuehrten Dateien.
% ======================================================================
%
% Paragraphs that are common for several chapters of the KOMA-Script guide
% Maintained by Markus Kohm
%
% ----------------------------------------------------------------------
%
% Absaetze, die mehreren Kapiteln der KOMA-Script-Anleitung gemeinsam sind
% Verwaltet von Markus Kohm
%
% ======================================================================

\ProvidesFile{common-5.tex}[2012/01/03 KOMA-Script guide (common paragraphs)]
\translator{Gernot Hassenpflug\and Markus Kohm}

% Date of the translated German file: 2012/01/01

\makeatletter
\@ifundefined{ifCommonmaincls}{\newif\ifCommonmaincls}{}%
\@ifundefined{ifCommonscrextend}{\newif\ifCommonscrextend}{}%
\@ifundefined{ifCommonscrlttr}{\newif\ifCommonscrlttr}{}%
\@ifundefined{ifIgnoreThis}{\newif\ifIgnoreThis}{}%
\makeatother


\section{Text Markup}
\label{sec:\csname label@base\endcsname.textmarkup}%
\ifshortversion\IgnoreThisfalse\IfCommon{scrlttr2}{\IgnoreThistrue}\fi%
\ifIgnoreThis %++++++++++++++++++++++++++++++++++++++++++++++ nur scrlttr2 +
It applies, mutatis mutandis, what is described in
\autoref{sec:maincls.textmarkup}.
\else %---------------------------------------------------- nicht scrlttr2 -
\BeginIndex{}{text>markup}%
\BeginIndex{}{font}%

{\LaTeX} offers different possibilities for logical and direct text
markup\Index{logical markup}\Index{markup} of text. Beside selection of the
font family commands, e.\,g., choosing the font size and width is
supported. More information about the standard font facilities may be found at
\cite{lshort}, \cite{latex:usrguide}, and \cite{latex:fntguide}.


\begin{Declaration}
  \Macro{textsuperscript}\Parameter{Text}\\
  \Macro{textsubscript}\Parameter{Text}
\end{Declaration}
\BeginIndex{Cmd}{textsubscript}%
\BeginIndex{Cmd}{textsuperscript}%
The \LaTeX-Kern already defines the command
\Macro{textsuperscript}\IndexCmd{textsuperscript} to
superscript\Index{text>superscript}\Index{superscript} text. Unfortunately,
{\LaTeX} itself does not offer a command to produce text in
subscript\Index{text>subscript}\Index{subscript} instead of
superscript\Index{text>superscript}\Index{superscript}. {\KOMAScript}
defines \Macro{textsubscript} for this purpose.

\IfCommon{scrlttr2}{You may find an example of usage at
  \autoref{sec:maincls.textmarkup},
  \autopageref{desc:maincls.textsubscript.example}.}
\IfCommon{scrextend}{You may find an example of usage at
  \autoref{sec:maincls.textmarkup},
  \autopageref{desc:maincls.textsubscript.example}.}
%
\ifCommonmaincls
\begin{Example}
  \phantomsection\label{desc:maincls.textsubscript.example}%
  You are writing a text on human metabolism. From time to time you
  have to give some simple chemical formulas in which the numbers are
  in subscript. For enabling logical markup you first define in the
  document preamble or in a separate package:
\begin{lstcode}
  \newcommand*{\molec}[2]{#1\textsubscript{#2}}
\end{lstcode}
  \newcommand*{\molec}[2]{#1\textsubscript{#2}}
  Using this you then write:
\begin{lstcode}
  The cell produces its energy partly from reaction of \molec C6\molec
  H{12}\molec O6 and \molec O2 to produce \molec H2\Molec O{} and
  \molec C{}\molec O2.  However, arsenic (\molec{As}{}) has a quite
  detrimental effect on the metabolism.
\end{lstcode}
  The output looks as follows:
  \begin{ShowOutput}
    The cell produces its energy from reaction of \molec C6\molec
    H{12}\molec O6 and \molec O2 to produce \molec H2\molec O{} and
    \molec C{}\molec O2.  However, arsenic (\molec{As}{}) has a quite
    detrimental effect on the metabolism.
  \end{ShowOutput}
  Some time later you decide that the chemical formulas should be
  typeset in sans serif. Now you can see the advantages of using
  logical markup. You only have the redefine the \Macro{molec}
  command:
\begin{lstcode}
  \newcommand*{\molec}[2]{\textsf{#1\textsubscript{#2}}}
\end{lstcode}
  \renewcommand*{\molec}[2]{\textsf{#1\textsubscript{#2}}}
  Now the output in the whole document changes to:
  \begin{ShowOutput}
    The cell produces its energy partly from reaction of \molec
    C6\molec H{12}\molec O6 and \molec O2 to produce \molec H2\molec
    O{} and \molec C{}\molec O2.  However, arsenic (\molec{As}{}) has
    a quite detrimental effect on the metabolism.
  \end{ShowOutput}
\end{Example}
\iftrue
\begin{Explain}
  In the example above, the notation ``\verb|\molec C6|'' is used. This
  makes use of the fact that arguments consisting of only one
  character do not have to be enclosed in parentheses. That is why
  ``\verb|\molec C6|'' is similar to ``\verb|\molec{C}{6}|''. You
  might already know this from indices or powers in mathematical
  environments, such as ``\verb|$x^2$|'' instead of ``\verb|$x^{2}$|''
  for ``$x^2$''.
\end{Explain}
\else % vielleicht in einer spaeteren Auflage
F\"ur Experten ist in \autoref{sec:experts.knowhow},
\autopageref{desc:experts.macroargs} dokumentiert, warum das Beispiel
funktioniert, obwohl teilweise die Argumente von \Macro{Molek} nicht in
geschweifte Klammern gesetzt wurden.%
\fi
\fi% \ifCommonmaincls
%
\EndIndex{Cmd}{textsuperscript}%
\EndIndex{Cmd}{textsubscript}%


\begin{Declaration}
  \Macro{setkomafont}\Parameter{element}\Parameter{commands}\\
  \Macro{addtokomafont}\Parameter{element}\Parameter{commands}\\
  \Macro{usekomafont}\Parameter{element}
\end{Declaration}%
\BeginIndex{Cmd}{setkomafont}%
\BeginIndex{Cmd}{addtokomafont}%
\BeginIndex{Cmd}{usekomafont}%
With\ChangedAt{v2.8p}{\Class{scrbook}\and \Class{scrreprt}\and
  \Class{scrartcl}} the help of the two commands \Macro{setkomafont} and
\Macro{addtokomafont} it is possible to define the \PName{commands} that
change the characteristics of a given \PName{element}. Theoretically all
possible statements including literal text could be used as \PName{commands}.
You should\textnote{Attention!} however absolutely limit yourself to those
statements that really switch only one font attribute. This will usually be
the commands \Macro{normalfont}, \Macro{rmfamily}, \Macro{sffamily},
\Macro{ttfamily}, \Macro{mdseries}, \Macro{bfseries}, \Macro{upshape},
\Macro{itshape}, \Macro{slshape}, and \Macro{scshape} as well as the font size
commands \Macro{Huge}, \Macro{huge}, \Macro{LARGE}, \Macro{Large},
\Macro{large}, \Macro{normalsize}, \Macro{small}, \Macro{footnotesize},
\Macro{scriptsize}, and \Macro{tiny}. The description of these commands can be
found in \cite{lshort}, \cite{latex:usrguide} or \cite{latex:fntguide}. Color
switching commands like \Macro{normalcolor} (see \cite{package:graphics} and
\cite{package:xcolor}) are also acceptable.  The behavior when using other
commands, specially those that make redefinitions or generate output, is not
defined. Strange behavior is possible and does not represent a bug.

The command \Macro{setkomafont } provides a font switching command with a
completely new definition. In contrast to this the \Macro{addtokomafont}
command merely extends an existing definition. It is recommended to not use
both commands inside the document body, but only in the document preamble.
Usage examples can be found in the paragraphs on the corresponding element.
Names and meanings of the individual items are listed in %
\fi %************************************************* Ende nicht scrlttr2 *
\IfNotCommon{scrextend}{\autoref{tab:\csname
    label@base\endcsname.elementsWithoutText}}%
\IfCommon{scrextend}{\autoref{tab:maincls.elementsWithoutText},
  \autopageref{tab:maincls.elementsWithoutText}}%
.  %
\IfCommon{scrextend}{However only the listed elements for the document title,
  dicta, footnotes, and the \Environment{labeling} environment are
  supported. Though element \FontElement{disposition} exists, it will also be
  used for the document title only. This has been done for compatibility with
  the \KOMAScript{} classes. }%
The default values are shown in the corresponding paragraphs.

%%% Local Variables:
%%% mode: latex
%%% coding: us-ascii
%%% TeX-master: "../guide"
%%% End:
