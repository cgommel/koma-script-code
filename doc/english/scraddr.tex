% ======================================================================
% scraddr.tex
% Copyright (c) Markus Kohm, 2001-2007
%
% This file is part of the LaTeX2e KOMA-Script bundle.
%
% This work may be distributed and/or modified under the conditions of
% the LaTeX Project Public License, version 1.3b of the license.
% The latest version of this license is in
%   http://www.latex-project.org/lppl.txt
% and version 1.3b or later is part of all distributions of LaTeX 
% version 2005/12/01 or later and of this work.
%
% This work has the LPPL maintenance status "author-maintained".
%
% The Current Maintainer and author of this work is Markus Kohm.
%
% This work consists of all files listed in manifest.txt.
% ----------------------------------------------------------------------
% scraddr.tex
% Copyright (c) Markus Kohm, 2001-2007
%
% Dieses Werk darf nach den Bedingungen der LaTeX Project Public Lizenz,
% Version 1.3b, verteilt und/oder veraendert werden.
% Die neuste Version dieser Lizenz ist
%   http://www.latex-project.org/lppl.txt
% und Version 1.3b ist Teil aller Verteilungen von LaTeX
% Version 2005/12/01 oder spaeter und dieses Werks.
%
% Dieses Werk hat den LPPL-Verwaltungs-Status "author-maintained"
% (allein durch den Autor verwaltet).
%
% Der Aktuelle Verwalter und Autor dieses Werkes ist Markus Kohm.
% 
% Dieses Werk besteht aus den in manifest.txt aufgefuehrten Dateien.
% ======================================================================
%
% Chapter about scraddr of the KOMA-Script guide
% Maintained by Jens-Uwe Morawski (with help from Markus Kohm)
%
% ----------------------------------------------------------------------
%
% Kapitel über scraddr in der KOMA-Script-Anleitung
% Verwaltet von Jens-Uwe Morawski (mit Unterstützung von Markus Kohm)
%
% ======================================================================

\ProvidesFile{scraddr.tex}[2005/11/27 KOMA-Script guide (chapter: scraddr)]
% Date of translated german file: 2002/10/15

\chapter{Access to Address Files with \Package{scraddr}}%
\labelbase{scraddr}%
\BeginIndex{Package}{scraddr}

\section{Overview}\label{sec:scraddr.overview}
The package \Package{scraddr} is a small extension to
the \KOMAScript{} letter class.
Its aim is to make access to the data of address files
more flexible and easier.
Basically the package implements a new loading mechanism
for address files, which contain address entries of the kind
described in the previous chapter.

\begin{Declaration}
\Macro{InputAddressFile}\Parameter{file name}
\end{Declaration}%
\BeginIndex{Cmd}{InputAddressFile}%
%
The command \Macro{InputAddressFile} reads the content
of the address file, given as its parameter.
If the file does not exist the command returns an
error message.

For every entry in the address file the command generates a
set of macros for accessing the data. 
First a short reminder of how an address entry is structured.

\begin{Declaration}%
  \Macro{adrentry}\Parameter{Lastname}\Parameter{Firstname}%
  \Parameter{Address}\Parameter{Telephone}\Parameter{F1}\Parameter{F2}%
  \Parameter{Comment}\Parameter{Key}\\
  \Macro{addrentry}\Parameter{Lastname}\Parameter{Firstname}%
  \Parameter{Address}\Parameter{Telephone}\Parameter{F1}\Parameter{F2}%
  \Parameter{F3}\Parameter{F4}\Parameter{Key}
\end{Declaration}%
The last parameter is the identifier of an entry, thus the
\PName{Key} has to be unique and not blank.  If the file contains more
than one entry with the same \PName{Key} value, the last occurrence
will be used.  The \PName{Key} should only be composed of letters.
\begin{Declaration}
\Macro{Name}\Parameter{Key}\\
\Macro{FirstName}\Parameter{Key}\\
\Macro{LastName}\Parameter{Key}\\
\Macro{Address}\Parameter{Key}\\
\Macro{Telephone}\Parameter{Key}\\
\Macro{FreeI}\Parameter{Key}\\
\Macro{FreeII}\Parameter{Key}\\
\Macro{Comment}\Parameter{Key}\\
\Macro{FreeIII}\Parameter{Key}\\
\Macro{FreeIV}\Parameter{Key}
\end{Declaration}%
\BeginIndex{Cmd}{Name}\BeginIndex{Cmd}{FirstName}\BeginIndex{Cmd}{LastName}%
\BeginIndex{Cmd}{Address}\BeginIndex{Cmd}{Telephone}\BeginIndex{Cmd}{FreeI}%
\BeginIndex{Cmd}{FreeII}\BeginIndex{Cmd}{FreeIII}\BeginIndex{Cmd}{FreeIV}%
\BeginIndex{Cmd}{Comment}%
%
These commands give access to data of your address file.
The parameter \PName{Key} is the same as in the
\Macro{adrentry} or \Macro{addrentry} command.
In the address of letters often both, first-name and
last-name, are required.
The command \Macro{Name}\PParameter{Key} is an abridgement for
\Macro{FirstName}\PParameter{Key} \Macro{LastName}\PParameter{Key}.

\EndIndex{Cmd}{InputAddressFile}%
\EndIndex{Cmd}{Name}\EndIndex{Cmd}{FirstName}\EndIndex{Cmd}{LastName}%
\EndIndex{Cmd}{Address}\EndIndex{Cmd}{Telephone}\EndIndex{Cmd}{FreeI}%
\EndIndex{Cmd}{FreeII}\EndIndex{Cmd}{FreeIII}\EndIndex{Cmd}{FreeIV}%
\EndIndex{Cmd}{Comment}%

\section{Usage}\label{sec:scraddr.usage}

First of all, we need an address file with valid address entries.
In this example the file has the name \emph{lotr.adr} and contains
the following entries.
\begin{lstlisting}
  \addrentry{Baggins}{Frodo}%
            {The Hill\\ Bag End/Hobbiton in the Shire}{}%
            {Bilbo Baggins}{pipe-weed}%
            {the Ring-bearer}{Bilbo's heir}{FRODO}
  \adrentry{Gamgee}{Samwise}%
            {Bagshot Row 3\\Hobbiton in the Shire}{}%
            {Rosie Cotton}{taters}%
            {the Ring-bearer's faithful servant}{SAM}
  \adrentry{Bombadil}{Tom}%
            {The Old Forest}{}%
            {Goldberry}{trill queer songs}%
            {The Master of Wood, Water and Hill}{TOM}
\end{lstlisting}

The 4\textsuperscript{th} parameter, the telephone number, has been left
blank. If you know the story behind these addresses you
will agree that a telephone number makes no sense here.
The command \Macro{InputAddressFile} is used to load
the address file shown above.
\begin{lstlisting}
  \InputAddressFile{lotr}
\end{lstlisting}


With the help of the commands introduced in this chapter
we can now write a letter to the old \textsc{Tom Bombadil}.
In this letter we ask him, if he can remember two
fellow-travelers from Elder Days.
\begin{lstlisting}
  \begin{letter}{\Name{TOM}\\\Address{TOM}}
     \opening{Dear \FirstName{TOM} \LastName{TOM},}
     
      or \FreeIII{TOM}, how your delightful \FreeI{TOM} calls
      you.
      Can you remember Mr.\,\LastName{FRODO},
      strictly speaking \Name{FRODO}, since there was
      Mr.\,\FreeI{FRODO} too.
      He was \Comment{FRODO} in the Third Age
      and \FreeIV{FRODO}
      \Name{SAM}, \Comment{SAM}, has attended him.
      
      Their pasions were very wordly.
      \FirstName{FRODO} enjoyed to smoke \FreeII{FRODO}, 
      his attendant has appreciate a good meal with
      \FreeII{SAM}.

      Do you remember? Certainly Mithrandir has told you much
      about their deeds and adventures .
    \closing{``O spring-time and summer-time
                and spring again after!\\
               O wind on the waterfall,
                and the leaves' laughter!''}
  \end{letter}
\end{lstlisting}
The 5\textsuperscript{th} and 6\textsuperscript{th} parameter of the
\Macro{adrentry} or \Macro{adrentry} command are for free use.  They are
accessible with the commands \Macro{FreeI} and \Macro{FreeII}.  In this
example the 5\textsuperscript{th} parameter contains the name of a person who
is the most important in the life of the entry's person, the
6\textsuperscript{th} contains the person's passion.  The
7\textsuperscript{th} parameter is a comment or in general also a free
parameter. The commands \Macro{Comment} or \Macro{FreeIII} give access to the
data. Using \Macro{FreeIV} is only valid for \Macro{addrentry} entries; for
\Macro{adrentry} entries it results in a warning.
\EndIndex{Package}{scraddr}

\section{Package Warning Options}

As mentioned above the command \Macro{FreeIV} leads
to a fault if it is used for \Macro{adrentry} entries.

Therefore \Package{scraddr} supports package options
in order to give the user the possibility to
choose how the package should react in such situation.
It is possible to choose different settings between
\emph{ignore} and \emph{rupture} of the \LaTeX{} run.

\begin{Declaration}
\Option{adrFreeIVempty}\\
\Option{adrFreeIVshow}\\
\Option{adrFreeIVwarn}\\
\Option{adrFreeIVstop}
\end{Declaration}
%
\BeginIndex{Option}{adrFreeIVempty}\BeginIndex{Option}{adrFreeIVshow}
\BeginIndex{Option}{adrFreeIVwarn}\BeginIndex{Option}{adrFreeIVstop}
%
One of these options can be given in the optional argument
of the \Macro{usepackage} command. The default setting is
\Option{adrFreeIVshow}.

\begin{labeling}[\,--]{\Option{adrFreeIVempty}}
\item[\Option{adrFreeIVempty}] 
        the command \Macro{FreeIV} will be ignored
\item[\Option{adrFreeIVshow}] 
        ``(entry FreeIV undefined at \PName{Key})'' will be
        written as warning in the text
\item[\Option{adrFreeIVwarn}]
        writes a warning in the log-file
\item[\Option{adrFreeIVstop}]
        the \LaTeX{} run will be interrupted with an error message
\end{labeling}
\EndIndex{Option}{adrFreeIVempty}\EndIndex{Option}{adrFreeIVshow}
\EndIndex{Option}{adrFreeIVwarn}\EndIndex{Option}{adrFreeIVstop}

%%% Local Variables: 
%%% mode: latex
%%% TeX-master: "../guide"
%%% End: 
