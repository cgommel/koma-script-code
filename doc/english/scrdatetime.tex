% ======================================================================
% scrdatetime.tex
% Copyright (c) Markus Kohm, 2001-2016
%
% This file is part of the LaTeX2e KOMA-Script bundle.
%
% This work may be distributed and/or modified under the conditions of
% the LaTeX Project Public License, version 1.3c of the license.
% The latest version of this license is in
%   http://www.latex-project.org/lppl.txt
% and version 1.3c or later is part of all distributions of LaTeX 
% version 2005/12/01 or later and of this work.
%
% This work has the LPPL maintenance status "author-maintained".
%
% The Current Maintainer and author of this work is Markus Kohm.
%
% This work consists of all files listed in manifest.txt.
% ----------------------------------------------------------------------
% scrdatetime.tex
% Copyright (c) Markus Kohm, 2001-2016
%
% Dieses Werk darf nach den Bedingungen der LaTeX Project Public Lizenz,
% Version 1.3c, verteilt und/oder veraendert werden.
% Die neuste Version dieser Lizenz ist
%   http://www.latex-project.org/lppl.txt
% und Version 1.3c ist Teil aller Verteilungen von LaTeX
% Version 2005/12/01 oder spaeter und dieses Werks.
%
% Dieses Werk hat den LPPL-Verwaltungs-Status "author-maintained"
% (allein durch den Autor verwaltet).
%
% Der Aktuelle Verwalter und Autor dieses Werkes ist Markus Kohm.
% 
% Dieses Werk besteht aus den in manifest.txt aufgefuehrten Dateien.
% ======================================================================
%
% Chapter about scrpage2 of the KOMA-Script guide
% Maintained by Markus Kohm
%
% ----------------------------------------------------------------------
%
% Kapitel ueber scrpage2 in der KOMA-Script-Anleitung
% Verwaltet von Markus Kohm
%
% ============================================================================

\KOMAProvidesFile{scrdatetime.tex}
                 [$Date$
                  KOMA-Script guide (chapter: scrdate, scrtime)]
\translator{Markus Kohm\and Gernot Hassenpflug}

% Date of the translated German file: 2016-11-14

\chapter{The Day of the Week Using \Package{scrdate}}
\labelbase{scrdate}
\BeginIndexGroup
\BeginIndex{Package}{scrdate}

With version~3.05a the functionality of this package enhanced a lot. Beside of
the current day of the week\Index{day>of the week} this package provides the
day of the week of every date of the Gregorian calendar now.

\begin{Declaration}
  \Macro{CenturyPart}\Parameter{year}\\%
  \Macro{DecadePart}\Parameter{year}%
\end{Declaration}%
The\ChangedAt{v3.05a}{\Package{scrdate}} command \Macro{CenturyPart} offers
the value of the century digits\,---\,hundreds and thousands\,---\,of a
\PName{year}. The command \Macro{DecadePart} in difference offers the other
digits which are the units and tens. The number of digits of \PName{year} does
not care. The value may be assigned to a counter or may be used for
calculations, i.\,e., using \Macro{numexpr}\IndexCmd{numexpr}. For
output\textnote{Attention!} of an Arabic number of the value prefix it with
\Macro{the}\IndexCmd{the}.

\begin{Example}
  You want to calculate and output the century of the current year.
\begin{lstcode}
  The year \the\year\ is the year \the\DecadePart{\year}
  of the \engord{\numexpr\CenturyPart{\year}+1\relax} century.
\end{lstcode}
  The result would be:
  \begin{quote}
    The year \the\year\ is the year \the\DecadePart{\year}
    of the \engordnumber{\numexpr\CenturyPart{\year}+1\relax} century.
  \end{quote}
  Package \Package{engord}\IndexPackage{engord} has been used for this
  example. See \cite{package:engord} for more information.
\end{Example}

Please\textnote{Attention!} note, that within used method of counting the
year~2000 is the year~0\,---\,and therefore the first year\,---\,of the
21st~century.%
\EndIndexGroup


\begin{Declaration}
  \Macro{DayNumber}\Parameter{year}\Parameter{month}\Parameter{day}\\%
  \Macro{ISODayNumber}\Parameter{ISO-date}%
\end{Declaration}%
These\ChangedAt{v3.05a}{\Package{scrdate}} two commands offers the value of
the number of the day of the week\Index{day>of the week} of any date. The
differ only in the kind of date declaration. Command \Macro{DayNumber} needs
\PName{year}, \PName{month}, and \PName{day} as separate parameters. Command
\Macro{ISODayNumber} expects an \PName{ISO-date} as a single argument. The
expected format of the \PName{ISO-date} is:
\PName{year}\texttt{-}\PName{month}\texttt{-}\PName{day}. It does not matter
whether \PName{month} or \PName{day} have one or two digits. The result of
both commands may be assigned to a counter or used for calculations, i.\,e.,
using \Macro{numexpr}\IndexCmd{numexpr}. For output\textnote{Attention!} of an
Arabic number of the value prefix it with \Macro{the}\IndexCmd{the}.

\begin{Example}
  You want to know the number of the day of the week of the 1st~May~2027.
\begin{lstcode}
  The 1st~May~2027 has \the\ISODayNumber{2027-5-1}
  as the number of the day of the week.
\end{lstcode}
  The result will be:
  \begin{quote}
    The 1st~May~2027 has \the\ISODayNumber{2027-5-1}
    as the number of the day of the week.
  \end{quote}
\end{Example}

A special feature is to walk a number of days into future or past from a given
date.
\begin{Example}
  You want to know the number of the day of the week, that will be in 12~days
  and that will be 24~days before the 24th~December~2027.
\begin{lstcode}
  In 12~days the number of the day of the week
  will be \the\DayNumber{\year}{\month}{\day+12} and
  24~days before the 24th~December~2027 it will be
  \the\ISODayNumber{2027-12-24-24}.
\end{lstcode}
  The result may be, e.\,g.:
  \begin{quote}
    In 12~days the number of the day of the week
    will be \the\DayNumber{\year}{\month}{\day+12} and
    24~days before the 24th~December~2027 it will be
    \the\ISODayNumber{2027-12-24-24}.
  \end{quote}
\end{Example}

The days of the week are numbered: Sunday\,=\,0, Monday\,=\,1, Tuesday\,=\,2,
Wednesday\,=\,3, Thursday\,=\,4, Friday\,=\,5, and Saturday\,=\,6.%
%
\EndIndexGroup


\begin{Declaration}
  \Macro{DayNameByNumber}\Parameter{number of the day of the week}\\%
  \Macro{DayName}\Parameter{year}\Parameter{month}\Parameter{day}\\%
  \Macro{ISODayName}\Parameter{ISO-date}%
\end{Declaration}%
Usually\ChangedAt{v3.05a}{\Package{scrdate}} you do not want to know the number
of the day of the week, but the name of the day of the week. Because of this,
the command \Macro{DayNameByNumber} offers the name of the day of the week
corresponding with a number. The number may be the result of \Macro{DayNumber}
or \Macro{ISODayNumber}. The two commands \Macro{DayName} and
\Macro{ISODayName} directly offer the name of the day of the week of a given
date.

\begin{Example}
  You want to know the name of the day of the week of the 24th~December~2027.
\begin{lstcode}
  Please pay you bill until \ISODayName{2027-12-24},
  24th~December~2027.
\end{lstcode}
  The result will be:
  \begin{quote}
    Please pay you bill until \ISODayName{2027-12-24},
    24th~December~2027.
  \end{quote}
\end{Example}

Again a special feature is to make some calculations inside the argument of
\Macro{DayName}.
\begin{Example}
  You want to know the names of the days of the week, that will be in 12~days
  and that will be 24~days before the 24th~December~2027.
\begin{lstcode}
  In 12~days the name of the day of the week
  will be \DayName{\year}{\month}{\day+12} and
  24~days before the 24th~December~2027 it will be
  \ISODayName{2027-12-24-24}. Nevertheless two weeks
  and three days after a Wednesday a
  \DayNameByNumber{3+2*7+3} will follow.
\end{lstcode}
  The result may be, e.\,g.:
  \begin{quote}
    In 12~days the name of the day of the week
    will be \DayName{\year}{\month}{\day+12} and
    24~days before the 24th~December~2027 it will be
    \ISODayName{2027-12-24-24}. Nevertheless two weeks
    and three days after a Wednesday a
    \DayNameByNumber{3+2*7+3} will follow.
  \end{quote}
\end{Example}%
%
\EndIndexGroup


\begin{Declaration}
  \Macro{ISOToday}%
  \Macro{IsoToday}%
  \Macro{todaysname}%
  \Macro{todaysnumber}%
\end{Declaration}%
In the prior examples the current date have been given cumbersomely and
explicitly using the \TeX{} registers \Macro{year}\IndexCmd{year},
\Macro{month}\IndexCmd{month}, and \Macro{day}\IndexCmd{day}. The commands
\Macro{ISOToday}\ChangedAt{v3.05a}{\Package{scrdate}} and \Macro{IsoToday}
offers the current date in ISO-notation directly. These commands differ in the
number of digits for numbers less than 10 only. \Macro{ISOToday} prefixes
numbers less than 10 for the month and day with a 0. In opposite to this
\Macro{IsoToday} will show numbers less than 10 for the month and day with one
digit only. Command \Macro{todaysname} directly offers the name of the current
day of the week. Command \Macro{todaysnumber} offers the number of that name
instead. More information about usability of this value may be found at
previous description of \DescRef{scrdate.cmd.DayNumber} and \DescRef{scrdate.cmd.ISODayNumber}.

\begin{Example}
  I want to show you the name of the weekday in which this document has been
  type-set:
  \begin{lstlisting}
    I have done the {\LaTeX} run of this document 
    on a \todaysname.
  \end{lstlisting}
  This will result in, e.\,g.:
  \begin{quote}
    I have done the {\LaTeX} run of this document 
    on a \todaysname.
  \end{quote}
\end{Example}

Note that the package is not able to decline words. The known terms
are the nominative singular that may be used, e.\,g., in the date of a
letter. Given this limitation, the example above can work correctly
only for some languages.

\begin{Explain}
  The\textnote{Hint!} names of the weekdays are saved in capitalized form,
  i.\,e., the first letter is a capital letter, all the others are lowercase
  letters. But for some languages you may need the names completely in
  lowercase. You may achieve this using the standard {\LaTeX} command
  \Macro{MakeLowercase}\IndexCmd{MakeLowercase}%
  \important{\Macro{MakeLowercase}}, e.\,g.:
  % Umbruchkorrektur: listings
\begin{lstcode}
  \MakeLowercase{\todaysname}
\end{lstcode}
  This converts the whole argument into lower case letters. Of course, this
  may be done also using previous described commands
  \DescRef{scrdate.cmd.DayNameByNumber}\IndexCmd{DayNameByNumber},
  \DescRef{scrdate.cmd.DayName}\IndexCmd{DayName} and
  \DescRef{scrdate.cmd.ISODayName}\IndexCmd{ISODayName}.%
\end{Explain}%
\EndIndexGroup


\begin{Declaration}
  \Macro{nameday}\Parameter{name}
\end{Declaration}%
Analogous to how the output of \Macro{today} can be modified using
\DescRef{maincls.cmd.date}\IndexCmd{date}, so the output of \DescRef{scrdate.cmd.todaysname}
can be changed to \PName{name} by using \Macro{nameday}.
\begin{Example}
  You change the current date to a fixed value using
  \DescRef{maincls.cmd.date}. You are not interested in the actual name of
  the day, but want only to show that it is a workday. So you set:
\begin{lstlisting}
  \nameday{workday}
\end{lstlisting}
  After this the previous example will result in:
  \begin{quote}\nameday{workday}
    I have done the {\LaTeX} run of this document on a \todaysname.
  \end{quote}
\end{Example}
There's no such command for changing the result of
\DescRef{scrdate.cmd.ISOToday}\IndexCmd{ISOToday} or \DescRef{scrdate.cmd.IsoToday}\IndexCmd{IsoToday}.
\EndIndexGroup


\begin{Declaration}
  \Macro{newdaylanguage}\Parameter{language}%
                        \Parameter{Monday}\Parameter{Tuesday}%
                        \Parameter{Wednesday}\Parameter{Thursday}%
                        \Parameter{Friday}\Parameter{Saturday}
                        \Parameter{Sunday}%
\end{Declaration}
Currently the package \Package{scrdate} knows the following languages:
\begin{itemize}\setlength{\itemsep}{.5\itemsep}
\item Croatian (\PValue{croatian}),
\item Czech (\PValue{czech}\ChangedAt{v3.13}{\Package{scrdate}}),
\item Danish (\PValue{danish}),
\item Dutch (\PValue{dutch}),
\item English (\PValue{american}\ChangedAt{v3.13}{\Package{scrdate}},
  \PValue{australian}, \PValue{british}, \PValue{canadian}, \PValue{english},
  \PValue{UKenglish}, and USenglish),
\item Finnish (\PValue{finnish}),
\item French (\PValue{acadian}, \PValue{canadien},
  \PValue{francais}\ChangedAt{v3.13}{\Package{scrdate}}, and \PValue{french}),
\item German (\PValue{austrian}\ChangedAt{v3.08b}{\Package{scrdate}},
  \PValue{german}, \PValue{naustrian}, \PValue{ngerman},
  \PValue{nswissgerman}, and
  \PValue{swissgerman}\ChangedAt{v3.13}{\Package{scrdate}}),
\item Italian (\PValue{italian}),
\item Norwegian (\PValue{norsk}),
\item Polish (\PValue{polish}\ChangedAt{v3.13}{\Package{scrdate}}),
\item Spanish (\PValue{spanish}),
\item Swedish (\PValue{swedish}).
\end{itemize}
If you want to configure it for other languages, you may use this command. The
first argument is the name of the language and the other arguments are the
names of the corresponding days.

In the current implementation it does not matter whether you load
\Package{scrdate} before or after
\Package{german}\IndexPackage{german},
\Package{ngerman}\IndexPackage{ngerman},
\Package{babel}\IndexPackage{babel} or similar packages. In both cases the
correct language will be used.

\begin{Explain}
  To explain a little bit more exactly: while you are using a language
  selection which works in a compatible way to
  \Package{babel}\IndexPackage{babel} or
  \Package{ngerman}\IndexPackage{ngerman}, the correct language will be used
  by \Package{scrdate}. If you are using another language selection you will
  get (US-)English names.
\end{Explain}

By the way: If you found a suitable definition for a language, that has not
been defined before, please mail it to the \KOMAScript{} author. There is a
good chance, that he will add support for the language to \Package{scrdate}.%
\EndIndexGroup
%
\EndIndexGroup


\chapter{Getting the Time with Package \Package{scrtime}}
\labelbase{scrtime}
\BeginIndexGroup
\BeginIndex{Package}{scrtime}

This package provides an answer to the question of the current time. Since
version~3.05 this package uses the option interface already known from
the \KOMAScript{} classes and several other \KOMAScript{} packages. See for
example \autoref{sec:typearea.options} for more information.

\begin{Declaration}%
  \Macro{thistime}\OParameter{delimiter}%
  \Macro{thistime*}\OParameter{delimiter}
\end{Declaration}%
\Macro{thistime} results in the current time\Index{time}. The
delimiter between the values of hour, minutes and seconds can be given
in the optional argument. The default symbol of the delimiter is
``\PValue{:}''.

\Macro{thistime*} works in almost the same way as \Macro{thistime}.
The only difference is that unlike with \Macro{thistime}, with
\Macro{thistime*} the value of the minute field is not preceded by a
zero when its value is less than 10. Thus, with \Macro{thistime} the
minute field has always two places.
\begin{Example}
  The line
\begin{lstlisting}
  Your train departs at \thistime.
\end{lstlisting}
  results, for example, in:
  \begin{quote}
    Your train departs at \thistime.
  \end{quote}
  or:
  \begin{quote}
    Your train departs at 23:09.
  \end{quote}
  \bigskip
  In contrast to the previous example a line like:
\begin{lstlisting}
  This day is already \thistime*[\ hours and\ ] minutes old.
\end{lstlisting}
  results in:
  \begin{quote}
    This day is already \thistime*[\ hours and\ ] minutes old.
  \end{quote}
  or:
  \begin{quote}
    This day is already 12 hours and 25 minutes old.
  \end{quote}
\end{Example}
\EndIndexGroup


\begin{Declaration}%
 \Macro{settime}\Parameter{time}
\end{Declaration}%
\Macro{settime} sets the output of \DescRef{scrtime.cmd.thistime} and
\DescRef{scrtime.cmd.thistime*} to the value \PName{time}.  Now the optional
parameter of \DescRef{scrtime.cmd.thistime} or \DescRef{scrtime.cmd.thistime*} is ignored, since
the result of \DescRef{scrtime.cmd.thistime} or \DescRef{scrtime.cmd.thistime*} was completely
determined using \Macro{settime}.%
\EndIndexGroup


\begin{Declaration}
  \OptionVName{12h}{simple-switch}%
\end{Declaration}%
\BeginIndex{Option}{24h}%
With option \Option{12h}\ChangedAt{v3.05a}{\Package{scrtime}} one can select
whether the result of \DescRef{scrtime.cmd.thistime} and \DescRef{scrtime.cmd.thistime*} is in 12- or in
24-hour format. The option understands the values for simple-switch listed in
\autoref{tab:truefalseswitch}, \autopageref{tab:truefalseswitch}. The option
without a value is same like \OptionValue{12h}{true} and therefore
12-hour-format will be used. The default is \Option{24h}.

You may use this option either as a global option in the optional argument of
\DescRef{typearea.cmd.documentclass}, as a package option in the optional
argument of \DescRef{typearea.cmd.usepackage} or after loading the package
using \DescRef{typearea.cmd.KOMAoptions} or \DescRef{typearea.cmd.KOMAoption}
(see, e.\,g., \autoref{sec:typearea.options},
\DescPageRef{typearea.cmd.KOMAoptions}).  The option has no effect on the
results of \DescRef{scrtime.cmd.thistime} and \DescRef{scrtime.cmd.thistime*}
if \DescRef{scrtime.cmd.settime} is used.

Only\textnote{Attention!} for compatibility with former releases of
\Package{scrtime} also option \Option{24h} will switch to 24-hour format if
used in the optional argument of \Macro{documentclass} or
\Macro{usepackage}. Nevertheless, you should not use this option any longer.%
\EndIndexGroup
%
\EndIndexGroup

\endinput

%%% Local Variables: 
%%% mode: latex
%%% coding: us-ascii
%%% TeX-master: "../guide"
%%% End: 
