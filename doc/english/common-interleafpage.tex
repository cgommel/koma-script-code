% ======================================================================
% common-interleafpage.tex
% Copyright (c) Markus Kohm, 2001-2017
%
% This file is part of the LaTeX2e KOMA-Script bundle.
%
% This work may be distributed and/or modified under the conditions of
% the LaTeX Project Public License, version 1.3c of the license.
% The latest version of this license is in
%   http://www.latex-project.org/lppl.txt
% and version 1.3c or later is part of all distributions of LaTeX 
% version 2005/12/01 or later and of this work.
%
% This work has the LPPL maintenance status "author-maintained".
%
% The Current Maintainer and author of this work is Markus Kohm.
%
% This work consists of all files listed in manifest.txt.
% ----------------------------------------------------------------------
% common-interleafpage.tex
% Copyright (c) Markus Kohm, 2001-2017
%
% Dieses Werk darf nach den Bedingungen der LaTeX Project Public Lizenz,
% Version 1.3c, verteilt und/oder veraendert werden.
% Die neuste Version dieser Lizenz ist
%   http://www.latex-project.org/lppl.txt
% und Version 1.3c ist Teil aller Verteilungen von LaTeX
% Version 2005/12/01 oder spaeter und dieses Werks.
%
% Dieses Werk hat den LPPL-Verwaltungs-Status "author-maintained"
% (allein durch den Autor verwaltet).
%
% Der Aktuelle Verwalter und Autor dieses Werkes ist Markus Kohm.
% 
% Dieses Werk besteht aus den in manifest.txt aufgefuehrten Dateien.
% ======================================================================
%
% Paragraphs that are common for several chapters of the KOMA-Script guide
% Maintained by Markus Kohm
%
% ----------------------------------------------------------------------
%
% Absaetze, die mehreren Kapiteln der KOMA-Script-Anleitung gemeinsam sind
% Verwaltet von Markus Kohm
%
% ======================================================================

\KOMAProvidesFile{common-interleafpage.tex}%
                 [$Date$
                  KOMA-Script guide (common paragraphs: Interleaf Pages)]
\translator{Markus Kohm\and Gernot Hassenpflug\and Krickette Murabayashi}

% Date of the translated German file: 2017-01-02

\section{Interleaf Pages}
\seclabel{emptypage}%
\BeginIndexGroup
\BeginIndex{}{interleaf page}%%
\BeginIndex{}{page>style}%

\IfThisCommonFirstRun{}{%
  What is described in \autoref{sec:\ThisCommonFirstLabelBase.emptypage}
  applies, mutatis mutandis. So if you have alread read and understood
  \autoref{sec:\ThisCommonFirstLabelBase.emptypage} you can switch to
  \autoref{sec:\ThisCommonLabelBase.emptypage.next},
  \autopageref{sec:\ThisCommonLabelBase.emptypage.next}.%
}

Interleaf pages are pages that are intended to stay blank. Originally these
pages were really completely white. \LaTeX{}, on the other hand, by default
sets those pages with the current valid page style. So those pages may have a
head and a pagination. \KOMAScript{} provides several extensions to this.

Interleaf pages may be found in books mostly. Because chapters in books
commonly start on odd pages, sometimes a left page without contents has to be
added before. This is also the reason that interleaf pages only exist in
double-sided printing. The unused back sides of the one-sided printings are
not interleaf pages, really, although they may seem to be such pages.

\IfThisCommonLabelBase{scrlttr2}{%
  At letters interleaf pages are unusual. This may be benefited by the case,
  that real two-sided letters are seldom, because binding of letters is not
  done often. Nevertheless \Class{scrlttr2} supports interleaf pages in the
  case of two-sided letters. Because the following described commands are
  seldom used in letters no examples are shown. If you need examples, please
  note them at \autoref{sec:maincls.emptypage} from
  \autopageref{sec:maincls.emptypage} upward.%
}{}%

\begin{Declaration}
  \OptionVName{cleardoublepage}{page style}
  \OptionValue{cleardoublepage}{current}
\end{Declaration}%
With this option, %
\IfThisCommonLabelBase{maincls}{%
  \ChangedAt{v3.00}{\Class{scrbook}\and \Class{scrreprt}\and
    \Class{scrartcl}}%
}{%
  \IfThisCommonLabelBase{scrlttr2}{%
    \ChangedAt{v3.00}{\Class{scrlttr2}}%
  }{}%
} %
you may define the page style of the interleaf pages created by the
\DescRef{\LabelBase.cmd.cleardoublepage},
\DescRef{\LabelBase.cmd.cleardoubleoddpage}, or
\DescRef{\LabelBase.cmd.cleardoubleevenpage} to break until the wanted
page. Every already defined \PName{page style} (see
\autoref{sec:\ThisCommonLabelBase.pagestyle} from
\autopageref{sec:\ThisCommonLabelBase.pagestyle} and
\autoref{cha:scrlayer-scrpage} from \autopageref{cha:scrlayer-scrpage}) may be
used. Besides this, \OptionValue{cleardoublepage}{current} is valid. This case
is the default until \KOMAScript~2.98c and results in interleaf page without
changing the page style. Since \KOMAScript~3.00%
\IfThisCommonLabelBase{maincls}{%
  \ChangedAt{v3.00}{\Class{scrbook}\and \Class{scrreprt}\and
    \Class{scrartcl}}%
}{%
  \IfThisCommonLabelBase{scrlttr2}{%
    \ChangedAt{v3.00}{\Class{scrlttr2}}%
  }{}%
} %
the default follows the recommendation of most typographers and has been
changed to blank interleaf pages with page style \PageStyle{empty} unless you
switch compatibility to an earlier version (see option
\DescRef{\ThisCommonLabelBase.option.version},
\autoref{sec:\ThisCommonLabelBase.compatibilityOptions},
\DescPageRef{\ThisCommonLabelBase.option.version}).
\IfThisCommonLabelBase{maincls}{\iftrue}{\csname iffalse\endcsname}
  \begin{Example}
    \phantomsection\xmpllabel{option.cleardoublepage}%
    Assume you want interleaf pages almost empty but with pagination. This
    means you want to use page style \PageStyle{plain}. You may use following
    to achieve this:
\begin{lstcode}
  \KOMAoptions{cleardoublepage=plain}
\end{lstcode}
    More information about page style \PageStyle{plain} may be found at
    \autoref{sec:maincls.pagestyle},
    \DescPageRef{maincls.pagestyle.plain}.
  \end{Example}
\else
  \IfThisCommonLabelBase{scrextend}{\iftrue}{\csname iffalse\endcsname}
    \begin{Example}
      \phantomsection\xmpllabel{option.cleardoublepage}%
      Assume you want interleaf pages almost empty but with pagination. This
      means you want to use page style \PageStyle{plain}. You may use
      following to achieve this:
\begin{lstcode}
  \KOMAoptions{cleardoublepage=plain}
\end{lstcode}
      More information about page style \PageStyle{plain} may be found at
      \autoref{sec:maincls.pagestyle},
      \DescPageRef{maincls.pagestyle.plain}.
    \end{Example}%
  \fi%
\fi%
\EndIndexGroup


\begin{Declaration}
  \Macro{clearpage}%
  \Macro{cleardoublepage}%
  \Macro{cleardoublepageusingstyle}\Parameter{page style}%
  \Macro{cleardoubleemptypage}%
  \Macro{cleardoubleplainpage}%
  \Macro{cleardoublestandardpage}%
  \Macro{cleardoubleoddpage}%
  \Macro{cleardoubleoddpageusingstyle}\Parameter{page style}%
  \Macro{cleardoubleoddemptypage}%
  \Macro{cleardoubleoddplainpage}%
  \Macro{cleardoubleoddstandardpage}%
  \Macro{cleardoubleevenpage}%
  \Macro{cleardoubleevenpageusingstyle}\Parameter{page style}%
  \Macro{cleardoubleevenemptypage}%
  \Macro{cleardoubleevenplainpage}%
  \Macro{cleardoubleevenstandardpage}
\end{Declaration}%
The {\LaTeX} kernel contains the \Macro{clearpage} command, which takes
care that all not yet output floats are output, and then starts a new
page.  There exists the instruction \Macro{cleardoublepage} which
works like \Macro{clearpage} but which, in the double-sided layouts
(see layout option \Option{twoside} in
\autoref{sec:typearea.options},
\DescPageRef{typearea.option.twoside}) starts a new right-hand
page.  An empty left page in the current page style is output if
necessary.

With%
\IfThisCommonLabelBase{maincls}{%
  \ChangedAt{v3.00}{\Class{scrbook}\and \Class{scrreprt}\and
    \Class{scrartcl}}%
}{%
  \IfThisCommonLabelBase{scrlttr2}{%
    \ChangedAt{v3.00}{\Class{scrlttr2}}%
  }{}%
} %
\Macro{cleardoubleoddstandardpage}, {\KOMAScript} works as described above.
The \Macro{cleardoubleoddplainpage} command changes the page style of the
empty left page to \PageStyle{plain}\IndexPagestyle{plain} in order to
suppress the \IfThisCommonLabelBase{scrlttr2}{page}{running} head.
Analogously, the page style \PageStyle{empty}\IndexPagestyle{empty} is applied
to the empty page with \Macro{cleardoubleoddemptypage}, suppressing the page
number as well as the \IfThisCommonLabelBase{scrlttr2}{page}{running}
head. The page is thus entirely empty. If another \PName{page style} is wanted
for the interleaf page is may be set with the argument of
\Macro{cleardoubleoddusingpagestyle}. Every already defined \PName{page style}
(see \autoref{cha:scrlayer-scrpage}) may be used.

\IfThisCommonLabelBase{scrlttr2}{}{%
  Sometimes chapters should not start on the right-hand page but the left-hand
  page. This is in contradition to the classic typography; nevertheless, it
  may be suitable, e.\,g., if the double-page spread of the chapter start is
  of special contents. \KOMAScript{} therefor provides the commands
  \Macro{cleardoubleevenstandardpage}, \Macro{cleardoubleevenplainpage},
  \Macro{cleardoubleevenemptypage}, and \Macro{cleardoubleevenpageusingstyle},
  which are equivalent to the odd-page commands.%
}

However, the approach used by the \KOMAScript{} commands
\Macro{cleardoublestandardpage}, \Macro{cleardoubleemptypage},
\Macro{cleardoubleplainpage}, and \Macro{cleardoublepageusingstyle} is
dependent on the option
\Option{cleardoublepage}\important{\Option{cleardoublepage}} described above
and is similar to one of the corresponding commands above. The same is valid for
the standard command \Macro{cleardoublepage}, that may be either
\Macro{cleardoubleoddpage} or \Macro{cleardoubleevenpage}.

\IfThisCommonLabelBase{scrlttr2}{%
  In \Class{scrlttr2} the other commands are there only for completeness. More
  information about them may be found at
  \autoref{sec:maincls.emptypage},
  \DescPageRef{maincls.cmd.cleardoublepage} if needed.%
}{}%
\IfThisCommonLabelBase{scrlttr2}{\iffalse}{\csname iftrue\endcsname}
\begin{Example}
    \phantomsection\xmpllabel{cmd.cleardoublepage}%
    Assume you want to set next in your document a double-page spread
    with a picture at the left-hand page and a chapter start at the
    right-hand page. The picture should have the same size as the text
    area without any head line or pagination. If the last chapter 
    ends with a left-hand page, an interleaf page has to be added, which
    should be completely empty.

    First you will use
\begin{lstcode}
  \KOMAoptions{cleardoublepage=empty}
\end{lstcode}
    to make interleaf pages empty. You may use this setting at the document
    preamble already. As an alternative you may set it as the optional
    argument of \DescRef{\ThisCommonLabelBase.cmd.documentclass}.

    At the relevant place in your document, you'll write:
\begin{lstcode}
  \cleardoubleevenemptypage
  \thispagestyle{empty}
  \includegraphics[width=\textwidth,%
                   height=\textheight,%
                   keepaspectratio]%
                  {picture}
  \chapter{Chapter Headline}
\end{lstcode}
    The first of these lines switches to the next left page. If needed it also
    adds a completely blank right-hand page. The second line makes sure that
    the following left-hand page will be set using page style
    \PageStyle{empty} too. From third down to sixth line, an external picture
    of wanted size will be loaded without deformation. Package
    \Package{graphicx}\IndexPackage{graphicx} will be needed for this
    command. The last line starts a new chapter on the next page which will be
    a right-hand one.
  \end{Example}%
\fi%

The commands \Macro{cleardoubleoddpage} respective \Macro{cleardoubleevenpage}
leads to the next odd respectively even page. The page style of an interleaf
page will be set depending on option
\DescRef{\LabelBase.option.cleardoublepage}.%
%
\EndIndexGroup
%
\EndIndexGroup

%%% Local Variables:
%%% mode: latex
%%% mode: flyspell
%%% coding: us-ascii
%%% ispell-local-dictionary: "en_GB"
%%% TeX-master: "../guide"
%%% End:

%  LocalWords:  mutatis mutandis Interleaf interleaf
