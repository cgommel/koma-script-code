% ======================================================================
% common-7.tex
% Copyright (c) Markus Kohm, 2001-2012
%
% This file is part of the LaTeX2e KOMA-Script bundle.
%
% This work may be distributed and/or modified under the conditions of
% the LaTeX Project Public License, version 1.3c of the license.
% The latest version of this license is in
%   http://www.latex-project.org/lppl.txt
% and version 1.3c or later is part of all distributions of LaTeX 
% version 2005/12/01 or later and of this work.
%
% This work has the LPPL maintenance status "author-maintained".
%
% The Current Maintainer and author of this work is Markus Kohm.
%
% This work consists of all files listed in manifest.txt.
% ----------------------------------------------------------------------
% common-7.tex
% Copyright (c) Markus Kohm, 2001-2012
%
% Dieses Werk darf nach den Bedingungen der LaTeX Project Public Lizenz,
% Version 1.3c, verteilt und/oder veraendert werden.
% Die neuste Version dieser Lizenz ist
%   http://www.latex-project.org/lppl.txt
% und Version 1.3c ist Teil aller Verteilungen von LaTeX
% Version 2005/12/01 oder spaeter und dieses Werks.
%
% Dieses Werk hat den LPPL-Verwaltungs-Status "author-maintained"
% (allein durch den Autor verwaltet).
%
% Der Aktuelle Verwalter und Autor dieses Werkes ist Markus Kohm.
% 
% Dieses Werk besteht aus den in manifest.txt aufgefuehrten Dateien.
% ======================================================================
%
% Paragraphs that are common for several chapters of the KOMA-Script guide
% Maintained by Markus Kohm
%
% ----------------------------------------------------------------------
%
% Abs�tze, die mehreren Kapiteln der KOMA-Script-Anleitung gemeinsam sind
% Verwaltet von Markus Kohm
%
% ======================================================================

\ProvidesFile{common-7.tex}[2009/01/04 KOMA-Script guide (common paragraphs)]
\translator{Gernot Hassenpflug}

\makeatletter
\@ifundefined{ifCommonmaincls}{\newif\ifCommonmaincls}{}%
\@ifundefined{ifCommonscrextend}{\newif\ifCommonscrextend}{}%
\@ifundefined{ifCommonscrlttr}{\newif\ifCommonscrlttr}{}%
\@ifundefined{ifIgnoreThis}{\newif\ifIgnoreThis}{}%
\makeatother


\section{Erkennung von rechten und linken Seiten}
\label{sec:\csname label@base\endcsname.oddOrEven}%
\ifshortversion\IgnoreThisfalse\IfNotCommon{maincls}{\IgnoreThistrue}\fi%
\ifIgnoreThis%
Es gilt sinngem��, was in \autoref{sec:maincls.oddOrEven} geschrieben
wurde.
\else
\BeginIndex{}{Seite>gerade}%
\BeginIndex{}{Seite>ungerade}%

Bei doppelseitigen Dokumenten wird zwischen linken und rechten Seiten
unterschieden. Dabei hat eine linke Seite immer eine gerade Nummer und eine
rechte Seite immer eine ungerade Nummer. %
\IfCommon{maincls}{Die Erkennung von rechten und linken Seiten ist damit
  gleichbedeutend mit der Erkennung von Seiten mit gerader oder ungerader
  Nummer. In \iffree{dieser Anleitung}{diesem Buch} ist vereinfachend von
  ungeraden und geraden Seiten die Rede.%

% Umbruchkorrekturtext
  \iftrue%
  Bei einseitigen Dokumenten existiert die Unterscheidung zwischen linken und
  rechten Seiten nicht. Dennoch gibt es nat�rlich auch bei einseitigen
  Dokumenten sowohl Seiten mit gerader als auch Seiten mit ungerader Nummer.%
  \fi}%
\IfCommon{scrlttr2}{In der Regel werden Briefe einseitig
gesetzt. Sollen Briefe mit einseitigem Layout jedoch auf Vorder- und R�ckseite
gedruckt, oder ausnahmsweise tats�chlich doppelseitige Briefe erstellt werden,
kann unter Umst�nden das Wissen, ob man sich auf einer Vorder- oder einer
R�ckseite befindet, n�tzlich sein.}%

\begin{Declaration}
  \Macro{ifthispageodd}\Parameter{Dann-Teil}\Parameter{Sonst-Teil}
\end{Declaration}%
\BeginIndex{Cmd}{ifthispageodd}%
Will man bei \KOMAScript{} feststellen, ob ein Text auf einer geraden oder
einer ungeraden Seite ausgegeben wird, so verwendet man die Anweisung
\Macro{ifthispageodd}. Dabei wird das Argument \PName{Dann-Teil} nur dann
ausgef�hrt, wenn man sich gerade auf einer ungeraden Seite
befindet. Anderenfalls kommt das Argument \PName{Sonst-Teil} zur Anwendung.

\begin{Example}
  Angenommen, Sie wollen einfach nur ausgeben, ob ein Text auf einer geraden
  oder ungeraden Seite ausgegeben wird. Sie k�nnten dann beispielsweise mit
  der Eingabe
  \begin{lstcode}
  Dies ist eine Seite mit \ifthispageodd{un}{}gerader
  Seitenzahl.
  \end{lstcode}
  die Ausgabe
  \begin{quote}
    Dies ist eine Seite mit \ifthispageodd{un}{}gerader Seitenzahl.
  \end{quote}
  erhalten. Beachten Sie, dass in diesem Beispiel das Argument
  \PName{Sonst-Teil} leer geblieben ist.
\end{Example}

Da die Anweisung \Macro{ifthispageodd} mit einem Mechanismus arbeitet, der
einem Label und einer Referenz darauf sehr �hnlich ist, werden nach jeder
Text�nderung mindestens zwei \LaTeX-Durchl�ufe ben�tigt. Erst dann ist die
Entscheidung korrekt. Im ersten Durchlauf wird eine Heuristik f�r die
Entscheidung verwendet.

N�heres zur Problematik der Erkennung von linken und rechten Seiten oder
geraden und ungeraden Seitennummern ist f�r Experten in
\autoref{sec:maincls-experts.addInfos},
\autopageref{desc:maincls-experts.cmd.ifthispageodd} zu finden.%
%
\EndIndex{Cmd}{ifthispageodd}%
%
\EndIndex{}{Seite>ungerade}%
\EndIndex{}{Seite>gerade}
\fi % IgnoreThis


%%% Local Variables:
%%% mode: latex
%%% coding: iso-latin-1
%%% TeX-master: "../guide"
%%% End:
