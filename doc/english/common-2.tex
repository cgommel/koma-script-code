% ======================================================================
% common-2.tex
% Copyright (c) Markus Kohm, 2001-2009
%
% This file is part of the LaTeX2e KOMA-Script bundle.
%
% This work may be distributed and/or modified under the conditions of
% the LaTeX Project Public License, version 1.3c of the license.
% The latest version of this license is in
%   http://www.latex-project.org/lppl.txt
% and version 1.3c or later is part of all distributions of LaTeX 
% version 2005/12/01 or later and of this work.
%
% This work has the LPPL maintenance status "author-maintained".
%
% The Current Maintainer and author of this work is Markus Kohm.
%
% This work consists of all files listed in manifest.txt.
% ----------------------------------------------------------------------
% common-2.tex
% Copyright (c) Markus Kohm, 2001-2009
%
% Dieses Werk darf nach den Bedingungen der LaTeX Project Public Lizenz,
% Version 1.3c, verteilt und/oder veraendert werden.
% Die neuste Version dieser Lizenz ist
%   http://www.latex-project.org/lppl.txt
% und Version 1.3c ist Teil aller Verteilungen von LaTeX
% Version 2005/12/01 oder spaeter und dieses Werks.
%
% Dieses Werk hat den LPPL-Verwaltungs-Status "author-maintained"
% (allein durch den Autor verwaltet).
%
% Der Aktuelle Verwalter und Autor dieses Werkes ist Markus Kohm.
% 
% Dieses Werk besteht aus den in manifest.txt aufgefuehrten Dateien.
% ======================================================================
%
% Paragraphs that are common for several chapters of the KOMA-Script guide
% Maintained by Markus Kohm
%
% ----------------------------------------------------------------------
%
% Abs�tze, die mehreren Kapiteln der KOMA-Script-Anleitung gemeinsam sind
% Verwaltet von Markus Kohm
%
% ======================================================================

\ProvidesFile{common-2.tex}[2009/01/04 KOMA-Script guide (common paragraphs)]
\translator{Gernot Hassenpflug}

\makeatletter
\@ifundefined{ifCommonmaincls}{\newif\ifCommonmaincls}{}%
\@ifundefined{ifCommonscrextend}{\newif\ifCommonscrextend}{}%
\@ifundefined{ifCommonscrlttr}{\newif\ifCommonscrlttr}{}%
\@ifundefined{ifIgnoreThis}{\newif\ifIgnoreThis}{}%
\makeatother


\section{Entwurfsmodus}
\label{sec:\csname label@base\endcsname.draft}%
\ifshortversion\IgnoreThisfalse\IfNotCommon{maincls}{\IgnoreThistrue}\fi%
\ifIgnoreThis%
Es gilt sinngem��, was in \autoref{sec:maincls.draft} geschrieben wurde.
\else
\BeginIndex{}{Entwurf}%

Viele Klassen und viele Pakete kennen neben dem normalen Satzmodus auch einen
Entwurfsmodus. Die Unterschiede zwischen diesen beiden sind so vielf�ltig wie
die Klassen und Pakete, die diese Unterscheidung anbieten.

\begin{Declaration}
  \KOption{draft}\PName{Ein-Aus-Wert}
\end{Declaration}%
\BeginIndex{Option}{draft~=\PName{Ein-Aus-Wert}}%
Mit dieser Option\ChangedAt{v3.00}{\Class{scrbook}\and \Class{scrreprt}\and
  \Class{scrartcl}\and \Class{scrlttr2}} wird zwischen Dokumenten im
Entwurfsstadium und fertigen Dokumenten\Index{Endfassung} unterschieden. Als
\PName{Ein-Aus-Wert} kann einer der Standardwerte f�r einfache Schalter aus
\autoref{tab:truefalseswitch}, \autopageref{tab:truefalseswitch} verwendet
werden. Bei Aktivierung der Option werden im Falle �berlanger Zeilen am
Zeilenende kleine, schwarze K�stchen ausgegeben. Diese K�stchen erleichtern
dem unge�bten Auge, Abs�tze ausfindig zu machen, die manueller Nachbearbeitung
bed�rfen. Demgegen�ber erscheinen in der Standardeinstellung
\OptionValue{draft}{false} keine solchen K�stchen.%
%
\EndIndex{Option}{draft~=\PName{Ein-Aus-Wert}}%
%
\EndIndex{}{Entwurf}

\fi % IgnoreThis


%%% Local Variables:
%%% mode: latex
%%% coding: iso-latin-1
%%% TeX-master: "../guide"
%%% End:
