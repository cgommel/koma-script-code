% ======================================================================
% common-textmarkup.tex
% Copyright (c) Markus Kohm, 2001-2018
%
% This file is part of the LaTeX2e KOMA-Script bundle.
%
% This work may be distributed and/or modified under the conditions of
% the LaTeX Project Public License, version 1.3c of the license.
% The latest version of this license is in
%   http://www.latex-project.org/lppl.txt
% and version 1.3c or later is part of all distributions of LaTeX 
% version 2005/12/01 or later and of this work.
%
% This work has the LPPL maintenance status "author-maintained".
%
% The Current Maintainer and author of this work is Markus Kohm.
%
% This work consists of all files listed in manifest.txt.
% ----------------------------------------------------------------------
% common-textmarkup.tex
% Copyright (c) Markus Kohm, 2001-2018
%
% Dieses Werk darf nach den Bedingungen der LaTeX Project Public Lizenz,
% Version 1.3c, verteilt und/oder veraendert werden.
% Die neuste Version dieser Lizenz ist
%   http://www.latex-project.org/lppl.txt
% und Version 1.3c ist Teil aller Verteilungen von LaTeX
% Version 2005/12/01 oder spaeter und dieses Werks.
%
% Dieses Werk hat den LPPL-Verwaltungs-Status "author-maintained"
% (allein durch den Autor verwaltet).
%
% Der Aktuelle Verwalter und Autor dieses Werkes ist Markus Kohm.
% 
% Dieses Werk besteht aus den in manifest.txt aufgefuehrten Dateien.
% ======================================================================
%
% Paragraphs that are common for several chapters of the KOMA-Script guide
% Maintained by Markus Kohm
%
% ----------------------------------------------------------------------
%
% Absaetze, die mehreren Kapiteln der KOMA-Script-Anleitung gemeinsam sind
% Verwaltet von Markus Kohm
%
% ======================================================================

\KOMAProvidesFile{common-textmarkup.tex}
                 [$Date$
                  KOMA-Script guide (common paragraphs)]
\translator{Gernot Hassenpflug\and Markus Kohm\and Krickette Murabayashi\and
	Karl Hagen}

% Date of the translated German file: 2018-02-06

\section{Text Markup}
\seclabel{textmarkup}%
\BeginIndexGroup%
\BeginIndex{}{text>markup}%
\BeginIndex{}{font}%

\IfThisCommonFirstRun{}{%
  The information in in \autoref{sec:\ThisCommonFirstLabelBase.textmarkup}
  largely applies to this chapter. So if you have already read and understood
  \autoref{sec:\ThisCommonFirstLabelBase.textmarkup.next}, you can
  \IfThisCommonLabelBaseOneOf{scrextend,scrjura,scrlayer-notecolumn}{}{%
    limit yourself to examining
    \autoref{tab:\ThisCommonLabelBase.fontelements},
    \autopageref{tab:\ThisCommonLabelBase.fontelements} and then }%
  skip ahead to \autoref{sec:\ThisCommonLabelBase.textmarkup.next},
  \autopageref{sec:\ThisCommonLabelBase.textmarkup.next}.%
  \IfThisCommonLabelBase{scrextend}{\ In this case, however,
    note\textnote{limitation} that \Package{scrextend} supports only the
    elements for the document title, the dictum, the footnotes, and the
    \DescRef{maincls.env.labeling} environment.  from
    \autoref{tab:maincls.fontelements},
    \autopageref{tab:maincls.fontelements}.  Although the
    \DescRef{maincls.fontelement.disposition} element exists,
    \Package{scrextend} uses it only for the document title.%
  }{}%
}

% Umbruchkorrektur
\IfThisCommonLabelBase{scrlayer-scrpage}{}{%
  {\LaTeX} offers different possibilities for logical and direct
  markup\Index{logical markup}\Index{markup} of text. %
  \IfThisCommonLabelBaseOneOf{scrlttr2}{}{%
    In addition to the choice of the font, this includes commands for choosing
    the font size and orientation. %
  } For more information about the standard font facilities, see
  \cite{lshort}, \cite{latex:usrguide}, and \cite{latex:fntguide}.}

\IfThisCommonLabelBaseOneOf{scrlayer-scrpage,scrjura,scrlayer-notecolumn}{%
  \iffalse}{%
  \csname iftrue\endcsname}%
  \begin{Declaration}
    \Macro{textsuperscript}\Parameter{text}%
    \Macro{textsubscript}\Parameter{text}
  \end{Declaration}
  The \LaTeX{} kernel defines the command
  \Macro{textsuperscript}\IndexCmd{textsuperscript} to put text in
  superscript\Index{text>superscript}\Index{superscript}. Unfortunately,
  \LaTeX{}\textnote{\Latex~2015/01/01} itself did not offer a command to
  produce text in subscript\Index{text>subscript}\Index{subscript} until
  release 2015/01/01. \KOMAScript{} defines \Macro{textsubscript} for this
  purpose. %
  \ifthiscommonfirst
    \begin{Example}
      \phantomsection
      \xmpllabel{cmd.textsubscript}%
      You are writing a text on human metabolism. From time to time you
      have to give some simple chemical formulas in which the numbers are
      in subscript. To allow for logical markup, you first define in the
      document preamble or in a separate package:
\begin{lstcode}
  \newcommand*{\molec}[2]{#1\textsubscript{#2}}
\end{lstcode}
      \newcommand*{\molec}[2]{#1\textsubscript{#2}}
      Using this you then write:
\begin{lstcode}
  The cell produces its energy partly from the reaction of \molec C6\molec
  H{12}\molec O6 and \molec O2 to produce \molec H2\Molec O{} and
  \molec C{}\molec O2.  However, arsenic (\molec{As}{}) has quite a
  detrimental effect on the metabolism.
\end{lstcode}
      The output looks as follows:
      \begin{ShowOutput}
        The cell produces its energy partly from the reaction of \molec C6\molec
        H{12}\molec O6 and \molec O2 to produce \molec H2\molec O{} and
        \molec C{}\molec O2.  However, arsenic (\molec{As}{}) has quite a
        detrimental effect on the metabolism.
      \end{ShowOutput}

      Some time later you decide that the chemical formulas should be
      typeset in sans serif. Now you can see the advantages of using
      logical markup. You only have the redefine the \Macro{molec}
      command:
\begin{lstcode}
  \newcommand*{\molec}[2]{\textsf{#1\textsubscript{#2}}}
\end{lstcode}
      \renewcommand*{\molec}[2]{\textsf{#1\textsubscript{#2}}}
      Now the output in the whole document changes to:
      \begin{ShowOutput}
        The cell produces its energy partly from the reaction of \molec
        C6\molec H{12}\molec O6 and \molec O2 to produce \molec H2\molec
        O{} and \molec C{}\molec O2.  However, arsenic (\molec{As}{}) has
        quite a detrimental effect on the metabolism.
      \end{ShowOutput}
    \end{Example}
    \iftrue
      \begin{Explain}
        The example above uses the notation ``\verb|\molec C6|''. 
        This makes use of the fact that arguments consisting of only one
        character do not have to be enclosed in parentheses. That is why
        ``\verb|\molec C6|'' is similar to ``\verb|\molec{C}{6}|''. You
        may already be familiar with this notation from indices or powers in
        mathematical environments, such as ``\verb|$x^2$|'' instead of
        ``\verb|$x^{2}$|''
        for ``$x^2$''.
      \end{Explain}
    \else % maybe some time I've made an English book
      Advanced users can find information about the reason the example above
      does work unless you put all arguments of \Macro{molec} into braces in
      \autoref{sec:experts.knowhow},
      \DescPageRef{experts.macroargs}.%
    \fi%
  \else%
    You can find a usage example at
    \autoref{sec:\ThisCommonFirstLabelBase.textmarkup},
    \PageRefxmpl{\ThisCommonFirstLabelBase.cmd.textsubscript}.
  \fi%
  \EndIndexGroup%
\fi


\begin{Declaration}
  \Macro{setkomafont}\Parameter{element}\Parameter{commands}%
  \Macro{addtokomafont}\Parameter{element}\Parameter{commands}%
  \Macro{usekomafont}\Parameter{element}
\end{Declaration}%
With%
\IfThisCommonLabelBase{maincls}{%
  \ChangedAt{v2.8p}{\Class{scrbook}\and \Class{scrreprt}\and
    \Class{scrartcl}}%
}{} the help of the \Macro{setkomafont} and \Macro{addtokomafont}
commands, you can attach particular font styling \PName{commands} that change
the appearance of a given \PName{element}. Theoretically, all statements,
including literal text, can be used as \PName{commands}. You
should\textnote{Attention!}, however, limit yourself to those statements that
really change font attributes only. These are usually commands like
\Macro{rmfamily}, \Macro{sffamily}, \Macro{ttfamily}, \Macro{upshape},
\Macro{itshape}, \Macro{slshape}, \Macro{scshape}, \Macro{mdseries},
\Macro{bfseries}, \Macro{normalfont}, as well as the font size commands
\Macro{Huge}, \Macro{huge}, \Macro{LARGE}, \Macro{Large}, \Macro{large},
\Macro{normalsize}, \Macro{small}, \Macro{footnotesize}, \Macro{scriptsize},
and \Macro{tiny}. You can find these commands explained in \cite{lshort},
\cite{latex:usrguide}, or \cite{latex:fntguide}. Colour switching commands
like \Macro{normalcolor} (see \cite{package:graphics} and
\cite{package:xcolor}) are also acceptable.%
\iffalse % Umbruchkorrekturtext
  \ The behaviour when using other commands, especially those that lead to
  redefinitions or generate output, is undefined. Strange behaviour is possible
  and does not represent a bug.
\else
  \ The use of other commands, in particular those that redefine things or
  or lead to output, is not supported. Strange behaviour is possible in these
  cases and does not represent a bug.
\fi

The command \Macro{setkomafont} provides an element with a completely new
definition of its font styling. In contrast, the \Macro{addtokomafont} command
merely extends an existing definition. It is recommended not to use either
command inside the document body but only in the preamble. For examples of
their use, refer to the sections for the respective element.%
\IfThisCommonLabelBase{scrlayer-notecolumn}{}{%
  \ The name and meaning of each element
  \IfThisCommonLabelBaseOneOf{scrlayer-scrpage,scrjura}{, as well as their
    defaults,}{} are listed in \IfThisCommonLabelBase{scrextend}{%
    \autoref{tab:maincls.fontelements}, \autopageref{tab:maincls.fontelements}
  }{%
    \autoref{tab:\ThisCommonLabelBase.fontelements} %
  }.%
  \IfThisCommonLabelBase{scrextend}{ %
    However, in \Package{scrextend} only\textnote{limitation} the listed
    elements for the document title, dictum, footnotes, and the
    \DescRef{maincls.env.labeling} environment are supported. Although the
    \DescRef{maincls.fontelement.disposition} element exists,
    \Package{scrextend} uses it only for the document title.%
  }{%
    \IfThisCommonLabelBase{scrlayer-scrpage}{ %
      The specified defaults apply only if the corresponding element has not
      already been defined before loading \Package{scrlayer-scrpage}. For
      example, the \KOMAScript classes define
      \DescRef{maincls.fontelement.pageheadfoot}, and then
      \Package{scrlayer-scrpage} uses the setting it finds.%
    }{%
      \IfThisCommonLabelBase{scrjura}{}{ %
        The default values can be found in the corresponding sections.%
      }%
    }%
  }%
}%

\IfThisCommonLabelBaseOneOf{scrlttr2,scrextend}{% Umbruchvarianten
  The \Macro{usekomafont} command can be used to switch the current font style
  to the specified \PName{Element}.%
}{%
  With the \Macro{usekomafont} command, the current font style can be changed
  to the one defined for the specified \PName{element}.%
}

\IfThisCommonLabelBase{maincls}{\iftrue}{\csname iffalse\endcsname}
  \begin{Example}
    \phantomsection\xmpllabel{cmd.setkomafont}%
    Suppose you want to use the same font specification for the element
    \DescRef{\ThisCommonLabelBase.fontelement.captionlabel}
    that is used with
    \DescRef{\ThisCommonLabelBase.fontelement.descriptionlabel}. This can be
    easily done with:
\begin{lstcode}
  \setkomafont{captionlabel}{%
    \usekomafont{descriptionlabel}%
  }
\end{lstcode}
    You can find other examples in the explanation of each element.
  \end{Example}

  \begin{desclist}
    \desccaption{%
      Elements whose font style can be changed in \Class{scrbook},
      \Class{scrreprt} or \Class{scrartcl} with \Macro{setkomafont} and
      \Macro{addtokomafont}%
      \label{tab:maincls.fontelements}%
      \label{tab:scrextend.fontelements}%     
    }{%
      Elements whose font style can be changed (\emph{continued})%
    }%
    \feentry{author}{%
      \ChangedAt{v3.12}{\Class{scrbook}\and \Class{scrreprt}\and
        \Class{scrartcl}\and \Package{scrextend}}%
      author of the document in the title, i.\,e., the argument of
      \DescRef{\ThisCommonLabelBase.cmd.author} when
      \DescRef{\ThisCommonLabelBase.cmd.maketitle} is used (see
      \autoref{sec:maincls.titlepage}, \DescPageRef{maincls.cmd.author})}%
    \feentry{caption}{text of a figure or table caption (see
      \autoref{sec:maincls.floats}, \DescPageRef{maincls.cmd.caption})}%
    \feentry{captionlabel}{label of a figure or table caption; applied in
      addition to the \DescRef{\ThisCommonLabelBase.fontelement.caption}
      element (see \autoref{sec:maincls.floats},
      \DescPageRef{maincls.cmd.caption})}%
    \feentry{chapter}{title of the sectioning command
      \DescRef{\ThisCommonLabelBase.cmd.chapter} (see
      \autoref{sec:maincls.structure}, \DescPageRef{maincls.cmd.chapter})}%
    \feentry{chapterentry}{%
      table of contents entry for the sectioning command
      \DescRef{\ThisCommonLabelBase.cmd.chapter} (see
      \autoref{sec:maincls.toc}, \DescPageRef{maincls.cmd.tableofcontents})}%
    \feentry{chapterentrydots}{%
      \ChangedAt{v3.15}{\Class{scrbook}\and \Class{scrreprt}}%
      optional points connecting table-of-content entries for the
      \DescRef{\ThisCommonLabelBase.cmd.chapter} level, differing from the
      \DescRef{\ThisCommonLabelBase.fontelement.chapterentrypagenumber}
      element (see \autoref{sec:maincls.toc},
      \DescPageRef{maincls.cmd.tableofcontents})}%
    \feentry{chapterentrypagenumber}{%
      page number of the table of contents entry for the sectioning command
      \DescRef{\ThisCommonLabelBase.cmd.chapter}, differing from the element
      \DescRef{\ThisCommonLabelBase.fontelement.chapterentry} (see
      \autoref{sec:maincls.toc}, \DescPageRef{maincls.cmd.tableofcontents})}%
    \feentry{chapterprefix}{%
      label, e.\,g., ``Chapter'', appearing before the chapter number in both
      \OptionValueRef{maincls}{chapterprefix}{true} and
      \OptionValueRef{maincls}{appendixprefix}{true} (see
      \autoref{sec:maincls.structure},
      \DescPageRef{maincls.option.chapterprefix})}%
    \feentry{date}{%
      \ChangedAt{v3.12}{\Class{scrbook}\and \Class{scrreprt}\and
        \Class{scrartcl}\and \Package{scrextend}}%
      date of the document in the main title, i.\,e., the argument of
      \DescRef{\ThisCommonLabelBase.cmd.date} when
      \DescRef{\ThisCommonLabelBase.cmd.maketitle} is used (see
      \autoref{sec:maincls.titlepage}, \DescPageRef{maincls.cmd.date})}%
    \feentry{dedication}{%
      \ChangedAt{v3.12}{\Class{scrbook}\and \Class{scrreprt}\and
        \Class{scrartcl}\and \Package{scrextend}}%
      dedication page after the main title, i.\,e., the argument of
      \DescRef{\ThisCommonLabelBase.cmd.dedication} when
      \DescRef{\ThisCommonLabelBase.cmd.maketitle} is used (see
      \autoref{sec:maincls.titlepage}, \DescPageRef{maincls.cmd.dedication})}%
    \feentry{descriptionlabel}{labels, i.\,e., the optional argument of
      \DescRef{\ThisCommonLabelBase.cmd.item.description} in the
      \DescRef{\ThisCommonLabelBase.env.description} environment (see
      \autoref{sec:maincls.lists}, \DescPageRef{maincls.env.description})}%
    \feentry{dictum}{dictum or epigraph (see \autoref{sec:maincls.dictum},
      \DescPageRef{maincls.cmd.dictum})}%
    \feentry{dictumauthor}{author of a dictum or epigraph; applied in addition
      to the element \DescRef{\ThisCommonLabelBase.fontelement.dictum} (see
      \autoref{sec:maincls.dictum}, \DescPageRef{maincls.cmd.dictum})}%
    \feentry{dictumtext}{alternative name for
      \DescRef{\ThisCommonLabelBase.fontelement.dictum}}%
    \feentry{disposition}{all sectioning command titles, i.\,e., the arguments
      of \DescRef{\ThisCommonLabelBase.cmd.part} down to
      \DescRef{\ThisCommonLabelBase.cmd.subparagraph} and
      \DescRef{\ThisCommonLabelBase.cmd.minisec}, including the title of the
      abstract; applied before the element of the respective unit (see
      \autoref{sec:maincls.structure}, \autopageref{sec:maincls.structure})}%
    \feentry{footnote}{footnote text and marker (see
      \autoref{sec:maincls.footnotes}, \DescPageRef{maincls.cmd.footnote})}%
    \feentry{footnotelabel}{marker for a footnote; applied in addition to the
      element \DescRef{\ThisCommonLabelBase.fontelement.footnote} (see
      \autoref{sec:maincls.footnotes}, \DescPageRef{maincls.cmd.footnote})}%
    \feentry{footnotereference}{footnote reference in the text (see
      \autoref{sec:maincls.footnotes}, \DescPageRef{maincls.cmd.footnote})}%
    \feentry{footnoterule}{%
      horizontal rule\ChangedAt{v3.07}{\Class{scrbook}\and
        \Class{scrreprt}\and \Class{scrartcl}} above the footnotes at the end
      of the text area (see \autoref{sec:maincls.footnotes},
      \DescPageRef{maincls.cmd.setfootnoterule})}%
    \feentry{labelinglabel}{labels, i.\,e., the optional argument of
      \DescRef{\ThisCommonLabelBase.cmd.item.labeling} in the
      \DescRef{\ThisCommonLabelBase.env.labeling} environment (see
      \autoref{sec:maincls.lists}, \DescPageRef{maincls.env.labeling})}%
    \feentry{labelingseparator}{separator, i.\,e., the optional argument of
      the \DescRef{\ThisCommonLabelBase.env.labeling} environment; applied in
      addition to the element
      \DescRef{\ThisCommonLabelBase.fontelement.labelinglabel} (see
      \autoref{sec:maincls.lists}, \DescPageRef{maincls.env.labeling})}%
    \feentry{minisec}{title of \DescRef{\ThisCommonLabelBase.cmd.minisec} (see
      \autoref{sec:maincls.structure} ab \DescPageRef{maincls.cmd.minisec})}%
    \feentry{pagefoot}{only used if package \Package{scrlayer-scrpage} has
      been loaded (see \autoref{cha:scrlayer-scrpage},
      \DescPageRef{scrlayer-scrpage.fontelement.pagefoot})}%
    \feentry{pagehead}{alternative name for
      \DescRef{\ThisCommonLabelBase.fontelement.pageheadfoot}}%
    \feentry{pageheadfoot}{the header and footer of a page (see
      \autoref{sec:maincls.pagestyle} from
      \autopageref{sec:maincls.pagestyle})}%
    \feentry{pagenumber}{page number in the header or footer (see
      \autoref{sec:maincls.pagestyle})}%
    \feentry{pagination}{alternative name for
      \DescRef{\ThisCommonLabelBase.fontelement.pagenumber}}%
    \feentry{paragraph}{title of the sectioning command
      \DescRef{\ThisCommonLabelBase.cmd.paragraph} (see
      \autoref{sec:maincls.structure}, \DescPageRef{maincls.cmd.paragraph})}%
    \feentry{part}{title of the \DescRef{\ThisCommonLabelBase.cmd.part}
      sectioning command, without the line containing the part number (see
      \autoref{sec:maincls.structure}, \DescPageRef{maincls.cmd.part})}%
    \feentry{partentry}{%
      table of contents entry for the sectioning command
      \DescRef{\ThisCommonLabelBase.cmd.part} (see \autoref{sec:maincls.toc},
      \DescPageRef{maincls.cmd.tableofcontents})}%
    \feentry{partentrypagenumber}{%
      page number of the table of contents entry for the sectioning command
      \DescRef{\ThisCommonLabelBase.cmd.part}; applied in addition to the
      element \DescRef{\ThisCommonLabelBase.fontelement.partentry} (see
      \autoref{sec:maincls.toc}, \DescPageRef{maincls.cmd.tableofcontents})}%
    \feentry{partnumber}{line containing the part number in a title of the
      sectioning command \DescRef{\ThisCommonLabelBase.cmd.part} (see
      \autoref{sec:maincls.structure}, \DescPageRef{maincls.cmd.part})}%
    \feentry{publishers}{%
      \ChangedAt{v3.12}{\Class{scrbook}\and \Class{scrreprt}\and
        \Class{scrartcl}\and \Package{scrextend}}%
      publishers of the document in the main title, i.\,e., the argument of
      \DescRef{\ThisCommonLabelBase.cmd.publishers} when
      \DescRef{\ThisCommonLabelBase.cmd.maketitle} is used (see
      \autoref{sec:maincls.titlepage}, \DescPageRef{maincls.cmd.publishers})}%
    \feentry{section}{title of the sectioning command
      \DescRef{\ThisCommonLabelBase.cmd.section} (see
      \autoref{sec:maincls.structure}, \DescPageRef{maincls.cmd.section})}%
    \feentry{sectionentry}{%
      table of contents entry for sectioning command
      \DescRef{\ThisCommonLabelBase.cmd.section} (only available in
      \Class{scrartcl}, see \autoref{sec:maincls.toc},
      \DescPageRef{maincls.cmd.tableofcontents})}%
    \feentry{sectionentrypagenumber}{%
      page number of the table of contents entry for the sectioning command
      \DescRef{\ThisCommonLabelBase.cmd.section}; applied in addition to
      element \DescRef{\ThisCommonLabelBase.fontelement.sectionentry} (only
      available in \Class{scrartcl, see \autoref{sec:maincls.toc},
        \DescPageRef{maincls.cmd.tableofcontents}})}%
    \feentry{sectioning}{alternative name for
      \DescRef{\ThisCommonLabelBase.fontelement.disposition}}%
    \feentry{subject}{%
      topic of the document, i.\,e., the argument of
      \DescRef{\ThisCommonLabelBase.cmd.subject} on the main title page (see
      \autoref{sec:maincls.titlepage}, \DescPageRef{maincls.cmd.subject})}%
    \feentry{subparagraph}{title of the sectioning command
      \DescRef{\ThisCommonLabelBase.cmd.subparagraph} (see
      \autoref{sec:maincls.structure},
      \DescPageRef{maincls.cmd.subparagraph})}%
    \feentry{subsection}{title of the sectioning command
      \DescRef{\ThisCommonLabelBase.cmd.subsection} (see
      \autoref{sec:maincls.structure}, \DescPageRef{maincls.cmd.subsection})}%
    \feentry{subsubsection}{title of the sectioning command
      \DescRef{\ThisCommonLabelBase.cmd.subsubsection} (see
      \autoref{sec:maincls.structure},
      \DescPageRef{maincls.cmd.subsubsection})}%
    \feentry{subtitle}{%
      subtitle of the document, i.\,e., the argument of
      \DescRef{\ThisCommonLabelBase.cmd.subtitle} on the main title page (see
      \autoref{sec:maincls.titlepage}, \DescPageRef{maincls.cmd.title})}%
    \feentry{title}{main title of the document, i.\,e., the argument of
      \DescRef{\ThisCommonLabelBase.cmd.title} (for details about the title
      size see the additional note in the text of
      \autoref{sec:maincls.titlepage} from \DescPageRef{maincls.cmd.title})}%
    \feentry{titlehead}{%
      \ChangedAt{v3.12}{\Class{scrbook}\and \Class{scrreprt}\and
        \Class{scrartcl}\and \Package{scrextend}}%
      heading above the main title of the document, i.\,e., the argument of
      \DescRef{\ThisCommonLabelBase.cmd.titlehead} when
      \DescRef{\ThisCommonLabelBase.cmd.maketitle} is used (see
      \autoref{sec:maincls.titlepage}, \DescPageRef{maincls.cmd.titlehead})}%
  \end{desclist}
\else
  \IfThisCommonLabelBase{scrextend}{\iftrue}{\csname iffalse\endcsname}
    \begin{Example}
      Suppose you want to print the document title in a red serif font.
      You can do this using:
\begin{lstcode}
  \setkomafont{title}{\color{red}}
\end{lstcode}
      You will need the \Package{color} or the \Package{xcolor} package for
      the \Macro{color}\PParameter{red} command. Using \Macro{normalfont} is
      unnecessary in this case because it is already part of the definition of
      the title itself. This\textnote{Attention!} example also needs the
      \OptionValueRef{scrextend}{extendedfeature}{title} option (see
      \autoref{sec:scrextend.optionalFeatures}, 
      \DescPageRef{scrextend.option.extendedfeature}).
    \end{Example}
  \else
    \IfThisCommonLabelBase{scrlttr2}{%
      A general example for the usage of both \Macro{setkomafont} and
      \Macro{usekomafont} can be found in \autoref{sec:maincls.textmarkup} at
      \PageRefxmpl{maincls.cmd.setkomafont}.

      \begin{desclist}
        \desccaption{%
          Elements whose font style can be changed in the \Class{scrlttr2}
          class or the \Package{scrletter} package with the
          \Macro{setkomafont} and \Macro{addtokomafont}
          commands\label{tab:scrlttr2.fontelements}%
        }{%
          Elements whose font style can be changed (\emph{continued})%
        }%
        \feentry{addressee}{name and address in address field %
          (\autoref{sec:scrlttr2.firstpage},
          \DescPageRef{scrlttr2.option.addrfield})}%
        \feentry{backaddress}{%
          return address for a window envelope %
          (\autoref{sec:scrlttr2.firstpage},
          \DescPageRef{scrlttr2.option.backaddress})}%
        \feentry{descriptionlabel}{%
          label, i.\,e., the optional argument of
          \DescRef{\ThisCommonLabelBase.cmd.item.description}, in a
          \DescRef{\ThisCommonLabelBase.env.description} environment %
          (\autoref{sec:scrlttr2.lists},
          \DescPageRef{scrlttr2.env.description})}%
        \feentry{foldmark}{%
          foldmark on the stationery; allows change of line colour %
          (\autoref{sec:scrlttr2.firstpage},
          \DescPageRef{scrlttr2.option.foldmarks})}%
        \feentry{footnote}{%
          footnote text and marker %
          (\autoref{sec:scrlttr2.footnotes},
          \DescPageRef{scrlttr2.cmd.footnote})}%
        \feentry{footnotelabel}{%
          footnote marker; applied in addition to the element
          \DescRef{\ThisCommonLabelBase.fontelement.footnote} %
          (\autoref{sec:scrlttr2.footnotes},
          \DescPageRef{scrlttr2.cmd.footnote})}%
        \feentry{footnotereference}{%
          footnote reference in the text %
          (\autoref{sec:scrlttr2.footnotes},
          \DescPageRef{scrlttr2.cmd.footnote})}%
        \feentry{footnoterule}{%
          horizontal rule\ChangedAt{v3.07}{\Class{scrlttr2}} above the
          footnotes at the end of the text area %
          (\autoref{sec:maincls.footnotes},
          \DescPageRef{maincls.cmd.setfootnoterule})}%
        \feentry{fromaddress}{sender's address in the letterhead
          (\autoref{sec:scrlttr2.firstpage},
          \DescPageRef{scrlttr2.variable.fromaddress})}%
        \feentry{fromname}{sender's name in the letterhead other than
          \PValue{fromaddress} (\autoref{sec:scrlttr2.firstpage},
          \DescPageRef{scrlttr2.variable.fromname})}%
        \feentry{fromrule}{Horizontal rule in the letterhead; intended for
          colour changes (\autoref{sec:scrlttr2.firstpage},
          \DescPageRef{scrlttr2.option.fromrule})}%
        \feentry{labelinglabel}{%
          labels, i.\,e., the optional argument of
          \DescRef{\ThisCommonLabelBase.cmd.item.labeling} in the
          \DescRef{\ThisCommonLabelBase.env.labeling} environment %
          (see \autoref{sec:scrlttr2.lists},
          \DescPageRef{scrlttr2.env.labeling})}%
        \feentry{labelingseparator}{%
          separator, i.\,e., the optional argument of the
          \DescRef{\ThisCommonLabelBase.env.labeling} environment; applied
          in addition to the element
          \DescRef{\ThisCommonLabelBase.fontelement.labelinglabel} %
          (see \autoref{sec:scrlttr2.lists},
          \DescPageRef{scrlttr2.env.labeling})}%
        \feentry{pagefoot}{%
          used after element
          \DescRef{\ThisCommonLabelBase.fontelement.pageheadfoot} for the page
          footer that has been defined with variable
          \DescRef{scrlttr2.variable.nextfoot}\IndexVariable{nextfoot}, or for
          the page footer of the \Package{scrlayer-scrpage} package
          (\autoref{cha:scrlayer-scrpage},
          \DescPageRef{scrlayer-scrpage.fontelement.pagefoot})}%
        \feentry{pagehead}{%
          alternative name for
          \DescRef{\ThisCommonLabelBase.fontelement.pageheadfoot}}%
        \feentry{pageheadfoot}{%
          the header and footer of a page for all page styles
          that have been defined using \KOMAScript{}
          (\autoref{sec:maincls.pagestyle},
          \DescPageRef{\ThisCommonLabelBase.fontelement.pageheadfoot})}%
        \feentry{pagenumber}{%
          page number in the header or footer %
          (\autoref{sec:maincls.pagestyle},
          \DescPageRef{\ThisCommonLabelBase.fontelement.pagenumber})}%
        \feentry{pagination}{%
          another name for
          \DescRef{\ThisCommonLabelBase.fontelement.pagenumber}}%
        \feentry{placeanddate}{%
          \ChangedAt{v3.12}{\Class{scrlttr2}}%
          place and date, if a date line will be used instead of a normal
          reference line (\autoref{sec:scrlttr2.firstpage},
          \DescPageRef{scrlttr2.variable.placeseparator})}%
        \feentry{refname}{%
          description or title of the fields in the reference line %
          (\autoref{sec:scrlttr2.firstpage},
          \DescPageRef{scrlttr2.variable.yourref})}%
        \feentry{refvalue}{%
          content of the fields in the reference line %
          (\autoref{sec:scrlttr2.firstpage},
          \DescPageRef{scrlttr2.variable.yourref})}%
        \feentry{specialmail}{%
          mode of dispatch in the address field %
          (\autoref{sec:scrlttr2.firstpage},
          \DescPageRef{scrlttr2.variable.specialmail})}%
        \feentry{lettersubject}{%
          \ChangedAt{v3.17}{\Class{scrlttr2}\and \Package{scrletter}}%
          subject in the opening of the letter %
          (\autoref{sec:scrlttr2.firstpage},
          \DescPageRef{scrlttr2.variable.subject})}%
        \feentry{lettertitle}{%
          \ChangedAt{v3.17}{\Class{scrlttr2}\and \Package{scrletter}}%
          title in the opening of the letter %
          (\autoref{sec:scrlttr2.firstpage},
          \DescPageRef{scrlttr2.variable.title})}%
        \feentry{toaddress}{%
          variation of the element
          \DescRef{\ThisCommonLabelBase.fontelement.addressee} for setting the
          addressee address (not including the name) in the address field %
          (\autoref{sec:scrlttr2.firstpage},
          \DescPageRef{scrlttr2.variable.toaddress})}%
        \feentry{toname}{%
          variation of the element
          \DescRef{\ThisCommonLabelBase.fontelement.addressee} for the name
          of the addressee in the address field %
          (\autoref{sec:scrlttr2.firstpage},
          \DescPageRef{scrlttr2.variable.toname})}%
      \end{desclist}
    }{%
      \IfThisCommonLabelBase{scrlayer-scrpage}{%
        \begin{desclist}
          \desccaption[{Elements of \Package{scrlayer-scrpage} whose font
            styles can be changed with the \Macro{setkomafont} and
            \Macro{addtokomafont} commands}]%
          {Elements of \Package{scrlayer-scrpage} whose font styles can be
            changed with the \Macro{setkomafont} and \Macro{addtokomafont}
            commands, and their defaults, if they have not been defined
            before loading \Package{scrlayer-scrpage}%
            \label{tab:scrlayer-scrpage.fontelements}%
          }%
          {Elements whose font style can be changed (\emph{continued})}%
          \feentry{footbotline}{%
            Horizontal line below the footer of a page style defined using
            \Package{scrlayer-scrpage}. The font will be applied after
            \Macro{normalfont}\IndexCmd{normalfont} and the fonts of elements
            \DescRef{\ThisCommonLabelBase.fontelement.pageheadfoot}%
            \IndexFontElement{pageheadfoot} and
            \DescRef{\ThisCommonLabelBase.fontelement.pagefoot}%
            \IndexFontElement{pagefoot}. It is recommended to use this element
            for colour changes only.\par
            \mbox{Default: \emph{empty}}%
          }%
          \feentry{footsepline}{%
            Horizontal line above the footer of a page style defined using
            \Package{scrlayer-scrpage}. The font will be applied after
            \Macro{normalfont}\IndexCmd{normalfont} and the fonts of elements
            \DescRef{\ThisCommonLabelBase.fontelement.pageheadfoot}%
            \IndexFontElement{pageheadfoot} and
            \DescRef{\ThisCommonLabelBase.fontelement.pagefoot}%
            \IndexFontElement{pagefoot}. It is recommended to use this element
            for colour changes only.\par
            \mbox{Default: \emph{empty}}%
          }%
          \feentry{headsepline}{%
            Horizontal line below the header of a page style defined using
            \Package{scrlayer-scrpage}. The font will be applied after
            \Macro{normalfont}\IndexCmd{normalfont} and the fonts of elements
            \DescRef{\ThisCommonLabelBase.fontelement.pageheadfoot}%
            \IndexFontElement{pageheadfoot} and
            \DescRef{scrlayer-scrpage.fontelement.pagehead}%
            \IndexFontElement{pagehead}. It is recommended to use this element
            for colour changes only.\par
            Default: \emph{empty}%
          }%
          \feentry{headtopline}{%
            Horizontal line above the header of a page style defined using
            \Package{scrlayer-scrpage}. The font will be applied after
            \Macro{normalfont}\IndexCmd{normalfont} and the fonts of elements
            \DescRef{\ThisCommonLabelBase.fontelement.pageheadfoot}%
            \IndexFontElement{pageheadfoot} and
            \DescRef{scrlayer-scrpage.fontelement.pagehead}%
            \IndexFontElement{pagehead}. It is recommended to use this element
            for colour changes only.\par
            \mbox{Default: \emph{empty}}%
          }%
          \feentry{pagefoot}{%
            Contents of the page footer of a page style defined using
            \Package{scrlayer-scrpage}. The font will be applied after
            \Macro{normalfont}\IndexCmd{normalfont} and the font of element
            \DescRef{\ThisCommonLabelBase.fontelement.pageheadfoot}%
            \IndexFontElement{pageheadfoot}.\par
            \mbox{Default: \emph{empty}}%
          }%
          \feentry{pagehead}{%
            Contents of the page header of a page style defined using
            \Package{scrlayer-scrpage}. The font will be applied after
            \Macro{normalfont}\IndexCmd{normalfont} and the font of element
            \DescRef{\ThisCommonLabelBase.fontelement.pageheadfoot}%
            \IndexFontElement{pageheadfoot}.\par
            \mbox{Default: \emph{empty}}%
          }%
          \feentry{pageheadfoot}{%
            Contents of the page header or footer of a page style defined
            using \Package{scrlayer-scrpage}. The font will be applied after
            \Macro{normalfont}\IndexCmd{normalfont}.\par
            \mbox{Default: \Macro{normalcolor}\Macro{slshape}}%
          }%
          \feentry{pagenumber}{%
            Pagination set with
            \DescRef{\ThisCommonLabelBase.cmd.pagemark}. If you redefine
            \DescRef{\ThisCommonLabelBase.cmd.pagemark}, you have to be sure
            that your redefinition also uses
            \Macro{usekomafont}\PParameter{pagenumber}!\par
            \mbox{Default: \Macro{normalfont}}%
          }%
        \end{desclist}
      }{%
        \IfThisCommonLabelBase{scrjura}{%
          \begin{table}
            \caption{Elements whose \Package{scrjura} font styles can be
              changed with \Macro{setkomafont} and \Macro{addtokomafont},
              including their default settings}%
            \label{tab:scrjura.fontelements}%
            \begin{desctabular}
              \feentry{Clause}{%
                Alias for \FontElement{\PName{environment name}.Clause}
                within a contract environment, for example
                \FontElement{contract.Clause} within
                \DescRef{\ThisCommonLabelBase.env.contract}; if no
                corresponding element is defined,
                \FontElement{contract.Clause} is used%
              }%
              \feentry{contract.Clause}{%
                Heading of a paragraph within
                \DescRef{\ThisCommonLabelBase.env.contract} (see
                \autoref{sec:\ThisCommonLabelBase.contract},
                \DescPageRef{\ThisCommonLabelBase.fontelement.contract.Clause});
                \par
                \mbox{Default: \Macro{sffamily}\Macro{bfseries}\Macro{large}}%
              }%
              \entry{\DescRef{\ThisCommonLabelBase.fontelement./Name/.Clause}}{%
                \IndexFontElement[indexmain]{\PName{name}.Clause}%
                Heading of a paragraph within a \PName{name} environment
                defined with
                \DescRef{\ThisCommonLabelBase.cmd.DeclareNewJuraEnvironment}
                as long as the setting was made with \Option{ClauseFont} or
                the item was subsequently defined (see
                \autoref{sec:\ThisCommonLabelBase.newenv},
                \DescPageRef{\ThisCommonLabelBase.fontelement./Name/.Clause});
                \par
                \mbox{Default: \emph{none}}%
              }%
              \feentry{parnumber}{%
                Paragraph numbers within a contract environment (see
                \autoref{sec:\ThisCommonLabelBase.par},
                \DescPageRef{\ThisCommonLabelBase.fontelement.parnumber});\par
                \mbox{Default: \emph{empty}}%
              }%
            \end{desctabular}
          \end{table}
        }{%
          \IfThisCommonLabelBase{scrlayer-notecolumn}{}{%
            \InternalCommonFileUsageError%
          }%
        }%
      }%
    }%
  \fi%
\fi
\EndIndexGroup


\begin{Declaration}
  \Macro{usefontofkomafont}\Parameter{element}%
  \Macro{useencodingofkomafont}\Parameter{element}%
  \Macro{usesizeofkomafont}\Parameter{element}%
  \Macro{usefamilyofkomafont}\Parameter{element}%
  \Macro{useseriesofkomafont}\Parameter{element}%
  \Macro{useshapeofkomafont}\Parameter{element}
\end{Declaration}
Sometimes\ChangedAt{v3.12}{\Class{scrbook}\and \Class{scrreprt}\and
  \Class{scrartcl}\and \Package{scrextend}}, although this is not recommended,
the font setting of an element is used for settings that are not actually
related to the font. If you want to apply only the font setting of an element
but not those other settings, you can use \Macro{usefontofkomafont} instead of
\DescRef{\ThisCommonLabelBase.cmd.usekomafont}. This will activate the font
size and baseline skip, the font encoding, the font family, the font series,
and the font shape of an element, but no further settings as long as those
further settings are local.

You can also switch to a single one of those attributes using one of the other
commands. Note that \Macro{usesizeofkomafont} uses both the font
size and the baseline skip.%
%
\IfThisCommonLabelBase{scrextend}{% Umbruchvariante!
}{%
  \IfThisCommonLabelBase{scrjura}{%
    \par%
    However, the misuse of the font settings is strongly discouraged (see
    \autoref{sec:maincls-experts.experts},
    \DescPageRef{maincls-experts.cmd.addtokomafontrelaxlist})!%
  }{%
    \par%
    However, you should not take these commands as legitimizing the insertion
    of arbitrary commands in an element's font setting. To do so can lead
    quickly to errors (see \autoref{sec:maincls-experts.experts},
    \DescPageRef{maincls-experts.cmd.addtokomafontrelaxlist}).%
  }%
}%
\EndIndexGroup
%
\EndIndexGroup

%%% Local Variables:
%%% mode: latex
%%% mode: flyspell
%%% coding: us-ascii
%%% ispell-local-dictionary: "en_GB"
%%% TeX-master: "../guide"
%%% End:
