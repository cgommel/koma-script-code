% ======================================================================
% scrbookreportarticle.tex
% Copyright (c) Markus Kohm, 2001-2012
%
% This file is part of the LaTeX2e KOMA-Script bundle.
%
% This work may be distributed and/or modified under the conditions of
% the LaTeX Project Public License, version 1.3c of the license.
% The latest version of this license is in
%   http://www.latex-project.org/lppl.txt
% and version 1.3c or later is part of all distributions of LaTeX 
% version 2005/12/01 or later and of this work.
%
% This work has the LPPL maintenance status "author-maintained".
%
% The Current Maintainer and author of this work is Markus Kohm.
%
% This work consists of all files listed in manifest.txt.
% ----------------------------------------------------------------------
% scrbookreportarticle.tex
% Copyright (c) Markus Kohm, 2001-2012
%
% Dieses Werk darf nach den Bedingungen der LaTeX Project Public Lizenz,
% Version 1.3c, verteilt und/oder veraendert werden.
% Die neuste Version dieser Lizenz ist
%   http://www.latex-project.org/lppl.txt
% und Version 1.3c ist Teil aller Verteilungen von LaTeX
% Version 2005/12/01 oder spaeter und dieses Werks.
%
% Dieses Werk hat den LPPL-Verwaltungs-Status "author-maintained"
% (allein durch den Autor verwaltet).
%
% Der Aktuelle Verwalter und Autor dieses Werkes ist Markus Kohm.
% 
% Dieses Werk besteht aus den in manifest.txt aufgefuehrten Dateien.
% ======================================================================
%
% Chapter about scrbook, scrreprt, and scrartcl of the KOMA-Script guide
% Maintained by Markus Kohm
%
% ----------------------------------------------------------------------
%
% Kapitel �ber scrbook, scrreprt und scrartcl in der KOMA-Script-Anleitung
% Verwaltet von Markus Kohm
%
% ============================================================================

\ProvidesFile{scrbookreportarticle.tex}[2012/02/07 KOMA-Script guide (chapter:
scrbook, scrreprt, scrartcl)]
\translator{Jens-Uwe Morawski\and Gernot Hassenpflug\and Markus Kohm}

% Date of translated german file: 2007-09-04

\chapter{The Main Classes \Class{scrbook}, \Class{scrreprt} and
  \Class{scrartcl}}
\labelbase{maincls}

%\AddSeeIndex{command}{gen}{\GuidecmdIndexShort}{cmd}% <-- set automaticly
\AddSeeIndex{macro}{gen}{\GuidecmdIndexShort}{cmd}%

The main classes of the {\KOMAScript} bundle are designed as counterparts to
the standard {\LaTeX} classes. This means that the {\KOMAScript} bundle
contains replacements for the three standard classes
\Class{book}\IndexClass{book}, \Class{report}\IndexClass{report} and
\Class{article}\IndexClass{article}. There is also a replacement for the
standard class \Class{letter}\IndexClass{letter}. The document class for
letters is described in a separate chapter, because it is fundamentally
different from the three main classes (see \autoref{cha:scrlttr2}).

\iffalse% Umbruchkorrekturtext
  The names of the {\KOMAScript} classes are composed of the prefix
  ``\texttt{scr}'' and the abbreviated name of the corresponding standard
  class. In order to restrict the length of the names to eight letters, the
  vowels, starting with the last one, are left off as necessary. The
  \autoref{tab:maincls.overview} shows an overview of the correspondence
  between the standard classes and the {\KOMAScript} classes.
\fi

The simplest way to use a {\KOMAScript} class instead of a standard one is to
substitute the class name in the \verb|\documentclass| command according to
\autoref{tab:maincls.overview}. For example you may replace
\Macro{documentclass}\PParameter{book} by
\Macro{documentclass}\PParameter{scrbook}.  Normally, the document should be
processed without errors by {\LaTeX}, just like before the substitution. The
look however should be different. Additionally, the {\KOMAScript} classes
provide new possibilities and options that are described in the following
sections.

\begin{table}
%  \centering
  \KOMAoptions{captions=topbeside}
  \setcapindent{0pt}
%  \caption
  \begin{captionbeside}
    [Class correspondence]{\label{tab:maincls.overview}Correspondence between
      standard classes and {\KOMAScript} classes}% and \Script{} styles.}
    [l]
    \begin{tabular}[t]{ll}
      \toprule
      standard class & {\KOMAScript} class \\%& \Script-Stil (\LaTeX2.09)\\
      \midrule
      \Class{article} & \Class{scrartcl}   \\%& \File{script\textunderscore s} \\
      \Class{report}  & \Class{scrreprt}   \\%& \File{script}   \\
      \Class{book}    & \Class{scrbook}    \\%& \File{script}   \\
      \Class{letter}  & \Class{scrlttr2}   \\%& \File{script\textunderscore l} \\
      \bottomrule
    \end{tabular}
  \end{captionbeside}
\end{table}

Allow me an observation before proceeding with the descriptions of the
options. It is often the case that at the beginning of a document one is often
unsure which options to choose for that specific document. Some options, for
instance the choice of paper size, may be fixed from the beginning. But
already the question of the size of the text area and the margins could be
difficult to answer initially. On the other hand, the main business of an
author\,---\,planning the document structure, writing the text, preparing
figures, tables, lists, index and other data\,---\,should be almost
independent of those settings. As an author you should concentrate initially
on this work. When that is done, you can concentrate on the fine points of
presentation. Besides the choice of options, this means correcting
hyphenation, optimizing page breaks and the placement of tables and figures.

\LoadCommon{0}% \section{Early or Late Selection of Options}

\LoadCommon{1}% \section{Compatibility with Earlier Versions of KOMA-Script}

\LoadCommon{2}% \section{Draft Mode}

\LoadCommon{3}% \section{Page Layout}

\begin{Declaration}
  \Macro{flushbottom}\\
  \Macro{raggedbottom}
\end{Declaration}
\BeginIndex{Cmd}{raggedbottom}%
\BeginIndex{Cmd}{flushbottom}%
\begin{Explain}
  Especially at double-sided documents it's preferred, if not only the first
  lines at the text areas of a double-side spread have the same visual
  baseline but also the last lines of double-side spreads. If pages consist in
  text without paragraphs or headlines only, this is the result in
  general. But using a paragraph distance of half of a line would be already
  enough to not achieve this, if the differences in the number of paragraphs
  at two pages is odd. In this case at least some of the paragraph distances
  needed to be shrunken or stretched to fit the rule again. \TeX{} knows
  shrinkable and stretchable distances for this purpose. \LaTeX{} provides an
  automatism for this kind of vertical adjustment\Index{adjustment>vertical}.
\end{Explain}

Using double-side typesetting with option \Option{twoside} (see
\autoref{sec:typearea.options}, \autopageref{desc:typearea.option.twoside}),
switches on vertical adjustment also. Alternatively vertical adjustment may be
switched on at any time valid from the current page using
\Macro{flushbottom}. In opposite \Macro{raggedbottom} would switch off
vertical adjustment from the current page on. This is also the default at
one-side typesetting.

By the way: \KOMAScript{} uses a slightly modified kind of abdication of
vertical adjustment. This has been done to move footnotes to the bottom of the
text area instead of having them close to the last used text line.\iffree{}{
  More information about this may be found at
  \autoref{sec:maincls-experts.addInfos},
  \autopageref{desc:maincls-experts.cmd.footnoterule}.}%
%
\EndIndex{Cmd}{flushbottom}%
\EndIndex{Cmd}{raggedbottom}%
%
\EndIndex{}{page>layout}

\section{Selection of the Document Font Size}
\LoadCommon{4}

The default at \Class{scrbook}, \Class{scrreprt}, and \Class{scrartcl} is
\OptionValue{fontsize}{11pt}.\textnote{\KOMAScript{} vs. standard classes} In
contrast the default of the standard classes would be \Option{10pt}. You
may attend to this if you switch from a standard class to a \KOMAScript{}
class.%
%
\EndIndex{Option}{fontsize~=\PName{size}}%
%
\EndIndex{}{font>size}

\LoadCommon{5}% \section{Text Markup}

The command \Macro{usekomafont} can change the current font specification to
the one currently used with the specified \PName{element}.

\begin{Example}
  \phantomsection\label{desc:maincls.setkomafont.example}%
  Assume that you want to use for the element \FontElement{captionlabel} the
  same font specification that is used with
  \FontElement{descriptionlabel}. This can be easily done with:
\begin{lstcode}
  \setkomafont{captionlabel}{%
    \usekomafont{descriptionlabel}%
  }
\end{lstcode}
  You can find other examples in the paragraphs on each element.
\end{Example}

\begin{desclist}
  \desccaption{%
    Elements, whose type style can be changed with the {\KOMAScript} command
    \Macro{setkomafont} or
    \Macro{addtokomafont}\label{tab:maincls.elementsWithoutText}%
  }{%
    Elements, whose type style can be changed (\emph{continuation})%
  }%
  \feentry{caption}{text of a table or figure caption (see
    \autoref{sec:maincls.floats}, \autopageref{desc:maincls.cmd.caption})}%
  \feentry{captionlabel}{label of a table or figure caption; used according to
    the element \FontElement{caption} (see \autoref{sec:maincls.floats},
    \autopageref{desc:maincls.cmd.caption})}%
  \feentry{chapter}{title of the sectional unit \Macro{chapter} (see
    \autoref{sec:maincls.structure}, \autopageref{desc:maincls.cmd.chapter})}%
  \feentry{chapterentry}{%
    table of contents entry of the sectional unit \Macro{chapter} (see
    \autoref{sec:maincls.toc},
    \autopageref{desc:maincls.cmd.tableofcontents})}%
  \feentry{chapterentrypagenumber}{%
    page number of the table of contents entry of the sectional unit
    \Macro{chapter}, variation on the element \FontElement{chapterentry} (see
    \autoref{sec:maincls.toc},
    \autopageref{desc:maincls.cmd.tableofcontents})}%
  \feentry{chapterprefix}{%
    chapter number line at setting \OptionValue{chapterprefix}{true} or
    \OptionValue{appendixprefix}{true} (see \autoref{sec:maincls.structure},
    \autopageref{desc:maincls.options.chapterprefix})}%
  \feentry{descriptionlabel}{labels, i.\,e., the optional argument of
    \Macro{item} in the \Environment{description} environment (see
    \autoref{sec:maincls.lists}, \autopageref{desc:maincls.env.description})}%
  \feentry{dictum}{dictum, wise saying, or smart slogan (see
    \autoref{sec:maincls.dictum}, \autopageref{desc:maincls.cmd.dictum})}%
  \feentry{dictumauthor}{author of a dictum, wise saying, or smart slogan;
    used according to the element \FontElement{dictumtext} (see
    \autoref{sec:maincls.dictum}, \autopageref{desc:maincls.cmd.dictum})}%
  \feentry{dictumtext}{another name for \FontElement{dictum}}%
  \feentry{disposition}{all sectional unit titles, i.\,e., the arguments of
    \Macro{part} down to \Macro{subparagraph} and \Macro{minisec}, including
    the title of the abstract; used before the element of the corresponding
    unit (see \autoref{sec:maincls.structure} ab
    \autopageref{sec:maincls.structure})}%
  \feentry{footnote}{footnote text and marker (see
    \autoref{sec:maincls.footnotes},
    \autopageref{desc:maincls.cmd.footnote})}%
  \feentry{footnotelabel}{mark of a footnote; used according to the element
    \FontElement{footnote} (see \autoref{sec:maincls.footnotes},
    \autopageref{desc:maincls.cmd.footnote})}%
  \feentry{footnotereference}{footnote reference in the text (see
    \autoref{sec:maincls.footnotes},
    \autopageref{desc:maincls.cmd.footnote})}%
  \feentry{footnoterule}{%
    horizontal rule\ChangedAt{v3.07}{\Class{scrbook}\and \Class{scrreprt}\and
      \Class{scrartcl}} above the footnotes at the end of the text area (see
    \autoref{sec:maincls.footnotes},
    \autopageref{desc:maincls.cmd.setfootnoterule})}%
  \feentry{labelinglabel}{labels, i.\,e., the optional argument of
    \Macro{item} in the \Environment{labeling} environment (see
    \autoref{sec:maincls.lists}, \autopageref{desc:maincls.env.labeling})}%
  \feentry{labelingseparator}{separator, i.\,e., the optional argument of the
    \Environment{labeling} environment; used according to the element
    \FontElement{labelinglabel} (see \autoref{sec:maincls.lists},
    \autopageref{desc:maincls.env.labeling})}%
  \feentry{minisec}{title of \Macro{minisec} (see
    \autoref{sec:maincls.structure} ab
    \autopageref{desc:maincls.cmd.minisec})}%
  \feentry{pagefoot}{only used, if package \Package{scrpage2} has been loaded
    (see \autoref{cha:scrpage},
    \autopageref{desc:scrpage.fontelement.pagefoot})}%
  \feentry{pagehead}{another name for \FontElement{pageheadfoot}}%
  \feentry{pageheadfoot}{the head of a page, but also the foot of a page (see
    \autoref{sec:maincls.pagestyle} ab \autopageref{sec:maincls.pagestyle})}%
  \feentry{pagenumber}{page number in the header or footer (see
    \autoref{sec:maincls.pagestyle})}%
  \feentry{pagination}{another name for \FontElement{pagenumber}}%
  \feentry{paragraph}{title of the sectional unit \Macro{paragraph} (see
    \autoref{sec:maincls.structure},
    \autopageref{desc:maincls.cmd.paragraph})}%
  \feentry{part}{title of the \Macro{part} sectional unit, without the line
    containing the part number (see \autoref{sec:maincls.structure},
    \autopageref{desc:maincls.cmd.part})}%
  \feentry{partentry}{%
    table of contents entry of the secitonal unit \Macro{part} (see
    \autoref{sec:maincls.toc},
    \autopageref{desc:maincls.cmd.tableofcontents})}%
  \feentry{partentrypagenumber}{%
    Page number of the table of contents entry of the sectional unit
    \Macro{part} variation on the element \FontElement{partentry} (see
    \autoref{sec:maincls.toc},
    \autopageref{desc:maincls.cmd.tableofcontents})}%
  \feentry{partnumber}{line containing the part number in a title of the
    sectional unit \Macro{part} (see \autoref{sec:maincls.structure},
    \autopageref{desc:maincls.cmd.part})}%
  \feentry{section}{title of the sectional unit \Macro{section} (see
    \autoref{sec:maincls.structure}, \autopageref{desc:maincls.cmd.section})}%
  \feentry{sectionentry}{%
    table of contents entry of sectional unit \Macro{section} (only available
    in \Class{scrartcl}, see \autoref{sec:maincls.toc},
    \autopageref{desc:maincls.cmd.tableofcontents})}%
  \feentry{sectionentrypagenumber}{%
    page number of the table of contents entry of the sectional unit
    \Macro{section}, variation on element \FontElement{sectionentry} (only
    available in \Class{scrartcl, see \autoref{sec:maincls.toc},
      \autopageref{desc:maincls.cmd.tableofcontents}})}%
  \feentry{sectioning}{another name for \FontElement{disposition}}%
  \feentry{subject}{%
    categorization of the document, i.\,e., the argument of \Macro{subject} on
    the main title page (see \autoref{sec:maincls.titlepage},
    \autopageref{desc:maincls.cmd.subject})}%
  \feentry{subparagraph}{title of the sectional unit \Macro{subparagraph} (see
    \autoref{sec:maincls.structure},
    \autopageref{desc:maincls.cmd.subparagraph})}%
  \feentry{subsection}{title of the sectional unit \Macro{subsection} (see
    \autoref{sec:maincls.structure},
    \autopageref{desc:maincls.cmd.subsection})}%
  \feentry{subsubsection}{title of the sectional unit \Macro{subsubsection}
    (see \autoref{sec:maincls.structure},
    \autopageref{desc:maincls.cmd.subsubsection})}%
  \feentry{subtitle}{%
    subtitle of the document, i.\,e., the argument of \Macro{subtitle} on the
    main title page (see \autoref{sec:maincls.titlepage},
    \autopageref{desc:maincls.cmd.title})}%
  \feentry{title}{main title of the document, i.\,e., the argument of
    \Macro{title} (for details about the title size see the additional note in
    the text of \autoref{sec:maincls.titlepage} from
    \autopageref{desc:maincls.cmd.title})}%
\end{desclist}
%
\EndIndex{Cmd}{setkomafont}%
\EndIndex{Cmd}{addtokomafont}%
\EndIndex{Cmd}{usekomafont}%
%
\EndIndex{}{font}%
\EndIndex{}{text>markup}%

\LoadCommon{14} %\section{Document Title Pages}


\section{Abstract}
\label{sec:maincls.abstract}
\BeginIndex{}{summary}%
\BeginIndex{}{abstract}%

Particularly\OnlyAt{\Class{scrartcl}\and\Class{scrreprt}} with
articles, more rarely with reports, there is a summary\Index{summary}
directly under the title and before the table of contents. Using a in-page
title this summary is normally a kind of left and right indented block. In
contrast to this a kind of chapter or section is printed using title pages.

\begin{Declaration}
  \KOption{abstract}\PName{simple switch}
\end{Declaration}%
\BeginIndex{Option}{abstract~=\PName{simple switch}}%
\ChangedAt{v3.00}{\Class{scrreprt}\and \Class{scrartcl}}%
In\OnlyAt{\Class{scrreprt}\and\Class{scrartcl}} the standard classes the
\Environment{abstract} environment sets the text ``\abstractname'' centered
before the summary text\Index[indexmain]{summary}. This was normal practice in
the past. In the meantime, newspaper reading has trained readers to recognize
a displayed text at the beginning of an article or report as the
abstract. This is even more true when the text comes before the table of
contents. It is also surprising when precisely this title appears small and
centered. {\KOMAScript} provides the possibility of including or excluding the
abstract's title with the options \Option{abstract}. For \PName{simplex
  switch} any value from
\autoref{tab:truefalseswitch},\autopageref{tab:truefalseswitch} may be used.

Books typically use another type of summary. In that case there is usually a
dedicated summary chapter at the beginning or end of the book. This chapter is
often combined with the introduction or a description of wider
prospects. Therefore, the class \Class{scrbook} has no \Environment{abstract}
environment. A\textnote{Hint!} summary chapter is also recommended for reports in a wider
sense, like a Master's or Ph.D.  thesis.%
%
\EndIndex{Option}{abstract~=\PName{simple switch}}%


\begin{Declaration}
  \XMacro{begin}\PParameter{\Environment{abstract}}\\
  \quad\dots\\
  \XMacro{end}\PParameter{abstract}
\end{Declaration}%
\OnlyAt{\Class{scrartcl}\and \Class{scrreprt}}%
Some {\LaTeX} classes offer a special environment for this summary, the
\Environment{abstract} environment. This is output directly, at it is not a
component of the titles set by \Macro{maketitle}.  Please\textnote{Attention!}
note that \Environment{abstract} is an environment, not a command. Whether the
summary has a heading or not is determined by the option \Option{abstract}
(see above).

With books (\Class{scrbook}) the summary is frequently a component of the
introduction or a separate chapter at the end of the document.  Therefore no
\Environment{abstract} environment is provided. When using the class
\Class{scrreprt} it is surely worth considering whether one should not proceed
likewise. See commands \Macro{chapter*}\IndexCmd{chapter*} and
\Macro{addchap}\IndexCmd{addchap} or \Macro{addchap*} at
\autoref{sec:maincls.structure} from \autopageref{desc:maincls.cmd.chapter*}
onwards.

Using a in-page title\Index{title>in-page} (see option \Option{titlepage},
\autoref{sec:maincls.titlepage}, \autopageref{desc:maincls.option.titlepage})
the abstract is set using environment
\Environment{quotation}\IndexEnv{quotation} (see \autoref{sec:maincls.lists},
\autopageref{desc:maincls.env.quotation}) internally. Thereby paragraphs will
be set with intention of the first line. If that first paragraph of the
abstract should not be intended, this indent may be disabled using
\Macro{noindent}\IndexCmd{noindent}\important{\Macro{noindent}}\iffree{just
  after \Macro{begin}\PParameter{abstract}}{at the begin of the environment}.%
%
\EndIndex{Env}{abstract}
%
\EndIndex{}{summary}%
\EndIndex{}{abstract}


\section{Table of Contents}
\label{sec:maincls.toc}
\BeginIndex{}{table of contents}

Next to the document title and an optional existing abstract the table of
contents will be set normally. Often one may find additional lists of floating
environments, e.\,g. the list of tables and the list of figures, after the
table of contents (see \autoref{sec:maincls.floats}).

\begin{Declaration}
  \KOption{toc}\PName{selection}
\end{Declaration}
\BeginIndex{Option}{toc~=\PName{selection}}%
It is becoming increasingly common to find entries in the table of contents
for the lists of tables and figures, for the bibliography, and, sometimes,
even for the index. This is surely also related to the recent trend of putting
lists of figures and tables at the end of the document. Both lists are
similiar to the table of contents in structure and intention. I'm therefore
sceptical of this evolution.  Since\important{\OptionValue{toc}{listof}} it
makes no sense to include only one of the lists of tables and figures in the
table of contents, there exists only one
\PName{selection}\ChangedAt{v3.00}{\Class{scrbook}\and \Class{scrreprt}\and
  \Class{scrartcl}} \PValue{listof}\IndexOption{toc~=\PValue{listof}} that
causes entries for both types of lists to be included. This also includes any
lists produced with version~1.2e or later of the
\Package{float}\IndexPackage{float} package (see \cite{package:float}) or the
\Package{floatrow} (see \cite{package:floatrow}).
All\important{\OptionValue{toc}{listofnumbered}} these lists are unnumbered,
since they contain entries that reference other sections of the document. If
one wants to ignore this general agreement, one may use \PName{selection}
\PValue{listofnumbered}%
\IndexOption{toc~=\PValue{listofnumbered}}.

\leavevmode\phantomsection\nobreak
\label{desc:maincls.option.toc.index}\nobreak
The\important{\OptionValue{toc}{index}} option \OptionValue{index}{totoc}
causes an entry for the index to be included in the table of contents. The
index is unnumbered since it too only includes references to the contents of
the other sectional units. \KOMAScript{} does not support to ignore this
general agreement.

\leavevmode\phantomsection\nobreak
\label{desc:maincls.option.toc.bibliography}\nobreak
The bibliography is a different kind of listing. It does not list the contents
of the present document but refers instead to external
documents. For\important{\begin{tabular}{@{}r@{}}
    \multicolumn{1}{@{}l@{}}{\Option{toc=}}\\
    ~\PValue{bibliographynumbered}\\
  \end{tabular}} that
reason, it could be argued that it qualifies as a chapter (or section) and, as
such, should be numbered. The option \OptionValue{toc}{bibliographynumbered}%
\IndexOption{toc~=\PValue{bibliographynumbered}}
has this effect, including the generation of the corresponding entry in the
table of contents. I personally think that this reasoning would lead us to
consider a classical list of sources also to be a separate chapter. On the
other hand, the bibliography is finally not something that was written by the
document's author. In\important{\OptionValue{toc}{bibliography}} view of this, the bibliography merits nothing more than
an unnumbered entry in the table of contents, and that can be achieved with
\OptionValue{toc}{bibliography}\IndexOption{toc~=\PValue{bibliography}}.

\leavevmode\phantomsection\nobreak
\label{desc:maincls.option.toc.graduated}\nobreak
The table of contents is normally\ChangedAt{v2.8q}{%
  \Class{scrbook}\and \Class{scrreprt}\and \Class{scrartcl}}%
\important{\OptionValue{toc}{graduated}} set up so that different sectional
units have different indentations. The section number is set left-justified in
a fixed-width field. This default setup is selected with the option
\ChangedAt{v3.00}{\Class{scrbook}\and \Class{scrreprt}\and \Class{scrartcl}}
\OptionValue{toc}{graduated}\IndexOption{toc~=\PValue{graduated}}.

\leavevmode\phantomsection\nobreak
\label{desc:maincls.option.toc.flat}\nobreak
When there are many sections, the corresponding numbering tends to become very
wide, so that the reserved field overflows. The German FAQ \cite{DANTE:FAQ}
suggests that the table of contents should be redefined in such a
case. {\KOMAScript}\important{\OptionValue{toc}{flat}} offers an alternative
format that avoids the problem completely. If the option
\OptionValue{toc}{flat}\IndexOption{toc~=\PValue{flat}} is selected, then no variable indentation is applied
to the titles of the sectional units. Instead, a table-like organisation is
used, where all unit numbers and titles, respectively, are set in a
left-justified column.  The space necessary for the unit numbers is thus
determined automatically.%

The \autoref{tab:maincls.toc} shows an overview of possible values for
\PName{selection} of \Option{toc}.

\begin{desclist}
  \desccaption[{Possible values of option \Option{toc}}]{%
    Possible values of option \Option{toc} to set form and contents of the
    table of contents\label{tab:maincls.toc}%
  }{%
    Possible values of option \Option{toc} (\emph{continuation})%
  }%
  \entry{\PValue{bibliography}, \PValue{bib}}{%
    The bibliography will be represented by an entry at the table of contents,
    but will not be numbered.%
    \IndexOption{toc~=\PValue{bibliography}}}%
  \entry{\PValue{bibliographynumbered}, \PValue{bibnumbered},
    \PValue{numberedbibliography}, \PValue{numberedbib}}{%
    The bibliography will be represented by an entry at the table of contents
    and will be numbered.%
    \IndexOption{toc~=\PValue{bibliographynumbered}}}%
  \entry{\PValue{flat}, \PValue{left}}{%
    The table of contents will be set in table form. The numbers of the
    headings will be at the first column, the heading text at the second
    column, and the page number at the third column. The amount of space
    needed for the numbers of the headings will be determined by the detected
    needed amount of space at the previous \LaTeX{} run.%
    \IndexOption{toc~=\PValue{flat}}}%
  \entry{\PValue{graduated}, \PValue{indent}, \PValue{indented}}{%
    The table of contents will be set in hierarchical form. The amount of
    space for the heading numbers is limited.%
    \IndexOption{toc~=\PValue{graduated}}}%
  \entry{\PValue{index}, \PValue{idx}}{%
    The index will be represented by an entry at the table of contents, but
    will not be numbered.%
    \IndexOption{toc~=\PValue{index}}}%
  \pventry{listof}{%
    The lists of floating environments, e.\,g., the list of figures and the
    list of tables, will be represented by entries at the table of contents,
    but will not be numbered.%
    \IndexOption{toc~=\PValue{listof}}}%
  \entry{\PValue{listofnumbered}, \PValue{numberedlistof}}{%
    The lists of floating environments, e.\,g., the list of figures and the
    list of tables, will be represented by entries at the table of contents
    and will be numbered.%
    \IndexOption{toc~=\PValue{listofnumbered}}}%
  \entry{\PValue{nobibliography}, \PValue{nobib}}{%
    The bibliography will not be represented by an entry at the table of
    contents.%
    \IndexOption{toc~=\PValue{nobibliography}}}%
  \entry{\PValue{noindex}, \PValue{noidx}}{%
    The index will not be represented by an entry at the table of
    contents.%
    \IndexOption{toc~=\PValue{noindex}}}%
  \pventry{nolistof}{%
    The lists of floating environments, e.\,g., the list of figures and the
    list of tables, will not be represented by entries at the table of
    contents.%
    \IndexOption{toc~=\PValue{nolistof}}}%
\end{desclist}
%
\EndIndex{Option}{toc~=\PName{selection}}%

\begin{Declaration}
  \Macro{tableofcontents}
\end{Declaration}%
\BeginIndex{Cmd}{tableofcontents}%
The production of the table of contents is done by the \Macro{tableofcontents}
command.  To get a correct table of contents, at least two {\LaTeX} runs are
necessary after every change. The contents and the form of the table of
contents may be influenced with the above described option \Option{toc}. After
changing the settings of this options at least two \LaTeX{} runs are needed
again.

The entry for the highest sectional unit below \Macro{part}, i.\,e.,
\Macro{chapter} with \Class{scrbook}\IndexClass{scrbook} and
\Class{scrreprt}\IndexClass{scrreprt} or \Macro{section} with
\Class{scrartcl}\IndexClass {scrartcl} is not indented. There are no dots
between the text of the sectional unit heading and the page number. The
typographic reasons for this are that the font is usually different, and the
desire for appropriate emphasis. The table of contents of this manual is a
good example of these considerations. The font
style\ChangedAt{v2.97c}{\Class{scrbook}\and \Class{scrreprt}\and
  \Class{scrartcl}}\important[i]{\FontElement{partentry}\\
  \FontElement{chapterentry}\\
  \FontElement{sectionentry}} is however affected by the settings of the
element \FontElement{partentry}\IndexFontElement{partentry} and for classes
\Class{scrbook} and \Class{scrreprt} by
\FontElement{chapterentry}\IndexFontElement{chapterentry} respectively for
class \Class{scrartcl} by
\FontElement{sectionentry}\IndexFontElement{sectionentry}. The font style of the
page numbers by set dissenting from these elements using
\FontElement{partentrypagenumber}\IndexFontElement{partentrypagenumber} and
\FontElement{chapterentrypagenumber}\IndexFontElement{chapterentrypagenumber}
respectively 
\FontElement{sectionentrypagenumber}\IndexFontElement{sectionentrypagenumber}
(see \autoref{sec:maincls.textmarkup},
\autopageref{desc:maincls.cmd.setkomafont}, and
\autoref{tab:maincls.elementsWithoutText},
\autopageref{tab:maincls.elementsWithoutText}). The default settings of the
elements may be found at \autoref{tab:maincls.tocelements}.
\begin{table}
%  \centering
%  \caption
  \KOMAoptions{captions=topbeside}%
  \setcapindent{0pt}%
  \begin{captionbeside}
    [Font style defaults of the elements of the table of contents]
    {\label{tab:maincls.tocelements}%
      Font style defaults of the elements of the table of contents}%
    [l]
    \setlength{\tabcolsep}{.9\tabcolsep}% Umbruchoptimierung!
  \begin{tabular}[t]{ll}
    \toprule
    Element & Default font style \\
    \midrule
    \FontElement{partentry} &
    \Macro{usekomafont}\PParameter{disposition}\Macro{large} \\
    \FontElement{partentrypagenumber} & \\
    \FontElement{chapterentry} & \Macro{usekomafont}\PParameter{disposition}
    \\
    \FontElement{chapterentrypagenumber} & \\
    \FontElement{sectionentry} & \Macro{usekomafont}\PParameter{disposition}
    \\
    \FontElement{sectionentrypagenumber} & \\
    \bottomrule
  \end{tabular}
  \end{captionbeside}
\end{table}
%
\EndIndex{Cmd}{tableofcontents}%

\begin{Declaration}
  \Counter{tocdepth}
\end{Declaration}%
\BeginIndex{Counter}{tocdepth}%
Normally, the units included in the table of contents are all the units from
\Macro{part} to \Macro{subsection} for the classes \Class{scrbook} and
\Class{scrreprt} or from \Macro{part} to \Macro{subsubsection} for the class
\Class{scrartcl}.  The inclusion of a sectional unit in the table of contents
is controlled by the counter \Counter{tocdepth}. This has the value \(-\)1 for
\Macro{part}, 0 for \Macro{chapter}, and so on. By setting, incrementing
or decrementing the counter, one can choose the lowest sectional unit level to
be included in the table of contents.  The same happens with the standard
classes.

The user of the \Package{scrpage2}\IndexPackage{scrpage2} package (see
\autoref{cha:scrpage}) does not need to remember the numerical
values of each sectional unit. They are given by the values of the
macros \Macro{chapterlevel}, \Macro{sectionlevel} and so on down to
\Macro{subparagraphlevel}.
\begin{Example}
  Assume that you are preparing an article that uses the sectional
  unit \Macro{subsubsection}. However, you don't want this sectional
  unit to appear in the table of contents. The preamble of your
  document might contain the following:
\begin{lstcode}
  \documentclass{scrartcl}
  \setcounter{tocdepth}{2}
\end{lstcode}
  You set the counter \Counter{tocdepth} to 2 because you know that
  this is the value for \Macro{subsection}. If you know that
  \Class{scrartcl} normally includes all levels down to
  \Macro{subsubsection} in the table of contents, you can simply
  decrement the counter \Counter{tocdepth} by one:
\begin{lstcode}
  \documentclass{scrartcl}
  \addtocounter{tocdepth}{-1}
\end{lstcode}
  How much you should add to or subtract from the \Counter{tocdepth} counter
  can also be found by looking at the table of contents after the first
  {\LaTeX} run.
\end{Example}
A\textnote{Hint!} small hint in order that you do not need to remember which
sectional unit has which number: in the table of contents count the number of
units required extra or less and then, as in the above example, use
\Macro{addtocounter} to add or subtract that number to or from
\Counter{tocdepth}.%
%
\EndIndex{Counter}{tocdepth}%
%
\EndIndex{}{table of contents}


\LoadCommon{6}% \section{Paragraph Markup}

\LoadCommon{7}% \section{Detection of Odd and Even Pages}


\section{Head and Foot Using Predefined Page Styles}
\label{sec:maincls.pagestyle}
\BeginIndex{}{page>style}

One of the general characteristics of a document is the page style. In
{\LaTeX} this means mostly the contents of headers and footers.

\begin{Declaration}
  \KOption{headsepline}\PName{simple switch}\\
  \KOption{footsepline}\PName{simple switch}
\end{Declaration}%
\BeginIndex{Option}{headsepline~=\PName{simple switch}}%
\BeginIndex{Option}{footsepline~=\PName{simple switch}}%
In\ChangedAt{v3.00}{\Class{scrbook}\and \Class{scrreprt}\and \Class{scrartcl}}
order to have or not to have a rule separating the header from the text body
use the option \Option{headsepline} with any value shown in
\autoref{tab:truefalseswitch}, \autopageref{tab:truefalseswitch}). Activation
of the option will result in such a separation line. Like this activation of
option \Option{footsepline} switches on a rule above the foot
line. Deactivation of any of the options will deactivate the corresponding
rule.

These options have no effect with the page styles \PValue{empty} and
\PValue{plain}, because there is no header in this case. Such a line always
has the effect of visually bringing header and text body closer together. That
doesn't mean that the header must now be moved farther from the text
body. Instead, the header should be considered as belonging to the text body
for the purpose of page layout calculations. {\KOMAScript} takes this into
account by automatically passing the option \Option{headinclude} to the
\Package{typearea} package whenever the \Option{headsepline} option is
used. \KOMAScript{} behaves similar with \Option{footinclude} using
\Option{footsepline}. Package \Package{scrpage2} (see \autoref{cha:scrpage}
adds additional features to this.%
%
\EndIndex{Option}{headsepline~=\PName{simple switch}}%
\EndIndex{Option}{footsepline~=\PName{simple switch}}%

\begin{Declaration}
  \Macro{pagestyle}\PParameter{page style}\\
  \Macro{thispagestyle}\Parameter{local page style}
\end{Declaration}%
\BeginIndex{Cmd}{pagestyle}%
\BeginIndex{Cmd}{thispagestyle}%
Usually one distinguishes four different page styles:
\begin{description}
\item[empty\BeginIndex{Pagestyle}{empty}] is the page
  style with entirely empty headers and footers. In {\KOMAScript} this is
  completely identical to the standard classes.
\item[headings\BeginIndex{Pagestyle}{headings}] is the page style with running
  headings in the header. These are headings for which titles are
  automatically inserted into the header.
  \OnlyAt{\Class{scrbook}\and\Class{scrreprt}}With the classes
  \Class{scrbook}\IndexClass{scrbook} and
  \Class{scrreprt}\IndexClass{scrreprt} the titles of chapters and sections
  are repeated in the header for double-sided layout\,---\,with {\KOMAScript}
  on the outer side, with the standard classes on the inner side.  The page
  number is set on the outer side of the footer with {\KOMAScript}, with the
  standard classes it is set on the inner side of the header.  In one-sided
  layouts only the titles of the chapters are used and are, with
  {\KOMAScript}, centered in the header. The page numbers are set centered in
  the footer with {\KOMAScript}.
  \OnlyAt{\Class{scrartcl}}\Class{scrartcl}\IndexClass{scrartcl} behaves
  similarly, but starting a level deeper in the section hierarchy with
  sections and subsections, because the chapter level does not exist in this
  case.

  While the standard classes automatically set running headings always
  in capitals, {\KOMAScript} applies the style of the title. This has
  several typographic reasons. Capitals as a decoration are actually
  far too strong. If one applies them nevertheless, they should be set
  in a one point smaller type size and with tighter spacing. The
  standard classes do not take these points in consideration.

  Beyond this {\KOMAScript} classes support rules below the head and above the
  foot using options \Option{headsepline} and \Option{footsepline} which are
  described above.
\item[myheadings\BeginIndex{Pagestyle}{myheadings}] corresponds mostly to the
  page style \PValue{headings}, but the running headings are not automatically
  produced, but have to be defined by the user. The commands \Macro
  {markboth}\IndexCmd{markboth} and \Macro{markright}\IndexCmd{markright} can
  be used for that purpose (see below).
\item[plain\BeginIndex{Pagestyle}{plain}] is the page style with empty header
  and only a page number in the footer. With the standard classes this page
  number is always centered in the footer. With {\KOMAScript} the page number
  appears on double-sided\Index {double-sided} layout on the outer side of the
  footer. The one-sided page style behaves like the standard setup.
\end{description}

The page style can be set at any time with the help of the \Macro{pagestyle}
command and takes effect with the next page that is output. If\textnote{Hint!}
you one uses \Macro{pagestyle} just before a command, that results in an
implicit page break and if the new page style should be used at the resulting
new page first, a \Macro{cleardoublepage} just before \Macro{pagestyle} will
be useful. But usually one sets the page style only once at the beginning of
the document or in the preamble.

To\important{\Macro{thispagestyle}} change the page style of the current page
only, one uses the \Macro{thispagestyle} command. This also happens
automatically at some places in the document. For example, the instruction
\Macro{thispagestyle}\PParameter{\Macro{chapterpagestyle}} is issued
implicitly on the first page of a chapter.

Please\textnote{Attention!} note that the change between automatic and manual
running headings is no longer performed by page style changes when using the
\Package{scrpage2} package, but instead via special instructions. The page
styles \PValue{headings} and \PValue{myheadings} should not be used together
with this package (see \autoref{cha:scrpage},
\autopageref{desc:scrpage.pagestyle.useheadings}).%
%
\EndIndex{Pagestyle}{empty}% Darueber erfahren wir nun nichts mehr

\BeginIndex[indexother]{}{font>style}%
\BeginIndex{FontElement}{pageheadfoot}%
\BeginIndex{FontElement}{pagefoot}%
\BeginIndex{FontElement}{pagenumber}%
In order to change the font style used in the header, footer or for the page
number\ChangedAt{v2.8p}{%
  \Class{scrbook}\and \Class{scrreprt}\and \Class{scrartcl}}%
, please use the interface described in \autoref{sec:maincls.textmarkup},
\autopageref{desc:maincls.cmd.setkomafont}. The same element is used for header
and footer, which you can designate with
\FontElement{pageheadfoot}\IndexFontElement{pageheadfoot}%
\important{\FontElement{pageheadfoot}}. The element for
the page number within the header or footer is called
\FontElement{pagenumber}\IndexFontElement{pagenumber}%
\important{\FontElement{pagenumber}}.
The element \FontElement{pagefoot}\IndexFontElement{pagefoot}, that is
additionally supported by the \KOMAScript{} classes, will be used only if
a page style has been defined, that has text at the foot line, using package
\Package{scrpage2}\IndexPackage{scrpage2} (see \autoref{cha:scrpage},
\autopageref{desc:scrpage.fontelement.pagefoot}).

The default settings can be found in
\autoref{tab:maincls.defaultFontsHeadFoot}.%
%
\begin{table}
%  \centering%
  \KOMAoptions{captions=topbeside}%
  \setcapindent{0pt}%
%  \caption
  \begin{captionbeside}
    [{Default values for the elements of a page style}]
    {\label{tab:maincls.defaultFontsHeadFoot}%
      Default values for the elements of a page style}
    [l]
  \begin{tabular}[t]{ll}
    \toprule
    Element & Default value \\
    \midrule
    \FontElement{pagefoot}\IndexFontElement{pagefoot} &
    \\
    \FontElement{pageheadfoot}\IndexFontElement{pagefoothead} &
    \Macro{normalfont}\Macro{normalcolor}\Macro{slshape} \\
    \FontElement{pagenumber}\IndexFontElement{pagenumber} &
    \Macro{normalfont}\Macro{normalcolor}\\
    \bottomrule
  \end{tabular}
  \end{captionbeside}
\end{table}
%
\begin{Example}
  \leavevmode\phantomsection\label{sec:maincls.pageStyle.example}%
  Assume that you want to set header and footer in a smaller type size
  and in italics. However, the page number should not be set in
  italics but bold. Apart from the fact that the result will look
  horrible, you can obtain this as follows:
\begin{lstcode}
  \setkomafont{pageheadfoot}{%
    \normalfont\normalcolor\itshape\small
  }
  \setkomafont{pagenumber}{\normalfont\bfseries}
\end{lstcode}
  If you want only that in addition to the default slanted variant a smaller
  type size is used, it is sufficient to use the following:
\begin{lstcode}
  \addtokomafont{pagehead}{\small}
\end{lstcode}
  As you can see, the last example uses the element
  \FontElement{pagehead}\important{\FontElement{pagehead}}. You can achieve
  the same result using \PValue{pageheadfoot} instead (see
  \autoref{tab:maincls.elementsWithoutText} on
  \autopageref{tab:maincls.elementsWithoutText}).
\end{Example}
It is not possible to use these methods to force capitals to be used
automatically for the running headings. For that, please use the
\Package{scrpage2} package (see \autoref{cha:scrpage},
\autopageref{desc:scrpage.option.markuppercase}).

If you define your own page styles, the commands
\Macro{usekomafont}\PParameter{pageheadfoot}, \Macro{usekomafont}\PParameter
{pagenumber}, and \Macro{usekomafont}\PParameter{pagefoot} can be useful.  If
you do not use the {\KOMAScript} package \Package{scrpage2} (see
\autoref{cha:scrpage}) for that, but, for example, the package
\Package{fancyhdr}\IndexPackage{fancyhdr} (see \cite{package:fancyhdr}), you
can use these commands in your definitions.  Thereby you can remain compatible
with {\KOMAScript} as much as possible. If you do not use these commands in
your own definitions, changes like those shown in the previous examples have
no effect. The package \Package{scrpage2}\IndexPackage {scrpage2} takes care
to keep the maximum possible compatibility with other packages itself.
%
\EndIndex{FontElement}{pagenumber}%
\EndIndex{FontElement}{pagefoot}%
\EndIndex{FontElement}{pageheadfoot}%
\EndIndex{Pagestyle}{plain}%
\EndIndex{Cmd}{pagestyle}%
\EndIndex{Cmd}{thispagestyle}%
%
\EndIndex[indexother]{}{font>style}%

\begin{Declaration}
  \Macro{markboth}\Parameter{left mark}\Parameter{right mark}\\
  \Macro{markright}\Parameter{right mark}
\end{Declaration}
\BeginIndex{Cmd}{markboth}%
\BeginIndex{Cmd}{markright}%
At page style \Pagestyle{myheadings}\important{\Pagestyle{myheadings}}%
\IndexPagestyle{myheadings} there's no automatic setting of the running head.
Instead of this one would set it with help of commands \Macro{markboth} and
\Macro{markright}. Thereby \PName{left mark} normally will be used at the head
of even pages and \PName{right mark} at the heads of odd pages. At one-side
printing only the \PName{right mark} exists. Using package
\Package{scrpage2}\IndexPackage{scrpage2}\important{\Package{scrpage2}} the
additional command
\Macro{markleft}\IndexCmd{markleft}\important{\Macro{markleft}} exists.

The commands may be used with other page style too. Combination with automatic
running head, i.\,g., with page style \Pagestyle{headings}, limits the effect
of the commands until the next automatic setting of the corresponding marks.%
%
\EndIndex{Cmd}{markright}%
\EndIndex{Cmd}{markboth}%
%
\EndIndex{Pagestyle}{myheadings}%
\EndIndex{Pagestyle}{headings}%

\begin{Declaration}
  \Macro{titlepagestyle}\\
  \Macro{partpagestyle}\\
  \Macro{chapterpagestyle}\\
  \Macro{indexpagestyle}
\end{Declaration}%
\BeginIndex{Cmd}{titlepagestyle}\Index{title>page style}%
\BeginIndex{Cmd}{partpagestyle}\Index{part>page style}%
\BeginIndex{Cmd}{chapterpagestyle}\Index{chapter>page style}%
\BeginIndex{Cmd}{indexpagestyle}\Index{index>page style}%
For some pages a different page style is chosen with the help of the
command \Macro{thispagestyle}. Which page style this actually is, is
defined by these four macros, of which \Macro{partpagestyle} and
\Macro{chapterpagestyle}\OnlyAt{\Class{scrbook}\and\Class{scrreprt}}
are found only with classes \Class{scrbook} and \Class{scrreprt}, but
not in \Class{scrartcl}. The default value for all four cases is
\PValue{plain}. The meaning of these macros can be taken from
\autoref{tab:specialpagestyles}.
%
\begin{table}
  \centering
  \caption{Macros to set up page style of special pages}
  \label{tab:specialpagestyles}
  \begin{desctabular}
    \mentry{titlepagestyle}{Page style for a title page when using
      \emph{in-page} titles.}%
    \mentry{partpagestyle}{Page style for the pages with \Macro{part}
      titles.}%
    \mentry{chapterpagestyle}{Page style for the first page of a chapter.}%
    \mentry{indexpagestyle}{Page style for the first page of the index.}%
  \end{desctabular}
\end{table}
%
The page styles can be redefined with the \Macro{renewcommand} macro.
\begin{Example}
  Assume that you want the pages with a \Macro{part} heading to have
  no number. Then you can use the following command, for example in
  the preamble of your document:
\begin{lstcode}
  \renewcommand*{\partpagestyle}{empty}
\end{lstcode}
  As mentioned previously on \autopageref{desc:maincls.pagestyle.empty},
  the page style \PValue{empty} is exactly what is required in this
  example. Naturally you can also use a user-defined page style.

  Assume you have defined your own page style for initial chapter pages
  with the package \Package{scrpage2} (see
  \autoref{cha:scrpage}). You have given to this page style the
  fitting name \PValue{chapter}. To actually use this style, you must
  redefine the macro \Macro{chapterpagestyle} accordingly:
\begin{lstcode}
  \renewcommand*{\chapterpagestyle}{chapter}
\end{lstcode}

  Assume that you want that the table of contents\Index{table of
    contents}\textnote{table of contents} of a book to have no page
  numbers. However, everything after the table of contents should work again
  with the page style \PValue{headings}, as well as with \PValue{plain} on
  every first page of a chapter. You can use the following commands:
\begin{lstcode}
  \clearpage
  \pagestyle{empty}
  \renewcommand*{\chapterpagestyle}{empty}
  \tableofcontents
  \clearpage
  \pagestyle{headings}
  \renewcommand*{\chapterpagestyle}{plain}
\end{lstcode}
  Instead of the above you may do a local redefinition using a group. The
  advantage will be that you don't need to know the current page style before
  the change to switch back at the end.
\begin{lstcode}
  \clearpage
  \begingroup
    \pagestyle{empty}
    \renewcommand*{\chapterpagestyle}{empty}
    \tableofcontents
    \clearpage
  \endgroup
\end{lstcode}

  But\important{Attention!} notice that you never should put a numbered head
  into a group. Otherwise you may get funny results with commands like
  \Macro{label}.
\end{Example}

\begin{Explain}
  Whoever thinks that it is possible to put running headings on the
  first page of a chapter by using the command
\begin{lstcode}
  \renewcommand*{\chapterpagestyle}{headings}
\end{lstcode}
  should read more about the background of \Macro{rightmark} at
  \autoref{sec:maincls-experts.addInfos},
  \autopageref{desc:maincls-experts.cmd.rightmark}.
\end{Explain}
%
\EndIndex{Cmd}{titlepagestyle}%
\EndIndex{Cmd}{partpagestyle}%
\EndIndex{Cmd}{chapterpagestyle}%
\EndIndex{Cmd}{indexpagestyle}%

\begin{Declaration}
  \Macro{pagenumbering}\Parameter{numbering style}
\end{Declaration}
\BeginIndex{Cmd}{pagenumbering}%
This command works the same way in {\KOMAScript} as in the standard
classes. More precisely it is a feature neither of the standard classes nor of
the \KOMAScript{} classes but of the {\LaTeX} kernel.  You can specify with
this command the \PName{numbering style} of page numbers.

The changes take effect immediately, hence starting with the
page that contains the command. It is recommended to use
\Macro{cleardoubleoddpage} to close the last page and start a new odd page
before. The possible settings can be found in
\autoref{tab:numberKind}.  

Using\textnote{Attention!} the command \Macro{pagenumbering} also resets the
page counter\Index{page>counter}\Index{page>number}.  Thus the page number of
the next page which {\TeX} outputs will have the number 1 in the style
\PName{numbering style}.
%
\begin{table}
%  \centering
  \KOMAoptions{captions=topbeside}%
  \setcapindent{0pt}%
%  \caption
  \begin{captionbeside}
    {\label{tab:numberKind}%
      Available numbering styles of page numbers}
  \begin{tabular}[t]{lll}
    \toprule
    numbering style & example & description \\
    \midrule
    \PValue{arabic} & 8 & Arabic numbers \\
    \PValue{roman}  & viii & lower-case Roman numbers \\
    \PValue{Roman}  & VIII & upper-case Roman numbers \\
    \PValue{alph}   & h    & letters \\
    \PValue{Alph}   & H    & capital letters \\
    \bottomrule
  \end{tabular}
  \end{captionbeside}
\end{table}
%
\EndIndex{Cmd}{pagenumbering}%
%
\EndIndex{}{page>style}


\LoadCommon{8}% \section{Interleaf Pages}

\LoadCommon{9}% \section{Footnotes}
\LoadCommon{10}% -"-
\LoadCommon{11}% -"-

\section[Demarcation]{Demarcation\protect\OnlyAt{\Class{scrbook}}}
\label{sec:maincls.separation}

\BeginIndex{}{front matter}%
\BeginIndex{}{main matter}%
\BeginIndex{}{back matter}%
\BeginIndex{}{matter>front}%
\BeginIndex{}{matter>main}%
\BeginIndex{}{matter>back}%

Sometimes books are roughly separated into front matter, main matter and
back matter. \KOMAScript{} provides this for \Class{scrbook} also.

\begin{Declaration}
  \Macro{frontmatter}\\
  \Macro{mainmatter}\\
  \Macro{backmatter}
\end{Declaration}%
\BeginIndex{Cmd}{frontmatter}%
\BeginIndex{Cmd}{mainmatter}%
\BeginIndex{Cmd}{backmatter}%
The macro \Macro{frontmatter}\important{\Macro{frontmatter}} introduces the
front matter in which roman numerals are used for the page numbers. Chapter
headings in a front matter are not numbered.  The section titles would be
numbered, start at chapter 0, and would be consecutively numbered across
chapter boundaries. However, this is of no import, as the front matter is used
only for the title pages, table of contents\Index{table of contents}, lists of
figures\Index{list of>figures}\Index{figures>list of} and
tables\Index{tables>list of}, and a foreword\Index{foreword}. The foreword can
thus be set as a normal chapter.  A foreword should never be divided into
sections but kept as short as possible.  Therefore in the foreword there is no
need for a deeper structuring than the chapter level.

In case the user sees things differently and wishes to use numbered
sections\Index{section>number} in the chapters of the front matter, as of
version~2.97e\ChangedAt{v2.97e}{\Class{scrbook}}%
\important{\OptionValue{version}{2.97e}} the section numbering no longer
contains the chapter number. This change only takes effect when the
compatibility option is set to at least version~2.97e (see option
\Option{version}, \autoref{sec:maincls.compatibilityOptions},
\autopageref{desc:maincls.option.version}). It is explicity noted that this
creates a confusion with chapter numbers! The use of \Macro{addsec} and
\Macro{section*} (see \autoref{sec:maincls.structure},
\autopageref{desc:maincls.cmd.section*} and
\autopageref{desc:maincls.cmd.addsec}) are thus, in the author's opinion, far
more preferable.

As of version~2.97e\ChangedAt{v2.97e}{\Class{scrbook}} the numbering
of float environments, such as tables\Index{table>number} and
figures\Index{figure>number}, and equation
numbers\Index{equation>number} in the front matter also contain no
chapter number part. To take effect this too requires the
corresponding compatibility setting (see option \Option{version},
\autoref{sec:maincls.compatibilityOptions},
\autopageref{desc:maincls.option.version}).

\Macro{mainmatter}\important{\Macro{mainmatter}} introduces the main matter
with the main text. If there is no front matter then this command can be
omitted. The default page numbering in the main matter uses Arabic numerals
(re)starting in the main matter at 1.

The back matter is introduced with
\Macro{backmatter}\important{\Macro{backmatter}}. Opinions differ in what
should be part of the back matter. So in some cases you will find only the
bibliography\Index{bibliography}, in some cases only the index\Index{index},
and in other cases both of these as well as the appendices.  The chapters in
the back matter are similar to the chapters in the front matter, but page
numbering is not reset. If you do require separate page numbering you may use
the command \Macro{pagenumbering} from \autoref{sec:maincls.pagestyle},
\autopageref{desc:maincls.cmd.pagenumbering}.
%
\EndIndex{Cmd}{frontmatter}%
\EndIndex{Cmd}{mainmatter}%
\EndIndex{Cmd}{backmatter}%
%
\EndIndex{}{front matter}%
\EndIndex{}{main matter}%
\EndIndex{}{back matter}%
\EndIndex{}{matter>front}%
\EndIndex{}{matter>main}%
\EndIndex{}{matter>back}%


\section{Structuring of Documents}
\label{sec:maincls.structure}

Structuring of documents means to divide them into parts, chapters, sections
and several other structural elements.

\begin{Declaration}
  \KOption{open}\PName{method}
\end{Declaration}%
\BeginIndex{Option}{open~=\PName{method}}%
\OnlyAt{\Class{scrbook}\and\Class{scrreprt}}%
\KOMAScript{} classes \Class{scrbook} and \Class{scrreprt} give you the choice
where to start\Index{chapter>start} a new chapter with double-side
printing. By default \Class{scrreprt} starts a new chapter at the next
page. This is same like \PName{method} \PValue{any}. In opposite to this
\Class{scrbook} starts new chapters at the next right-hand page. This is
\PName{method} \PValue{right} and most usual at books. But sometimes chapters
should start a the left-hand page of a double-page spread. This would be
\PName{method} \PValue{left} \ChangedAt{v3.00}{\Class{scrbook}\and
  \Class{scrreprt}}. An overview of the supported methods may be found at
\autoref{tab:maincls.open}.

\begin{table}
  \caption[{Available values for option \Option{open}}]{Available values for
    option \Option{open} to select page breaks with interleaf pages}
  \begin{desctabular}
    \pventry{any}{Commands
      \Macro{cleardoublepageusingstyle},
      \Macro{cleardoublestandardpage},
      \Macro{cleardoubleplainpage},
      \Macro{cleardoubleemptypage},
      and
      \Macro{cleardoublepage} 
      result in a single page break and therefor are same like
      \Macro{clearpage}.%
      \IndexOption{open~=\PValue{any}}}%
    \pventry{left}{Commands
      \Macro{cleardoublepageusingstyle},
      \Macro{cleardoublestandardpage},
      \Macro{cleardoubleplainpage},
      \Macro{cleardoubleemptypage},
      and
      \Macro{cleardoublepage} 
      result in a page break and add an interleaf page\Index{page>interleaf}
      if needed to reach the next left-hand page.%
      \IndexOption{open~=\PValue{left}}}%
    \pventry{right}{Commands
      \Macro{cleardoublepageusingstyle},
      \Macro{cleardoublestandardpage},
      \Macro{cleardoubleplainpage},
      \Macro{cleardoubleemptypage},
      and
      \Macro{cleardoublepage} 
      result in a page break and add an interleaf page\Index{page>interleaf}
      if needed to reach the next right-hand page.%
      \IndexOption{open~=\PValue{right}}}%
  \end{desctabular}
  \label{tab:maincls.open}
\end{table}

Beside the implicit usage of \Macro{cleardoublepage} at chapter starts the
option influences also the explicit usage of the commands
\Macro{cleardoublepage}, \Macro{cleardoublepageusingstyle},
\Macro{cleardoublestandardpage}, \Macro{cleardoubleplainpage}, and
\Macro{cleardoubleemptypage}. See \autoref{sec:maincls.pagestyle},
\autopageref{desc:maincls.cmd.cleardoublepage} for more information about
these. While these \LaTeX{} doesn't differ left-hand and right-hand pages at
single-side printing, the options doesn't have any influence in that case.

At class \Class{scrartcl} the section is the first structural element below
the part. Because of this \Class{scrartcl} doesn't support this option.
%
\EndIndex{Option}{open~=\PName{method}}%

\begin{Declaration}
  \KOption{chapterprefix}\PName{simple switch}\\
  \KOption{appendixprefix}\PName{simple switch}
\end{Declaration}%
\BeginIndex{Option}{chapterprefix~=\PName{simple switch}}%
\BeginIndex{Option}{appendixprefix~=\PName{simple switch}}%
With\OnlyAt{\Class{scrbook}\and\Class{scrreprt}} the standard classes
\Class{book} and \Class{report} a chapter
title\Index[indexmain]{chapter>title} consists of a line with the word
``Chapter''\footnote{When using another language the word ``Chapter'' is
  naturally translated to the appropriate language.}
followed by the chapter number. The title itself is set left-justified on the
following lines. The same effect is obtained in {\KOMAScript} with the option
\Option{chapterprefix}. Any value from table \autoref{tab:truefalseswitch},
\autopageref{tab:truefalseswitch} may be used as \PName{simple switch}.  The
default however is \OptionValue{chapterprefix}{false} in opposite to the
behaviour of the standard classes, that would correspond to
\OptionValue{chapterprefix}{truw}.  These options also affect the automatic
running titles in the headers (see \autoref{sec:maincls.pagestyle},
\autopageref{desc:maincls.pagestyle.headings}).

Sometimes one wishes to have the
chapter titles in simplified form according to
\OptionValue{chapterprefix}{false}. But at the same time, one wishes a title
of an appendix\Index{appendix} to be preceded by a line with ``Appendix''
followed by the appendix letter. This is achieved by using the
\Option{appendixprefix} option (see \autoref{tab:truefalseswitch},
\autopageref{tab:truefalseswitch}). Since this results in an inconsistent
document layout, I advise against using this option.

The font style of the chapter number line using
\OptionValue{chapterprefix}{true} or \OptionValue{appendixprefix}{true} may be
changes with element
\FontElement{chapterprefix}\IndexFontElement{chapterprefix}%
\ChangedAt{v2.96a}{\Class{scrbook}\and \Class{scrreprt}} using commands
\Macro{setkomafont} and \Macro{addtokomafont} (see
\autoref{sec:maincls.textmarkup},
\autopageref{desc:maincls.cmd.setkomafont}). Default is the usage of element
\FontElement{chapter}\IndexFontElement{chapter} (see
\autopageref{desc:maincls.cmd.chapter} as well as
\autoref{tab:maincls.structureElementsFont},
\autopageref{tab:maincls.structureElementsFont}).%
%
\EndIndex{Option}{appendixprefix~=\PName{simple switch}}%
\EndIndex{Option}{chapterprefix~=\PName{simple switch}}%

\begin{Declaration}
  \KOption{headings}\PName{selection}
\end{Declaration}%
\BeginIndex{Option}{headings~=\PName{selection}}%
The font size used for the titles\index{title}\index{document structure} is
relatively big, both with the standard classes and with {\KOMAScript}. Not
everyone likes this choice; moreover it is specially problematic for small
paper sizes. Consequently, {\KOMAScript} provides, besides the large title
font size defined by the
\OptionValue{headings}{big}\IndexOption{headings~=\PValue{big}}%
\ChangedAt{v3.00}{\Class{scrbook}\and \Class{scrreprt}\and \Class{scrartcl}}%
\important[i]{\OptionValue{headings}{big}\\
  \OptionValue{headings}{normal}\\
  \OptionValue{headings}{small}} option, the two options
\OptionValue{headings}{normal}\IndexOption{headings~=\PValue{normal}} and
\OptionValue{headings}{small}\IndexOption{headings~=\PValue{small}}, that
allow for smaller title font sizes. The font sizes for headings resulting from
these options for \Class{scrbook} and \Class{scrreprt} are shown in
\autoref{tab:maincls.structureElementsFont},
\autopageref{tab:maincls.structureElementsFont}. For \Class{scrartcl} smaller
font sizes are generally used. \OnlyAt{\Class{scrbook}\and
  \Class{scrreprt}}The spacing before and after chapter titles is also
influenced by these options.

Chapter\important[i]{\begin{tabular}[t]{@{}l@{}}
    \Option{headings=}\\ \quad\PValue{twolinechapter}\end{tabular}\\
  \begin{tabular}[t]{@{}l@{}}
    \Option{headings=}\\ \quad\PValue{onelinechapter}\end{tabular}\\
  \begin{tabular}[t]{@{}l@{}}
    \Option{headings=}\\ \quad\PValue{twolineappendix}\end{tabular}}
titles are also influenced by the options
\OptionValue{headings}{twolinechapter} and
\OptionValue{headings}{onelinechapter}, that are same like \OptionValue{chapterprefix}{true} and
\OptionValue{chapterprefix}{false} (see above). The appendix titles are
influenced by \OptionValue{headings}{twolineappendix} and
\OptionValue{headings}{onelineappendix}\important[i]{%
  \begin{tabular}[b]{@{}l@{}}
    \Option{headings=}\\ \quad\PValue{onelineappendix}
  \end{tabular}}, that are same like the options
\OptionValue{appendixprefix}{true} and \OptionValue{appendixprefix}{false}
(see also above). 

The\OnlyAt{\Class{scrbook}\and \Class{scrreprt}} method of beginning new
chapters may be switched by \OptionValue{headings}{openany}\important[i]{%
  \OptionValue{headings}{openany}\\
  \OptionValue{headings}{openright}\\
  \OptionValue{headings}{openleft}}, \OptionValue{headings}{openright}, and
\OptionValue{headings}{openleft} alternatively to option \Option{open} with
the values \PValue{any},
\PValue{right}, and \PValue{left} (see above).

Another\ChangedAt{v3.10}{\Class{scrbook}\and \Class{scrreprt}\and
  \Class{scrartcl}} special feature of \KOMAScript{} is the handling of the
optional argument of the structural commands \Macro{part}, \Macro{chapter}
etc. down to \Macro{subparagraph}. Function and meaning\important[i]{\begin{tabular}[t]{@{}l@{}}
    \Option{headings=}\\\quad\PValue{optiontohead}\end{tabular}\\
  \OptionValue{headings}{optiontotoc}\\
  \begin{tabular}[t]{@{}l@{}}
    \Option{headings=}\\\quad\PValue{optiontoheadandtoc}\end{tabular}} may be
influenced by the options \OptionValue{headings}{optiontohead}%
\IndexOption{headings~=\PValue{optiontohead}},
\OptionValue{headings}{optiontotoc}\IndexOption{headings~=\PValue{optiontotoc}},
and \OptionValue{headings}{optiontoheadandtoc}%
\IndexOption{headings~=\PValue{optiontoheadandtoc}}.

A summarize of all the available selections of option \Option{headings} bay be found in \autoref{tab:maincls.headings}. Examples are at the following
description of the structural commands.

\begin{desclist}
  \renewcommand*{\abovecaptionskipcorrection}{-\normalbaselineskip}%
  \desccaption[{Available values for option \Option{headings}}]{%
    Available values for option \Option{headings} to select different kinds of
    structural headings%
    \label{tab:maincls.headings}%
  }{%
    Available values for option \Option{headings} (\emph{continuation})%
  }%
  \pventry{big}{%
    Use very large headings with large distances above and below.
    \IndexOption{headings~=\PValue{big}}}%
  \pventry{normal}{%
    Use mid-size headings with medium large distances above and below.
    \IndexOption{headings~=\PValue{normal}}}%
  \entry{\PValue{onelineappendix}, \PValue{noappendixprefix},
    \PValue{appendixwithoutprefix}, \PValue{appendixwithoutprefixline}%
    \IndexOption{headings~=\PValue{onelineappendix}}}{%
    Chapter headings at the appendix will be set like other headings too.%
  }%
  \entry{\PValue{onelinechapter}, \PValue{nochapterprefix},
    \PValue{chapterwithoutprefix}, \PValue{chapterwithoutprefixline}%
    \IndexOption{headings~=\PValue{onelinechapter}}}{%
    Chapter headings will be set like other headings too.%
  }%
  \pventry{openany}{%
    Parts, chapter, index and back matter use \Macro{clearpage} instead of
    \Macro{cleardoublepage}.%
    \IndexOption{headings~=\PValue{openany}}}%
  \pventry{openleft}{%
    The commands \Macro{cleardoublepageusingstyle},
    \Macro{cleardoublestandardpage}, \Macro{cleardoubleplainpage},
    \Macro{cleardoubleemptypage}, and \Macro{cleardoublepage} generate a page
    break and if needed insert a interleaf page to reach the next left-hand
    page at double-page printing. Part, chapter, index and back matter use
    \Macro{cleardoublepage}.%
    \IndexOption{headings~=\PValue{openleft}}}%
  \pventry{openright}{%
    The commands \Macro{cleardoublepageusingstyle},
    \Macro{cleardoublestandardpage}, \Macro{cleardoubleplainpage},
    \Macro{cleardoubleemptypage}, and \Macro{cleardoublepage} generate a page
    break and if needed insert a interleaf page to reach the next right-hand
    page at double-page printing. Part, chapter, index and back matter use
    \Macro{cleardoublepage}.%
    \IndexOption{headings~=\PValue{openright}}}%
  \pventry{optiontohead}{%
    The\ChangedAt{v3.10}{\Class{scrbook}\and \Class{scrreprt}\and
      \Class{scrartcl}} advanced functionality of the optional argument of the
    structural commands \Macro{part} down to \Macro{subparagraph} will be
    activated. By default the optional argument will be used for the running
    head only.%
    \IndexOption{headings=~optiontohead}%
  }%
  \entry{\PValue{optiontoheadandtoc}, \PValue{optiontotocandhead}%
    \IndexOption{headings~=\PValue{optiontoheadandtoc}}}{%
    The\ChangedAt{v3.10}{\Class{scrbook}\and \Class{scrreprt}\and
      \Class{scrartcl}} advanced functionality of the optional argument of the
    structural commands \Macro{part} down to \Macro{subparagraph} will be
    activated. By default the optional argument will be used for the running
    head and the table of contents.%
  }%
  \pventry{optiontotoc}{%
    The\ChangedAt{v3.10}{\Class{scrbook}\and \Class{scrreprt}\and
      \Class{scrartcl}} advanced functionality of the optional argument of the
    structural commands \Macro{part} down to \Macro{subparagraph} will be
    activated. By default the optional argument will be used for the table of
    contents only.%
    \IndexOption{headings=~optiontohead}%
  }%
  \pventry{small}{%
    Use small headings with small distances above and below.%
    \IndexOption{headings~=\PValue{small}}}%
  \entry{\PValue{twolineappendix}, \PValue{appendixprefix},
    \PValue{appendixwithprefix}, \PValue{appendixwithprefixline}%
    \IndexOption{headings~=\PValue{twolineappendix}}}{%
    Chapters at the appendix will be set with a number line with the contents
    of \Macro{chapterformat}\IndexCmd{chapterformat}.%
  }%
  \entry{\PValue{twolinechapter}, \PValue{chapterprefix},
    \PValue{chapterwithprefix}, \PValue{chapterwithprefixline}%
    \IndexOption{headings~=\PValue{twolinechapter}}}{%
    Chapters will be set with a number line with the contents
    of \Macro{chapterformat}\IndexCmd{chapterformat}.%
  }%
\end{desclist}
%
\EndIndex{Option}{headings~=\PName{selection}}%

\begin{Declaration}
  \KOption{numbers}{selection}
\end{Declaration}%
\BeginIndex{Option}{numbers~=\PName{selection}}%
\begin{Explain}%
  In German, according to {\small DUDEN}, the numbering of sectional
  units should have no period at the end if only arabic numbers are used
  (see \cite[R\,3]{DUDEN}). On the other hand, if roman numerals or
  letters are appear in the numbering, then a period should appear at the
  end of the numbering (see \cite[R\,4]{DUDEN}). {\KOMAScript} has an
  internal mechanisms that tries to implement this somewhat complex
  rule. The resulting effect is that, normally, after the sectional
  commands \Macro{part} and \Macro{appendix} a switch is made to
  numbering with an ending period. The information is saved in the
  \File{aux} file and takes effect on the next {\LaTeX} run.
\end{Explain}

In some cases the mechanism for placing or leaving off the ending period may
fail, or other languagues may have different
rules. Therefore\important[i]{\OptionValue{numbers}{endperiod}\\
  \OptionValue{numbers}{noendperiod}} it is possible to activate the use of
the ending period manually with the option
\OptionValue{numbers}{endperiod}\IndexOption{numbers~=\PValue{endperiod}} or
to deactivate it with
\OptionValue{numbers}{noendperiod}\OptionValue{numbers}{noendperiod}. Default
is \OptionValue{numbers}{autoendperiod}%
\IndexOption{numbers~=\PValue{autoendperiod}} with auto detection whether to
set the period or not.

Please\textnote{Attention!} note that the mechanism only takes effect on the
next {\LaTeX} run.  Therefore, before trying to use these options to forcibly
control the numbering format, a further run without changing any options
should be made.

The\textnote{Attention!} available values are summarized in
\autoref{tab:maincls.numbers}. In opposite to most other selections, this
option may be changed at the document preamble, before
\Macro{begin}\PParameter{document}, only.

\begin{table}
  \caption[{Available values of option \Option{numbers}}]{Available values of
    option \Option{numbers} for selection of the period at the end of numbers
    of structural headings}
  \label{tab:maincls.numbers}
  \begin{desctabular}
    \entry{\PValue{autoendperiod}, \PValue{autoenddot}, \PValue{auto}}{%
      \KOMAScript{} decides, whether or not to set the period at the end of
      the numbers. The numbers consists in Arabic digits only, the period will
      be omitted. If there are alphabetic characters or roman numbers the
      period will always be set. References to numbers will be set without
      ending period always.%
      \IndexOption{numbers~=\PValue{autoendperiod}}}%
    \entry{\PValue{endperiod}, \PValue{withendperiod}, \PValue{periodatend},
      \PValue{enddot}, \PValue{withenddot}, \PValue{dotatend}}{%
      All numbers of structural commands and all dependent numbers will be set
      with ending period. Only references will be set without the ending
      period.% 
      \IndexOption{numbers~=\PValue{endperiod}}}%
    \entry{\PValue{noendperiod}, \PValue{noperiodatend},
      \PValue{noenddot}, \PValue{nodotatend}}{%
      All the numbers are without ending period.%
      \IndexOption{numbers~=\PValue{noendperiod}}}%
  \end{desctabular}
\end{table}
%
\EndIndex{Option}{numbers~=\PName{selection}}%

\begin{Declaration}
  \Option{chapteratlists}\\
  \XOption{chapteratlists}=\PName{value}
\end{Declaration}%
\BeginIndex{Option}{chapteratlists}%
\BeginIndex{Option}{chapteratlists~=\PName{value}}%
As mentioned in \autoref{sec:maincls.floats},
\autopageref{desc:maincls.option.listof}\OnlyAt{\Class{scrbook}\and
  \Class{scrreprt}} normally, every chapter entry generated with
\Macro{chapter} introduces vertical spacing into the lists of floats. Since
version~2.96a\ChangedAt{v2.96a}{\Class{scrbook}\and \Class{scrreprt}} this
applies also for the command \Macro{addchap}, if no compatibility option to an
earlier version was chosen (see option \Option{version} in
\autoref{sec:maincls.compatibilityOptions},
\autopageref{desc:maincls.option.version}).

Furthermore, now the option \Option{chapteratlists} can be used to change the
spacing, by passing the desired distance as \PName{value}. The default setting
with \OptionValue{listof}{chaptergapsmall}%
\IndexOption{listof~=\PValue{chaptergapsmall}}%
is 10\,pt. If \OptionValue{chapteratlists}{entry}%
\IndexOption{chapteratlists~=\PValue{entry}}%
\important{\OptionValue{chapteratlists}{entry}} or \Option{chapteratlists}
without value is specified, then instead of a vertical distance the chapter
entry itself will be entered into the lists. This will be done even, if
there's no floating environment inside of the chapter.

Please\textnote{Attention!} note that changes to the option will only become
effective in the lists following two more {\LaTeX} runs.%
%
\EndIndex{Option}{chapteratlists~=\PName{value}}%
\EndIndex{Option}{chapteratlists}%


\begin{Declaration}
  \Macro{part}\OParameter{short version}\Parameter{heading}\\
  \Macro{chapter}\OParameter{short version}\Parameter{heading}\\
  \Macro{section}\OParameter{short version}\Parameter{heading}\\
  \Macro{subsection}\OParameter{short version}\Parameter{heading}\\
  \Macro{subsubsection}\OParameter{short version}\Parameter{heading}\\
  \Macro{paragraph}\OParameter{short version}\Parameter{heading}\\
  \Macro{subparagraph}\OParameter{short version}\Parameter{heading}
\end{Declaration}%
\BeginIndex{Cmd}{part}\Index[indexmain]{part}%
\BeginIndex{Cmd}{chapter}\Index[indexmain]{chapter}%
\BeginIndex{Cmd}{section}\Index[indexmain]{section}%
\BeginIndex{Cmd}{subsection}%
\BeginIndex{Cmd}{subsubsection}%
\BeginIndex{Cmd}{paragraph}%
\BeginIndex{Cmd}{subparagraph}%
The standard sectioning commands in {\KOMAScript} work in a similar
fashion to those of the standard classes. Thus, an alternative entry
for the table of contents and running headings can be specified as an
optional argument to the sectioning commands.

In difference to this with\ChangedAt{v3.10}{\Class{scrbook}\and
  \Class{scrreprt}\and \Class{scrartcl}}\textnote{\KOMAScript{} vs. standard
  classes} option \important{\Option{headings}}
\OptionValue{headings}{optiontohead}%
\IndexOption[indexmain]{headings~=\PValue{optiontohead}} \KOMAScript{} uses
the optional argument \PName{short version} not longer at the table of
contents but only for the running head. Nevertheless, such a running head
needs an appropriate page style. See \autoref{sec:maincls.pagestyle} and
\autoref{cha:scrpage} about this. With option
\OptionValue{headings}{optiontotoc}%
\IndexOption[indexmain]{headings~=\PValue{optiontotoc}} \KOMAScript{} uses the
optional argument \PName{short version} not longer for the running head but
only at the table of contents. Nevertheless the entry will be shown only, if
counter \Counter{tocdepth} (see \autoref{sec:maincls.toc},
\autopageref{desc:maincls.counter.tocdepth}) is great enough. With option
\OptionValue{headings}{optiontoheadandtoc}%
\IndexOption[indexmain]{headings~=\PValue{optiontoheadandtoc}} \KOMAScript{}
uses the optional argument \PName{short version} at both, table of contents
and running head. All these three selections will also activate the extended
interpretation of the optional argument \PName{short version}, that isn't
active by default.

The \ChangedAt{v3.10}{\Class{scrbook}\and \Class{scrreprt}\and
  \Class{scrartcl}} extended interpretation of the optional argument
determines, whether there's a equality sign in
\PName{short version}. If so, the optional argument will be interpreted as
\PName{option list} instead of simple \PName{short version}. Thereby the two
options \KOption{head}\PName{running head}\important{\Option{head}} and
\KOption{tocentry}\PName{table of contents entry} are supported. Commas or
equality signs inside of the values of those options will be accepted only, if
the are enclosed by braces.

Please\textnote{Attention!} note, that this mechanism is only as long
functional as \KOMAScript{} controls the described commands. From using a
package, that controls the sectioning commands or the internal \LaTeX kernel
commands for sectioning commands, \KOMAScript cannot longer provide this
extended mechanism. This is also valid for the always active extension of
\KOMAScript{} to not create entries to the table of contents if the heading
text if the entry is empty. If you really want an entry with empty heading
text, you may use an invisible entry like \lstinline|\mbox{}| instead.

\begin{Example}
  Assumed, you're writing a document with some very extensive chapter
  headings. These headings should be shown in the table of contents too. But
  for the running head you want only single-line short headings. You will do
  this using the optional argument of \Macro{chapter}.
\begin{lstcode}
  \chapter[short version of chapter heading]
          {The Structural Sectioning Command
           for Chapters Supports not only the
           Heading Text itself but also a
           Short Version with Selectable
           Usage}
\end{lstcode}

  Sometimes later you become aware, that the automatic line breaking of this
  heading is somehow inappropriate. Therefor you want to make the breaking
  yourself. Nevertheless the automatic line breaking should be still used at
  the table of contents. With
\begin{lstcode}
  \chapter[head={short version of chapter heading},
           tocentry={The Structural Sectioning
             Command for Chapters Supports not
             only the Heading Text itself but
             also a Short Version with 
             Selectable Usage}]
          {The Structural\\
            Sectioning Command for Chapters\\
            Supports not only\\
            the Heading Text itself\\
            but also\\
            a Short Version\\
            with Selectable Usage}
\end{lstcode}
  you use independent entries for table of contents, running head and the
  chapter heading itself. The arguments of the options \Option{head} and
  \Option{tocentry} have been enclosed into braces, so the contents of the
  options cannot influence the interpretation of the optional argument.

  The recommendation of the braces in the example above will make more sense
  at one more example. Assumed you're using option
  \OptionValue{headings}{optiontotoc} and know have a heading:
\begin{lstcode}
  \section[head=\emph{value}]
          {Option head=\emph{value}}
\end{lstcode}
  This would result to the entry ``Option head=\emph{value}'' at the table of
  contents but ``\emph{value}'' at the running head. But surely you wanted the
  entry ``head=\emph{value}'' at the table of contents and the complete
  heading text at the running head. You may do this using braces:
\begin{lstcode}
  \section[head{=}\emph{value}]
          {Option head=\emph{value}}
\end{lstcode}

  A similar case would be a comma. With the same \Option{headings} option like
  before:
\begin{lstcode}
  \section[head=0, 1, 2, 3, \dots]
          {Natural Numbers Including the Zero}
\end{lstcode}
  would result in an error, because the comma would be interpreted as the
  separator between the single options of the option list %
  ``\lstinline|0, 1, 2, 3, \dots|''. But writing
\begin{lstcode}
  \section[head={0, 1, 2, 3, \dots}]
          {Natural Numbers Including the Zero}
\end{lstcode}
  will change ``\lstinline|0, 1, 2, 3, \dots|'' into the argument of
  option \Option{head}.
\end{Example}

The title of the level part\important{\Macro{part}} (\Macro{part}) is
distinguished from other sectioning levels by being numbered independently
from the other parts. This means that the chapter level (in \Class{scrbook} or
\Class{scrreprt}), or the section level (in \Class{scrartcl}) is numbered
consecutively over all parts. Furthermore, for classes
\Class{scrbook}\OnlyAt{\Class{scrbook}\and \Class{scrreprt}} and
\Class{scrreprt} the title of the part level together with the corresponding
preamble (see \Macro{setpartpreamble},
\autopageref{desc:maincls.cmd.setpartpreamble}) is set on a separate page.

\Macro{chapter}\important{\Macro{chapter}}\OnlyAt{\Class{scrbook}\and
  \Class{scrreprt}} only exists in book or report classes, that is, in classes
\Class{book}, \Class{scrbook}, \Class{report} and \Class{scrreport}, but not
in the article classes \Class{article} and \Class{scrartcl}. In addition to
this, the command \Macro{chapter} in {\KOMAScript} differs substantially from
the version in the standard class. In the standard classes the chapter number
is used together with the prefix ``Chapter'', or the corresponding word in the
appropriate language, on a separate line above the actual chapter title
test. This overpowering\important[i]{\Option{chapterprefix}\\
  \Option{appendixprefix}} style is replaced in {\KOMAScript} by a simple
chapter number before the chapter heading text, can however be reverted by the
option \Option{chapterprefix} (see
\autopageref{desc:maincls.option.chapterprefix}).

Please\textnote{Attention!} note that \Macro{part} and \Macro{chapter} in
classes \Class{scrbook} and \Class{scrreprt}
\OnlyAt{\Class{scrbook}\and\Class{scrreprt}} change the page style for one
page. The applied page style in {\KOMAScript} is defined in the macros
\Macro{partpagestyle} and \Macro{chapterpagestyle} (see
\autoref{sec:maincls.pagestyle},
\autopageref{desc:maincls.cmd.titlepagestyle}).

\BeginIndex[indexother]{}{font}%
\BeginIndex[indexother]{}{font>style}%
\BeginIndex[indexother]{}{font>size}%
The font of all headings\ChangedAt{v2.8p}{\Class{scrbook}\and
  \Class{scrreprt}\and \Class{scrartcl}} can be changed with the commands
\Macro{setkomafont}\IndexCmd{setkomafont} and
\Macro{addtokomafont}\IndexCmd{addtokomafont} (see
\autoref{sec:maincls.textmarkup},
\autopageref{desc:maincls.cmd.setkomafont}). In doing this, generally the
element \FontElement{disposition}\IndexFontElement{disposition}%
\important{\FontElement{disposition}} is used, followed by a specific
element\important[s]{%
  \FontElement{part}\\
  \FontElement{chapter}\\
  \FontElement{section}\\
  \FontElement{subsection}\\
  \FontElement{paragraph}\\
  \FontElement{subparagraph}} for every section level (see
\autoref{tab:maincls.elementsWithoutText},
\autopageref{tab:maincls.elementsWithoutText}). The font for the element
\FontElement{disposition}\IndexFontElement{disposition} is predefined as
\Macro{normalcolor}\linebreak[2]\Macro{sffamily}\linebreak[2]\Macro{bfseries}.
The default font size for the specific elements depends on the options
\OptionValue{headings}{big}, \OptionValue{headings}{normal} and
\OptionValue{headings}{small} (see
\autopageref{desc:maincls.option.headings}). The defaults are listed in
\autoref{tab:maincls.structureElementsFont}.
%
\begin{table}
%  \centering%
  \KOMAoptions{captions=topbeside}%
  \setcapindent{0pt}%
%  \caption
  \begin{captionbeside}[{Default font sizes for different levels of document
    structuring}]{Default font sizes for different levels of document
    structuring in \Class{scrbook} and \Class{scrreprt}}[l]
  \begin{tabular}[t]{lll}
    \toprule
    class option & element & default\\
    \midrule
    \OptionValue{headings}{big}
      & \FontElement{part}\IndexFontElement{part}
      & \Macro{Huge} \\
      & \FontElement{partnumber}\IndexFontElement{partnumber}
      & \Macro{huge} \\
      & \FontElement{chapter}\IndexFontElement{chapter}
      & \Macro{huge} \\
      & \FontElement{section}\IndexFontElement{section}
      & \Macro{Large} \\
      & \FontElement{subsection}\IndexFontElement{subsection}
      & \Macro{large} \\
      & \FontElement{subsubsection}%
        \IndexFontElement{subsubsection}
      & \Macro{normalsize} \\
      & \FontElement{paragraph}\IndexFontElement{paragraph}
      & \Macro{normalsize} \\
      & \FontElement{subparagraph}\IndexFontElement{subparagraph}
      & \Macro{normalsize} \\[1ex]
    \OptionValue{headings}{normal}
      & \FontElement{part}          & \Macro{huge} \\
      & \FontElement{partnumber}    & \Macro{huge} \\
      & \FontElement{chapter}       & \Macro{LARGE} \\
      & \FontElement{section}       & \Macro{Large} \\
      & \FontElement{subsection}    & \Macro{large} \\
      & \FontElement{subsubsection} & \Macro{normalsize} \\
      & \FontElement{paragraph}     & \Macro{normalsize} \\
      & \FontElement{subparagraph}  & \Macro{normalsize} \\[1ex]
    \OptionValue{headings}{small}
      & \FontElement{part}          & \Macro{LARGE} \\
      & \FontElement{partnumber}    & \Macro{LARGE} \\
      & \FontElement{chapter}       & \Macro{Large} \\
      & \FontElement{section}       & \Macro{large} \\
      & \FontElement{subsection}    & \Macro{normalsize} \\
      & \FontElement{subsubsection} & \Macro{normalsize} \\
      & \FontElement{paragraph}     & \Macro{normalsize} \\
      & \FontElement{subparagraph}  & \Macro{normalsize}\\
    \bottomrule
  \end{tabular}
  \end{captionbeside}
  \label{tab:maincls.structureElementsFont}
\end{table}

\begin{Example}
  Suppose you are using the class option \OptionValue{headings}{big} and
  notice that the very big headings of document parts are too
  bold. You could change this as follows:
\begin{lstcode}
  \setkomafont{disposition}{\normalcolor\sffamily}
  \part{Appendices}
  \addtokomafont{disposition}{\bfseries}
\end{lstcode}
Using the command above you only switch off the font attribute
\textbf{bold} for a heading ``Appendices''. A much more comfortable
and elegant solution is to change all \Macro{part} headings at once.
This is done either by:
\begin{lstcode}
  \addtokomafont{part}{\normalfont\sffamily}
  \addtokomafont{partnumber}{\normalfont\sffamily}
\end{lstcode}
  or simply using:
\begin{lstcode}
  \addtokomafont{part}{\mdseries}
  \addtokomafont{partnumber}{\mdseries}
\end{lstcode}
The last version is to be preferred because it gives you the correct
result even when you make changes to the \FontElement{disposition}
element\IndexFontElement{disposition}, for instance:
\begin{lstcode}
  \setkomafont{disposition}{\normalcolor\bfseries}
\end{lstcode}
  With this change it is possible to set all section levels at once to no
  longer use sans serif fonts.
\end{Example}

Please be warned of misusing the possibilities of font switching to
mix fonts, font sizes and font attributes excessively. Picking the
most suitable font for a given task is a hard task even for
professionals and has almost nothing to do with the personal tastes of
non-experts. Please refer to the citation at the end of
\autoref{sec:typearea.tips}, \autopageref{sec:typearea.tips.cite} and
to the following explanation.

\begin{Explain}
  It is possible to use different font types for different section
  levels in {\KOMAScript}. Non-experts in typography should for very
  good typographical reasons refrain absolutely from using these
  possibilities.

  There is a rule in typography which states that one should mix as
  few fonts as possible. Using sans serif for headings already seems
  to be a breach of this rule. However, one should know that bold,
  large serif letters are much to heavy for headings. Strictly
  speaking, one would then have to at least use a normal instead of a
  bold or semi bold font. However, in deeper levels of the structuring
  a normal font may then appear too lightly weighted. On the other
  hand, sans serif fonts in headings have a very pleasant appearance
  and in fact find acceptance almost solely for headings. That is why
  sans serif is the carefully chosen default in {\KOMAScript}.

  More variety should however be avoided. Font mixing is only for
  professionals. In case you want to use other fonts than the standard
  {\TeX} fonts\,---\,regardless of whether these are CM \Index{CM
    fonts}, EC \Index{EC fonts} or LM fonts \Index{LM fonts}\,---\,you
  should consult an expert, or for safety's sake redefine the font for
  the element \FontElement{disposition}\IndexFontElement{disposition}
  as seen in the example above. The author of this documentation
  considers the commonly encountered combinations Times and Helvetica
  or Palatino with Helvetica as unfavourable.
\end{Explain}
\EndIndex[indexother]{}{font>style}%
\EndIndex[indexother]{}{font}%
\EndIndex[indexother]{}{font>size}%
%
\EndIndex{Cmd}{part}%
\EndIndex{Cmd}{chapter}%
\EndIndex{Cmd}{section}%
\EndIndex{Cmd}{subsection}%
\EndIndex{Cmd}{subsubsection}%
\EndIndex{Cmd}{paragraph}%
\EndIndex{Cmd}{subparagraph}%

\begin{Declaration}
  \Macro{part*}\Parameter{Heading}\\
  \Macro{chapter*}\Parameter{Heading}\\
  \Macro{section*}\Parameter{Heading}\\
  \Macro{subsection*}\Parameter{Heading}\\
  \Macro{subsubsection*}\Parameter{Heading}\\
  \Macro{paragraph*}\Parameter{Heading}\\
  \Macro{subparagraph*}\Parameter{Heading}
\end{Declaration}%
\BeginIndex{Cmd}{part*}%
\BeginIndex{Cmd}{chapter*}%
\BeginIndex{Cmd}{section*}%
\BeginIndex{Cmd}{subsection*}%
\BeginIndex{Cmd}{subsubsection*}%
\BeginIndex{Cmd}{paragraph*}%
\BeginIndex{Cmd}{subparagraph*}%
All disposition commands have starred versions, which are
unnumbered\Index{numbering}, and produce section headings which do not
show up in the table of contents\Index{table>of contents}\Index{contents>table
  of} or in in the
running heading\Index{header}. The absence of a running heading often
has an unwanted side effect. For example, if a chapter which is set
using \Macro{chapter*} spans several pages, then the running heading
of the previous chapter suddenly reappears. {\KOMAScript} offers a
solution for this which is described
below. \OnlyAt{\Class{scrbook}\and\Class{scrreprt}}\Macro{chapter*}
only exists in book and report classes, that is, \Class{book},
\Class{scrbook}, \Class{report} and \Class{scrreport}, but not the
article classes \Class{article} and \Class{scrartcl}.

Please\textnote{Attention!} note that \Macro{part} and \Macro{chapter} change
the page style for one page. The applied style is defined in the macros
\Macro{partpagestyle} and \Macro{chapterpagestyle} in {\KOMAScript} (see
\autoref{sec:maincls.pagestyle},
\autopageref{desc:maincls.cmd.titlepagestyle}).

As for the possibilities of font switching\ChangedAt{v2.8p}{%
  \Class{scrbook}\and\Class{scrreprt}\and\Class{scrartcl}}, %
the same explanations apply as were given above for the unstarred
variants. The structuring elements are named the same since they do
not indicate variants but structuring levels.%
%
\EndIndex{Cmd}{part*}%
\EndIndex{Cmd}{chapter*}%
\EndIndex{Cmd}{section*}%
\EndIndex{Cmd}{subsection*}%
\EndIndex{Cmd}{subsubsection*}%
\EndIndex{Cmd}{paragraph*}%
\EndIndex{Cmd}{subparagraph*}%

\iftrue% Umbruchkorrekturtext
In the standard classes\textnote{\KOMAScript{} vs. standard classes} there are
no further structuring commands. In particular, there are no commands which
can produce unnumbered chapters or sections which show up in the table of
contents and in the running heading.%
\fi

\begin{Declaration}
  \Macro{addpart}\OParameter{Short version}\Parameter{Heading}\\
  \Macro{addpart*}\Parameter{Heading}\\
  \Macro{addchap}\OParameter{Short version}\Parameter{Heading}\\
  \Macro{addchap*}\Parameter{Heading}\\
  \Macro{addsec}\OParameter{Short version}\Parameter{Heading}\\
  \Macro{addsec*}\Parameter{Heading}\textnote[n]{\KOMAScript{} vs. standard classes}
\end{Declaration}%
\BeginIndex{Cmd}{addpart}%
\BeginIndex{Cmd}{addpart*}%
\BeginIndex{Cmd}{addchap}%
\BeginIndex{Cmd}{addchap*}%
\BeginIndex{Cmd}{addsec}%
\BeginIndex{Cmd}{addsec*}%
In addition to the commands of the standard classes {\KOMAScript}
offers the new commands \Macro{addsec} and \Macro{addchap}. They are
similar to the standard commands \Macro{chapter} and
\Macro{section} except that they are unnumbered. They thus produce
both a running heading and an entry in the table of contents. 

The starred variants \Macro{addchap*} and \Macro{addsec*} are similar to the
standard commands \Macro{chapter*} and \Macro{section*} except for a tiny but
important difference: The running headings are deleted. This eliminates the
side effect of obsolete headers mentioned above. Instead, the running headings
on following pages remain empty. \OnlyAt{\Class{book}\and
  \Class{scrreprt}}\Macro{addchap} and \Macro{addchap*} of course only exist
in book and report classes, namely \Class{book}, \Class{scrbook},
\Class{report} and \Class{scrreport}, but not in the article classes
\Class{article} and \Class{scrartcl}.

Similarly, the command \Macro{addpart} produces an unnumbered document
part with an entry in the table of contents. Since the running
headings are already deleted by \Macro{part} and \Macro{part*} the
problem of obsolete headers does not exist. The starred version
\Macro{addpart*} is thus identical to \Macro{part*} and is only
defined for consistency reasons.

Please note\textnote{Attention!} that \Macro{addpart} and \Macro{addchap} and
their starred versions change the page style for one page. The particular page
style is defined in the macros \Macro{partpagestyle} and
\Macro{chapterpagestyle} (see \autoref{sec:maincls.pagestyle},
\autopageref{desc:maincls.cmd.titlepagestyle}).

As for the possibilities of font switching\ChangedAt{v2.8p}{%
  \Class{scrbook}\and\Class{scrreprt}\and\Class{scrartcl}}, %
the same explanations apply as given above for the normal structuring
commands. The elements are named the same since they describe not
variants but structuring levels.%
%
\EndIndex{Cmd}{addpart}%
\EndIndex{Cmd}{addpart*}%
\EndIndex{Cmd}{addchap}%
\EndIndex{Cmd}{addchap*}%
\EndIndex{Cmd}{addsec}%
\EndIndex{Cmd}{addsec*}%


\begin{Declaration}
  \Macro{minisec}\Parameter{Heading}
\end{Declaration}%
\BeginIndex{Cmd}{minisec}%
Sometimes a heading\Index{heading} is wanted which is highlighted but
also closely linked to the following text. Such a heading should not
be separated by a large vertical skip.

The command \Macro{minisec} is designed for this situation. This
heading is not associated with any structuring level. Such a
\emph{mini section} does not produce an entry in the table of contents
nor does it receive any numbering.

\BeginIndex{FontElement}{minisec}%
\BeginIndex{FontElement}{disposition}%
The font type of the structuring command \Macro{minisec} be changed using the
element \FontElement{disposition}\IndexFontElement{disposition} (see
\autoref{tab:maincls.elementsWithoutText},
\autopageref{tab:maincls.elementsWithoutText}) and
\FontElement{minisec}\ChangedAt{2.96a}{%
  \Class{scrbook} \and\Class{scrreprt} \and\Class{scrartcl}}. Default setting
of element \FontElement{minisec} is empty, so the default of the element
\FontElement{disposition} is active.
%
\EndIndex{FontElement}{disposition}%
\EndIndex{FontElement}{minisec}%

\begin{Example}
  You have developed a kit for building a mouse trap and want the
  documentation separated into a list of necessary items and an
  assembly description. You could write the following:
\begin{lstcode}
  \minisec{Items needed}

  \begin{flushleft}
    1 plank ($100\times 50 \times 12$)\\
    1 spring-plug of a beer-bottle\\
    1 spring of a ball-point pen\\
    1 drawing pin\\
    2 screws\\
    1 hammer\\
    1 knife
  \end{flushleft}

  \minisec{Assembly}
  At first one searches the mouse-hole and puts the drawing pin
  directly behind the hole.  Thus the mouse cannot escape during the
  following actions.

  Then one knocks the spring-plug with the hammer into the mouse-hole.
  If the spring-plug's size is not big enough in order to shut the
  mouse-hole entirely, then one can utilize the plank instead and
  fasten it against the front of the mouse-hole utilizing the two
  screws and the knife.  Instead of the knife one can use a
  screw-driver instead.
\end{lstcode}
  Which gives:
  \begin{ShowOutput}[\baselineskip]\setlength{\parindent}{1em}
    \minisec{Items needed}

  \begin{flushleft}
    1 plank ($100\times 50 \times 12$)\\
    1 spring-plug of a beer-bottle\\
    1 spring of a ball-point pen\\
    1 drawing pin\\
    2 screws\\
    1 hammer\\
    1 knife
  \end{flushleft}

  \minisec{Assembly}
  At first one searches the mouse-hole and puts the drawing pin
  directly behind the hole.  Thus the mouse cannot escape during the
  following actions.

  Then one knocks the spring-plug with the hammer into the
  mouse-hole.  If the spring-plug's size is not big enough in order
  to shut the mouse-hole entirely, then one can utilize the plank
  instead and fasten it against the front of the mouse-hole
  utilizing the two screws and the knife.  Instead of the knife one
  can use a screw-driver instead.
  \end{ShowOutput}
\end{Example}
%
\EndIndex{Cmd}{minisec}%

\begin{Declaration}
  \Macro{raggedsection}\\
  \Macro{raggedpart}
\end{Declaration}%
\BeginIndex{Cmd}{raggedsection}%
\BeginIndex{Cmd}{raggedpart}%
In the standard classes\textnote{\KOMAScript{} vs. standard classes} headings
are set as justified text. That means that hyphenated words can occur and
headings with more than one line are stretched up to the text border. This is
a rather uncommon approach in typography. {\KOMAScript} therefore formats the
headings left aligned with hanging indentation using \Macro{raggedsection}
with the definition:
\begin{lstcode}[belowskip=\dp\strutbox]
  \newcommand*{\raggedsection}{\raggedright}
\end{lstcode}
This command can be redefined with \Macro{renewcommand}.
\begin{Example}
  You prefer justified headings, so you write in the preamble of your document:
\begin{lstcode}
  \renewcommand*{\raggedsection}{}
\end{lstcode}
  or more compactly:
\begin{lstcode}
  \let\raggedsection\relax
\end{lstcode}
  You will get a formatting of the headings which is very close to that
  of the standard classes. It will become even closer when you combine
  this change with the change of the element
  \FontElement{disposition}\IndexFontElement{disposition} mentioned
  above.
\end{Example}
In difference to all others the headings of parts (\Macro{part}) will be
horizontally centered instead of set ragged right. This is because command
\Macro{raggedpart} is defined as:
\begin{lstcode}[belowskip=\dp\strutbox]
  \let\raggedpart\centering
\end{lstcode}
You may also redefine this using \Macro{renewcommand} too.
\begin{Example}
  You don't want different alignment at headings of \Macro{part}. So you
  put
\begin{lstcode}
  \renewcommand*{\raggedpart}{\raggedsection}
\end{lstcode}
  into the preamble of your document. In\textnote{Hint!} this case and in
  opposite to the example above \Macro{let} has not been used, because
  \Macro{let} would give \Macro{raggedpart} the current meaning of
  \Macro{raggedsection}. Further changes of \Macro{raggedsection} would then
  stay disregarded at the usage of \Macro{raggedpart}. Doing the redefinition
  using \Macro{renewcommand} give \Macro{raggedpart} the meaning of
  \Macro{raggedsection} not at definition time, but each time
  \Macro{raggedpart} will be used.
\end{Example}%
%
\EndIndex{Cmd}{raggedpart}%
\EndIndex{Cmd}{raggedsection}%


\begin{Declaration}
  \Macro{partformat}\\
  \Macro{chapterformat}\\
  \Macro{othersectionlevelsformat}\Parameter{sectioning
    name}\Parameter{}\Parameter{counter output}\\
  \Macro{autodot}
\end{Declaration}%
\BeginIndex{Cmd}{partformat}\Index{part>number}%
\BeginIndex{Cmd}{chapterformat}\Index{chapter>number}%
\BeginIndex{Cmd}{othersectionlevelsformat}\Index{section>number}%
\BeginIndex{Cmd}{autodot}%
{\KOMAScript} has added a further logical level on the top of
\Macro{the\PName{sectioning name}} to the output of the sectioning
numbers. The counters for the respective heading are not merely output. They
are formatted using the commands \Macro{partformat}, \Macro{chapterformat},
and the command \Macro{othersectionlevelsformat}, that expects three
arguments.  \OnlyAt{\Class{scrbook}\and\Class{scrreprt}}Of course the command
\Macro{chapterformat} like \Macro{thechapter} does not exist in the class
\Class{scrartcl} but only in the classes \Class{scrbook} and \Class{scrreprt}.

As described for option \Option{numbers}\important{\Option{numbers}} at the
beginning of this section (see \autopageref{desc:maincls.option.numbers}),
periods in section numbers should be handled for the German-speaking region
according to the rules given in \cite{DUDEN}. The command \Macro{autodot} in
{\KOMAScript} ensures that these rules are being followed. In all levels
except for \Macro{part} a dot is followed by a further \Macro{enskip}. This
corresponds to a horizontal skip of 0.5\Unit{em}.

The command \Macro{othersectionlevelsformat} takes as first parameter the
name of the section level, such as \PValue{section},
\PValue{subsection}, \PValue{subsection},
\PValue{subsubsection}, \PValue{paragraph} and \PValue{subparagraph}. The
third parameter should be the output of the corresponding counter, usually
\Macro{thesection}, \Macro{thesubsection}, \Macro{thesubsubsection},
\Macro{theparagraph}, or \Macro{thesubparagraph}.

Per default therefore, only the levels
\Macro{part} and \Macro{chapter} have formatting commands of their
own, while all other section levels are covered by one general
formatting command. This has historical reasons. At the time that
Werner Lemberg suggested a suitable extension of {\KOMAScript} for his
\Package{CJK} package, only this differentiation was needed. Users should
ignore the third argument of \Macro{othersectionlevelsformat} completely.

The formatting commands can be redefined using \Macro{renewcommand} to
fit them to your personal needs. The following original definitions
are used by the {\KOMAScript} classes:
\begin{lstcode}
  \newcommand*{\partformat}{\partname~\thepart\autodot}
  \newcommand*{\chapterformat}{%
    \chapappifchapterprefix{\ }\thechapter\autodot\enskip}
  \newcommand*{\othersectionlevelsformat}[3]{%
    #3\autodot\enskip}
\end{lstcode}

\begin{Example}
  Assume that when using \Macro{part} you do not want the word
  ``Part'' written in front of the part number.  You could use the
  following command in the preamble of your document:
\begin{lstcode}
  \renewcommand*{\partformat}{\thepart\autodot}
\end{lstcode}
  Strictly speaking, you could do without \Macro{autodot} at this point and
  insert a fixed dot instead. As \Macro{part} is numbered with roman numerals,
  according to \cite{DUDEN} a period has to be applied. However, you thereby
  give up the possibility to use one of the options
  \OptionValue{numbers}{endperiod} and \OptionValue{numbers}{noendperiod} and
  optionally depart from the rules. More details concerning class options can
  be found at \autopageref{desc:maincls.option.numbers}.
  
  An additional possibility could be to place the section numbers in the
  left margin in such a way that the heading text is left aligned with
  the surrounding text.  This can be accomplished with:
\begin{lstcode}
  \renewcommand*{\othersectionlevelsformat}[3]{%
      \llap{\#3\autodot\enskip}}
\end{lstcode}
  The little known {\TeX} command \Macro{llap} in the definition above
  puts its argument left of the current position without changing the
  position thereby. A much better {\LaTeX} solution would be:
\begin{lstcode}
  \renewcommand*{\othersectionlevelsformat}[3]{%
    \makebox[0pt][r]{%
      #3\autodot\enskip}}
\end{lstcode}
  See \cite{latex:usrguide} for more information about the optional arguments
  of \Macro{makebox}.
\end{Example}
%
\EndIndex{Cmd}{partformat}%
\EndIndex{Cmd}{chapterformat}%
\EndIndex{Cmd}{othersectionlevelsformat}%
\EndIndex{Cmd}{autodot}%


\begin{Declaration}
  \Macro{chapappifchapterprefix}\Parameter{additional text}\\
  \Macro{chapapp}
\end{Declaration}%
\BeginIndex{Cmd}{chapappifchapterprefix}%
\BeginIndex{Cmd}{chapapp}%
These\OnlyAt{\Class{scrbook}\and\Class{scrreprt}}%
\ChangedAt{v2.8o}{\Class{scrbook}\and\Class{scrreprt}} two commands
are not only used internally by {\KOMAScript} but are also provided to
the user. Later it will be shown how they can be used for example to
redefine other commands. 

Using the layout option
\OptionValue{chapterprefix}{true}\important{\Option{chapterprefix}} (see
\autopageref{desc:maincls.option.chapterprefix})
\Macro{chapappifchapterprefix} outputs the word
``Chapter''\Index{chapter>heading} in the main part of the document in the
current language, followed by \PName{additional text}. In the
appendix\Index{appendix}, the word ``Appendix'' in the current
language is output instead, followed by \PName{additional text}. If the
option \OptionValue{chapterprefix}{false} is set, then nothing is output.

The command \Macro{chapapp} always outputs the word ``Chapter'' or
``Appendix''.  In this case the selection of option \Option{chapterprefix} has
no effect.

Since chapters only exist in the classes \Class{scrbook} and
\Class{scrreprt} these commands only exist in these classes.%
%
\EndIndex{Cmd}{chapappifchapterprefix}%
\EndIndex{Cmd}{chapapp}%


\begin{Declaration}
  \Macro{chaptermark}\Parameter{running head}\\
  \Macro{sectionmark}\Parameter{running head}\\
  \Macro{subsectionmark}\Parameter{running head}\\
  \Macro{chaptermarkformat}\\
  \Macro{sectionmarkformat}\\
  \Macro{subsectionmarkformat}
\end{Declaration}%
\BeginIndex{Cmd}{chaptermark}%
\BeginIndex{Cmd}{sectionmark}%
\BeginIndex{Cmd}{subsectionmark}%
\BeginIndex{Cmd}{chaptermarkformat}%
\BeginIndex{Cmd}{sectionmarkformat}%
\BeginIndex{Cmd}{subsectionmarkformat}%
\begin{Explain}%
  As mentioned in \autoref{sec:maincls.pagestyle} the page style
  \PValue{headings} works with automatic running
  heads\Index{running head}. For this, the commands
  \Macro{chaptermark} and \Macro{sectionmark}, or \Macro{sectionmark}
  and \Macro{subsectionmark}, respectively, are defined. Every
  structuring command (\Macro{chapter}, \Macro{section} \dots)
  automatically carries out the respective \Macro{\dots mark} command.
  The parameter passed contains the text of the section
  heading\Index{heading}. The respective section number is added
  automatically in the \Macro{\dots mark} command. The formatting is
  done according to the section level with one of the three commands
  \Macro{chaptermarkformat}, \Macro{sectionmarkformat} or
  \Macro{subsectionmarkformat}.

  \OnlyAt{\Class{scrbook}\and\Class{scrreprt}}Of course there is no
  command \Macro{chaptermark} or \Macro{chaptermarkformat} in
  \Class{scrartcl}. \OnlyAt{\Class{scrartcl}}Accordingly,
  \Macro{subsectionmark} and \Macro{subsectionmarkformat} exist only
  in \Class{scrartcl}. This changes when you use the
  \Package{scrpage2} package (see \autoref{cha:scrpage}).\par
\end{Explain}
Similar to \Macro{chapterformat} and \Macro{othersectionlevelsformat},
the commands \Macro{chaptermarkformat} (not in \Class{scrartcl}),
\Macro{sectionmarkformat} and \Macro{subsectionmarkformat} (only in
\Class{scrartcl}) define the formatting of the sectioning numbers in
the automatic running heads. They can be adapted to your personal
needs with \Macro{renewcommand}. The original definitions for the
{\KOMAScript} classes are:
\begin{lstcode}
  \newcommand*{\chaptermarkformat}{%
    \chapappifchapterprefix{\ }\thechapter\autodot\enskip}
  \newcommand*{\sectionmarkformat}{\thesection\autodot\enskip}
  \newcommand*{\subsectionmarkformat}{%
    \thesubsection\autodot\enskip}
\end{lstcode}
\begin{Example}
  Suppose you want to prepend the word ``Chapter'' to the chapter
  number in the running heading. For example you could insert the
  following definition in the preamble of your document :
\begin{lstcode}
  \renewcommand*{\chaptermarkformat}{%
    \chapapp~\thechapter\autodot\enskip}
\end{lstcode}
\end{Example}
As you can see, both the commands \Macro{chapappifchapterprefix} and
\Macro{chapapp} explained above are used here.%
%
\EndIndex{Cmd}{chaptermark}%
\EndIndex{Cmd}{sectionmark}%
\EndIndex{Cmd}{subsectionmark}%
\EndIndex{Cmd}{chaptermarkformat}%
\EndIndex{Cmd}{sectionmarkformat}%
\EndIndex{Cmd}{subsectionmarkformat}%


\begin{Declaration}
  \Counter{secnumdepth}
\end{Declaration}%
\BeginIndex{Counter}{secnumdepth}\BeginIndex{}{numbering}%
Per default, in the classes \Class{scrbook}\IndexClass{scrbook} and
\Class{scrreprt}\IndexClass{scrreprt} the section levels from
\Macro{part}\IndexCmd{part}\IndexCmd{chapter}\IndexCmd{section} down
to \Macro{subsection}\IndexCmd{subsection} and in the class
\Class{scrartcl}\IndexClass{scrartcl} the levels from \Macro{part}
down to \Macro{subsubsection}\IndexCmd{subsubsection} are numbered.
This is controlled by the {\LaTeX} counter \Counter{secnumdepth}. The
value \(-\)1 represents \Macro{part}, 0 the level \Macro{chapter},
and so on. By defining, incrementing or decrementing this counter you
can determine down to which level the headings are numbered. The same
applies in the standard classes. Please refer also to the explanation
concerning the counter \Counter{tocdepth} in
\autoref{sec:maincls.toc},
\autopageref{desc:maincls.counter.tocdepth}.
%
\EndIndex{Counter}{secnumdepth}\EndIndex{}{numbering}


\begin{Declaration}
  \Macro{setpartpreamble}%
  \OParameter{position}\OParameter{width}\Parameter{preamble}\\
  \Macro{setchapterpreamble}%
  \OParameter{position}\OParameter{width}\Parameter{preamble}
\end{Declaration}%
\BeginIndex{Cmd}{setpartpreamble}%
\BeginIndex{Cmd}{setchapterpreamble}%
Parts\OnlyAt{\Class{scrbook}\and
  \Class{scrreprt}}\Index{part>preamble}\Index{chapter>preamble} and chapters
in {\KOMAScript} can be started with a \PName{preamble}. This is particularly
useful when you are using a two column layout with the class option
\Option{twocolumn}\IndexOption{twocolumn}, since the heading together with the
\PName{preamble} is always set in a one column layout.  The \PName{preamble}
can comprise more than one paragraph. The command to output the
\PName{preamble} has to be placed before the respective \Macro{part},
\Macro{addpart}, \Macro{chapter} or \Macro{addchap} command.
\begin{Example}
  You are writing a report about the condition of a company. You
  organize the report in such a way that every department gets its own
  partial report.  Every one of these parts should be introduced by an
  abstract on the corresponding title page. You could write the
  following:
\begin{lstcode}
  \setpartpreamble{%
    \begin{abstract}
      This is a filler text. It serves merely to demonstrate the
      capabilities of {\KOMAScript}. If you read this text, you will
      get no information.
    \end{abstract}
  }
  \part{Department for Word Processing}
\end{lstcode}
  Depending on the settings for the heading font size\Index{heading}
  (see \autopageref{desc:maincls.option.headings}) and the options for
  the \Environment{abstract} environment\IndexEnv{abstract} (see
  \autoref{sec:maincls.abstract},
  \autopageref{desc:maincls.option.abstract}), the result would look
  similar to:
  \begin{ShowOutput}\centering
    {\LARGE\usekomafont{disposition} Part III.\par\vspace{20pt}}
    {\LARGE\usekomafont{disposition} Department for Word Processing\strut\par}
    \begin{quote}\small
      \vspace{4ex}
      \begin{center}
        \usekomafont{disposition}\abstractname
      \end{center}
      \vspace{2ex}       
      This is a filler text. It serves merely to demonstrate the
      capabilities of {\KOMAScript}. If you read this text, you will
      get no information.
    \end{quote}
  \end{ShowOutput}
\end{Example}
Please\textnote{Attention!} note that it is \emph{you} who is responsible for
the spaces between the heading, preamble and the following text. Please note
also that there is no \Environment{abstract} environment in the class
\Class{scrbook} (see \autoref{sec:maincls.abstract},
\autopageref{desc:maincls.env.abstract}).

The\ChangedAt{v2.8p}{\Class{scrbook}\and\Class{scrreprt}} first
optional argument \PName{position} determines the position at which
the preamble is placed with the help of one or two letters. For the
vertical placement there are two possibilities at present:
\begin{labeling}[~--]{\quad\PValue{o}}\itemsep=0pt
\item [\quad\texttt{o}] above the heading
\item [\quad\texttt{u}] below the heading
\end{labeling}
You can insert one preamble above and another below a heading. For the
horizontal placement you have the choice between three alignments:
\begin{labeling}[~--]{\quad\PValue{o}}\itemsep=0pt
\item [\quad\texttt{l}] left-aligned
\item [\quad\texttt{r}] right-aligned
\item [\quad\texttt{c}] centered
\end{labeling}
However, this does not output the text of the \PName{preamble} in such
a manner, but inserts a box whose width is determined by the second
optional argument \PName{width}. If you leave out this second argument
the whole text width is used. In that case the option for horizontal
positioning will have no effect. You can combine exactly one letter
from the vertical with one letter from the horizontal positioning.

A more often usage of \Macro{setchapterpreamble} would be something like a
smart slogan or dictum to a heading. The command
\Macro{dictum}\IndexCmd{dictum}, that may be used for this, will be described
at the next section. You will also find an example there.

Please note\textnote{Attention!} that a above the chapter headings placed
\PName{preamble} will be set into the already existing vertical space above
the heading. The heading will not be moved down. It is you, who is responsible
that the preamble is small enough and the space is sufficient. See also
\Macro{chapterheadstartvskip} in \autoref{sec:maincls-experts.experts},
\autopageref{desc:maincls-experts.cmd.chapterheadstartvskip} for this.%
%
\EndIndex{Cmd}{setpartpreamble}%
\EndIndex{Cmd}{setchapterpreamble}%


\section{Dicta\protect\footnote{This section is still missing. Translators
    from German to English would be welcome!}}
\label{sec:maincls.dictum}

\mbox{}

\section{Lists\protect\footnote{This section is still missing. Translators
    from German to English would be welcome!}}
\label{sec:maincls.lists}

\mbox{}

\section{Math\protect\footnote{This section is still missing. Translators
    from German to English would be welcome!}}
\label{sec:maincls.math}

\mbox{}

\section{Floating Environments of Tables and Figures\protect\footnote{This
    section is still missing. Translators from German to English would be
    welcome!}}
\label{sec:maincls.floats}

\mbox{}

\section{Marginal Notes\protect\footnote{This section is still
    missing. Translators from German to English would be welcome!}}
\label{sec:maincls.marginNotes}

\mbox{}

\section{Appendix\protect\footnote{This section is still missing. Translators
    from German to English would be welcome!}}
\label{sec:maincls.appendix}

\mbox{}

\section{Bibliography\protect\footnote{This section is still
    missing. Translators from German to English would be welcome!}}
\label{sec:maincls.bibliography}

\mbox{}

\section{Index\protect\footnote{This section is still missing. Translators
    from German to English would be welcome!}}
\label{sec:maincls.index}

\mbox{}

\section{Other Options\protect\footnote{This section is deprecated
    and should be replaced by new sections using the structure of the German
    guide. Translators from English to German would be welcome! You may find
    additional information about obsolete or deprecated options at
    \autoref{cha:maincls-experts}.}}
%\label{sec:maincls.options}

This section describes the global options of the three main classes.
The majority of the options can also be found in the standard classes.
Since experience shows that many options of the standard classes are
unknown, their description is included here. This is a departure from
the rule that the \File{scrguide} should only describe those aspects
whose implementation differs from the standard one.

Table~\ref{tab:maincls.stdOptions} lists those options that are set by
default in at least one of the {\KOMAScript} classes. The table shows
for each {\KOMAScript} main class if the option is set by default and
if it is even defined for that class. An undefined option cannot be
set, either by default or by the user.

\begin{table}[htbp]
  \centering
  \caption{Default options of the {\KOMAScript} classes}
  \begin{tabular}{llll}
    Option                 &
      \Class{scrbook} & \Class{scrreprt} & \Class{scrartcl} \\
    \hline\rule{0pt}{2.7ex}%
    \KOption{abstract}   &
      \emph{undefined} & \PValue{false} & \PValue{false} \\
    \KOption{captions} &
      \PValue{tablesignature} & \PValue{tablesignature} & \PValue{tablesignature} \\
    \KOption{chapteratlists}   &
      \PValue10{pt} & \PValue{10pt} & \emph{undefined} \\
    \KOption{chapterprefix} &
      \PValue{false} & \PValue{false} & \emph{undefined} \\
    \KOption{draft}         &
      \PValue{false} & \PValue{false} & \PValue{false} \\
    \KOption{fontsize}          &
      \PValue{11pt} & \PValue{11pt} & \PValue{11pt} \\
    \KOption{footsepline} &
      \PValue{false} & \PValue{false} & \PValue{false} \\
    \KOption{headings}   &
      \PValue{big} & \PValue{big} & \PValue{big} \\
    \KOption{headsepline} &
      \PValue{false} & \PValue{false} & \PValue{false} \\
    \KOption{listof} &
      \PValue{graduated} & \PValue{graduated} & \PValue{graduated} \\
    \KOption{open}       &
      \PValue{right}   & \PValue{any} & \emph{undefined} \\
    \KOption{paper}       &
      \PValue{a4} & \PValue{a4} & \PValue{a4} \\
    \Option{parindent}     &
      default & default & default \\
    \KOption{titlepage}     &
      \PValue{true} & \PValue{true} & \PValue{false}              \\
    \KOption{toc} &
      \PValue{graduated} & \PValue{graduated} & \PValue{graduated} \\
    \KOption{twocolumn}     &
      \PName{false} & \PName{false} & \PName{false} \\
    \KOption{twoside}       &
      \PValue{true} & \PValue{false}  & \PValue{false} \\
    \KOption{version}       &
      \PValue{first} & \PValue{first} & \PValue{first} \\
  \end{tabular}
  \label{tab:maincls.stdOptions}
\end{table}

\begin{Explain}
  Allow me an observation before proceeding with the descriptions of
  the options. It is often the case that at the beginning of a
  document one is often unsure which options to choose for that
  specific document. Some options, for instance the choice of paper
  size, may be fixed from the beginning. But already the question of
  which \Var{DIV} value to use could be difficult to answer initially.
  On the other hand, this kind of information should be initially
  irrelevant for the main tasks of an author: design of the document
  structure, text writing, preparation of figures, tables and index.
  As an author you should concentrate initially on the contents. When
  that is done, you can concentrate on the fine points of
  presentation. Besides the choice of options, this means correcting
  things like hyphenation, page breaks, and the distribution of tables
  and figures. As an example consider
  \autoref{tab:maincls.stdOptions}, which I moved repeatedly between
  the beginning and the end of this section. The choice of the actual
  position will only be made during the final production of the
  document.
\end{Explain}

\subsection{Options for Page Layout}
\label{sec:maincls.typeareaOptions}

With the standard classes the page layout\Index{page>layout} is
established by the option files \File{size10.clo}, \File{size11.clo},
\File{size12.clo} (or \File{bk10.clo}, \File{bk11.clo},
\File{bk12.clo} for the book class) and by fixed values in the class
definitions. The {\KOMAScript} classes, however, do not use a fixed
page layout, but one that depends on the paper format\Index{paper
  format} and font size. For this task all three main classes use the
\Package{typearea} package\IndexPackage{typearea} (see
\autoref{cha:typearea}).  The package is automatically loaded by the
{\KOMAScript} main classes.  Therefore it is not necessary to load the
package using \Macro{usepackage}\PParameter{typearea}. If a {\LaTeX}
run results in an error ``\texttt{Option clash for package
  typearea}'', then this is most likely owing to the use of an
explicit command \Macro{usepackage}\OParameter{package
  options}\PParameter{typearea}.


\begin{Declaration}
  \Option{letterpaper} \\
  \Option{legalpaper} \\
  \Option{executivepaper} \\
  \Option{a\Var{X}paper} \\
  \Option{b\Var{X}paper} \\
  \Option{c\Var{X}paper} \\
  \Option{d\Var{X}paper} \\
  \Option{landscape}
\end{Declaration}%
\BeginIndex{Option}{letterpaper}%
\BeginIndex{Option}{legalpaper}%
\BeginIndex{Option}{executivepaper}%
\BeginIndex{Option}{a0paper}%
\BeginIndex{Option}{b0paper}%
\BeginIndex{Option}{c0paper}%
\BeginIndex{Option}{d0paper}%
\BeginIndex{Option}{landscape}%
The basic options for the choice of paper format are not processed
directly by the classes. Instead, they are automatically processed by
the \Package{typearea} package as global options (see
\autoref{sec:typearea.options},
\autopageref{desc:typearea.option.letterpaper}). The options
\Option{a5paper}, \Option{a4paper}, \Option{letterpaper},
\Option{legalpaper} and \Option{executivepaper} correspond to the
likewise-named options of the standard classes and define the same
paper format.  The page layout calculated for each is different,
however.
%
\EndIndex{Option}{letterpaper}%
\EndIndex{Option}{legalpaper}%
\EndIndex{Option}{executivepaper}%
\EndIndex{Option}{a0paper}%
\EndIndex{Option}{b0paper}%
\EndIndex{Option}{c0paper}%
\EndIndex{Option}{d0paper}%
\EndIndex{Option}{landscape}

\begin{Explain}
  The reason that the options for the A, B, C or D format are not
  processed by the \Package{typearea} is not because they are global
  options, but because the {\KOMAScript} classes explicitly pass them
  to the \Package{typearea} package. This is caused by the way option
  processing is implemented in the \Package{typearea} package and by
  the operation of the underlying option passing and processing
  mechanism of {\LaTeX}.

  This is also valid for the options, described subsequently, that set
  the binding correction, the divisor and the number of header lines.
\end{Explain}


\subsection{Options for Document Layout}
\label{sec:maincls.layoutOptions}

This subsection deals with all the options that affect the document
layout in general and not only the page layout. Strictly speaking, of
course, all page layout options (see
\autoref{sec:maincls.typeareaOptions}) are also document layout
options. The reverse is also partially true.


\begin{Declaration}
  \OptionValue{captions}{oneline}\\
  \OptionValue{captions}{nooneline}
\end{Declaration}
\BeginIndex{Option}{captions~=\PValue{oneline}}%
\BeginIndex{Option}{captions~=\PValue{nooneline}}%
The standard classes differentiate between one-line and multi-line
table or figure captions. One-line captions are centered while
multi-line captions are left-justified. This behavior, which is also
the default with \KOMAScript, corresponds to the option
\OptionValue{captions}{oneline}. There is no special handling of one-line
captions when the \OptionValue{captions}{nooneline} option is given.

\begin{Explain}
  The avoidance of a special treatment for the caption has an
  additional effect that is sometimes highly desirable. Footnotes that
  appear inside a \Macro{caption} command often have a wrong number
  assigned to them. This happens because the footnote counter is
  incremented once as soon as the text is measured to determine if it
  will be one line or more. When the \OptionValue{captions}{nooneline} option
  is used no such measurement is made. The footnote numbers are
  therefore correct.

  But since {\KOMAScript} version~2.9 you don't need the option
  \OptionValue{captions}{nooneline} to avoid the above described
  effect. {\KOMAScript} classes contain a workaround, so you can have
  footnotes inside captions. It should be mentioned though that when
  using footnotes inside floating environments, the contents of the
  floating environment should be encapsulated inside a minipage. That
  way it is guaranteed that floating environment and footnote are
  inseparable.
\end{Explain}
%
\EndIndex{Option}{captions~=\PValue{oneline}}%
\EndIndex{Option}{captions~=\PValue{nooneline}}%


\subsection{Options for Lists of Floats}
\label{sec:maincls.listsOptions}

The best known lists of floats are the list of figures and the list of
tables.  Additionally, with help from the
\Package{float}\IndexPackage{float} package, for instance, it is
possible to produce new float environments with corresponding lists.

\begin{Explain}
  Whether {\KOMAScript} options have any effect on lists of floats
  produced by other packages depends mainly on those packages. This is
  generally the case with the lists of floats produced by the
  \Package{float}\IndexPackage{float} package.

  Besides the options described here, there are others that affect the
  lists of floats though not their formatting or contents. Instead
  they affect what is included in the table of contents. The
  corresponding descriptions can therefore be found in
  \autoref{sec:maincls.tocOptions}.
\end{Explain}


\subsection{Options Affecting the Formatting}
\label{sec:maincls.formattingOptions}

Formatting options are all those options that affect the form or
formatting of the document and cannot be assigned to other sections.
They are therefore the \emph{remaining options}.

\begin{Declaration}
  \Option{leqno}
\end{Declaration}%
\BeginIndex{Option}{leqno}%
Equations\Index{equation} are normally numbered on the right.  The
standard option \Option{leqno} causes the standard option file
\File{leqno.clo} to be loaded. The equations are then numbered on the
left.
%
\EndIndex{Option}{leqno}%


\begin{Declaration}
  \Option{fleqn}
\end{Declaration}%
\BeginIndex{Option}{fleqn}%
Displayed equations are normally centered. The standard option
\Option{fleqn} causes the standard option file \File{fleqn.clo} to be
loaded. Displayed equations are then left-justified. This option may not be
used at the argument of \Macro{KOMAoptions} but at the optional argument of
\Macro{documentclass}.
%
\EndIndex{Option}{fleqn}%

\begin{Declaration}
  \OptionValue{captions}{tablesignature}\\
  \OptionValue{captions}{tableheading}
\end{Declaration}%
\BeginIndex{Option}{captions~=\PValue{tablesignature}}%
\BeginIndex{Option}{captions~=\PValue{tableheading}}%
As described in \autoref{sec:maincls.floats},
\autopageref{desc:maincls.cmd.caption}, the
\Macro{caption}\IndexCmd{caption} command acts with figures like the
\Macro{captionbelow}\IndexCmd{captionbelow} command. The behaviour
with tables, however, depends on these two options.  In the default
setting, \OptionValue{captions}{tablesignature}, the \Macro{caption} macro acts
also with tables like the \Macro{captionbelow} command. With the
\OptionValue{captions}{tableheading} option, \Macro{caption} acts like the
\Macro{captionabove}\IndexCmd{captionabove} command.

Note that using any of these options does not change the position of
the caption from above the top of the table to below the bottom of the
table or vica versa. It only affects whether the text is formatted as
a caption for use above or below a table. Whether the text is in fact
placed above or below a table is set through the position of the
\Macro{caption} command inside the \Environment{table} environment.

Note\OnlyAt{\Package{float}}\IndexPackage{float} that when using the
\Package{float} package, the options \OptionValue{captions}{tablesignature} and
\OptionValue{captions}{tableheading} cease to act correctly when
\Macro{restylefloat} is applied to tables. More details of the
\Package{float} package and \Macro{restylefloat} can be found in
\cite{package:float}. Additional support in {\KOMAScript} for the
\Package{float} package may be found at the explanation of
\PValue{komaabove} in \autoref{sec:maincls.floats},
\autopageref{desc:maincls.floatstyle.komaabove}.
%
\EndIndex{Option}{captions~=\PValue{tablesignature}}%
\EndIndex{Option}{captions~=\PValue{tableheafins}}%


\begin{Explain}%
\begin{Declaration}
  \Option{origlongtable}
\end{Declaration}%
\BeginIndex{Option}{origlongtable}%
The package\OnlyAt{\Package{longtable}}
\Package{longtable}\IndexPackage{longtable} (see
\cite{package:longtable}) sets table captions internally by calling
the command \Macro{LT@makecaption}.  In order to ensure that these
table captions match the ones used with normal tables, the
{\KOMAScript} classes normally redefine that command. See
\autoref{sec:maincls.floats},
\autopageref{desc:maincls.cmd.caption.longtable} for more
details. The redefinition is performed with help of the command
\Macro{AfterPackage} immediately after the loading of package
\Package{longtable}.  If the package
\Package{caption2}\IndexPackage{caption2} (see \cite{package:caption})
has been previously loaded, the redefinition is not made in order not
to interfere with the \Package{caption2} package.
\end{Explain}
If the table captions produced by the \Package{longtable} package
should not be redefined by the {\KOMAScript} classes, activate the
\Option{origlongtable} option.
%
\EndIndex{Option}{origlongtable}%


\begin{Declaration}
  \Option{openbib}\\
  \OptionValue{bibliography}{openstyle}\\
  \OptionValue{bibliography}{oldstyle}
\end{Declaration}%
\BeginIndex{Option}{openbib}%
The standard option \Option{openbib} switches to an alternative
bibliography format. The effects are twofold: The first line of a
bibliography entry, normally containing the author's name, receives a
smaller indentation; and the command \Macro{newblock} is redefined to
produce a paragraph. Without this option, \Macro{newblock} introduces
only a stretchable horizontal space.
%
\EndIndex{Option}{openbib}


% %%%%%%%%%%%%%%%%%%%%%%%%%%%%%%%%%%%%%%%%%%%%%%%%%%%%%%%%%%%%%%%%%%%%%%

\section{Titles\protect\footnote{This section is deprecated
    and should be replaced by new sections using the structure of the German
    guide. Translators from English to German would be welcome! You may find
    additional information about obsolete or deprecated options at
    \autoref{cha:maincls-experts}.}}
\label{sec:maincls.titles}

\begin{Explain}
  After having described the options and some general issues, we begin
  the document where it usually begins: with the titles. The titles
  comprise everything that belongs in the widest sense to the title of
  a document. Like already mentioned in
  \autoref{sec:maincls.layoutOptions},
  \autopageref{desc:maincls.option.titlepage}, we can distinguish
  between title pages and \emph{in-page} titles. Article classes like
  \Class{article} or \Class{scrartcl} have by default \emph{in-page}
  titles, while classes like \Class{report}, \Class{book},
  \Class{scrreprt} and \Class{scrbook} have title pages as default.
  The defaults can be changed with the class option
  \Option{titlepage}.
\end{Explain}


% %%%%%%%%%%%%%%%%%%%%%%%%%%%%%%%%%%%%%%%%%%%%%%%%%%%%%%%%%%%%%%%%%%%%%%


\section{Lists of Floats\protect\footnote{This section is deprecated
    and should be replaced by new sections using the structure of the German
    guide. Translators from English to German would be welcome! You may find
    additional information about obsolete or deprecated options at
    \autoref{cha:maincls-experts}.}}

As a rule, the lists of floats\Index{floating environments}, e.\,g., list of
tables\Index[indexmain]{list>of tables} and list of
figures\Index[indexmain]{list>of figures}, can be found directly after the
table of contents.  In some documents, they can even be found in the appendix.
However, the author of this manual prefers their location after the table of
contents, therefore the explanation is given here.

\begin{Declaration}
  \Macro{listoftables}\\
  \Macro{listoffigures}\\
  \Macro{listtablename}\\
  \Macro{listfigurename}
\end{Declaration}
\BeginIndex{Cmd}{listoftables}%
\BeginIndex{Cmd}{listoffigures}%
\BeginIndex{Cmd}{listtablename}%
\BeginIndex{Cmd}{listfigurename}%
These commands generate a list of tables or figures.  Changes in the document
that modify these lists will require two {\LaTeX} runs in order to take
effect.  The layout of the lists can be influenced by the options
\OptionValue{listof}{graduated} and \OptionValue{listof}{flat} (see
\autoref{sec:maincls.listsOptions},
\autopageref{desc:maincls.option.listof}).  Moreover, the options
\OptionValue{listof}{totoc} and \OptionValue{listof}{numbered} have indirect
influence (see \autoref{sec:maincls.tocOptions},
\autopageref{desc:maincls.option.listof.totoc}).

\begin{Explain}
  The text of the titles of this tables are stored in the macros
  \Macro{listtablename} and \Macro{listfigurename}. If you use a
  language package like \Package{babel} and want to redefine these
  macros, you should read the documentation of the language package.
\end{Explain}
%
\EndIndex{Cmd}{listoftables}%
\EndIndex{Cmd}{listoffigures}%
\EndIndex{Cmd}{listtablename}%
\EndIndex{Cmd}{listfigurename}%


\section{Main Text\protect\footnote{This section is deprecated
    and should be replaced by new sections using the structure of the German
    guide. Translators from English to German would be welcome! You may find
    additional information about obsolete or deprecated options at
    \autoref{cha:maincls-experts}.}}
\label{sec:maincls.mainText}

This section explains everything provided by {\KOMAScript}
in order to write the main text. The main text is the
part that the author should focus on first.
Of course this includes tables, figures and comparable
information as well.


\subsection{Structuring the Document}\Index[indexmain]{structuring}
\label{sec:maincls.structure.obsolete}

There are several commands to structure a document into parts, chapters,
sections and so on.


\begin{Declaration}
  \Macro{dictum}\OParameter{author}\Parameter{dictum}\\
  \Macro{dictumwidth}\\
  \Macro{dictumauthorformat}\Parameter{author}\\
  \Macro{raggeddictum}\\
  \Macro{raggeddictumtext}\\
  \Macro{raggeddictumauthor}
\end{Declaration}%
\BeginIndex{Cmd}{dictum}%
\BeginIndex{Cmd}{dictumwidth}%
\BeginIndex{Cmd}{dictumauthorformat}%
\BeginIndex{Cmd}{raggeddictum}%
\BeginIndex{Cmd}{raggeddictumtext}%
\BeginIndex{Cmd}{raggeddictumauthor}%
Apart\OnlyAt{\Class{scrbook}\and\Class{scrreprt}}%
\ChangedAt{v2.8q}{\Class{scrbook}\and\Class{scrreprt}} from an
introductory paragraph you can use \Macro{setpartpreamble} or
\Macro{setchapterpreamble} for a kind of
\PName{aphorism}\Index{aphorism} (also known as ``dictum'') at the
beginning of a chapter or section. The command \Macro{dictum} inserts
such an aphorism. This macro can be used as obligatory argument of
either the command \Macro{setchapterpreamble} or
\Macro{setpartpreamble}. However, this is not obligatory.

The dictum together with an optional \PName{author} is inserted in a
\Macro{parbox}\IndexCmd{parbox} (see \cite{latex:usrguide}) of width
\Macro{dictumwidth}. Yet \Macro{dictumwidth} is not a length which can
be set with \Macro{setlength}. It is a macro that can be redefined
using \Macro{renewcommand}. Default setting is
\verb;0.3333\textwidth;, which is a third of the textwidth. The box
itself is positioned with the command \Macro{raggeddictum}. Default
here is \Macro{raggedleft}\IndexCmd{raggedleft}, that is, right
justified.  The command \Macro{raggeddictum} can be redefined using
\Macro{renewcommand}.

Within the box the \PName{dictum} is set using
\Macro{raggeddictumtext}.  Default setting is
\Macro{raggedright}\IndexCmd{raggedright}, that is, left
justified. Similarly to \Macro{raggeddictum} this can be redefined
with \Macro{renewcommand}.  The output uses the default font setting
for the element \FontElement{dictumtext}, which can be changed with
the commands from \autoref{sec:maincls.font}. Default settings are
listed in \autoref{tab:maincls.dictumfont}.

If there is an \PName{author} name, it is separated from the
\PName{dictum} by a line the full width of the \Macro{parbox}. This is
defined by the macro \Macro{raggeddictumauthor}. Default is
\Macro{raggedleft}. This command can also be redefined using
\Macro{renewcommand}. The format of the output is defined with
\Macro{dictumauthorformat}. This macro expects the \Macro{author} as
argument. As default \Macro{dictumauthorformat} is defined as:
\begin{lstcode}
  \newcommand*{\dictumauthorformat}[1]{(#1)}
\end{lstcode}
Thus the \PName{author} is set enclosed in rounded parenthesis. For
the element \FontElement{dictumauthor} a different font than for the
element \FontElement{dictumtext} can be defined. Default settings are
listed in \autoref{tab:maincls.dictumfont}. Changes can be made using
the commands from \autoref{sec:maincls.font}.
%
\begin{table}
  \centering%
  \caption{Default settings for the elements of a dictum}
  \begin{tabular}{ll}
    \toprule
    Element & Default \\
    \midrule
    \FontElement{dictumtext} &
    \Macro{normalfont}\Macro{normalcolor}\Macro{sffamily}\Macro{small}\\
    \FontElement{dictumauthor} &
    \Macro{itshape}\\
    \bottomrule
  \end{tabular}
  \label{tab:maincls.dictumfont}
\end{table}
%
If \Macro{dictum} is used within the macro \Macro{setchapterpreamble}
or \Macro{setpartpreamble} you have to take care of the following: the
horizontal positioning is always done with \Macro{raggeddictum}.
Therefore, the optional argument for horizontal positioning which is
implemented for these two commands has no effect. \Macro{textwidth} is
not the width of the whole text corpus but the actually used text
width.  If \Macro{dictumwidth} is set to \verb;.5\textwidth; and
\Macro{setchapterpreamble} has an optional width of
\verb;.5\textwidth; too, you will get a box with a width one quarter
of the text width.  Therefore, if you use \Macro{dictum} it is
recommended to refrain from setting the optional width for
\Macro{setchapterpreamble} or \Macro{setpartpreamble}.

If you have more than one dictum one under another, you should
separate them by an additional vertical space, easily accomplished
using the command \Macro{bigskip}\IndexCmd{bigskip}.

\begin{Example}
  You are writing a chapter on an aspect of weather forecasting. You
  have come across an aphorism which you would like to place at the
  beginning of the chapter beneath the heading. You could write:
\begin{lstcode}
  \setchapterpreamble[u]{%
    \dictum[Anonymous]{Forecasting is the art of saying
            what is going to happen and then to explain
            why it didn't.}}
  \chapter{Weather forecasting}
\end{lstcode}
  The output would look as follows:
  \begin{ShowOutput}
    {\usekomafont{disposition}\usekomafont{chapter}%
      17\enskip Weather forecasting\par} \vspace{\baselineskip}
    \dictum[Anonymous]{Forecasting is the art of saying what is going to
    happen and then to explain why it didn't.}
  \end{ShowOutput}

  If you would rather the dictum span only a quarter of the text width
  rather than one third you can redefine \Macro{dictumwidth}:
\begin{lstcode}
  \renewcommand*{\dictumwidth}{.25\textwidth}
\end{lstcode}
\end{Example}

For a somewhat more sophisticated formatting of left- or right-aligned
paragraphs including hyphenation you can use the
package~\Package{ragged2e}~\cite{package:ragged2e}.
\EndIndex{Cmd}{dictum}%
\EndIndex{Cmd}{dictumwidth}%
\EndIndex{Cmd}{dictumauthorformat}%
\EndIndex{Cmd}{raggeddictum}%
\EndIndex{Cmd}{raggeddictumtext}%
\EndIndex{Cmd}{raggeddictumauthor}%


\subsection{Lists}
\label{sec:maincls.lists.obsolete} 
\BeginIndex[indexother]{}{lists}

Both {\LaTeX} and the standard classes offer different environments
for lists. Though slightly changed or extended all these list are of
course offered in {\KOMAScript} as well. In general all
lists\,---\,even of different kind\,---\,can be nested up to four
levels. From a typographical view, anything more would make no sense,
as more than three levels can no longer be easily perceived. The
recommended procedure in such a case is to split the large list into
several smaller ones.

\begin{Declaration}
  \Environment{itemize}\\
  \Macro{item}\\
  \Macro{labelitemi}\\
  \Macro{labelitemii}\\
  \Macro{labelitemiii}\\
  \Macro{labelitemiv}
\end{Declaration}%
\BeginIndex{Env}{itemize}%
\BeginIndex{Cmd}{item}%
\BeginIndex{Cmd}{labelitemi}%
\BeginIndex{Cmd}{labelitemii}%
\BeginIndex{Cmd}{labelitemiii}%
\BeginIndex{Cmd}{labelitemiv}%
The simplest form of a list is an \Environment{itemize} list. The
users of a certain disliked word processing package often refer to
this form of a list as \emph{bulletpoints}.  Presumably, these users
are unable to envisage that, depending on the level, a different
symbol from a large dot could be used to introduce each
point. Depending on the level, {\KOMAScript} uses the following marks:
``{\labelitemi}'', ``{\labelitemii}'', ``{\labelitemiii}'' and
``{\labelitemiv}''. The definition of these symbols is specified in
the macros \Macro{labelitemi}, \Macro{labelitemii},
\Macro{labelitemiii} and \Macro{labelitemiv}, all of which can be
redefined using \Macro{renewcommand}. Every item is introduced with
\Macro{item}.
\begin{Example}
  You have a simple list which is nested in several levels. You write
  for example:
\begin{lstcode}
  \minisec{Vehicles}
  \begin{itemize}
    \item aeroplanes
    \begin{itemize}
      \item biplane
      \item jets
      \item transport planes
      \begin{itemize}
        \item single-engined
        \begin{itemize}
          \item jet-driven
          \item propeller-driven
        \end{itemize}
        \item multi-engined
      \end{itemize}
      \item helicopters
    \end{itemize}
    \item automobiles
    \begin{itemize}
      \item racing cars
      \item private cars
      \item lorries
    \end{itemize}
    \item bicycles
  \end{itemize}
\end{lstcode}
  As output you get:
  \begin{ShowOutput}[\baselineskip]
  \minisec{Vehicles}
  \begin{itemize}
    \item aeroplanes
    \begin{itemize}
      \item biplanes
      \item jets
      \item transport planes
      \begin{itemize}
        \item single-engined
        \begin{itemize}
          \item jet-driven
          \item propeller-driven
        \end{itemize}
        \item multi-engined
      \end{itemize}
      \item helicopters
    \end{itemize}
%% wahrsheinlich Platzfiller in Deutscher Ausgabe?
%    \item motorcycles
%    \begin{itemize}
%      \item historically accurate
%      \item futuristic, not real
%    \end{itemize}
    \item automobiles
    \begin{itemize}
      \item racing cars
      \item private cars
      \item lorries
    \end{itemize}
    \item bicycles
  \end{itemize}
  \end{ShowOutput}
\end{Example}
%
\EndIndex{Env}{itemize}%
\EndIndex{Cmd}{item}%
\EndIndex{Cmd}{labelitemi}%
\EndIndex{Cmd}{labelitemii}%
\EndIndex{Cmd}{labelitemiii}%
\EndIndex{Cmd}{labelitemiv}%


\begin{Declaration}
  \Environment{enumerate}\\
  \XMacro{item}\\
  \Macro{theenumi}\\
  \Macro{theenumii}\\
  \Macro{theenumiii}\\
  \Macro{theenumiv}\\
  \Macro{labelenumi}\\
  \Macro{labelenumii}\\
  \Macro{labelenumiii}\\
  \Macro{labelenumiv}
\end{Declaration}%
\BeginIndex{Env}{enumerate}%
\BeginIndex{Cmd}{item}%
\BeginIndex{Cmd}{theenumi}%
\BeginIndex{Cmd}{theenumii}%
\BeginIndex{Cmd}{theenumiii}%
\BeginIndex{Cmd}{theenumiv}%
\BeginIndex{Cmd}{labelenumi}%
\BeginIndex{Cmd}{labelenumii}%
\BeginIndex{Cmd}{labelenumiii}%
\BeginIndex{Cmd}{labelenumiv}%
Another form of a list often used is a numbered list which is already
implemented by the {\LaTeX} kernel. Depending on the level, the
numbering\Index{numbering} uses the following characters: Arabic
numbers, small letters, small roman numerals and capital letters. The
kind of numbering is defined with the macros \Macro{theenumi} down to
\Macro{theenumiv}. The output format is determined by the macros
\Macro{labelenumi} to \Macro{labelenumiv}. While the small letter of
the second level is followed by a round parenthesis, the values of all
other levels are followed by a dot. Every item is introduced with
\Macro{item}.
\begin{Example}
  Replacing every occurrence of an \Environment{itemize} environment
  with an \Environment{enumerate} environment in the example above we
  get the following result:
  \begin{ShowOutput}[\baselineskip]
  \minisec{Vehicles}
  \begin{enumerate}
    \item aeroplanes
    \begin{enumerate}
      \item biplanes
      \item jets
      \item transport planes
      \begin{enumerate}
        \item single-engined
        \begin{enumerate}
          \item jet-driven\label{xmp:maincls.jets}
          \item propeller-driven
        \end{enumerate}
        \item multi-engined
      \end{enumerate}
      \item helicopters
    \end{enumerate}
%% wahrsheinlich Platzfiller in Deutscher Ausgabe?
%    \item motorcycles
%    \begin{enumerate}
%      \item historically accurate
%      \item futuristic, not real
%    \end{enumerate}
    \item automobiles
    \begin{enumerate}
      \item racing cars
      \item private cars
      \item lorries
    \end{enumerate}
    \item bicycles
  \end{enumerate}
  \end{ShowOutput}
  Using \Macro{label} within a list you can set labels which are
  referenced with \Macro{ref}. In the example above, a label was set
  after the jet-driven, single-engined transport planes with
  \Macro{label}\PParameter{xmp:jets}. The \Macro{ref} value is then
  \ref{xmp:maincls.jets}.
\end{Example}
%
\EndIndex{Env}{enumerate}%
\EndIndex{Cmd}{item}%
\EndIndex{Cmd}{theenumi}%
\EndIndex{Cmd}{theenumii}%
\EndIndex{Cmd}{theenumiii}%
\EndIndex{Cmd}{theenumiv}%
\EndIndex{Cmd}{labelenumi}%
\EndIndex{Cmd}{labelenumii}%
\EndIndex{Cmd}{labelenumiii}%
\EndIndex{Cmd}{labelenumiv}%


\begin{Declaration}
  \Environment{description}\\
  \XMacro{item}\OParameter{item}
\end{Declaration}%
\BeginIndex{Env}{description}%
\BeginIndex{Cmd}{item}%
A further list form is the description list. Its main use is the
description of several items. The item itself is an optional parameter
in \Macro{item}. The font\Index{font}\ChangedAt{v2.8p}{%
  \Class{scrbook}\and\Class{scrreprt}\and\Class{scrartcl}}%
which is responsible for emphasizing the item can be changed with the
commands for the element
\FontElement{descriptionlabel}\IndexFontElement{descriptionlabel} (see
\autoref{tab:maincls.elementsWithoutText},
\autopageref{tab:maincls.elementsWithoutText}) described in
\autoref{sec:maincls.font}. Default setting is
\Macro{sffamily}\Macro{bfseries}.
\begin{Example}
  Instead of items in sans serif and bold you want them printed in the
  standard font in bold. Using
\begin{lstcode}
  \setkomafont{descriptionlabel}{\normalfont\bfseries}
\end{lstcode}
  you redefine the font accordingly.

  An example for a description list is the output of the page styles
  listed in \autoref{sec:maincls.pageStyle}. The heavily
  abbreviated source code is:
\begin{lstcode}
  \begin{description}
  \item[empty] is the page style without any header or footer.
    \item[plain] is the page style without headings.
    \item[headings] is the page style with running headings.
    \item[myheadings] is the page style for manual headings.
  \end{description}
\end{lstcode}
  This abbreviated version gives:
  \begin{ShowOutput}
    \begin{description}
    \item[empty] is the page style without any header or footer.
    \item[plain] is the page style without headings.
    \item[headings] is the page style with running headings.
    \item[myheadings] is the page style for manual headings.
    \end{description}
  \end{ShowOutput}
\end{Example}
%
\EndIndex{Env}{description}%
\EndIndex{Cmd}{item}%


\begin{Declaration}
  \Environment{labeling}\OParameter{delimiter}\Parameter{widest pattern}\\
  \XMacro{item}\OParameter{keyword}
\end{Declaration}%
\BeginIndex{Env}{labeling}%
\BeginIndex{Cmd}{item}%
An additional form of a description list is only available in the
{\KOMAScript} classes: the \Environment{labeling} environment. Unlike the
\Environment{description} environment, you can provide a pattern whose length
determines the indentation of all items. Furthermore, you can put an optional
\PName{delimiter} between the item and its description.  The
font\Index{font}\ChangedAt{v3.01}{%
  \Class{scrbook}\and \Class{scrreprt}\and \Class{scrartcl}\and
  \Package{scrextend}}%
which is responsible for emphasizing the item and the separator can be changed
with the commands for the element
\FontElement{labelinglabel}\IndexFontElement{labelinglabel} and
\FontElement{labelingseparator}\IndexFontElement{labelingseparator} (see
\autoref{tab:maincls.elementsWithoutText},
\autopageref{tab:maincls.elementsWithoutText}) described in
\autoref{sec:maincls.font}.
\begin{Example}
  Slightly changing the example from the \Environment{description}
  environment, we could write:
\begin{lstcode}
  \setkomafont{labelinglabel}{\ttfamily}
  \setkomafont{labelingseparator}{\normalfont}
  \begin{labeling}[~--]{myheadings}
    \item[empty]
      Page style without header and footer
    \item[plain]
      Page style for chapter beginnings without headings
    \item[headings]
      Page style for running headings
    \item[myheadings]
      Page style for manual headings
  \end{labeling}
\end{lstcode}
  As result we get:
  \begin{ShowOutput}
    \setkomafont{labelinglabel}{\ttfamily}
    \setkomafont{labelingseparator}{\normalfont}
    \begin{labeling}[~--]{myheadings}
    \item[empty]
      Page style without header and footer
    \item[plain]
      Page style for chapter beginnings without headings
    \item[headings]
      Page style for running headings
    \item[myheadings]
      Page style for manual headings
    \end{labeling}
  \end{ShowOutput}
  As can be seen in this example, a font changing command may be set in the
  usual way. But if you don't want the font of the separator be changed in the
  same way like the font of the label, you have to set the font of the
  separator different.
\end{Example}
Originally this environment was implemented for things like
``Precondition, Assertion, Proof'', or ``Given, Required, Solution''
that are often used in lecture hand-outs.  By now this environment has
found many different applications. For example, the environment for
examples in this guide was defined with the \Environment{labeling}
environment.
%
\EndIndex{Env}{labeling}%
\EndIndex{Cmd}{item}%


\begin{Declaration}
  \Environment{verse}
\end{Declaration}%
\BeginIndex{Env}{verse}%
Usually the \Environment{verse} environment is not perceived as a list
environment because you do not work with \Macro{item}
commands. Instead, fixed line breaks are used within the
\Environment{flushleft} environment. Yet internally in both the
standard classes as well as {\KOMAScript} it is indeed a list
environment.

In general the \Environment{verse} environment is used for
poems\Index{poems}.  Lines are indented both left and
right. Individual lines of verse are ended by a fixed line break
\verb|\\|. Verses are set as paragraphs, separated by an empty
line. Often also \Macro{medskip}\IndexCmd{medskip} or
\Macro{bigskip}\IndexCmd{bigskip} is used instead. To avoid a page
break at the end of a line of verse you as usual insert \verb|\\*|
instead of \verb|\\|.
\begin{Example}
  As an example, the first lines of ``Little Red Riding Hood and the
  Wolf'' by Roald Dahl:
\begin{lstcode}
  \begin{verse}
    As soon as Wolf began to feel\\*
    that he would like a decent meal,\\*
    He went and knocked on Grandma's door.\\*
    When Grandma opened it, she saw\\*
    The sharp white teeth, the horrid grin,\\*
    And Wolfie said, 'May I come in?'
  \end{verse}
\end{lstcode}
  The result is as follows:
  \begin{ShowOutput}
  \begin{verse}
    As soon as Wolf began to feel\\*
    That he would like a decent meal,\\*
    He went and knocked on Grandma's door.\\*
    When Grandma opened it, she saw\\*
    The sharp white teeth, the horrid grin,\\*
    And Wolfie said, 'May I come in?'
  \end{verse}
  \end{ShowOutput}
  However, if you have very long lines of verse, for instance:
\begin{lstcode}
  \begin{verse}
    Both the philosopher and the house-owner
    have always something to repair.\\
    \bigskip
    Don't trust a man, my son, who tells you
    that he has never lied.
  \end{verse}
\end{lstcode}
where a line break occurs within a line of verse:
\begin{ShowOutput}
  \begin{verse}
    Both the philosopher and the house-owner
    have always something to repair.\\
    \bigskip
    Don't trust a man, my son, who tells you
    that he has never lied.
  \end{verse}
\end{ShowOutput}
there \verb|\\*| can not prevent a page break occurring within a verse
at such a line break. To prevent such a page break, a
\Macro{nopagebreak}\IndexCmd{nopagebreak} would have to be inserted
somewhere in the first line:
\begin{lstcode}
  \begin{verse}
    Both the philosopher and the house-owner\nopagebreak
    have always something to repair.\\
    \bigskip
    Don't trust a man, my son, who tells you\nopagebreak
    that he has never lied.
  \end{verse}
\end{lstcode}

In the above example, \Macro{bigskip} was used to separate the lines
of verse.
\end{Example}
%
\EndIndex{Env}{verse}%


\begin{Declaration}
  \Environment{quote}\\
  \Environment{quotation}
\end{Declaration}%
\BeginIndex{Env}{quote}%
\BeginIndex{Env}{quotation}%
These two environments are also list environments and can be found
both in the standard and the {\KOMAScript} classes. Both environments
use justified text which is indented both on the left and right side.
Usually they are used to separate long citations\Index{citations} from
the main text. The difference between these two lies in the manner in
which paragraphs are typeset. While \Environment{quote} paragraphs are
highlighted by vertical space, in \Environment{quotation} paragraphs
the first line is indented. This is also true for the first line of a
\Environment{quotation} environment. To prevent indentation you have
to insert a \Macro{noindent} command\IndexCmd{noindent} before the
text.
\begin{Example}
  You want to highlight a short anecdote. You write the following
  \Environment{quotation} environment for this:
  %
\begin{lstcode}
  A small example for a short anecdote:
  \begin{quotation}
    The old year was turning brown; the West Wind was
    calling;
        
    Tom caught the beechen leaf in the forest falling.
    ``I've caught the happy day blown me by the breezes!
    Why wait till morrow-year? I'll take it when me pleases.
    This I'll mend my boat and journey as it chances
    west down the withy-stream, following my fancies!''
    
    Little Bird sat on twig. ``Whillo, Tom! I heed you.
    I've a guess, I've a guess where your fancies lead you.
    Shall I go, shall I go, bring him word to meet you?''
  \end{quotation}
\end{lstcode}
  The result is:
  \begin{ShowOutput}
    A small example for a short anecdote:
    \begin{quotation}
    The old year was turning brown; the West Wind was
    calling;
    
    Tom caught the beechen leaf in the forest falling.
    ``I've caught the happy day blown me by the breezes!
    Why wait till morrow-year? I'll take it when me pleases.
    This I'll mend my boat and journey as it chances
    west down the withy-stream, following my fancies!''
    
    Little Bird sat on twig. ``Whillo, Tom! I heed you.
    I've a guess, I've a guess where your fancies lead you.
    Shall I go, shall I go, bring him word to meet you?''
    \end{quotation}
  \end{ShowOutput}
  %
  Using a \Environment{quote} environment instead you get:
  %
  \begin{ShowOutput}
    A small example for a short anecdote:
     \begin{quote}\setlength{\parskip}{4pt plus 2pt minus 2pt}
    The old year was turning brown; the West Wind was
    calling;

    Tom caught the beechen leaf in the forest falling.
    ``I've caught the happy day blown me by the breezes!
    Why wait till morrow-year? I'll take it when me pleases.
    This I'll mend my boat and journey as it chances
    west down the withy-stream, following my fancies!''
    
    Little Bird sat on twig. ``Whillo, Tom! I heed you.
    I've a guess, I've a guess where your fancies lead you.
    Shall I go, shall I go, bring him word to meet you?''
    \end{quote}
  \end{ShowOutput}
  %
\end{Example}
%
\EndIndex{Env}{quote}%
\EndIndex{Env}{quotation}%



\begin{Declaration}
  \Environment{addmargin}\OParameter{left indentation}\Parameter{indentation}\\
  \Environment{addmargin*}\OParameter{inner indentation}\Parameter{indentation}
\end{Declaration}
\BeginIndex{Env}{addmargin}%
Similar to \Environment{quote} and \Environment{quotation}, the
\Environment{addmargin} environment changes the margin\Index{margin}.
In contrast to the first two environments, with
\Environment{addmargin} the user can set the width of the
indentation. Besides this, this environment does not change the
indentation of the first line nor the vertical spacing between
paragraphs.

If only the obligatory argument \PName{indentation} is given, both the
left and right margin are expanded by this value. If the optional
argument \PName{left indentation} is given as well, then at the left
margin the value \PName{left indentation} is used instead of
\PName{indentation}.

The starred \Environment{addmargin*} only differs from the normal
version in a two-sided layout. Furthermore, the difference only occurs
if the optional argument \PName{inner indentation} is used. In this
case this value \PName{inner indentation} is added to the normal inner
indentation. For right-hand pages this is the left margin, for
left-hand pages the right margin. Then the value of
\PName{indentation} determines the width of the opposite margin.

Both versions of this environment take also negative values for all
parameters. This has the effect of expanding the environment into the
margin.
\begin{Example}
  Suppose you write a documentation which includes short source code
  examples. To highlight these you want them separated from the text
  by a horizontal line and shifted slightly into the outer
  margin. First you define the environment:
\begin{lstcode}
  \newenvironment{SourceCodeFrame}{%
    \begin{addmargin*}[1em]{-1em}%
      \begin{minipage}{\linewidth}%
        \rule{\linewidth}{2pt}%
  }{%
      \rule[.25\baselineskip]{\linewidth}{2pt}%
      \end{minipage}%
    \end{addmargin*}%
  }
\end{lstcode}
  If you now put your source code in such an environment it will show
  up as:
  \begin{ShowOutput}
  \newenvironment{SourceCodeFrame}{%
    \begin{addmargin*}[1em]{-1em}%
      \begin{minipage}{\linewidth}%
        \rule{\linewidth}{2pt}%
  }{%
      \rule[.25\baselineskip]{\linewidth}{2pt}%
      \end{minipage}%
    \end{addmargin*}%
  }
  You define yourself the following environment:
  \begin{SourceCodeFrame}
\begin{lstcode}
\newenvironment{\SourceCodeFrame}{%
  \begin{addmargin*}[1em]{-1em}%
    \begin{minipage}{\linewidth}%
      \rule{\linewidth}{2pt}%
}{%
    \rule[.25\baselineskip]{\linewidth}{2pt}%
    \end{minipage}%
  \end{addmargin*}%
}
\end{lstcode}
  \end{SourceCodeFrame}
  This may be feasible or not. In any case it shows the usage of this
  environment.
  \end{ShowOutput}
  The optional argument of the \Environment{addmargin*} environment
  makes sure that the inner margin is extended by 1\Unit{em}. In turn
  the outer margin is decreased by 1\Unit{em}. The result is a shift
  by 1\Unit{em} to the outside.  Instead of \PValue{1em} you can of
  course use a length, for example, \PValue{2\Macro{parindent}}.
\end{Example}
There is one problem with the \Environment{addmargin*} which you
should be aware of. If a page break occurs within an
\Environment{addmargin*} environment, the indentation on the following
page will be on the wrong side.  This means that suddenly the
\PName{inner indentation} is applied on the outside of the
page. Therefore it is recommended to prevent page breaks within this
environment. This can be achieved by using an additional
\Macro{parbox} or, as in the example above, a
\Environment{minipage}. This makes use of the fact that neither the
argument of a \Macro{parbox} nor the content of a
\Environment{minipage} breaks at the end of a page.  Unfortunately
this is not without another disadvantage: in some cases pages can no
longer be filled correctly, which has the effect of generating several
warnings.

Incidentally, whether a page is going to be on the left or right side
of the book can not be determined for certain in the first {\LaTeX}
run.  For details please refer to the explanation of the commands
\Macro{ifthispageodd} and \Macro{ifthispagewasodd} in
\autoref{sec:maincls.pageStyle},
\autopageref{desc:maincls.cmd.ifthispageodd}.
%
\EndIndex{Env}{addmargin}%


\begin{Explain}
  One concluding remark on list environments: on the internet and
  during support it is often asked why such an environment is followed
  by a indented\Index{indentation} paragraph. In fact, this is not the
  case but is the result of the user demanding a new paragraph. In
  {\LaTeX} empty lines are interpreted as a new paragraph. This is
  also the case before and after list environments. Thus, if you want
  a list environment to be set within a paragraph you have to omit
  empty lines before and after. To nevertheless separate the
  environment from the rest of your text in the {\LaTeX} source file,
  you can insert a comment line before and after, that is, lines which
  begin with a percent character and contain nothing more.
\end{Explain}
\EndIndex[indexother]{}{lists}

\subsection{Margin Notes}
\label{sec:maincls.marginNotes.obsolete}

\begin{Declaration}
  \Macro{marginpar}\OParameter{margin note left}\Parameter{margin note}\\
  \Macro{marginline}\Parameter{margin note}
\end{Declaration}%
\BeginIndex{Cmd}{marginpar}%
\BeginIndex{Cmd}{marginline}%
Usually margin notes\Index[indexmain]{margin>notes} in {\LaTeX} are
inserted with the command \Macro{marginpar}. They are placed in the
outer margin.  In documents with one-sided layout the right border is
used. Though \Macro{marginpar} can take an optional different margin
note argument in case the output is in the left margin, margin notes
are always set in justified layout.  However, experience has shown
that many users prefer left- or right-aligned margin notes instead.
To facilitate this, {\KOMAScript} offers the command
\Macro{marginline}.
\begin{Example}
  In the introduction, the class name \Class{scrartcl} can be found in
  the margin. This can be produced\footnote{In fact, instead of
    \Macro{texttt}, a semantic highlighting was used. To avoid
    confusion this was replaced in the example.} with:
\begin{lstcode}
  \marginline{\texttt{scrartcl}}
\end{lstcode}

Instead of \Macro{marginline} you could have used
\Macro{marginpar}. In fact the first command is implemented internally
as:
\begin{lstcode}
  \marginpar[\raggedleft\texttt{scrartcl}]
    {\raggedright\texttt{scrartcl}}
\end{lstcode}
Thus \Macro{marginline} is really only an abbreviated writing of the
code above.
\end{Example}
%
\begin{Explain}
  Unfortunately \Macro{marginpar} does not always work correctly in
  two-sided\Index{twoside} layout. Whether a margin note will end up
  in the left or right margin is already decided while evaluating the
  command \Macro{marginpar}. If the output routine now shifts the
  margin note onto the next page the formatting is no longer
  correct. This behaviour is deeply rooted within {\LaTeX} and was
  therefore declared a feature by the {\LaTeX}3 team. \Macro{marginline}
  suffers from this ``feature'' too. The package
  \Package{mparhack}\IndexPackage{mparhack} (see
  \cite{package:mparhack}) offers a standard solution for this problem
  which naturally benefits also \Macro{marginpar} and
  \Macro{marginline}.

  Note that you may not use \Macro{marginpar} or \Macro{marginline}
  within float environments such as tables or figures. Also, these
  commands will not function in displayed math formulas.
\end{Explain}
%
\EndIndex{Cmd}{marginpar}%
\EndIndex{Cmd}{marginline}%


\subsection{Tables and Figures}
\label{sec:maincls.floats.obsolete}

\begin{Explain}
  With the floating environments {\LaTeX} offers a very capable and
  comfortable mechanism for automatic placement of
  figures\Index{figure} and tables\Index{table}. But often these
  floating environments\Index[indexmain] {floating environments} are
  slightly misunderstood by beginners. They often ask for a fixed
  position of a table or figure within the text. However, since these
  floating environments are being referenced in the text this is not
  necessary in most cases. It is also not sensible because such an
  object can only be set on the page if there is enough space left for
  it. If this is not the case the object would have to be shifted onto
  the next page, thereby possibly leaving a huge blank space on the
  page before.
  
  Often one finds in a document for every floating object the same
  optional argument for positioning the object. This also makes no
  sense. In such cases one should rather change the standard parameter
  globally. For more details refer to \cite{DANTE:FAQ}.\par
\end{Explain}

One last important note before starting this section: most mechanisms
described here which extend the capabilities of the standard classes
no longer work correctly when used together with packages which modify
the typesetting of captions of figures and tables. This should be self
evident, but it is often not understood.

\begin{Declaration}
  \Macro{caption}\OParameter{entry}\Parameter{title}\\
  \Macro{captionbelow}\OParameter{entry}\Parameter{title}\\
  \Macro{captionabove}\OParameter{entry}\Parameter{title}
\end{Declaration}%
\BeginIndex{Cmd}{caption}%
\BeginIndex{Cmd}{captionabove}%
\BeginIndex{Cmd}{captionbelow}%
In the standard classes caption text \PName{title} of tables and
figures is inserted with the \Macro{caption} command below the table
or figure. In general this is correct for figures. Opinions differ as
to whether captions of tables are to be placed above or, consistent
with captions of figures\Index{caption>of figure}, below the
table\Index{caption>of table}. That is the reason why {\KOMAScript},
unlike the standard classes, offers \Macro{captionbelow} for captions
below and \Macro{captionabove} for captions above tables or
figures. Using \Macro{caption} for figures always produces captions
below the figure, whereas with tables the behaviour of \Macro{caption}
can be modified using the options
\OptionValue{captions}{tableheading} and
\OptionValue{captions}{tablesignature} (see
\autoref{sec:maincls.formattingOptions},
\autopageref{desc:maincls.option.captions.tableheading}). For
compatibility reasons the default behaviour of \Macro{caption} used
with tables is similar to \Macro{captionbelow}.
%
\begin{Example}
  Instead of using captions below a table you want to place your
  captions above it\Index{table>caption}, because you have tables
  which span more then one page. In the standard classes you could
  only write:
\begin{lstcode}
  \begin{table}
    \caption{This is an example table}
    \begin{tabular}{llll}
      This & is & an & example.\\\hline
      This & is & an & example.\\
      This & is & an & example.
    \end{tabular}
  \end{table}
\end{lstcode}
  Then you would get the unsatisfying result:
  \begin{ShowOutput}\centering
    {\usekomafont{caption}{\usekomafont{captionlabel}\tablename~30.2:}
      This is an example table.}\\
        \begin{tabular}{llll}
      This & is & an & example.\\\hline
      This & is & an & example.\\
      This & is & an & example.
        \end{tabular}
\end{ShowOutput}
  Using {\KOMAScript} you write instead:
\begin{lstcode}
  \begin{table}
    \captionabove{This is just an example table}
    \begin{tabular}{llll}
      This & is & an & example.\\\hline
      This & is & an & example.\\
      This & is & an & example.
    \end{tabular}
  \end{table}
\end{lstcode}
  Then you get:
  \begin{ShowOutput}\centering
    {\usekomafont{caption}{\usekomafont{captionlabel}\tablename~30.2:}
      This is just an example table}\\\vskip\abovecaptionskip
    \begin{tabular}{llll}
      This & is & an & example.\\\hline
      This & is & an & example.\\
      This & is & an & example.
    \end{tabular}
  \end{ShowOutput}
  Since you want all your tables typeset with captions above, you could
  of course use the option \OptionValue{captions}{tableheading} instead (see
  \autoref{sec:maincls.formattingOptions},
  \autopageref{desc:maincls.option.captions.tableheading}). Then you can use
  \Macro{caption} as you would in the standard classes. You will get
  the same result as with \Macro{captionabove}.
\end{Example}

\begin{Explain}
  Some would argue that you could achieve the same result using the
  \Macro{topcaption} command from the \Package{topcapt}
  package\IndexPackage{topcapt} (see \cite{package:topcapt}). However,
  that is not the case. The command \Macro{topcaption} is ignored by
  packages which directly redefine the \Macro{caption} macro. The
  \Package{hyperref} package (see \cite{package:hyperref}) is one such
  example. In {\KOMAScript}, \Macro{captionabove} and
  \Macro{captionbelow} are so implemented that changes have an effect
  on both of these commands as well.
  
  \phantomsection\label{desc:maincls.cmd.caption.longtable}%
  If the \Package{longtable} package\IndexPackage{longtable} is used,
  {\KOMAScript} ensures that captions above tables which are placed
  within a \Environment{longtable} environment have the same
  appearance as those in a normal \Environment{table}
  environment. This also means that you can apply the same settings as
  in a \Environment{table} environment. Please note that in the
  \Package{longtable} package the maximum width of a table caption can
  be limited and the default is set to 4\Unit{in} (see
  \cite{package:longtable}). Used together with {\KOMAScript} this
  mechanism in \Package{longtable} works only if the class option
  \Option{origlongtable} is set (see
  \autoref{sec:maincls.formattingOptions},
  \autopageref{desc:maincls.option.origlongtable}). If the
  \Package{caption2}\IndexPackage{caption2} or
  \Package{caption}\IndexPackage{caption} package (see
  \cite{package:caption}) is loaded, table captions are handled by
  this package.
  
  Please note that \Macro{captionabove} and \Macro{captionbelow}, if
  placed within a \Environment{float} environment which was defined
  using the \Package{float}\IndexPackage{float} package, have the
  exact same behaviour described in \cite{package:float} for the
  \Macro{caption} command.  In this case, the float style determines
  whether the caption will be set below or above the figure or table.
\end{Explain}


\begin{Declaration}
  \Environment{captionbeside}\OParameter{entry}%
    \Parameter{title}\OParameter{placement}\OParameter{width}%
    \OParameter{offset}\\%
  \XEnvironment{captionbeside}\OParameter{entry}%
    \Parameter{title}\OParameter{placement}\OParameter{width}%
    \OParameter{offset}\PValue{*}
\end{Declaration}
\BeginIndex{Env}{captionbeside}%
Apart\ChangedAt{v2.8q}{%
  \Class{scrbook}\and\Class{scrreprt}\and\Class{scrartcl}} from
captions above and below the figure, one often finds captions, in
particular with small figures, which are placed beside the figure. In
general in this case both the baseline of the figure and of the
caption are aligned at the bottom. With some fiddling and the use of
two \Macro{parbox} commands this could also be achieved in the
standard classes.  However, {\KOMAScript} offers a special environment
for this which can be used within the floating environment.  The first
optional parameter \PName{entry} and the obligatory parameter
\PName{title} mean the same as the corresponding parameters of
\Macro{caption}, \Macro{captionabove} or \Macro{captionbelow}. The
caption text \PName{title} is placed beside the content of the
environment in this case.

Whether the caption text \PName{title} is placed on the left or the
right can be determined by the parameter \PName{placement}. Exactly
one of the following letters is allowed:
\begin{labeling}[--~]{\PValue[o]}
\item[\PValue{l}] left
\item[\PValue{r}] right
\item[\PValue{i}] inner margin in two-sided layout
\item[\PValue{o}] outer margin in two-sided layout
\end{labeling}
Default setting is to the right of the content of the environment.  If
either \PValue{o} or \PValue{i} are used you may need to run {\LaTeX}
twice to obtain the correct placement.

Per default the content of the environment and the caption text
\PName{title} fill the entire available text width. However, using the
optional parameter \PName{width}, it is possible to adjust the width
used. This width could even be larger than the current text width.

When supplying a \PName{width} the used width is usually centered with
respect to the text width. Using the optional parameter
\PName{offset}, you can shift the environment relative to the left
margin. A positive value corresponds to a shift to the right, whereas
a negative value corresponds to a shift to the left. An \PName{offset}
of 0\Unit{pt} gives you a left-aligned output.

Adding a star to the optional parameter \PName{offset} makes the the
value mean a shift relative to the right margin on left hand pages in
two-sided layout. A positive value corresponds to a shift towards the
outer margin, whereas a negative value corresponds to a shift towards
the inner margin. An \PName{offset} of 0\Unit{pt} means alignment with
the inner margin. As mentioned before, in some cases it takes two
{\LaTeX} runs for this to work correctly.

\begin{Example}
  An example for the usage of the \Environment{captionbeside} environment
  can be found in \autoref{fig:maincls.captionbeside}.
  This figure was typeset with:
\begin{lstcode}
  \begin{figure}
    \begin{captionbeside}[Example for a figure description]%
      {A figure description which is neither above nor
       below, but beside the figure}[i][\linewidth][2em]*
      \fbox{%
        \parbox[b][5\baselineskip][c]{.25\textwidth}{%
          \hspace*{\fill}\KOMAScript\hspace*{\fill}\par}}
    \end{captionbeside}
    \label{fig:maincls.captionbeside}
  \end{figure}
\end{lstcode}
  \begin{figure}
    \begin{captionbeside}[Example for a figure description]%
      {A figure description which is neither above nor 
       below, but beside the figure}[i][\linewidth][2em]*
      \fbox{%
        \parbox[b][5\baselineskip][c]{.25\textwidth}{%
          \hspace*{\fill}\KOMAScript\hspace*{\fill}\par}}
    \end{captionbeside}
    \label{fig:maincls.captionbeside}
  \end{figure}
  The total width is thus the currently available width
  \PValue{\Macro{linewidth}}. However, this width is shifted
  \PValue{2em} to the outside. The caption text or description is
  placed on the inner side beside the figure. The figure itself is
  shifted 2\Unit{em} into the outer margin.
\end{Example}
%
\EndIndex{Env}{captionbeside}


\BeginIndex[indexother]{}{font>style}
The font style\ChangedAt{v2.8p}{%
  \Class{scrbook}\and\Class{scrreprt}\and\Class{scrartcl}} for the
description and the label\,---\,``Figure'' or ``Table'', followed by
the number and the delimiter\,---\,can be changed with the commands
described in \autoref{sec:maincls.font}. The respective elements for
this are \FontElement{caption}\IndexFontElement{caption} and
\FontElement{captionlabel}\IndexFontElement{captionlabel} (see
\autoref{tab:maincls.elementsWithoutText},
\autopageref{tab:maincls.elementsWithoutText}).  First the font style
for the element \FontElement{caption} is applied to the element
\FontElement{captionlabel} too.  After this the font style of
\FontElement{captionlabel} is applied on the respective element. The
default settings are listed in \autoref{tab:maincls.captionFont}.
%
\begin{table}
  \centering%
  \caption{Font defaults for the elements of figure or table captions}
  \begin{tabular}{ll}
    \toprule
    element & default \\
    \midrule
    \FontElement{caption} & \Macro{normalfont} \\
    \FontElement{captionlabel} & \Macro{normalfont}\\
    \bottomrule
  \end{tabular}
  \label{tab:maincls.captionFont}
\end{table}
%
\begin{Example}
  You want the table and figure descriptions typeset in a smaller font
  size. Thus you could write the following in the preamble of your
  document:
\begin{lstcode}
  \addtokomafont{caption}{\small}
\end{lstcode}
  Furthermore, you would like the labels to be printed in sans serif and
  bold. You add:
\begin{lstcode}
  \setkomafont{captionlabel}{\sffamily\bfseries}
\end{lstcode}
  As you can see, simple extensions of the default definitions are
  possible.
\end{Example}
\EndIndex[indexother]{}{font>style}
%
\EndIndex{Cmd}{caption}%
\EndIndex{Cmd}{captionabove}%
\EndIndex{Cmd}{captionbelow}%


\begin{Explain}
\begin{Declaration}
  \FloatStyle{komaabove}\\
  \FloatStyle{komabelow}
\end{Declaration}%
\BeginIndex{Floatstyle}{komaabove}%
\BeginIndex{Floatstyle}{komabelow}%
If you use\OnlyAt{\Package{float}} the
\Package{float}\IndexPackage{float} package the appearance of the
float environments is solely defined by the \emph{float} style. This
includes whether captions above or below are used. In the
\Package{float} package there is no predefined style which gives you
the same output and offers the same setting options (see below) as
{\KOMAScript}.  Therefore {\KOMAScript} defines the two additional
styles \PValue{komaabove} and \PValue{komabelow}.  When using the
\Package{float} package these styles can be activated just like the
styles \PValue{plain}\IndexFloatstyle{plain},
\PValue{boxed}\IndexFloatstyle{boxed} or
\PValue{ruled}\IndexFloatstyle{ruled} defined in \Package{float}.  For
details refer to \cite{package:float}.  The style \PValue{komaabove}
inserts \Macro{caption}, \Macro{captionabove} and \Macro{captionbelow}
above, whereas \PValue{komabelow} inserts them below the float
content.
%
\EndIndex{Floatstyle}{komaabove}%
\EndIndex{Floatstyle}{komabelow}%
\end{Explain}


\begin{Declaration}
  \Macro{captionformat}
\end{Declaration}%
\BeginIndex{Cmd}{captionformat}%
In {\KOMAScript} there are different ways to change the formatting of
the caption text. The definition of different font styles was already
explained above. This or the caption delimiter between the label and
the label text itself is specified in the macro \Macro{captionformat}.
In contrast to all other \Macro{\dots}format commands, in this case it
does not contain the counter but only the items which follow it. The
original definition is:
\begin{lstcode}
  \newcommand*{\captionformat}{:\ }
\end{lstcode}
This too can be changed with \Macro{renewcommand}.
\begin{Example}
  For some inexplicable reasons you want a dash with spaces before and
  after instead of a colon followed by a space as label delimiter. You
  define:
\begin{lstcode}
  \renewcommand*{\captionformat}{~--~}
\end{lstcode}
This definition should be put in the preamble of your document.
\end{Example}
%
\EndIndex{Cmd}{captionformat}%


\begin{Declaration}
  \Macro{figureformat}\\
  \Macro{tableformat}
\end{Declaration}%
\BeginIndex{Cmd}{figureformat}%
\BeginIndex{Cmd}{tableformat}%
It was already mentioned that \Macro{captionformat} does not contain
formatting for the label itself. This situation should under no
circumstances be changed using redefinitions of the commands for the
output of counters, \Macro{thefigure} or \Macro{thetable}. Such a
redefinition would have unwanted side effects on the output of
\Macro{ref} or the table of contents, list of figures and list of
tables. To deal with the situation, {\KOMAScript} offers two
\Macro{\dots format} commands instead. These are predefined as
follows:
\begin{lstcode}
  \newcommand*{\figureformat}{\figurename~\thefigure\autodot}
  \newcommand*{\tableformat}{\tablename~\thetable\autodot}
\end{lstcode}
They also can be adapted to your personal preferences with
\Macro{renewcommand}.
\begin{Example}
  From time to time captions without any label and of course without
  delimiter are desired. In {\KOMAScript} it takes only the following
  definitions to achieve this:
\begin{lstcode}
  \renewcommand*{\figureformat}{}
  \renewcommand*{\tableformat}{}
  \renewcommand*{\captionformat}{}
\end{lstcode}
It should be noted, however, that although no numbering is output, the
internal counters are nevertheless incremented. This becomes important
especially if this redefinition is applied only to selected
\Environment{figure} or \Environment{table} environments.
\end{Example}
%
\EndIndex{Cmd}{figureformat}%
\EndIndex{Cmd}{tableformat}%

\begin{Declaration}
  \Macro{setcapindent}\Parameter{indent}\\
  \Macro{setcapindent*}\Parameter{xindent}\\
  \Macro{setcaphanging}
\end{Declaration}%
\BeginIndex{Cmd}{setcapindent}%
\BeginIndex{Cmd}{setcapindent*}%
\BeginIndex{Cmd}{setcaphanging}%
As mentioned previously, in the standard classes the captions are set
in a non-hanging style, that is, in multi-line captions the second and
subsequent lines start directly beneath the label. The standard
classes provide no direct mechanism to change this behaviour. In
{\KOMAScript}, on the contrary, beginning at the second line all lines
are indented by the width of the label so that the caption text is
aligned.

This behaviour, which corresponds to the usage of
\Macro{setcaphanging}, can easily be changed by using the command
\Macro{setcapindent} or \Macro{setcapindent*}. Here the parameter
\PName{indent} determines the indentation of the second and subsequent
lines.

If you want a line break after the label and before the caption text,
then you can define the indentation \PName{xindent} of the caption
text with the starred version of the command instead:
\Macro{setcapindent*}.

Using a negative value of \PName{indent} instead, a line break is also
inserted before the caption text and only the first line of the
caption text but not subsequent lines are indented by
the absolute value of \PName{indent}.

Whether one-line captions are set as captions with more than one line
or are treated separately is specified with the class options
\OptionValue{captions}{oneline} and \OptionValue{captions}{nooneline}. For details
please refer to the explanations of these options in
\autoref{sec:maincls.layoutOptions},
\autopageref{desc:maincls.option.captions.nooneline}.

\begin{Example}
  For the examples please refer to
  figures~\ref{fig:maincls.caption.first} to
  \ref{fig:maincls.caption.last}. As you can see the usage of a fully
  hanging indentation is not advantageous when combined with narrow
  column width. To illustrate, the source code for the second figure
  is given here with a modified caption text:
\begin{lstcode}
  \begin{figure}
    \setcapindent{1em}
    \fbox{\parbox{.95\linewidth}{\centering{\KOMAScript}}}
    \caption{Example with slightly indented caption
             starting at the second line}
  \end{figure}
\end{lstcode}
As can be seen the formatting can also be changed locally within the
\Environment{figure} environment\IndexEnv{figure}. The change then
affects only the current figure. Following figures once again use the
default settings or global settings set, for example, in the preamble
of the document. This also of course applies to tables.
  \begin{figure}
    \typeout{^^J--- Ignore underfull and overfull \string\hbox:}
    \addtokomafont{caption}{\small}
    \addtokomafont{captionlabel}{\bfseries}
    \centering%
    \begin{minipage}{.9\linewidth}
      \begin{minipage}[t]{.48\linewidth}\sloppy
        \fbox{\parbox{.95\linewidth}{\centering{\KOMAScript}}}
        \caption[Example for figure caption]%
        {\sloppy Equivalent to the standard setting, similar to the
          usage of \Macro{setcaphanging}}
        \label{fig:maincls.caption.first}
      \end{minipage}
      \hspace{.02\linewidth}
      \begin{minipage}[t]{.48\linewidth}\sloppy
        \setcapindent{1em}
        \fbox{\parbox{.95\linewidth}{\centering{\KOMAScript}}}
        \caption[Example for figure caption]%
        {With slightly hanging indentation starting at the second line
         using \Macro{setcapindent}\PParameter{1em}}
      \end{minipage}
    \end{minipage}

    \vspace*{2ex}\noindent%
    \begin{minipage}{.9\linewidth}
      \begin{minipage}[t]{.48\linewidth}\sloppy
        \setcapindent*{1em}
        \fbox{\parbox{.95\linewidth}{\centering{\KOMAScript}}}
        \caption[Example for a figure caption]%
        {With hanging indentation starting at the second line and line
          break before the description using
          \Macro{setcapindent*}\PParameter{1em}}
      \end{minipage}
      \hspace{.02\linewidth}
      \begin{minipage}[t]{.48\linewidth}\sloppy
        \setcapindent{-1em}
        \fbox{\parbox{.95\linewidth}{\centering{\KOMAScript}}}
        \caption[Example for a figure caption]%
        {With indentation in the second line only and line break
          before the description using
          \Macro{setcapindent}\PParameter{-1em}}
                \label{fig:maincls.caption.last}
      \end{minipage}
    \end{minipage}
    \typeout{^^J--- Don't ignore underfull and overfull
      \string\hbox:^^J}
  \end{figure}
\end{Example}
%
\EndIndex{Cmd}{setcapindent}%
\EndIndex{Cmd}{setcapindent*}%
\EndIndex{Cmd}{setcaphanging}%


\begin{Declaration}
  \Macro{setcapwidth}\OParameter{justification}\Parameter{width}\\
  \Macro{setcapmargin}\OParameter{margin left}\Parameter{margin}\\
  \Macro{setcapmargin*}\OParameter{margin inside}\Parameter{margin}
\end{Declaration}
\BeginIndex{Cmd}{setcapwidth}%
\BeginIndex{Cmd}{setcapmargin}%
\BeginIndex{Cmd}{setcapmargin*}%
Using\ChangedAt{v2.8q}{%
  \Class{scrbook}\and\Class{scrreprt}\and\Class{scrartcl}} these three
commands you can specify the width and justification of the caption
text.  In general the whole text width or column width is available
for the caption.

With the command \Macro{setcapwidth} you can decrease this
\PName{width}.  The obligatory argument determines the maximum
\PName{width} of the caption.  As an optional argument you can supply
exactly one letter which specifies the horizontal justification. The
possible justifications are given in the following list.
\begin{labeling}[--~]{\PValue[o]}
\item[\PValue{l}] left-aligned
\item[\PValue{c}] centered
\item[\PValue{r}] right-aligned
\item[\PValue{i}] alignment at the inner margin in double-sided output
\item[\PValue{o}] alignment at the outer margin in double-sided output
\end{labeling}
The justification inside and outside corresponds to left-aligned and
right-aligned, respectively, in single-sided output. Within
\Package{longtable}\IndexPackage{longtable} tables the justification
inside or outside does not work correctly. In particular, the captions
on subsequent pages of such tables are aligned according to the format
of the caption on the first page of the table. This is a conceptual
problem in the implementation of \Package{longtable}.

With the command \Macro{setcapmargin} you can specify a \PName{margin}
which is to be left free next to the description in addition to the
normal text margin. If you want margins with different widths at the
left and right side you can specify these using the optional argument
\PName{margin left}. The starred version \Macro{setcapmargin*} defines
instead of a \PName{margin left} a \PName{margin inside} in a
double-sided layout. In case of
\Package{longtable}\IndexPackage{longtable} tables you have to deal
with the same problem with justification inside or outside as
mentioned with the macro \Macro{setcapwidth}.  Furthermore, the usage
of \Macro{setcapmargin} or \Macro{setcapmargin*} switches on the
option \OptionValue{captions}{nooneline} (see
\autoref{sec:maincls.layoutOptions},
\autopageref{desc:maincls.option.captions.nooneline}) for the captions
which are typeset with this margin setting.

\begin{Explain}
  \Package{longtable} places the caption in a box, which is issued
  again on subsequent pages as needed. When outputting a box, the
  macros needed for its creation are not reevaluated. That is the
  reason why it is not possible for {\KOMAScript} to swap margin
  settings for even pages in double-sided layout . This is what would
  be necessary in order to produce a justification which is shifted
  towards the outside or inside.
  
  You can also submit negative values for \PName{margin} and
  \PName{margin left} or \PName{margin inside}. This has the effect
  of the caption expanding into the margin.
\end{Explain}

\begin{Example}
  A rather odd problem is that of a figure caption which is required
  to be both centered and of the same width as the figure itself. If
  the width of the figure is known in advance, the solution with
  {\KOMAScript} is quite easy. Supposing the figure has a width of
  8\Unit{cm}, it only takes:
\begin{lstcode}
  \setcapwidth[c]{8cm}
\end{lstcode}
directly in front of \Macro{caption} or \Macro{captionbelow}. If the
width is unknown then you first have to define a length in the
preamble of your document:
\begin{lstcode}
  \newlength{\FigureWidth}
\end{lstcode}
  Having done this you can calculate the width directly with the
  {\LaTeX} command \Macro{settowidth} (see \cite{latex:usrguide})
 in many cases. A possible solution would look as follows:
\begin{lstcode}
  \begin{figure}
    \centering%
    \settowidth{\FigureWidth}{%
      \fbox{\quad\KOMAScript\quad}%
      }%
    \fbox{\quad\KOMAScript\quad}%
    \setcapwidth[c]{\FigureWidth}
    \caption{Example of a centered caption below the figure}
  \end{figure}
\end{lstcode}
However, it is awkward to write the content twice and to call
\Macro{setcapwidth} for every figure. Yet nothing is easier than
defining a new command in the preamble of your document which hides
the three steps of:
  \begin{enumerate}
  \item defining the width of the argument
  \item specifying the width of the caption
  \item outputting the argument
  \end{enumerate}
  in:
\begin{lstcode}
  \newcommand{\Figure}[1]{%
    \settowidth{\FigureWidth}{#1}%
    \setcapwidth[c]{\FigureWidth}%
    #1}
\end{lstcode}
  Using this command the example abbreviates to:
\begin{lstcode}
  \begin{figure}
    \centering%
    \Figure{\fbox{\quad\KOMAScript\quad}}%
    \caption{Example of a centered caption below the figure}
  \end{figure}
\end{lstcode}
  
However, commands have the disadvantage that errors in the macros of
the argument in case of arguments with more than one line are not
reported with the very accurate line numbers by {\LaTeX}. Thus in some
cases the use of an environment has advantages. Then, however, the
question arises of how the width of the content of the environment can
be determined. The solution involves the \Environment{lrbox}
environment, described in \cite{latex:usrguide}:
\begin{lstcode}
  \newsavebox{\FigureBox}
  \newenvironment{FigureDefinesCaptionWidth}{%
    \begin{lrbox}{\FigureBox}%
  }{%
    \end{lrbox}%
    \global\setbox\FigureBox=\box\FigureBox%
    \aftergroup\SetFigureBox%
  }
  \newcommand{\SetFigureBox}{%
    \Figure{\usebox{\FigureBox}}}
\end{lstcode}
  This definition uses the macro \Macro{Figure} defined above.  In
  the main text you write:
\begin{lstcode}
  \begin{figure}
    \centering%
    \begin{FigureDefinesCaptionWidth}
      \fbox{\hspace{1em}\KOMAScript\hspace{1em}}
    \end{FigureDefinesCaptionWidth}
    \caption{Example of a centered caption below the figure}
  \end{figure}
\end{lstcode}
Admittedly, the environment in this example is not necessary. However,
its definition using \Macro{global} is quite tricky. Most users would
probably not be able to define such an environment without help. Thus,
as this definition can be very useful, it was introduced in the above
example.
  
Even if the \Environment{captionbeside} environment did not exist you
could nevertheless place the figure caption beside the figure in a
quite simple way. For this \Macro{SetFigureBox} from the example above
would have to be redefined first:
\begin{lstcode}
  \renewcommand{\SetFigureBox}{%
    \settowidth{\captionwidth}{\usebox{\FigureBox}}%
    \parbox[b]{\captionwidth}{\usebox{\FigureBox}}%
    \hfill%
    \addtolength{\captionwidth}{1em}%
    \addtolength{\captionwidth}{-\hsize}%
    \setlength{\captionwidth}{-\captionwidth}%
    \setcapwidth[c]{\captionwidth}%
    }
\end{lstcode}
Finally you only have to put the \Macro{caption} command in a
\Macro{parbox} too:
\begin{lstcode}
  \begin{figure}
    \centering%
    \begin{FigureSetsCaptionWidth}
      \fbox{\rule{0pt}{5\baselineskip}%
        \hspace{1em}\KOMAScript\hspace{1em}}
    \end{FigureSetsCaptionWidth}
    \parbox[b]{\FigureWidth}{%
      \caption{Example of a centered caption
               below the figure}
    }
  \end{figure}
\end{lstcode}
The \Macro{rule} command in this example only serves as an invisible
support to produce an example figure with a greater vertical height.
\end{Example}
%
\EndIndex{Cmd}{setcapwidth}%
\EndIndex{Cmd}{setcapmargin}%
\EndIndex{Cmd}{setcapmargin*}%


% %%%%%%%%%%%%%%%%%%%%%%%%%%%%%%%%%%%%%%%%%%%%%%%%%%%%%%%%%%%%%%%%%%%%%%

\section{Appendix\protect\footnote{This section is deprecated
    and should be replaced by new sections using the structure of the German
    guide. Translators from English to German would be welcome! You may find
    additional information about obsolete or deprecated options at
    \autoref{cha:maincls-experts}.}}
\label{sec:maincls.appendix.obsolete}

The last part of a document usually contains the
appendix\Index{appendix}, the bibliography\Index{bibliography} and, if
necessary, the index\Index{index}.

\begin{Declaration}
  \Macro{appendix}
\end{Declaration}%
\BeginIndex{Cmd}{appendix}%
The appendix in the standard as well as the {\KOMAScript} classes is
introduced with \Macro{appendix}. This command switches, among other
things, the chapter numbering to upper case letters, also ensuring
that the rules according to \cite{DUDEN} are followed (for
German-speaking regions). These rules are explained in more detail in
the description of the class options \OptionValue{numbers}{enddot} and
\OptionValue{numbers}{noenddot} in \autoref{sec:maincls.formattingOptions},
\autopageref{desc:maincls.option.numbers}.

Please note that \Macro{appendix} is a command, \emph{not} an
environment!  This command does not expect any argument. Sectioning in
the appendix uses \Macro{chapter} and \Macro{section} just as does the
main text.

%
\EndIndex{Cmd}{appendix}


\begin{Declaration}
  \Macro{appendixmore}
\end{Declaration}%
\BeginIndex{Cmd}{appendixmore}%
There is a peculiarity within the \Macro{appendix} command in the
{\KOMAScript} classes. If the command \Macro{appendixmore} is defined,
this command is executed also by the \Macro{appendix}
command. Internally the {\KOMAScript} classes \Class{scrbook} and
\Class{scrreprt} take advantage of this behaviour to implement the
options \Option{appendixprefix} and \OptionValue{appendixprefix}{false} (see
\autoref{sec:maincls.layoutOptions},
\autopageref{desc:maincls.option.appendixprefix}). You should take
note of this in case you decide to define or redefine the
\Macro{appendixmore}. In case one of these options is set, you will
receive an error message when using
\verb|\newcommand{\appendixmore}{|\dots\verb|}|. This behaviour is
intended to prevent you from disabling options without noticing it.

\begin{Example}
  You do not want the chapters in the main part of the classes
  \Class{scrbook} or \Class{scrreprt} to be introduced by a prefix
  line (see layout options \Option{chapterprefix} and
  \OptionValue{chapterprefix}{false} in \autoref{sec:maincls.layoutOptions},
  \autopageref{desc:maincls.option.chapterprefix}). For consistency
  you also do not want such a line in the appendix either. Instead,
  you would like to see the word ``Chapter'' in the language of your
  choice written in front of the chapter letter and, simultaneously,
  in the page headings.  Instead of using the either layout option
  \Option{appendixprefix} or \OptionValue{appendixprefix}{false}, you would
  define in the document preamble:
%
\begin{lstcode}
  \newcommand*{\appendixmore}{%
    \renewcommand*{\chapterformat}{%
      \appendixname~\thechapter\autodot\enskip}
    \renewcommand*{\chaptermarkformat}{%
      \appendixname~\thechapter\autodot\enskip}
  }
\end{lstcode}
%  
  In case you subsequently change your mind and decide to use the option
  \Option{appendixprefix} at a later stage, you will get an error message
  because of the already defined \Macro{appendixmore} command.  This behaviour
  prevents the definition made above from invisibly changing the settings
  intended with the option.
  
  It is also possible to get a similar behaviour of the appendix for
  the class \Class{scrartcl}. You would write in the preamble of your
  document:
\begin{lstcode}
  \newcommand*{\appendixmore}{%
    \renewcommand*{\othersectionlevelsformat}[1]{%
      \ifthenelse{\equal{##1}{section}}{\appendixname~}{}%
      \csname the##1\endcsname\autodot\enskip}
    \renewcommand*{\sectionmarkformat}{%
      \appendixname~\thesection\autodot\enskip}
  }
\end{lstcode}
  In addition, the package \Package{ifthen}\IndexPackage{ifthen} (see
  \cite{package:ifthen}) is required.
  
  Redefined commands are explained in more detail in
  \autoref{sec:maincls.structure},
  \autopageref{desc:maincls.cmd.chapterformat} and
  \autopageref{desc:maincls.cmd.chaptermarkformat}.
\end{Example}
%
\EndIndex{Cmd}{appendixmore}%


\begin{Declaration}
  \Macro{setbibpreamble}\Parameter{preamble}
\end{Declaration}%
\BeginIndex{Cmd}{setbibpreamble}%
The command \Macro{setbibpreamble} can be used to set a preamble for the
bibliography\Index{bibliography}. This can be achieved by placing the preamble
before the command for issuing the bibliography.  However, it does not have to
be directly in front of it. For example, it could be placed at the beginning
of the document. Similar to the class options
\OptionValue{bibliography}{totoc} and
\OptionValue{bibliography}{totocnumbered}, this command can only be successful
if you have not loaded a package which prevents this by redefining the
\Environment{thebibliography} environment.  Even though the \Package{natbib}
package \IndexPackage{natbib} makes unauthorized use of internal macros of
{\KOMAScript} it could be achieved that \Macro{setbibpreamble} works with the
current version of \Package{natbib} (see \cite{package:natbib}).

\begin{Example}
  You want to point out that the sorting of the references in the
  bibliography is not according to their occurrence in the text, but
  in alphabetical order. You use the following command:
\begin{lstcode}
  \setbibpreamble{References are in alphabetical order.
    References with more than one author are sorted
    according to the first author.\par\bigskip}
\end{lstcode}
  The \Macro{bigskip}\IndexCmd{bigskip} command makes sure that the
  preamble and the first reference are separated by a large vertical
  space.
\end{Example}
%
\EndIndex{Cmd}{setbibpreamble}


\begin{Declaration}
  \Macro{setindexpreamble}\Parameter{preamble}
\end{Declaration}%
\BeginIndex{Cmd}{setindexpreamble}%
Similarly to the bibliography you can use a preamble to the
index. This is often the case if you have more than one index or if
you use different kinds of referencing by highlighting the page
numbers in different ways.

\begin{Example}
  You have a document in which terms are both defined and used. The
  page numbers of definitions are in bold. Of course you want to make
  your reader aware of this fact. Thus you insert a preamble for the
  index:
\begin{lstcode}
  \setindexpreamble{In \textbf{bold} printed page numbers are
    references to the definition of terms. Other page numbers indicate
    the use of a term.\par\bigskip}
\end{lstcode}
\end{Example}
%
Please note that the page style of the first page of the index is
changed. The applied page style is defined in the macro
\Macro{indexpagestyle} (see \autoref{sec:maincls.pageStyle},
\autopageref{desc:maincls.cmd.titlepagestyle}).

The production, sorting and output of the index is done by the
standard {\LaTeX} packages and additional programs. Similar to the
standard classes {\KOMAScript} only provides the basic macros and
environments.
%
\EndIndex{Cmd}{setindexpreamble}%



%%% Local Variables:
%%% mode: latex
%%% coding: iso-latin-1
%%% TeX-master: "../guide"
%%% End:
