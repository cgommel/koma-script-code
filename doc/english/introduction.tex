% ======================================================================
% introduction.tex
% Copyright (c) Markus Kohm, 2001-2009
%
% This file is part of the LaTeX2e KOMA-Script bundle.
%
% This work may be distributed and/or modified under the conditions of
% the LaTeX Project Public License, version 1.3c of the license.
% The latest version of this license is in
%   http://www.latex-project.org/lppl.txt
% and version 1.3c or later is part of all distributions of LaTeX 
% version 2005/12/01 or later and of this work.
%
% This work has the LPPL maintenance status "author-maintained".
%
% The Current Maintainer and author of this work is Markus Kohm.
%
% This work consists of all files listed in manifest.txt.
% ----------------------------------------------------------------------
% introduction.tex
% Copyright (c) Markus Kohm, 2001-2009
%
% Dieses Werk darf nach den Bedingungen der LaTeX Project Public Lizenz,
% Version 1.3c, verteilt und/oder veraendert werden.
% Die neuste Version dieser Lizenz ist
%   http://www.latex-project.org/lppl.txt
% und Version 1.3c ist Teil aller Verteilungen von LaTeX
% Version 2005/12/01 oder spaeter und dieses Werks.
%
% Dieses Werk hat den LPPL-Verwaltungs-Status "author-maintained"
% (allein durch den Autor verwaltet).
%
% Der Aktuelle Verwalter und Autor dieses Werkes ist Markus Kohm.
% 
% Dieses Werk besteht aus den in manifest.txt aufgefuehrten Dateien.
% ======================================================================
%
% Introduction of the KOMA-Script guide
% Maintained by Markus Kohm
%
% ----------------------------------------------------------------------
%
% Einleitung der KOMA-Script-Anleitung
% Verwaltet von Markus Kohm
%
% ======================================================================

\ProvidesFile{introduction.tex}[2007/09/26 KOMA-Script guide introduction]
\translator{Kevin Pfeiffer\and Gernot Hassenpflug\and Markus Kohm}

% Date of translated german file: 2006/07/05

\chapter{Introduction}
\labelbase{introduction}

\section{Preface}\label{sec:introduction.preface}

The {\KOMAScript} bundle is actually several packages and classes. It
provides counterparts or replacements for the standard {\LaTeX} classes such
as \emph{article}, \emph{book}, etc. (see \autoref{cha:maincls}), but offers
many additional features and its own unique look and feel.

The {\KOMAScript} user guide is intended to serve the advanced as well as the
inexperienced {\LaTeX} user and is accordingly quite large. The result is a
compromise and we hope that you will keep this in mind when using it.
Your suggestions for improvement are, of course, always welcome.


\section{Structure of the Guide}\label{sec:introduction.structure}

The {\KOMAScript} user guide is not intended to be a {\LaTeX} primer.
Those new to {\LaTeX} should look at \emph{The Not So Short Introduction
to {\LaTeXe}} \cite{lshort} or \emph{{\LaTeXe} for Authors}
\cite{latex:usrguide} or a {\LaTeX} reference book. You will also find
useful information in the many {\LaTeX} FAQs, including the \emph{{\TeX}
Frequently Asked Questions on the Web} \cite{UK:FAQ}.

\begin{Explain}
  In this guide you will find supplemental information about {\LaTeX} and 
  {\KOMAScript} in (sans serif) paragraphs like this one. 
  The information given in these explanatory  sections is not essential for 
  using {\KOMAScript}, but if you experience  problems you should 
  take a look at it\,---\,particularly before sending a bug report.
\end{Explain}

If you are only interested in using a single {\KOMAScript} class or
package you can probably successfully avoid reading the entire guide.
Each class and package typically has its own chapter; however, the three
main classes (\Class{scrbook}, \Class{scrrprt}, and \Class{scrartcl})
are introduced together in chapter three. Where an example or note only
applies to one or two of the three classes, it is called out in the
margin. \OnlyAt{\Class{Like this.}}

\begin{Explain} 
  The primary documentation for {\KOMAScript} is in German and has been
  translated for your convenience; like most of the {\LaTeX} world, its
  commands, environments, options, etc., are in English. In a few cases, the
  name of a command may sound a little strange, but even so, we hope and
  believe that with the help of this guide {\KOMAScript} will be usable
  and useful to you.
\end{Explain}


\section{History of {\KOMAScript}}\label{sec:introduction.history}

In the early 1990s, Frank Neukam needed a method to publish an
instructor's lecture notes. At that time {\LaTeX} was {\LaTeX}2.09 and there
was no distinction between classes and packages\,---\,there were only
\emph{styles}.  Frank felt that the standard document styles were not
good enough for his work; he wanted additional commands and
environments. At the same time he was interested in typography and,
after reading Tschichold's \emph{Ausgew�hlte Aufs�tze �ber Fragen der
Gestalt des Buches und der Typographie} (Selected Articles on the
Problems of Book Design and Typography) \cite{JTsch87}, he decided to
write his own document style\,---\,and not just a one-time solution to his
lecture notes, but an entire style family, one specifically designed for
European and German typography. Thus {\Script} was born.

Markus Kohm, the developer of {\KOMAScript}, came across {\Script} in
December 1992 and added an option to use the A5 paper format. This and
other changes were then incorporated into {\ScriptII}, released by Frank
in December 1993.

Beginning in mid-1994, {\LaTeXe} became available and brought with it
many changes. Users of {\ScriptII} were faced with either limiting their
usage to {\LaTeXe}'s compatibility mode or giving up {\Script}
altogether.  This situation led Markus to put together a new {\LaTeXe}
package, released on 7~July 1994 as {\KOMAScript}; a few months later
Frank declared {\KOMAScript} to be the official successor to {\Script}.
{\KOMAScript} originally provided no \emph{letter} class, but this
deficiency was soon remedied by Axel Kielhorn, and the result became part
of {\KOMAScript} in December 1994.  Axel also wrote the first true
German-language user guide, which was followed by an English-language
guide by Werner Lemberg.

Since then much time has passed. {\LaTeX} has changed in only minor
ways, but the {\LaTeX} landscape has changed a great deal; many new
packages and classes are now available and {\KOMAScript} itself has
grown far beyond what it was in 1994. The initial goal was to provide
good {\LaTeX} classes for German-language authors, but today its
primary purpose is to provide more-flexible alternatives to the
standard classes. {\KOMAScript}'s success has led to e-mail from users
all over the world, and this has led to many new macros\,---\,all
needing documentation; hence this ``small guide.''


\section{Special Thanks}\label{sec:introduction.thanks}

Acknowledgements in the introduction? No, the proper acknowledgements can be
found in the addendum. My comments here are not intended for the authors of
this guide\,---\,and those thanks should rightly come from you, the reader,
anyhow. I, the author of {\KOMAScript}, would like to extend my personal thanks
to Frank Neukam.  Without his {\Script} family, {\KOMAScript} would not have
come about.  I am indebted to the many persons who have contributed to
{\KOMAScript}, but with their indulgence, I would like to specifically mention
Jens-Uwe Morawski and Torsten Kr\"uger. The English translation of the guide
is, among many other things, due to Jens's untiring commitment. Torsten was
the best beta-tester I ever had. His work has particularly enhanced the
usability of \Class{scrlttr2} und \Class{scrpage2}. Many thanks to all who
encouraged me to go on, to make things better and less error-prone, or to
implement additional features.

Thanks go as well to DANTE, Deutschsprachige
Anwendervereinigung {\TeX}~e.V\kern-.18em, (the German-Language {\TeX} User Group).
Without the DANTE server, {\KOMAScript} could not have been released and
distributed. Thanks as well to everybody in the {\TeX} newsgroups and mailing
lists who answer questions and have helped me to provide support for
{\KOMAScript}.

\section{Legal Notes}\label{sec:introduction.legal}

{\KOMAScript} was released under the {\LaTeX} Project Public License. You
will find it in the file \File{lppl.txt}. An unofficial German-language
translation is also available in \File{lppl-de.txt} and is valid for all
German-speaking countries.

\iffree{This document and the {\KOMAScript} bundle are provided ``as is'' and
without warranty of any kind.}%
{Please note: the printed version of this guide is not free under the
conditions of the {\LaTeX} Project Public Licence. If you need a free version
of this guide, use the version that is provided as part of the {\KOMAScript}
bundle.}

\section{Installation}\label{sec:introduction.installation}
Installation information can be found in the file \File{INSTALL.txt}. You
should also read the documentation that comes with the {\TeX} distribution you
are using.

\section{Bugreports and Other Requests}
\label{sec:introduction.errors}

If you think you have found an error in the documentation or a bug in one of
the {\KOMAScript} classes, one of the {\KOMAScript} packages, or another
part of {\KOMAScript}, please do the following: first have a look on
CTAN to see if a newer version of {\KOMAScript} is available; in this case
install the applicable section and try again.

If you are using the most recent version of {\KOMAScript} and still have a
bug, please provide a short {\LaTeX} document that demonstrates the problem. You
should only use the packages and definitions needed to demonstrate the
problem; do not use any unusual packages.

By preparing such an example it often becomes clear whether the problem is
truly a {\KOMAScript} bug or something else. Please report {\KOMAScript} (only)
bugs to the author of {\KOMAScript}. Please use \File{komabug.tex}, an
interactive {\LaTeX} document, to generate your bug report and send it to the
address you may find at \File{komabug.tex}.

If you want to ask your question in a newsgroup or mailing list, you
should also include such an example as part of your question, but in this
case, using \File{komabug.tex} is not necessary. To find out the version
numbers of all packages in use, simply put \Macro{listfiles} in the preamble of
your example and read the end of the \File{log}-file.


\section{Additional Information}
\label{sec:introduction.moreinfos}

Once you become an experienced {\KOMAScript} user you may want to look at some
more advanced examples and information. These you will find on the
{\KOMAScript} documentation web site \cite{homepage}. The main
language of the site is German, but nevertheless English is
welcome.

\endinput
%%% Local Variables: 
%%% mode: latex
%%% coding: iso-latin-1
%%% TeX-master: "../guide.tex"
%%% End: 
