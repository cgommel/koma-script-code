% ======================================================================
% introduction.tex
% Copyright (c) Markus Kohm, 2001-2012
%
% This file is part of the LaTeX2e KOMA-Script bundle.
%
% This work may be distributed and/or modified under the conditions of
% the LaTeX Project Public License, version 1.3c of the license.
% The latest version of this license is in
%   http://www.latex-project.org/lppl.txt
% and version 1.3c or later is part of all distributions of LaTeX 
% version 2005/12/01 or later and of this work.
%
% This work has the LPPL maintenance status "author-maintained".
%
% The Current Maintainer and author of this work is Markus Kohm.
%
% This work consists of all files listed in manifest.txt.
% ----------------------------------------------------------------------
% introduction.tex
% Copyright (c) Markus Kohm, 2001-2012
%
% Dieses Werk darf nach den Bedingungen der LaTeX Project Public Lizenz,
% Version 1.3c, verteilt und/oder veraendert werden.
% Die neuste Version dieser Lizenz ist
%   http://www.latex-project.org/lppl.txt
% und Version 1.3c ist Teil aller Verteilungen von LaTeX
% Version 2005/12/01 oder spaeter und dieses Werks.
%
% Dieses Werk hat den LPPL-Verwaltungs-Status "author-maintained"
% (allein durch den Autor verwaltet).
%
% Der Aktuelle Verwalter und Autor dieses Werkes ist Markus Kohm.
% 
% Dieses Werk besteht aus den in manifest.txt aufgefuehrten Dateien.
% ======================================================================
%
% Introduction of the KOMA-Script guide
% Maintained by Markus Kohm
%
% ----------------------------------------------------------------------
%
% Einleitung der KOMA-Script-Anleitung
% Verwaltet von Markus Kohm
%
% ======================================================================

\ProvidesFile{introduction.tex}[2012/01/31 KOMA-Script guide introduction]
\translator{Kevin Pfeiffer\and Gernot Hassenpflug\and Markus Kohm}

% Date of translated german file: 2012/01/01

\chapter{Introduction}
\labelbase{introduction}

This chapter includes important information about the structure of the manual
and the history of KOMA-Script, which begins years before the first
version. You will also find information in case that you have not installed
KOMA-Script or encounter errors.

\section{Preface}\label{sec:introduction.preface}

\KOMAScript{} is very complex. This is already explained by the fact that it
not only a single class or a single package, but a bundle of many classes and
packages. Although the classes are designed as a counterpart to the standard
classes, that does not particularly mean that they only have the commands,
environments, and setting of the standard classes or immitate their
appearance. The capabilities of \KOMAScript{} surpass the capabilities of the
standard classes considerably. Some of them are to be regarded as a supplement
to the basic skills of \LaTeX{} kernel.

The foregoing mean that the documentation of \KOMAScript{} has to be
extensive. In addition, \KOMAScript{} usually is not taught. That means there
is no teacher who knows his students and can therefore choose the teaching and
learning materials and adapt them accordingly. It would be easy to write the
documentation for any specifiy audience. The difficulty is, however, that the
guide is requires for all potential audiences. We, the authors, have tried to
create a guide that is suited for the computer scientist as well as the
secretary of the fishmonger. We have tried, although this is actually dealing
with an impossible task. The result consists of several compromises and we
hope that you will keep this in mind when using it.  Your suggestions for
improvement are, of course, always welcome.

Despite the volume of the manual in case of problems it is recommended to
consult the documentation. Attention is dawn to the multi-part index at the
end of this document. In addition to this guide documentation includes all the
text documents that are part of the bundle. See \File{manifest.tex} for a list
of all of them.


\section{Structure of the Guide}\label{sec:introduction.structure}

This manual consists of several parts. There's a part for average users,
another part for advanced users and experts and an appendix with additional
examples and information for those, who always want to know more.

\expandafter\MakeUppercase\autoref{part:forAuthors} is recommended for all
\KOMAScript{} users. This means that you may find here even some information
for newcomers to \LaTeX. In particular, this part is enhanced by many examples
to the average user, that are intended to illustrate the explanations. Do not
be afraid to try these examples yourself and in modifying them to find out how
\KOMAScript{} works. Nevertheless the {\KOMAScript} user guide is not intended
to be a {\LaTeX} primer.  Those new to {\LaTeX} should look at \emph{The Not
  So Short Introduction to {\LaTeXe}} \cite{lshort} or \emph{{\LaTeXe} for
  Authors} \cite{latex:usrguide} or a {\LaTeX} reference book. You will also
find useful information in the many {\LaTeX} FAQs, including the \emph{{\TeX}
  Frequently Asked Questions on the Web} \cite{UK:FAQ}. The volume of
\cite{UK:FAQ} is significantly too. Nevertheless it is recommended to consult
it, if you have problems using \LaTeX{} or \KOMAScript.

\expandafter\MakeUppercase\autoref{part:forExperts} is recommended for
advanced \KOMAScript{} users. These are all of you, who already know \LaTeX{},
maybe worked with \KOMAScript{} for a while, and want to learn more about
\KOMAScript{} internals, interaction of \KOMAScript{} with other packages, and
how to use \KOMAScript{} as an answer to special demands. For this purpose we
will have a closer look to some aspects from
\autoref{part:forAuthors} again. In
addition some instructions will be documented, that have been implemented for
advanced users and experts especially. This is complemented by the
documentation of packages that are normally hidden to the user insofar as they
do their work under the surface of the classes and user packages. These
packages are specifically designed to be used by other authors of classes and
packages.

The appendix, which may be found only in the German book version, is aimed at
those who do not meet all this information. Advanced users may find background
information on issues of typography there to give them a basis for their own
decisions. In addition, they provide examples for aspiring authors of
packages. These examples are less intended to be simply transferred. Rather,
they convey knowledge of planning and implementation of projects as well as
some basic \LaTeX{} instructions for authors of packages.

If you are only interested in using a single {\KOMAScript} class or package
you can probably successfully avoid reading the entire guide.  Each class and
package typically has its own chapter; however, the three main classes
(\Class{scrbook}, \Class{scrrprt}, and \Class{scrartcl}) are introduced
together in \autoref{cha:maincls}. Where an example or note only
applies to one or two of the three classes, e.g.,
\Class{scrartcl}\OnlyAt{\Class{scrartcl}} it is called out in the margin. Like
shown here with \Class{scrartcl}.

\begin{Explain} 
  The primary documentation for {\KOMAScript} is in German and has been
  translated for your convenience; like most of the {\LaTeX} world, its
  commands, environments, options, etc., are in English. In a few cases, the
  name of a command may sound a little strange, but even so, we hope and
  believe that with the help of this guide {\KOMAScript} will be usable
  and useful to you.
\end{Explain}


\section{History of {\KOMAScript}}\label{sec:introduction.history}

In the early 1990s, Frank Neukam needed a method to publish an
instructor's lecture notes. At that time {\LaTeX} was {\LaTeX}2.09 and there
was no distinction between classes and packages\,---\,there were only
\emph{styles}.  Frank felt that the standard document styles were not
good enough for his work; he wanted additional commands and
environments. At the same time he was interested in typography and,
after reading Tschichold's \emph{Ausgew�hlte Aufs�tze �ber Fragen der
Gestalt des Buches und der Typographie} (Selected Articles on the
Problems of Book Design and Typography) \cite{JTsch87}, he decided to
write his own document style\,---\,and not just a one-time solution to his
lecture notes, but an entire style family, one specifically designed for
European and German typography. Thus {\Script} was born.

Markus Kohm, the developer of {\KOMAScript}, came across {\Script} in December
1992 and added an option to use the A5 paper format. At this time neither the
standard style nor {\Script} provide support for A5 paper. Therefore is did
not take long until Markus made the first changes to {\Script}. This and other
changes were then incorporated into {\ScriptII}, released by Frank in December
1993.

Beginning in mid-1994, {\LaTeXe} became available and brought with it
many changes. Users of {\ScriptII} were faced with either limiting their
usage to {\LaTeXe}'s compatibility mode or giving up {\Script}
altogether.  This situation led Markus to put together a new {\LaTeXe}
package, released on 7~July 1994 as {\KOMAScript}; a few months later
Frank declared {\KOMAScript} to be the official successor to {\Script}.
{\KOMAScript} originally provided no \emph{letter} class, but this
deficiency was soon remedied by Axel Kielhorn, and the result became part
of {\KOMAScript} in December 1994.  Axel also wrote the first true
German-language user guide, which was followed by an English-language
guide by Werner Lemberg.

Since then much time has passed. {\LaTeX} has changed in only minor
ways, but the {\LaTeX} landscape has changed a great deal; many new
packages and classes are now available and {\KOMAScript} itself has
grown far beyond what it was in 1994. The initial goal was to provide
good {\LaTeX} classes for German-language authors, but today its
primary purpose is to provide more-flexible alternatives to the
standard classes. {\KOMAScript}'s success has led to e-mail from users
all over the world, and this has led to many new macros\,---\,all
needing documentation; hence this ``small guide.''


\section{Special Thanks}
\label{sec:introduction.thanks}

Acknowledgements in the introduction? No, the proper acknowledgements can be
found in the addendum. My comments here are not intended for the authors of
this guide\,---\,and those thanks should rightly come from you, the reader,
anyhow. I, the author of {\KOMAScript}, would like to extend my personal thanks
to Frank Neukam.  Without his {\Script} family, {\KOMAScript} would not have
come about.  I am indebted to the many persons who have contributed to
{\KOMAScript}, but with their indulgence, I would like to specifically mention
Jens-Uwe Morawski and Torsten Kr\"uger. The English translation of the guide
is, among many other things, due to Jens's untiring commitment. Torsten was
the best beta-tester I ever had. His work has particularly enhanced the
usability of \Class{scrlttr2} und \Class{scrpage2}. Many thanks to all who
encouraged me to go on, to make things better and less error-prone, or to
implement additional features.

Thanks go as well to DANTE, Deutschsprachige
Anwendervereinigung {\TeX}~e.V\kern-.18em, (the German-Language {\TeX} User Group).
Without the DANTE server, {\KOMAScript} could not have been released and
distributed. Thanks as well to everybody in the {\TeX} newsgroups and mailing
lists who answer questions and have helped me to provide support for
{\KOMAScript}.


\section{Legal Notes}
\label{sec:introduction.legal}

{\KOMAScript} was released under the {\LaTeX} Project Public License. You
will find it in the file \File{lppl.txt}. An unofficial German-language
translation is also available in \File{lppl-de.txt} and is valid for all
German-speaking countries.

\iffree{This document and the {\KOMAScript} bundle are provided ``as is'' and
without warranty of any kind.}%
{Please note: the printed version of this guide is not free under the
conditions of the {\LaTeX} Project Public Licence. If you need a free version
of this guide, use the version that is provided as part of the {\KOMAScript}
bundle.}


\section{Installation}
\label{sec:introduction.installation}

The three most important \TeX{} distributions, Mac\TeX, MiK\TeX, and \TeX{}
Live, make \KOMAScript{} available by their package management software. It is
recommended to make installations and updates of \KOMAScript{} using these
tools. Nevertheless the manual installation without using the package managers
has been described in the file \File{INSTALL.txt}, that is part of every legal
\KOMAScript{} bundle. You should also read the documentation that comes with
the {\TeX} distribution you are using.


\section{Bugreports and Other Requests}
\label{sec:introduction.errors}

If you think you have found an error in the documentation or a bug in one of
the {\KOMAScript} classes, one of the {\KOMAScript} packages, or another part
of {\KOMAScript}, please do the following: first have a look on CTAN to see if
a newer version of {\KOMAScript} is available; in this case install the
applicable section and try again.

If you are using the most recent version of {\KOMAScript} and still have a
bug, please provide a short {\LaTeX} document that demonstrates the
problem. You should only use the packages and definitions needed to
demonstrate the problem; do not use any unusual packages.

By preparing such an example it often becomes clear whether the problem is
truly a {\KOMAScript} bug or something else.  To find out the version numbers
of all packages in use, simply put \Macro{listfiles} in the preamble of your
example and read the end of the \File{log}-file.

Please report {\KOMAScript} (only) bugs to
\href{mailto:komascript@gmx.info}{komascript@gmx.info}. If you want to ask
your question in a Usenet group, mailing list, or Internet forum, you should
also include such an example as part of your question.


\section{Additional Information}
\label{sec:introduction.moreinfos}

Once you become an experienced {\KOMAScript} user you may want to look at some
more advanced examples and information. These you will find on the
{\KOMAScript} documentation web site \cite{homepage}. The main language of the
site is German, but nevertheless English is welcome.

\endinput
%%% Local Variables: 
%%% mode: latex
%%% coding: iso-latin-1
%%% TeX-master: "../guide.tex"
%%% End: 
