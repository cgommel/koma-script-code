% ======================================================================
% tocbasic.tex
% Copyright (c) Markus Kohm, 2002-2016
%
% This file is part of the LaTeX2e KOMA-Script bundle.
%
% This work may be distributed and/or modified under the conditions of
% the LaTeX Project Public License, version 1.3c of the license.
% The latest version of this license is in
%   http://www.latex-project.org/lppl.txt
% and version 1.3c or later is part of all distributions of LaTeX 
% version 2005/12/01 or later and of this work.
%
% This work has the LPPL maintenance status "author-maintained".
%
% The Current Maintainer and author of this work is Markus Kohm.
%
% This work consists of all files listed in manifest.txt.
% ----------------------------------------------------------------------
% tocbasic.tex
% Copyright (c) Markus Kohm, 2002-2016
%
% Dieses Werk darf nach den Bedingungen der LaTeX Project Public Lizenz,
% Version 1.3c, verteilt und/oder veraendert werden.
% Die neuste Version dieser Lizenz ist
%   http://www.latex-project.org/lppl.txt
% und Version 1.3c ist Teil aller Verteilungen von LaTeX
% Version 2005/12/01 oder spaeter und dieses Werks.
%
% Dieses Werk hat den LPPL-Verwaltungs-Status "author-maintained"
% (allein durch den Autor verwaltet).
%
% Der Aktuelle Verwalter und Autor dieses Werkes ist Markus Kohm.
% 
% Dieses Werk besteht aus den in manifest.txt aufgefuehrten Dateien.
% ======================================================================
%
% Package tocbasic for Package and Class Authors
% Maintained by Markus Kohm
%
% ----------------------------------------------------------------------
%
% Paket tocbasic fuer Paket- und Klassenautoren
% Verwaltet von Markus Kohm
%
% ======================================================================

\KOMAProvidesFile{tocbasic.tex}
                 [$Date$
                  KOMA-Script guide (package tocbasic)]

% Date of translated German file: 2016/06/12

\translator{Markus Kohm\and Arndt Schubert}

\chapter[{Management of Tables and Lists of Contents Using
  \Package{tocbasic}}]
  {Management of Tables and Lists of Contents Using
  \Package{tocbasic}\protect\footnote{Translation of this chapter has been
    made by the package author and needs editing!}}
\labelbase{tocbasic}

\BeginIndex{Package}{tocbasic}%
\BeginIndex{}{table of contents}%
\BeginIndex{}{list>of contents}%
\BeginIndex{}{file>extension}%
The main purpose of package \Package{tocbasic} is to provide features for
authors of classes and packages to create own tables or lists of contents like
the list of figures and the list of tables and thereby allow other classes or
packages some types of control over these. For examples package
\Package{tocbasic} delegates language control of all these tables and lists of
contents to package \Package{babel}\IndexPackage{babel} (see
\cite{package:babel}). So automatic change of language will be provided
inside all these tables and lists of contents. Using \Package{tocbasic} will
exculpate authors of classes and packages from implementation of such
features.

\KOMAScript{} itself uses \Package{tocbasic} not only for the table of
contents but also for the already mentioned lists of figures and
tables. In\textnote{TOC} this chapter we call all kinds of tables of contents
or lists of contents simply TOC\Index[indexmain]{TOC}.

\section{Basic Commands}
\label{sec:tocbasic.basics}

Basic commands are used to handle a list of all file name
extensions\textnote{file name extension, table or list of contents} known for
files representing a table or list of contents. We call these auxiliary
files\Index{auxiliary file}\footnote{Here we do not talk about the
  \File{aux}-file but the auxiliary files used indirect via the
  \File{aux}-file, e.\,g., the \File{toc}-file, the \File{lof}-file, or the
  \File{lot}-file.}  TOC-files\textnote{TOC-file}\Index[indexmain]{TOC-file}
independent from the file name extension that is used. Entries to such files
are typically written using
\Macro{addtocontents}\important{\Macro{addtocontents},
  \Macro{addcontentsline}} or \Macro{addcontentsline}. Later in this chapter
you will learn to know recommended, additional commands.  There are also
commands to do something for all known extensions. Additionally, there are
commands to set or unset features of a file name extension or the file
represented by the extension.  Typically an extension also has an
owner\textnote{file owner}.  This owner may be a class or package or a term
decided by the author of the class or package using \Package{tocbasic},
e.\,g., \KOMAScript{} uses the owner \texttt{float} for list of figures and
list of tables, and the file name of the class file as owner for the table of
contents.

\begin{Declaration}
  \Macro{ifattoclist}\Parameter{extension}\Parameter{true
    part}\Parameter{false part}
\end{Declaration}
\BeginIndex{Cmd}{ifattoclist}%
This command may be used to ask, whether or not a file name \PName{extension}
is already a known extension.  If the \PName{extension} is already known the
\PName{true instructions} will be used, otherwise the \PName{false
  instructions} will be used.
\begin{Example}
  Maybe you want to know if the file name extension ``\File{foo}'' is already
  in use to report an error, if you can not use it:
\begin{lstcode}
  \ifattoclist{foo}{%
    \PackageError{bar}{%
      extension `foo' already in use%
    }{%
      Each extension may be used only
      once.\MessageBreak
      The class or another package already
      uses extension `foo'.\MessageBreak
      This error is fatal!\MessageBreak
      You should not continue!}%
  }{%
    \PackageInfo{bar}{using extension `foo'}%
  }
\end{lstcode}
\end{Example}
\EndIndex{Cmd}{ifattoclist}%

\begin{Declaration}
  \Macro{addtotoclist}\OParameter{owner}\Parameter{extension}
\end{Declaration}
\BeginIndex{Cmd}{addtotoclist}%
This command adds the \PName{extension} to the list of known extensions. But
if the \PName{extension} is a known one already, then an error will be
reported to avoid double usage of the same \PName{extension}.

If the optional argument, \OParameter{owner}, was given, this \PName{owner}
will be stored to be the owner of the \PName{extension}.  If the optional
argument has been omitted, \Package{tocbasic} tries to find out the file name
of the current processed class or package and stores this as owner.
This\textnote{Attention!} will fail if \Macro{addtotoclist} was not used,
loading a class or package but using a command of a class or package after
loading this class or package.  In this case the owner will be set to
``\PValue{.}''.

Please note\textnote{Attention!} that an empty \PName{owner} is not the same
like omitting the optional argument with the braces. An empty argument would
result in an empty owner.
\begin{Example}
  You want to add the extension ``\File{foo}'' to the list of known extension,
  while loading your package with file name ``\File{bar.sty}'':
\begin{lstcode}
  \addtotoclist{foo}
\end{lstcode}%
  This will add the extension ``\PValue{foo}'' with owner ``\PValue{bar.sty}''
  to the list of known extensions, if it was not already in the list of known
  extensions. If the class or another package already added the extension you
  will get the error:
\begin{lstoutput}
  Package tocbasic Error: file extension `foo' cannot be used twice

  See the tocbasic package documentation for explanation.
  Type  H <return>  for immediate help.
\end{lstoutput}
  and after typing \texttt{h} and pressing the return key you will get the
  help:
\begin{lstoutput}
  File extension `foo' is already used by a toc-file, while bar.sty
  tried to use it again for a toc-file.
  This may be either an incompatibility of packages, an error at a package,
  or a mistake by the user.
\end{lstoutput}

  Maybe your package has a command, that creates list of files dynamically. In 
  this case you should use the optional argument of \Macro{addtotoclist} to
  set the owner.
\begin{lstcode}
  \newcommand*{\createnewlistofsomething}[1]{%
    \addtotoclist[bar.sty]{#1}%
    % Do something more to make this list of something available
  }
\end{lstcode}
  If the user calls now, e.\,g.,
\begin{lstcode}
  \createnewlistofsomething{foo}
\end{lstcode}
  this would add the extension ``\PValue{foo}'' with the owner
  ``\PValue{bar.sty}'' to the list of known extension or report an error, if
  the extension is already in use.
\end{Example}
You may use any owner you want.  But it should be unique!  So, if you would
be, e.\,g., the author of package \Package{float} you could use for example
owner ``\PValue{float}'' instead of owner ``\PValue{float.sty}'', so the
\KOMAScript{} options for the list of figures and the list of tables will also
handle the lists of this package. Those are already added to the known
extensions when the option is used. This is because \KOMAScript{} already
registers file name extension ``\PValue{lof}'' for the list of figures and
file name extension ``\PValue{lot}'' for the list of tables with owner
``\PValue{float}'' and sets options for this owner. Package \Package{scrhack}
redefines some of package \Package{float}'s commands to do this.%
\EndIndex{Cmd}{addtotoclist}%

\begin{Declaration}
  \Macro{AtAddToTocList}\OParameter{owner}\Parameter{instructions}
\end{Declaration}
\BeginIndex{Cmd}{AtAddToTocList}%
This command adds the \PName{instructions} to an internal list of instructions
that will be processed whenever a file name extension with the given
\PName{owner} will be added to the list of known extensions using
\Macro{addtotoclist}.  The optional argument is handled in the same way as
with the command \Macro{addtotoclist}. With an empty \PName{owner} you may
add \Parameter{instructions}, that will be processed at every successful
\Macro{addtotoclist}, after processing the instructions for the individual
owner.  While processing the \PValue{instructions},
\Macro{@currext}\IndexCmd{@currext}\important{\Macro{@currext}} will be set to
the extension of the currently added extension.
\begin{Example}
  \Package{tocbasic} itself uses
\begin{lstcode}
  \AtAddToTocList[]{%
    \expandafter\tocbasic@extend@babel
    \expandafter{\@currext}%
  }
\end{lstcode}
  to add every extension to the \Package{tocbasic}-internal babel handling of
  files.
\end{Example}

The two \Macro{expandafter} commands are needed, because the argument of
\Macro{tocbasic@extend@babel} has to be expanded!  See the description of
\Macro{tocbasic@extend@babel} at \autoref{sec:tocbasic.internals}%
, \autopageref{desc:tocbasic.cmd.tocbasic@extend@babel} for more information.%
\EndIndex{Cmd}{AtAddToTocList}%

\begin{Declaration}
  \Macro{removefromtoclist}\OParameter{owner}\Parameter{extension}
\end{Declaration}
\BeginIndex{Cmd}{removefromtoclist}%
This command removes the \PName{extension} from the list of known extensions.
If the optional argument, \OParameter{owner}, was given, the \PName{extension}
will only be removed if it was added by this \PName{owner}. See description of
\Macro{addtotoclist} for information of omitting optional argument. Note that
an empty \PName{owner} is not the same like omitting the optional argument,
but removes the \PName{extension} without any owner test.%
\EndIndex{Cmd}{removefromtoclist}%

\begin{Declaration}
  \Macro{doforeachtocfile}\OParameter{owner}\Parameter{instructions}
\end{Declaration}
\BeginIndex{Cmd}{doforeachtocfile}%
Until now you've learned to know commands that result in more safety in
handling file name extensions, but also needs some additional effort. With
\Macro{doforeachtocfile} you will win for this. The command provides to
processes \PName{instructions} for every known file name extension of the
given \PName{owner}.  While processing the \PName{instructions}
\Macro{@currext} is the extension of the current file.  If you omit the
optional argument, \OParameter{owner}, every known file name extension
independent from the owner will be used. If the optional argument is empty,
only file name extensions with an empty owner will be processed.
\begin{Example}
  If you want to type out all known extensions, you may simply write:
\begin{lstcode}
  \doforeachtocfile{\typeout{\@currext}}
\end{lstcode}
  and if only the extensions of owner ``\PValue{foo}'' should be typed out:
\begin{lstcode}
  \doforeachtocfile[foo]{\typeout{\@currext}}
\end{lstcode}
\end{Example}
\EndIndex{Cmd}{doforeachtocfile}%

\begin{Declaration}
  \Macro{tocbasicautomode}
\end{Declaration}
\BeginIndex{Cmd}{tocbasicautomode}%
This command redefines \LaTeX{} kernel macro \Macro{@starttoc} to add all not
yet added extensions to the list of known extensions and use
\Macro{tocbasic@starttoc} instead of \Macro{@starttoc}. See
\autoref{sec:tocbasic.internals},
\autopageref{desc:tocbasic.cmd.tocbasic@starttoc} for more information about
\Macro{tocbasic@starttoc} and \Macro{@starttoc}.

This means that after using \Macro{tocbasicautomode} every table of contents
or list of something, that will be generated using \Macro{@starttoc} will be
at least partially under control of \Package{tocbasic}. Whether or not this
will make the wanted result, depends on the individual TOC. At least the
\Package{babel} control extension for all those TOCs will work. Nevertheless,
it would be better if the author of the corresponding class or package will
use \Package{tocbasic} explicitly. In that case additional advantages of
\Package{tocbasic} may be used that will be described at the following
sections.%
\EndIndex{Cmd}{tocbasicautomode}%

\section{Creating a Table or List of Contents}
\label{sec:tocbasic.toc}

In the previous section you've seen commands to handle a list of known file
name extensions and to trigger commands while adding a new extension to this
list. You've also seen a command to do something for all known extensions or
all known extensions of one owner. In this section you will learn to know
commands to handle the file corresponding with an extension or the list of
known extensions.

\begin{Declaration}
  \Macro{addtoeachtocfile}\OParameter{owner}\Parameter{content}
\end{Declaration}
\BeginIndex{Cmd}{addtoeachtocfile}%
This command writes \PName{content} to the TOC-files\Index{TOC} of every known
file name extension of \PName{owner} using \LaTeX{} kernel command
\Macro{addtocontents}. If you omit the optional argument, \PName{content} is
written to the files of every known file name extension. Furthermore, the
practical file name is built from \Macro{jobname} and the file name
extension. While writing the \PName{content},
\Macro{@currext}\IndexCmd{@currext}\important{\Macro{@currext}} is the
extension of the currently handled file.
\begin{Example}
  You may add a vertical space of one text line to all toc-files.
\begin{lstcode}
    \addtoeachtocfile{%
      \protect\addvspace{\protect\baselineskip}%
    }
\end{lstcode}
  And if you want to do this, only for the TOC-files of owner
  ``\PValue{foo}'':
\begin{lstcode}
    \addtoeachtocfile[foo]{%
      \protect\addvspace{\protect\baselineskip}%
    }
\end{lstcode}
\end{Example}
Commands, that shouldn't be expanded while writing, should be prefixed by
\Macro{protect} in the same way like they should be in the argument of
\Macro{addtocontents}.
\EndIndex{Cmd}{addtoeachtocfile}%

\begin{Declaration}
  \Macro{addxcontentsline}%
  \Parameter{extension}\Parameter{level}\OParameter{number}%
  \Parameter{text}
\end{Declaration}
\BeginIndex{Cmd}{addxcontentsline}%
The command \Macro{addxcontentsline} adds an entry of given \PName{level} to
TOC-file with file name \PName{extension}. If the \Parameter{number} is empty
or omitted the entry will not have number for the entry with the given
\PName{text}. Entries without number may be left aligned to the number of the
numbered entries of the same \PName{level} or indented like the text of the
numbered entries of the same \PName{level}, depending on the
\PValue{numberline} feature.

\begin{Example}
  Maybe you are not using a \KOMAScript{} class but need a not numbered
  chapter with entry to the table of contents. This may be done using
\begin{lstcode}
  \cleardoublepage
  \csname phantomsection\endcsname
  \addxcontentsline{toc}{chapter}
             {Chapters without Numbers}
  \chapter*{Chapters without Numbers}
  \markboth{Chapters without Numbers}{}
\end{lstcode}
  As you can see, you simply have to replace usual \Macro{addcontentsline} by
  \Macro{addxcontentsline} to support the \Package{tocbasic} feature
  \PValue{numberline}.
\end{Example}

Note that \Macro{addxcontentsline} uses
\Macro{add\PName{level}\PName{extension}entry} if such a macro exists and
\Macro{tocbasic@addxcontentsline} otherwise. Therefore you cannot define
a macro \Macro{add\PName{level}\PName{extension}entry} using
\Macro{addxcontentsline} but \Macro{tocbasic@addxcontentsline}.

It is recommended to use \Macro{addxcontentsline} instead of
\Macro{addcontentsline} whenever possible.%
\EndIndex{Cmd}{addxcontentsline}

\begin{Declaration}
  \Macro{addcontentslinetoeachtocfile}\OParameter{owner}\Parameter{level}%
  \Parameter{contentsline}\\
  \Macro{addxcontentslinetoeachtocfile}\OParameter{owner}%
  \Parameter{level}\OParameter{number}\Parameter{text}
\end{Declaration}
\BeginIndex{Cmd}{addcontentslinetoeachtocfile}%
\BeginIndex{Cmd}{addxcontentslinetoeachtocfile}%
The first command is something like \Macro{addcontentsline} from \LaTeX{}
kernel. In difference to that it writes the \PName{contentsline} not only into
one file, but into all files of all known file name extensions or of all known
file name extensions of a given owner.

The command \Macro{addxcontentslinetoeachtocfile} is similar but uses
\Macro{addxcontentsline} instead of \Macro{addcontentsline} and
therefore supports \Package{tocbasic} feature \PValue{numberline}.
\begin{Example}
  You are a class author and want to write the chapter entry not only to the
  table of contents TOC-file but to all TOC-files, while \texttt{\#1} is the
  title, that should be written to the files.
\begin{lstcode}
    \addxcontentslinetoeachtocfile
              {chapter}[\thechapter]{#1}%
\end{lstcode}
  In this case the current chapter number should be expanded while writing
  into the file. So it isn't protected from expansion using \Macro{protect}.
\end{Example}
While writing \Macro{@currext}\IndexCmd{@currext}\important{\Macro{@currext}}
is the file name extension of the file into which \PName{contentsline} will be
written.

It is recommended to use \Macro{addxcontentslinetoeachtocfile} instead
of \Macro{addcontentslinetoeachtocfile} whenever possible.
\EndIndex{Cmd}{addxcontentslinetoeachtocfile}%
\EndIndex{Cmd}{addcontentslinetoeachtocfile}%

\begin{Declaration}
  \Macro{listoftoc}\OParameter{list of title}\Parameter{extension}\\
  \Macro{listoftoc*}\Parameter{extension}\\
  \Macro{listofeachtoc}\OParameter{owner}\\
  \Macro{listof\PName{extension}name}
\end{Declaration}
\BeginIndex{Cmd}{listoftoc*}%
\BeginIndex{Cmd}{listoftoc}%
\BeginIndex{Cmd}{listofeachtoc}%
\BeginIndex{Cmd}{listof\PName{extension}name}%
These commands may be used to print the TOC corresponding to file name
\PName{extension}. The\important{\Macro{listoftoc*}} star version
\Macro{listoftoc*} needs only one argument, the \PName{extension} of the
file. It does setup the vertical and horizontal spacing of paragraphs, calls
before hooks, reads the file, and last but not least calls the after hooks.
You may interpret it as direct replacement of the \LaTeX{} kernel macro
\Macro{@starttoc}\IndexCmd{@starttoc}\important{\Macro{@starttoc}}.

The\important{\Macro{listoftoc}} version without star, prints the whole file
with title, optional table of contents entry, and running heads. If the
optional argument \OParameter{list of title} was given, it will be used as
title term, optional table of contents entry and running head. Please
note\textnote{Attention!}: If the optional argument is empty, this term will
be empty, too! If you omit the optional argument, but
\Macro{listof\PName{extension}name} was defined, that will be used. If that is
also not defined, a standard replacement name will be used and reported by a
warning message.

The\important{\Macro{listofeachtoc}} command \Macro{listofeachtoc} outputs all
lists of something of the given \PName{owner} or of all known file
extensions. Thereby\textnote{Attention!}
\Macro{listof\PName{extension}name} should be defined to get the correct
titles.

It\textnote{Hint!} is recommended to define
\Macro{listof\PName{extension}name} for all used file name extensions, because
the user itself may use \Macro{listofeachtoc}.
\begin{Example}
  Assumed, you have a new ``list of algorithms'' with extension \PValue{loa}
  and want to show it:
\begin{lstcode}
  \listoftoc[List of Algorithms]{loa}
\end{lstcode}
  will do it for you. But maybe the ``list of algorithms'' should not be set
  with a title. So you may use
\begin{lstcode}
  \listof*{loa}
\end{lstcode}
  Note that in this case no entry at the table of contents will be created,
  even if you'd used the setup command above.
  See command \Macro{setuptoc}
  at \autopageref{desc:tocbasic.cmd.setuptoc}
  for more information about the
  attribute of generating entries into the table of contents using
  \Macro{setuptoc}.

  If you've defined
\begin{lstcode}
  \newcommand*{\listofloaname}{%
    List of Algorithms%
  }
\end{lstcode}
  before, then
\begin{lstcode}
  \listoftoc{loa}
\end{lstcode}
  would be enough to print the list of algorithms with the wanted heading. For
  the user it may be easier to operate, if you'd define
\begin{lstcode}
  \newcommand*{\listofalgorithms}{\listoftoc{loa}}
\end{lstcode}
  additionally.
\end{Example}

Because\textnote{Attention!} \LaTeX{} normally opens a new file for each of
those lists of something immediately, the call of each of those commands may
result in an error like:
\begin{lstoutput}
  ! No room for a new \write .
  \ch@ck ...\else \errmessage {No room for a new #3}
                                                    \fi
\end{lstoutput}
if there are no more write handles left. Loading package
\Package{scrwfile}\important{\Package{scrwfile}}\IndexPackage{scrwfile}
(see \autoref{cha:scrwfile})
may solve this problem.
\EndIndex{Cmd}{listof\PName{extension}name}%
\EndIndex{Cmd}{listofeachtoc}%
\EndIndex{Cmd}{listoftoc*}%
\EndIndex{Cmd}{listoftoc}%

\begin{Declaration}
  \Macro{BeforeStartingTOC}\OParameter{extension}\Parameter{instructions}\\
  \Macro{AfterStartingTOC}\OParameter{extension}\Parameter{instructions}
\end{Declaration}
\BeginIndex{Cmd}{BeforeStartingTOC}%
\BeginIndex{Cmd}{AfterStartingTOC}%
Sometimes it's useful, to process \PName{instructions} immediately before
reading the auxiliary file of a TOC.  These commands may be used to process
\PName{instructions} before or after loading the file with given
\PName{extension} using \Macro{listoftoc*}, \Macro{listoftoc}, or
\Macro{listofeachtoc}.  If you omit the optional argument (or set an empty
one) the general hooks will be set. The general before hook will be called
before the individuel one and the general after hook will be called after the
individuel one. While calling the hooks
\Macro{@currext}\IndexCmd{@currext}\important{\Macro{@currext}} is the
extension of the TOC-file and should not be changed.

An example\textnote{Example} for usage of \Macro{AfterStartingTOC} may be
found in \autoref{sec:scrwfile.clonefilefeature} at
\autopageref{example:scrwfile.AfterStartingTOC}.
\EndIndex{Cmd}{AfterStartingTOC}%
\EndIndex{Cmd}{BeforeStartingTOC}%

\begin{Declaration}
  \Macro{BeforeTOCHead}\OParameter{extension}\Parameter{instructions}\\
  \Macro{AfterTOCHead}\OParameter{extension}\Parameter{instructions}
\end{Declaration}
\BeginIndex{Cmd}{BeforeTOCHead}%
\BeginIndex{Cmd}{AfterTOCHead}%
This commands may be used to process \PName{instructions} before or after
setting the title of a TOC corresponding to given file name \PName{extension}
using \Macro{listoftoc*} or \Macro{listoftoc}. If you omit the optional
argument (or set an empty one) the general hooks will be set. The general
before hook will be called before the individuel one and the general after
hook will be called after the individuel one. While calling the hooks
\Macro{@currext}IndexCmd{@currext}\important{\Macro{@currext}} is the
extension of the corresponding file and should not be changed.
\EndIndex{Cmd}{AfterTOCHead}%
\EndIndex{Cmd}{BeforeTOCHead}%

\begin{Declaration}
  \Macro{MakeMarkcase}\Parameter{text}
\end{Declaration}
\BeginIndex{Cmd}{MakeMarkcase}%
Whenever \Package{tocbasic} sets a mark for a running head, The text of the
mark will be an argument of \Macro{MakeMarkcase}. This command may be used, to
change the case of the letters at the running head if wanted. The default is,
to use
\Macro{@firstofone}\IndexCmd{@firstofone}\important{\Macro{@firstofone}} for
\KOMAScript{} classes. This means the text of the running head will be set
without change of case.
\Macro{MakeUppercase}\IndexCmd{MakeUppercase}\important{\Macro{MakeUppercase}}
will be used for all other classes. If you are the class author you may define
\Macro{MakeMarkcase} on your own. If \Package{scrlayer} or another package,
that defines \Macro{MakeMarkcase} will be used, \Package{tocbasic} will not
overwrite that definition.
\begin{Example}
  For incomprehensible reasons, you want to set the running heads in lower
  case letters only. To make this automatically for all running heads, that
  will be set by \Package{tocbasic}, you define:
\begin{lstcode}
  \let\MakeMarkcase\MakeLowercase
\end{lstcode}
\end{Example}
Please\textnote{Hint!} allow me some words about
\Macro{MakeUppercase}\IndexCmd{MakeUppercase}, First of all this command isn't
fully expandable. This means, that problems may occur using it in the context
of other commands. Beyond that typographers accord, that whenever setting
whole words or phrases in capitals, letter spacing is absolutely
necessary. But correct letter spacing of capitals should not be done with a
fix white space between all letters. Different pairs of letters need different
space between each other. Additional some letters build holes in the text,
that have to be taken into account. Packages like \Package{ulem} or
\Package{soul} doesn't provide this and \Macro{MakeUppercase} does not do
anything like this. Also automatic letter spacing using package
\Package{microtype} is only one step to a less-than-ideal solution, because it
cannot recognize and take into account the glyphs of the letters. Because of
this\textnote{Attention!} typesetting whole words and phrases is expert work
and almost ever must be hand made. So average users are recommended to not do
that or to use it only spare and not at exposed places like running heads.%
\EndIndex{Cmd}{MakeMarkcase}%

\begin{Declaration}
  \Macro{deftocheading}\Parameter{extension}\Parameter{definition}
\end{Declaration}
\BeginIndex{Cmd}{deftocheading}%
The package \Package{tocbasic} contains a standard definition for typesetting
headings of TOCs. This standard definition is configurable by several
features, described at \Macro{setuptoc} next. But if all those features are
not enough, an alternative heading command may be defined using
\Macro{deftocheading}. Thereby \PName{extension} is the file name extension of
the corresponding TOC-file. The \PName{definition} of the heading command may
use one single parameter \PValue{\#1}. While calling the newly defined command
inside of \Macro{listoftoc} or \Macro{listofeachtoc} that \PValue{\#1} will be
replaced by the corresponding heading term.%
\EndIndex{Cmd}{deftocheading}%

\begin{Declaration}
  \Macro{setuptoc}\Parameter{extension}\Parameter{feature list}\\
  \Macro{unsettoc}\Parameter{extension}\Parameter{feature list}
\end{Declaration}
\BeginIndex{Cmd}{setuptoc}%
\BeginIndex{Cmd}{unsettoc}%
This commands set up and unset features bound to a file name
\PName{extension}. The \PName{feature list} is a comma separated list of
single features. \Package{tocbasic} does know following features:
\begin{description}
\item[\texttt{leveldown}] uses not the top level heading below
  \Macro{part}\,---\,\Macro{chapter} if available, \Macro{section}
  otherwise\,---\,but the first sub level. This feature will be evaluated by
  the internal heading command. On the other hand, if an user defined heading
  command has been made with \Macro{deftocheading}, that user is responsible
  for the evaluation of the feature. The \KOMAScript{} classes set this
  feature using option
  \OptionValue{listof}{leveldown}important{\OptionValue{listof}{leveldown}}%
  \IndexOption{listof~=\PValue{leveldown}} for all file name extensions of the
  owner \PValue{float}.
\item[\texttt{nobabel}] prevents usage of the language switch of
  \Package{babel}\IndexPackage{babel} at the TOC-file with the corresponding
  file name \PName{extension}. This feature should be used only for auxiliary
  files, that contain text in one language only. Changes of the language
  inside of the document will not longer regarded at the TOC-file. Package
  \Package{scrwfile}\important{\Package{scrwfile}}\IndexPackage{scrwfile} uses
  this feature also for clone destinations, because those will get the
  language change from the clone source already.

  Please note\textnote{Attention!}, the feature is executed while adding a
  file extension to the list of known file extension. Changing the feature
  afterwards would not have any effect.
\item[\texttt{noparskipfake}] prevents\ChangedAt{v3.17}{\Package{tocbasic}}
  usage of an extra \Length{parskip} before switching \Length{parskip} off. In
  general, the consequence of this feature for documents using paragraph
  distance is less vertical space between heading and first entry than between
  normal headings and normal text.
\item[\texttt{noprotrusion}] prevents\ChangedAt{v3.10}{\Package{tocbasic}}
  disabling character protrusion at the TOC. Character protrusion at the TOCs
  will be disabled by default if package
  \Package{microtype}\IndexPackage{microtype} or another package, that
  supports \Macro{microtypesetup}\IndexCmd{microtypesetup}, was loaded. So if
  you want protrusion at a TOC, you have to set this feature. But
  note\textnote{Attention!}, with character protrusion TOC-entries may be
  printed wrong. This is a known issue of character protrusion.
\item[\texttt{numbered}] uses a numbered heading for the TOC and because of
  this also generates an entry to the table of contents. This feature will be
  evaluated by the internal heading command. On the other hand, if an user
  defined heading command has been made with \Macro{deftocheading}, that user
  is responsible for the evaluation of the feature. The \KOMAScript{} classes
  set this feature using option
  \OptionValue{listof}{numbered}\important{\OptionValue{listof}{numbered}}%
  \IndexOption{listof~=\PValue{numbered}} for all file name extensions of the
  owner \PValue{float}.
\item[\texttt{numberline}] \leavevmode\ChangedAt{v3.12}{\Package{tocbasic}}%
  redefines \Macro{nonumberline} to use \Macro{numberline}. With this the not
  numbered entries generated by \KOMAScript{} or using \Macro{nonumberline} at
  the very beginning of the last argument of \Macro{addcontentline} will also
  be indented like numbered entries of the same
  type. Using\ChangedAt{v3.20}{\Package{tocbasic}} \Macro{numberline} for
  entries without number\ChangedAt{v3.20}{\Package{tocbasic}} can have
  additional side effects if you use entry style \PValue{tocline}. See style
  attribute \Option{breakafternumber} and \Option{entrynumberformat} in
  \autoref{tab:tocbasic.tocstyle.attributes} from
  \autopageref{tab:tocbasic.tocstyle.attributes} downwards.

  \KOMAScript{} classes set this feature for the file name extensions of the
  owner \PValue{float} if you use option \OptionValue{listof}{numberline}%
  \important{\OptionValue{listof}{numberline}}%
  \IndexOption{listof~=\PValue{numberline}} and for file name extension
  \PValue{toc} if you use option
  \OptionValue{toc}{numberline}\important{\OptionValue{toc}{numberline}}%
  \IndexOption{toc~=\PValue{numberline}}. Analogous they reset this feature if
  you use \OptionValue{listof}{nonumberline}%
  \important{\OptionValue{listof}{nonumberline}}%
  \IndexOption{listof~=\PValue{nonumberline}} or
  \OptionValue{toc}{nonumberline}%
  \important{\OptionValue{toc}{nonumberline}}%
  \IndexOption{toc~=\PValue{nonumberline}}.
\item[\texttt{onecolumn}] \leavevmode\ChangedAt{v3.01}{\Package{tocbasic}}%
  typesets the corresponding TOC with internal one column mode of
  \Macro{onecolumn}\IndexCmd{onecolumn}. This\textnote{Attention!} will be
  done only, if the corresponding table of contents or list of something does
  not use feature \PValue{leveldown}\important{\PValue{leveldown}}. The
  \KOMAScript{} classes \Class{scrbook} and \Class{scrreprt} activate this
  feature with \Macro{AtAddToTocList} (see \autoref{sec:tocbasic.basics},
  \autopageref{desc:tocbasic.cmd.AtAddToTocList}) for all TOCs with owner
  \PValue{float} or with themselves as owner. With this, e.\,g., the table of
  contents, the list of figures and the list of tables of both classes will be
  single columned automatically. The multiple-column-mode of package
  \Package{multicol}\IndexPackage{multicol} will not be recognised or changed
  by this option.
\item[\texttt{totoc}] writes the title of the corresponding TOC to the table
  of contents. This feature will be evaluated by the internal heading
  command. On the other hand, if an user defined heading command has been made
  with \Macro{deftocheading}, that user is responsible for the evaluation of
  the feature. The \KOMAScript{} classes set this feature using option
  \OptionValue{listof}{totoc}\important{\OptionValue{listof}{totoc}}%
  \IndexOption{listof~=\PValue{totoc}} for all file name extensions of the
  owner \PValue{float}.
\end{description}
Classes and packages may know features, too, e.\,g, the \KOMAScript{} classes
know following additional features:
\begin{description}
\item[\texttt{chapteratlist}] activates special code to be put into the TOC at
  start of a new chapter. This code may either be vertical space or the
  heading of the chapter.  See
  \Option{listof}\IndexOption{listof}\important{\Option{listof}} in
  \autoref{sec:maincls.floats}, \autopageref{desc:maincls.option.listof} for
  more information about such features.
\end{description}
\begin{Example}
  Because \KOMAScript{} classes use \Package{tocbasic} for the list of figures
  and list of tables, there's one more way to remove chapter structuring at
  those:
\begin{lstcode}
  \unsettoc{lof}{chapteratlist}
  \unsettoc{lot}{chapteratlist}
\end{lstcode}

  And if you want to have the chapter structuring of the \KOMAScript{} classes
  at your own list of algorithms with file name \PName{extension}
  ``\PValue{loa}'' from the previous examples, you may use
\begin{lstcode}
  \setuptoc{loa}{chapteratlist}
\end{lstcode}
  And if classes with \Macro{chapter} should also force single column mode for
  the list of algorithms you may use
\begin{lstcode}
  \ifundefinedorrelax{chapter}{}{%
    \setuptoc{loa}{onecolumn}%
  }
\end{lstcode}
  Usage of \Macro{ifundefinedorrelax} presumes package \Package{scrbase} (see
  \autoref{sec:scrbase.if},
  \autopageref{desc:scrbase.cmd.ifundefinedorrelax}).

  It\textnote{Hint!} doesn't matter if you're package would be used with
  another class. You should never the less set this feature. And if the other
  class would also recognise the feature your package would automatically use
  the feature of that class.
\end{Example}
As you may see, packages that use \Package{tocbasic}, already provide several
interesting features, without the need of a lot of implementation effort. Such
an effort would be needed only without \Package{tocbasic} and because of this,
most packages currently lack of such features.%
\EndIndex{Cmd}{unsettoc}%
\EndIndex{Cmd}{setuptoc}%

\begin{Declaration}
  \Macro{iftocfeature}\Parameter{extension}\Parameter{feature}%
  \Parameter{true-instructions}\Parameter{false-instructions}
\end{Declaration}
\BeginIndex{Cmd}{iftocfeature}%
This command may be used, to test, if a \PName{feature} was set for file name
\PName{extension}. If so the \PName{true-instructions} will be processed,
otherwise the \PName{false-instruction} will be. This may be useful, e.\,g.,
if you define your own heading command using \Macro{deftocheading} but want to
support the features \PValue{totoc}, \PValue{numbered} or \PValue{leveldown}.%
\EndIndex{Cmd}{iftocfeature}%


\section{Configuration of Entries to a Table or List of Contents}
\seclabel{tocstyle}%
\BeginIndex{}{table of contents>entry}%
\Index{list of contents|\see{table of contents}}

Since\ChangedAt[2016/03]{v3.20}{\Package{tocbasic}} version~3.20 package
\Package{tocbasic} can not only configure the tables or lists of contents and
their auxiliary files but also influence their entries. To do so, you can
define new styles or you can use and configure one of the predefined
styles. In the medium term \Package{tocbasic} will replace the experimental
package \Package{tocstyle} that never became an official part of the
\KOMAScript{} bundle. The \KOMAScript{} classes intensively use the TOC-entry
styles of \Package{tocbasic} since version~3.20.

\begin{Declaration}
  \Macro{numberline}\Parameter{entry number}\\
  \Macro{usetocbasicnumberline}\OParameter{code}
\end{Declaration}
\BeginIndex{Cmd}{numberline}%
\BeginIndex{Cmd}{usetocbasicnumberline}%
Though\ChangedAt[2016/03]{v3.20}{\Package{tocbasic}} the \LaTeX{} kernel
already defines command \Macro{numberline}, the kernel definition is not
sufficient for \Package{tocbasic}. Therefore \Package{tocbasic} has its own
definition of \Macro{numberline}. The package uses
\Macro{usetocbasicnumberline} to activate this definition whenever a TOC-entry
needs it. Because of this, re-defining \Macro{numberline} often does not work
and even may result in warnings if you use \Package{tocbasic}.

You can use the definition of \Package{tocbasic} putting
\Macro{usetocbasicnumberline} into your document's preamble. The command first
of all checks, whether or not the current definition of \Macro{numberline}
uses essential, internal commands of \Package{tocbasic}. Only if this is not
the case \Macro{usetocbasicnumberline} redefines \Macro{numberline} and
executes \PName{code}. If you omit the optional argument, a
\Macro{PackageInfo} outputs a message about the successful re-definition. If
you just do not want such a message, use an empty optional argument.

Please note\textnote{Attention!}, as a side effect
\Macro{usetocbasicnumberline} can globally change the internal switch
\Macro{@tempswa}!%
\EndIndex{Cmd}{usetocbasicnumberline}%
\EndIndex{Cmd}{numberline}%


\begin{Declaration}
  \Macro{DeclareTOCStyleEntry}\OParameter{option list}\Parameter{style}%
                              \Parameter{entry level}
\end{Declaration}
\BeginIndex{Cmd}{DeclareTOCStyleEntry}%
This\ChangedAt[2016/03]{v3.20}{\Package{tocbasic}} command is used to define
or configure the TOC-entries of a given \PName{entry level}. Argument
\PName{entry level} is a symbolic name, e.\,g., \PValue{section} for the entry
to the table of contents of the section level with the same name or
\PValue{figure} for an entry of a figure to the list of figures. A \PName{style}
is assigned to each \PName{entry level}. The \PName{style} has to be defined
before using it as an argument of \Macro{DeclareTOCStyleEntry}. The
\PName{option list} is used to configure several \PName{style}-dependent
attributes of the \PName{entry level}.

Currently, \Package{tocbasic} defines the following entry styles:
\begin{description}
\item[\PValue{default}] defaults to a clone of style
  \PValue{dottedtocline}. It is recommended to class authors, who use
  \Package{tocbasic}, to change this style into the default style of the class
  using \Macro{CloneTOCStyle}. For example the \KOMAScript{} classes change
  \PValue{default} into a clone of \PValue{tocline}.
\item[\PValue{dottedtocline}] is similar to the style used by the standard
  classes \Class{book} and \Class{report} for the \PValue{section} down to
  \PValue{subparagraph} entry levels of the table of contents and for the
  entries at the list of figures or list of tables. It supports the attributes
  \PValue{level}, \PValue{indent}, and \PValue{numwidth}. The entries will be
  indented by the value of \PValue{indent} from the left. The width of the
  entry number is given by the value of \PValue{numwidth}. For multiline
  entries the indent will be increased by the value of \PValue{numwidth} for
  the second and following lines. The page number is printed using
  \Macro{normalfont}\IndexCmd{normalfont}. Entry text and page number are
  connected by a dotted line. \hyperref[fig:tocbasic.dottedtocline]%
  {Figure~\ref*{fig:tocbasic.dottedtocline}} illustrates the attributes of
  this style.
  \begin{figure}
    \centering
    \resizebox{.8\linewidth}{!}{%
      \begin{tikzpicture}
        \draw[color=lightgray] (0,2\baselineskip) -- (0,-2.5\baselineskip);
        \draw[color=lightgray] (\linewidth,2\baselineskip) --
                               (\linewidth,-2.5\baselineskip);
        \node (dottedtocline) at (0,0) [anchor=west,inner sep=0,outer sep=0]
        {%
          \hspace*{7em}%
          \parbox[t]{\dimexpr\linewidth-9.55em}{%
            \setlength{\parindent}{-3.2em}%
            \addtolength{\parfillskip}{-2.55em}%
            \makebox[3.2em][l]{1.1.10}%
            Text of an entry to the table of contents with style
            \PValue{dottedtocline} and more than one line%
            \leaders\hbox{$\csname m@th\endcsname
              \mkern 4.5mu\hbox{.}\mkern 4.5mu$}\hfill\nobreak
            \makebox[1.55em][r]{12}%
          }%
        };
        \draw[|-|,color=gray,overlay] (0,0) --
                              node [anchor=north,font=\small] {
                                \PValue{indent}
                              }
                              (3.8em,0);
        \draw[|-|,color=gray,overlay] (3.8em,\baselineskip) -- 
                              node [anchor=south,font=\small] {
                                \PValue{numwidth}
                              }
                              (7em,\baselineskip);
        \draw[|-|,color=gray,overlay] (\linewidth,\ht\strutbox) -- 
                              node [anchor=south,font=\small] { 
                                \Macro{@tocrmarg} 
                              }
                              (\linewidth-2.55em,\ht\strutbox);
        \draw[|-|,color=gray,overlay] (\linewidth,-\baselineskip) -- 
                              node [anchor=north,font=\small] { 
                                \Macro{@pnumwidth} 
                              } 
                              (\linewidth-1.55em,-\baselineskip);
      \end{tikzpicture}%
    }
    \caption{Illustrations of some attributes of a TOC-entry with style 
      \PValue{dottedtocline}}
    \label{fig:tocbasic.dottedtocline}
  \end{figure}
\item[\PValue{gobble}] is the most ordinary style. Independently from the
  setting of
  \Counter{tocdepth}\IndexCounter{tocdepth}\important{\Counter{tocdepth}},
  entries with this style will never be printed.  The style simply gobbles the
  entries. Nevertheless, it supports the standard attribute \PValue{level} but
  does ignore it.
\item[\PValue{largetocline}] is similar to the style used by the
  standard classes for the level \PValue{part}. It supports attributes
  \PValue{level} and \PValue{indent} only. The last one is already a variation
  of the standard classes that do not support an indent of the \PValue{part}
  entries.

  Before an entry a page break will be made easier. The entries will be
  indented by the value of \PValue{indent} from the left. They are printed
  using \Macro{large}\Macro{bfseries}. If \Macro{numberline} is used, the
  number width is 3\Unit{em}. \Macro{numberline} is not redefined. The
  standard classes do not use \Macro{numberline} for \PName{part} entries. The
  value of \PName{indent} even does not influence the indent from the second
  line of an entry.

  \hyperref[fig:tocbasic.largetocline]%
  {Figure~\ref*{fig:tocbasic.largetocline}} illustrates the attributes of
  this style. There you can also see that the style copies inconsistencies of
  the standard classes, e.\,g. the missing indent of the second and following
  lines of an entry or the different values of \Macro{@pnumwidth} that results
  from the font size dependency. This can even result in a to small distance
  between the entry text and the page number. Please note, the entry number
  width shown in the figure is only valid if \Macro{numberline} has been
  used. In contrast, the standard classes use a distance of 1\Unit{em} after
  the number.
  \begin{figure}
    \centering
    \resizebox{.8\linewidth}{!}{%
      \begin{tikzpicture}
        \draw[color=lightgray] (0,2\baselineskip) -- (0,-2.5\baselineskip);
        \draw[color=lightgray] (\linewidth,2\baselineskip) --
                               (\linewidth,-2.5\baselineskip);
        \node (largetocline) at (0,0) [anchor=west,inner sep=0,outer sep=0] {%
          \parbox[t]{\dimexpr \linewidth-1.55em\relax}{%
            \makebox[3em][l]{\large\bfseries I}%
            \large\bfseries
            Text of an entry to the table of contents with style
            \PValue{largetocline} and more than one line%
            \hfill
            \makebox[0pt][l]{\normalsize\normalfont
              \makebox[1.55em][r]{\large\bfseries 1}}%
          }%
        };
        \draw[|-|,color=gray] (0,\baselineskip) -- 
                              node [anchor=south] { 3\,em } 
                              (3em,\baselineskip);
        \draw[|-|,color=gray,overlay] (\linewidth,\ht\strutbox) -- 
                              node [anchor=south] { \Macro{@pnumwidth} }
                              (\linewidth-1.55em,\ht\strutbox);
        \large\bfseries
        \draw[|-|,color=gray,overlay] (\linewidth,-\baselineskip) -- 
                              node [anchor=north,font=\normalfont\normalsize] { 
                                \Macro{large}\Macro{@pnumwidth} 
                              }
                              (\linewidth-1.55em,-\baselineskip);
      \end{tikzpicture}%
    }
    \caption{Illustrations of some attributes of a TOC-entry with style 
      \PValue{largetocline}}
    \label{fig:tocbasic.largetocline}
  \end{figure}
\item[\PValue{tocline}] is a very flexible style. The \KOMAScript{} classes
  use this style by default for all kinds of entries. Classes \Class{scrbook}
  and \Class{scrreprt} respectively \Class{scrartcl} also define
  clones \PValue{part}, \PValue{chapter} and \PValue{section} respectively
  \PValue{section} and \PValue{subsection}, but add extra
  \PName{initial code} to the clones to change their defaults.

  The style supports attributes \PValue{level}, \PValue{beforeskip},
  \PValue{dynnumwidth}, \PValue{entryformat}, \PValue{entrynumberformat},
  \PValue{breakafternumber}, \PValue{indent}, \PValue{linefill},
  \PValue{numsep}, \PValue{numwidth}, \PValue{onstarthigherlevel},
  \PValue{onstartlowerlevel}, \PValue{onstartsamelevel},
  \PValue{pagenumberbox}, \PValue{pagenumberformat}, \PValue{raggedentrytext},
  and \PValue{raggedpagenumber}. The defaults of all these attributes depend
  on the name of the \PName{entry level}. They correspond to the results of
  the standard classes. So after loading \Package{tocbasic}, you can change
  the style of the standard classes entries to the table of contents into
  \PValue{tocline} using \Macro{DeclareTOCEntryStyle} without obvious visual
  changes unless you change exactly these attributes that should induce such
  changes. Same is valid for list of figures or list of tables of the standard
  classes.

  Because of the flexibility of this style it even could be used instead of
  the styles \PValue{dottedtocline}, \PValue{undottedtocline} or
  \PValue{largetocline}. However it needs more effort for configuration.

  {Figure~\ref*{fig:tocbasic.tocline}} illustrates some of the length
  attributes of this style. All attributes are described in
  \autoref{tab:tocbasic.tocstyle.attributes} from
  \autopageref{tab:tocbasic.tocstyle.attributes}.
  \begin{figure}
    \centering
    \resizebox{.8\linewidth}{!}{%
      \begin{tikzpicture}
        \coordinate (subsection) at (0,0);
        \coordinate (section) at ($(subsection)+(0,2\baselineskip)$);
        \coordinate (chapter) at ($(section)+(0,2\baselineskip)$);
        \coordinate (part)    at ($(chapter)+(0,2.4\baselineskip+1em)$);

        \draw[color=lightgray] 
          ($(part)+(0,2\baselineskip)$) -- 
          (0,-2.5\baselineskip);
        \draw[color=lightgray] 
          ($(part)+(\linewidth,2\baselineskip)$) --
          (\linewidth,-2.5\baselineskip);

        \coordinate (subsection) at (0,0);

        \node at (part) [anchor=west,inner sep=0,outer sep=0]
        {%
          \hspace*{3em}%
          \parbox[t]{\dimexpr\linewidth-5.55em}{%
            \setlength{\parindent}{-3em}%
            \addtolength{\parfillskip}{-2.55em}%
            \makebox[3em][l]{\large\bfseries I.}%
            \textbf{\large Text of a part entry with style
            \PValue{tocline} and with at least two lines of text}%
            \hfill
            \makebox[1.55em][r]{\bfseries 12}\large
          }%
        };
        \draw[|-|,color=gray,overlay] 
          (part) --
          ($(part)+(3em,0)$)
          node [anchor=north east,font=\small] {
            \PValue{numwidth}
          };
        \draw[|-|,color=gray,overlay] 
          ($(part)+(\linewidth,\ht\strutbox)$)
          node [anchor=north,font=\small] { 
            \Macro{@tocrmarg} 
          } --
          ($(part)+(\linewidth-2.55em,\ht\strutbox)$);
        \draw[|-|,color=gray,overlay] 
          ($(part)+(\linewidth,-\baselineskip)$) -- 
          node [anchor=north,font=\small] { 
            \Macro{@pnumwidth} 
          } 
          ($(part)+(\linewidth-1.55em,-\baselineskip)$);
        \node at (chapter) [anchor=west,inner sep=0,outer sep=0]
        {%
          \hspace*{1.5em}%
          \parbox[t]{\dimexpr\linewidth-4.05em}{%
            \setlength{\parindent}{-1.5em}%
            \addtolength{\parfillskip}{-2.55em}%
            \makebox[1.5em][l]{\bfseries 1.}%
            \textbf{Text of a chapter entry with style
            \PValue{tocline} and for demonstration purpose with more than one
            line of text}%
            \hfill
            \makebox[1.55em][r]{\bfseries 12}%
          }%
        };
        \draw[|-|,color=gray,overlay]
          ($(chapter)+(3em,\baselineskip)$) --
          node [anchor=west,font=\small] {
            \PValue{beforeskip}
          }
          ($(part)+(3em,-\baselineskip)$);
        \draw[|-|,color=gray,overlay] 
          (chapter) --
          ($(chapter)+(1.5em,0)$)
          node [anchor=north east,font=\small] {
            \PValue{numwidth}
          };
        \draw[|-|,color=gray,overlay] 
          ($(chapter)+(\linewidth,\ht\strutbox)$)
          node [anchor=north,font=\small] { 
            \Macro{@tocrmarg} 
          } --
          ($(chapter)+(\linewidth-2.55em,\ht\strutbox)$);
        \draw[|-|,color=gray,overlay] 
          ($(chapter)+(\linewidth,-\baselineskip)$)
          node [anchor=north,font=\small] { 
            \Macro{@pnumwidth} 
          } --
          ($(chapter)+(\linewidth-1.55em,-\baselineskip)$);
        \node at (section) [anchor=west,inner sep=0,outer sep=0]
        {
          \hspace*{3.8em}%
          \parbox[t]{\dimexpr\linewidth-6.35em}{%
            \setlength{\parindent}{-2.3em}%
            \addtolength{\parfillskip}{-2.55em}%
            \makebox[2.3em][l]{1.1.}%
            Text of a section entry with style
            \PValue{tocline} and for demonstration purpose with more than one
            line of text%
            \leaders\hbox{$\csname m@th\endcsname
              \mkern 4.5mu\hbox{.}\mkern 4.5mu$}\hfill\nobreak
            \makebox[1.55em][r]{3}%
          }%
        };
        \node at (subsection) [anchor=west,inner sep=0,outer sep=0]
        {%
          \hspace*{7em}%
          \parbox[t]{\dimexpr\linewidth-9.55em}{%
            \setlength{\parindent}{-3.2em}%
            \addtolength{\parfillskip}{-2.55em}%
            \makebox[3.2em][l]{1.1.10.}%
            Text of a subsection entry with style
            \PValue{tocline} and for demonstration purpose with more than one
            line of text%
            \leaders\hbox{$\csname m@th\endcsname
              \mkern 4.5mu\hbox{.}\mkern 4.5mu$}\hfill\nobreak
            \makebox[1.55em][r]{12}%
          }%
        };
        \draw[|-|,color=gray,overlay] 
          ($(subsection)+(0,\ht\strutbox)$) -- 
          node [anchor=north,font=\small] {
            \PValue{indent}
          }
          ($(subsection)+(3.8em,\ht\strutbox)$);
        \draw[|-|,color=gray,overlay] 
          ($(subsection)+(3.8em,0)$) --
          ($(subsection)+(7em,0)$)
          node [anchor=north east,font=\small] {
            \PValue{numwidth}
          };
        \draw[|-|,color=gray,overlay] 
          ($(subsection)+(\linewidth,\ht\strutbox)$)
          node [anchor=north,font=\small] { 
            \Macro{@tocrmarg} 
          } --
          ($(subsection)+(\linewidth-2.55em,\ht\strutbox)$);
        \draw[|-|,color=gray,overlay] 
          ($(subsection)+(\linewidth,-\baselineskip)$) -- 
          node [anchor=north,font=\small] { 
            \Macro{@pnumwidth} 
          } 
          ($(subsection)+(\linewidth-1.55em,-\baselineskip)$);
      \end{tikzpicture}%
    }
    \caption{Illustrations of some attributes of a TOC-entry with style 
      \PValue{tocline}}
    \label{fig:tocbasic.tocline}
  \end{figure}
\item[\PValue{undottedtocline}] is similar to the style used by the standard
  classes \Class{book} and \Class{report} for the \PValue{chapter} entry level
  or by \Class{article} for the \PValue{section} entry level of the table of
  contents. It supports the attributes \PValue{level}, \PValue{indent}, and
  \PValue{numwidth}. The last one is already a variation of the standard
  classes that do not support an indent of these entry levels.

  Before an entry, a page break will be made easier. The entries will be
  indented by the value of \PValue{indent} from the left. They are printed
  using \Macro{bfseries}. \Macro{numberline} is used unchanged. The width of
  the entry number is given by the value of \PValue{numwidth}. For multiline
  entries the indent will be increased by the value of \PValue{numwidth} for
  the second and following lines. \hyperref[fig:tocbasic.undottedtocline]%
  {Figure~\ref*{fig:tocbasic.undottedtocline}} illustrates the attributes of
  this style.
  \begin{figure}
    \centering
    \resizebox{.8\linewidth}{!}{%
      \begin{tikzpicture}
        \draw[color=lightgray] (0,2\baselineskip) -- (0,-2.5\baselineskip);
        \draw[color=lightgray] (\linewidth,2\baselineskip) --
                               (\linewidth,-2.5\baselineskip);
        \node (undottedtocline) at (0,0) [anchor=west,inner sep=0,outer sep=0]
        {%
          \makebox[1.5em][l]{\bfseries 1}%
          \parbox[t]{\dimexpr \linewidth-4.05em\relax}{%
            \bfseries
            Text of an entry to the table of contents with style
            \PValue{undottedtocline} and more than one line%
          }%
          \raisebox{-\baselineskip}{\makebox[2.55em][r]{\bfseries 3}}%
        };
        \draw[|-|,color=gray,overlay] (0,\baselineskip) -- 
                              node [anchor=south,font=\small] {
                                \PValue{numwidth}
                              }
                              (1.5em,\baselineskip);
        \draw[|-|,color=gray,overlay] (\linewidth,\ht\strutbox) -- 
                              node [anchor=south,font=\small] { 
                                \Macro{@tocrmarg} 
                              }
                              (\linewidth-2.55em,\ht\strutbox);
        \draw[|-|,color=gray,overlay] (\linewidth,-\baselineskip) -- 
                              node [anchor=north,font=\small] { 
                                \Macro{@pnumwidth} 
                              } 
                              (\linewidth-1.55em,-\baselineskip);
      \end{tikzpicture}%
    }
    \caption{Illustration of some attributes of style \PValue{undottedtocline}
      by the example of a chapter title}%
    \label{fig:tocbasic.undottedtocline}
  \end{figure}
\end{description}
\hyperref[tab:tocbasic.tocstyle.attributes]%
{Table~\ref*{tab:tocbasic.tocstyle.attributes}} describes all attributes of
all styles defined by
\Package{tocbasic}. If\ChangedAt[2016/06]{v3.21}{\Package{tocbasic}} you want
to use these attributes as options to \Macro{DeclareNewTOC} (see
\autopageref{desc:tocbasic.cmd.DeclareNewTOC}) you have to prefix the names of
the attribute by \PValue{tocentry}, e\,g., attribute \PValue{level} becomes
option \Option{tocentrylevel}.
If\ChangedAt[2016/06]{v3.20}[2015/12]{\Package{tocbasic}} you want to use
these attributes as options to \Macro{DeclareSectionCommand} (see
\autopageref{desc:maincls-experts.cmd.DeclareSectionCommand}) and similar
commands you have to prefix the names of the attributes by \PValue{toc},
e\,g., attribute \PValue{level} becomes option \Option{toclevel}.

Last but not least using \Macro{DeclareTOCStyleEntry} will define internal
command \Macro{l@\PName{entry level}}.

\begin{desclist}
  \desccaption{%
    Attributes of the predefined TOC-entry styles of \Package{tocbasic}%
    \label{tab:tocbasic.tocstyle.attributes}%
  }{%
    Attributes of the TOC-entry styles (\emph{continuation})%
  }%
  \entry{\KOption{beforeskip}\PName{length}}{%
    vertical distance, inserted before an entry of this level using style
    \PValue{tocline} (see \autoref{fig:tocbasic.tocline}). The distance is
    made using either \Macro{vskip} or \Macro{addvspace} depending on the
    \PName{entry level} to adapt the differences of the standard classes and
    former versions of \KOMAScript.

    At \PName{entry level} \PValue{part} the attribute will be initialised
    with \texttt{2.25em plus 1pt}, at \PValue{chapter} with \texttt{1em plus
      1pt}. If \PName{entry level} currently is unknown, rather
    \PValue{section} is initialised with \texttt{1em plus 1pt}. The initial
    value for all other levels is \texttt{0pt plus .2pt}.%
  }%
  \entry{\KOption{breakafternumber}\PName{switch}}{%
    \PName{switch} is one of the values for simple switches from 
    \autoref{tab:truefalseswitch}, \autopageref{tab:truefalseswitch}. If the
    switch is active with style \PValue{tocline}, there will be a line break
    after the entry number of \Macro{numberline}\IndexCmd{numberline}. The
    line after the entry number again starts left aligned with the number.

    This switch is not active by default at style \PValue{tocline}.

    If\textnote{Attention!} the feature \Option{numberline} of a list of
    something has been activated (see \Macro{setuptoc},
    \autoref{sec:tocbasic.toc}, \autopageref{desc:tocbasic.cmd.setuptoc}),
    i.\,e., if a \KOMAScript{} class with option
    \OptionValue{toc}{numberline}\IndexOption{toc~=\PValue{numberline}} is
    used, then the not numbered entries will nevertheless have a (by default
    empty) number line using the format code of \Option{entrynumberformat}.%
  }%
  \entry{\KOption{dynnumwidth}\PName{switch}}{%
    \PName{switch} is one of the values for simple switches from
    \autoref{tab:truefalseswitch}, \autopageref{tab:truefalseswitch}. If the
    switch is active with style \PValue{tocline}, attribute \PValue{numwidth}
    is ignored. Instead of that the maximum number width detected at the
    previous \LaTeX{} run increased by the value of \PValue{numsep} is used.%
  }%
  \entry{\KOption{entryformat}\PName{command}}{%
    This attributes makes the format of the entry. The value should be a
    \PName{command} with exactly one argument. The \PName{command} should not
    expect the argument to be fully expandable. Commands like
    \Macro{MakeUppercase}, that need a fully expandable argument, must no be
    used here. Font changes are allowed and are relative to
    \Macro{normalfont}\Macro{normalsize}. Please note that the output of
    \Option{linefill} and the page number are independent from
    \Option{entryformat}. See also attribute \Option{pagenumberformat}.

    The initial value of the attribute for \PName{entry level} \PValue{part}
    is printing the argument in \Macro{large}\Macro{bfseries} and for
    \PValue{chapter} printing the argument in \Macro{bfseries}. If currently
    no level \PValue{chapter} exists, \PValue{section} used
    \Macro{bfseries}. All other levels print the argument unchanged.%
  }%
  \entry{\KOption{entrynumberformat}\PName{command}}{%
    This attribute makes the format of the entry number within
    \Macro{numberline}. The value should be a \PName{command} with exactly one
    argument. Font changes are relative to the one of attribute
    \Option{entryformat}.

    The initial \PName{command} prints the argument unchanged. This means the
    entry number will be printed as it is.

    If\textnote{Attention!} the feature \Option{numberline} of a list of
    something has been activated (see \Macro{setuptoc},
    \autoref{sec:tocbasic.toc}, \autopageref{desc:tocbasic.cmd.setuptoc}),
    i.\,e., if a \KOMAScript{} class with option
    \OptionValue{toc}{numberline}\IndexOption{toc~=\PValue{numberline}} is
    used, then the not numbered entries will nevertheless execute the
    \PName{command}.%
  }%
  \entry{\KOption{indent}\PName{length}}{%
    Horizontal distance of the entry relative to the left margin (siehe
    \autoref{fig:tocbasic.dottedtocline} and \autoref{fig:tocbasic.tocline}).

    At style \PValue{tocline} all entry levels with a name that starts with
    ``\texttt{sub}'' are initialised with the sum of the values of
    \PValue{indent} and \PValue{numwidth} of the entry level without this
    prefix. At styles \PValue{dottedtocline}, \PValue{undottedtocline} and
    \PValue{tocline} the initial values of levels \PValue{part} down to
    \PValue{subparagraph} and the levels \PValue{figure} and \PValue{table}
    are compatible with the standard classes. All other levels do not have an
    initial value. Therefore you have to set an explicit value for such
    levels when they are defined first time.%
  }%
  \entry{\KOption{level}\PName{integer}}{%
    The numerical value of the \PName{entry level}. Only those entries are
    printed that have a numerical value less or equal to counter 
    \Counter{tocdepth}\important{\Counter{tocdepth}}\IndexCounter{tocdepth}.

    This attribute is mandatory for all styles and will be defined
    automatically at the declaration of the style.

    At style \PValue{tocline} all entry levels with a name starting with
    ``\texttt{sub}'', the initial value is the level value of the entry level
    without this prefix increased by one. At the styles \PValue{dottedtocline},
    \PValue{largetocline}, \PValue{tocline}, and \PValue{undottedtocline}
    entry levels \PValue{part} down to \PValue{subparagraph}, \PValue{figure},
    and \PValue{table} are initialised compatible with the standard
    classes. For all other levels the initialisation is done with the value of
    \Macro{\PName{entry level}numdepth} if this is defined.%
  }%
  \entry{\KOption{linefill}\PName{code}}{%
    At style \PValue{tocline} there can be a filler between the end of the
    entry text and the page number. The value of attribute \PName{linefill} is
    a \PName{code} that prints this filler. For \PName{entry level}
    \PValue{part} and \PValue{chapter} the attribute is initialised with
    \Macro{hfill}. If currently no \PName{entry level} \PValue{chapter} has
    been defined, \PValue{section} also uses \Macro{hfill}. All other entry
    levels are initialised with \Macro{TOCLineLeaderFill} (see
    \autopageref{desc:tocbasic.cmd.TOCLineLeaderFill}).

    If \PName{code} does not result in filling the distance, you should also
    activate attribute \PValue{raggedpagenumber}, to avoid ``\texttt{underfull
      \Macro{hbox}}'' messages.%
  }%
  \entry{\KOption{numsep}\PName{length}}{%
    Style \PValue{tocline} tries to ensure a minimum distance of
    \PName{length} between the entry number and the entry text. If
    \PValue{dynnumwidth} is active, it will correct the number width to achieve
    this. Otherwise it simply throws a warning, if the condition is missed.

    The initial \PName{length} is 0.4\Unit{em}.%
  }%
  \entry{\KOption{numwidth}\PName{length}}{%
    The reserved width for the entry number (see
    \autoref{fig:tocbasic.dottedtocline} until
    \autoref{fig:tocbasic.undottedtocline}). At the styles
    \PValue{dottedtocline}, \PValue{tocline}, and \PValue{undottedtocline}
    this \PName{length} will be added to the \PName{length} of attribute
    \PValue{indent} for the second and each following entry text line.

    At style \PValue{tocline} the initial \PName{length} of all entries with
    a name starting with ``\texttt{sub}'' is the value of the level without
    this prefix plus 0.9\Unit{em}. At the styles \PValue{dottedtocline},
    \PValue{undottedtocline} and \PValue{tocline} the initial \PName{length}
    of levels \PValue{part} down to \PValue{subparagraph} and levels
    \PName{figure} and \PName{table} is compatible to the standard
    classes. All other levels do not have an initial value. Therefore you
    have to explicitly set \PValue{numwidth} at the first definition of the
    entry.%
  }%
  \entry{\KOption{onstarthigherlevel}\PName{code}}{%
    Style \PValue{tocline} can execute \PName{code} at the start of an entry,
    if the numerical level is greater than the numerical level of the previous
    entry. Remember: The numerical level of, e.\,g., \PValue{section} is
    greater than the numerical level of \PValue{part}. Nevertheless
    \PValue{part} has the highest position in the entry hierarchy.

    Please note that the detection of the level of the previous entry depends
    on a valid unchanged value of \Macro{lastpenalty}.

    The initial \PName{code} is \Macro{LastTOCLevelWasLower} (see
    \autopageref{desc:tocbasic.cmd.LastTOCLevelWasLower}).%
  }%
  \entry{\KOption{onstartlowerlevel}\PName{code}}{%
    Style \PValue{tocline} can execute \PName{code} at the start of an entry,
    if the numerical level is lower than the numerical level of the previous
    entry. Remember: The numerical level of, e.\,g., \PValue{part} is
    lower than the numerical level of \PValue{section}. Nevertheless
    \PValue{part} has the highest position in the entry hierarchy.

    Please note that the detection of the level of the previous entry depends
    on a valid unchanged value of \Macro{lastpenalty}.

    The initial \PName{code} is \Macro{LastTOCLevelWasHigher} (see
    \autopageref{desc:tocbasic.cmd.LastTOCLevelWasHigher}).%
  }%
  \entry{\KOption{onstartsamelevel}\PName{code}}{%
    Style \PValue{tocline} can execute \PName{code} at the start of an entry,
    if the level is same like the level of the previous entry.

    Please note that the detection of the level of the previous entry depends
    on a valid unchanged value of \Macro{lastpenalty}.

    The initial \PName{code} is \Macro{LastTOCLevelWasSame} (see
    \autopageref{desc:tocbasic.cmd.LastTOCLevelWasSame}).%
  }%
  \entry{\KOption{pagenumberbox}\PName{command}}{%
    By default the page number of an entry is printed right aligned in a box
    of width \Macro{@pnumwidth}. At style \PValue{tocline} the \PName{command}
    to print the number can be changed using this attribute. The
    \PName{command} should have exactly one argument, the page number.%
  }%
  \entry{\KOption{pagenumberformat}\PName{command}}{%
    This attribute is the format of the page number of an entry. The
    \PName{command} should have exactly one argument, the page number. Font
    changes are relative to the font of \Option{entryformat} followed by
    \Macro{normalfont}\Macro{normalsize}.

    The initial \PName{command} of entry level \PValue{part} prints the
    argument in \Macro{large}\Macro{bfseries}. The initial \PName{command} of
    all other levels prints the argument in
    \Macro{normalfont}\Macro{normalcolor}.%
  }%
  \entry{\KOption{raggedentrytext}\PName{switch}}{%
    \ChangedAt{v3.21}{\Package{tocbasic}}%
     \PName{switch} is one of the values for simple switches from
    \autoref{tab:truefalseswitch}, \autopageref{tab:truefalseswitch}. If the
    switch is active, style \PValue{tocline} does print the text of an entry
    left-aligned instead of justified and only word, that are longer than a
    text line, are automatically hyphenated.

    This switch is not active by default.%
  }%
 \entry{\KOption{raggedpagenumber}\PName{switch}}{%
    \PName{switch} is one of the values for simple switches from
    \autoref{tab:truefalseswitch}, \autopageref{tab:truefalseswitch}. If the
    switch is active, style \PValue{tocline} does not force the page number to
    be right aligned.

    Depending on the value of \PValue{linefill}, the setting of this attribute
    could be needed for the wanted printing of the number, or only to avoid
    unwanted warning messages. So both attributes should correspond.

    By default the switch is not activated and therefore corresponds with an
    initial value \Macro{hfill} or \Macro{TOCLineLeaderFill} of attribute
    \PValue{linefill}.%
  }%
\end{desclist}%
%
\EndIndex{Cmd}{DeclareTOCStyleEntry}%

\begin{Declaration}
  \Macro{DeclareTOCEntryStyle}\Parameter{style}%
                              \OParameter{initial code}%
                              \Parameter{command code}\\
  \Macro{DefineTOCEntryOption}\Parameter{option}\OParameter{default value}%
                              \Parameter{code}\\
  \Macro{DefineTOCEntryBooleanOption}\Parameter{option}%
                                     \OParameter{default value}%
                                     \Parameter{prefix}%
                                     \Parameter{postfix}%
                                     \Parameter{description}\\%
                                     %\OParameter{initial code}\\
  \Macro{DefineTOCEntryCommandOption}\Parameter{option}%
                                     \OParameter{default value}%
                                     \Parameter{prefix}%
                                     \Parameter{postfix}%
                                     \Parameter{description}\\%
                                     %\OParameter{initial code}\\
  \Macro{DefineTOCEntryIfOption}\Parameter{option}%
                                     \OParameter{default value}%
                                     \Parameter{prefix}%
                                     \Parameter{postfix}%
                                     \Parameter{description}\\%
                                     %\OParameter{initial code}\\
  \Macro{DefineTOCEntryLengthOption}\Parameter{option}%
                                     \OParameter{default value}%
                                     \Parameter{prefix}%
                                     \Parameter{postfix}%
                                     \Parameter{description}\\%
                                     %\OParameter{initial code}\\
  \Macro{DefineTOCEntryNumberOption}\Parameter{option}%
                                     \OParameter{default value}%
                                     \Parameter{prefix}%
                                     \Parameter{postfix}%
                                     \Parameter{description}%
                                     %\OParameter{initial code}%
\end{Declaration}
\BeginIndex{Cmd}{DeclareTOCEntryStyle}%
\BeginIndex{Cmd}{DefineTOCEntryOption}%
\BeginIndex{Cmd}{DefineTOCEntryBooleanOption}%
\BeginIndex{Cmd}{DefineTOCEntryCommandOption}%
\BeginIndex{Cmd}{DefineTOCEntryIfOption}%
\BeginIndex{Cmd}{DefineTOCEntryLengthOption}%
\BeginIndex{Cmd}{DefineTOCEntryNumberOption}%
\Macro{DeclareTOCEntryStyle}\ChangedAt[2016/03]{v3.20}{\Package{tocbasic}} is
one of the most complex commands in \KOMAScript. It is addressed to \LaTeX{}
developers not the \LaTeX{} users. It provides the definition of a new
TOC-entry \PName{style}. Usually TOC-entries are made using
\Macro{addcontentsline}\IndexCmd{addcontentsline}, or, if you use
\Package{tocbasic}, with recommended
\Macro{addxcontentsline}\IndexCmd{addxcontentsline} (see
\autoref{sec:tocbasic.basics},
\autopageref{desc:tocbasic.cmd.addxcontentsline}). In both cases \LaTeX{}
writes a corresponding \Macro{contentsline}\IndexCmd{contentsline} into the
given auxiliary file. Reading this auxiliary file each \Macro{contentsline}
results in execution of a command \Macro{l@\PName{entry level}}.

Whenever you assign a \PName{style} to a TOC-entry level using
\Macro{DeclareTOCStyleEntry}, first of all the \PName{initial code} is
executed and then \Macro{l@\PName{entry level}} is defined to be
\PName{command code}. So \PName{command code} is the code that will be
expanded and executed by \Macro{l@\PName{entry level}}. Inside
\PName{command code} \texttt{\#1} is the name of the TOC-entry level and
\texttt{\#\#1} and \texttt{\#\#2} are the arguments of
\Macro{l@\PName{entry level}}.

The \PName{initial code} should initialise all attributes of the
\PName{style}. Developers are recommended to use \PName{initial code} to
initialise all internal macros of the \PName{style} without the need of
using an \PName{option list}. The second task of the \PName{initial code} is
the definition of options to setup the attributes of the \PName{style}. Option
\Option{level} is always defined automatically. The value of \Option{level}
can be got in \PName{command code} using
\Macro{@nameuse}\PParameter{\#1tocdepth}%
\important{\Macro{\PName{entry level}tocdepth}}, e.\,g., to compare it with
the counter \Counter{tocdepth}\IndexCounter{tocdepth}.

Commands \Macro{DefineTOCEntryBooleanOption},
\Macro{DefineTOCEntryCommandOption}, \Macro{DefineTOCEntryIfOption},
\Macro{DefineTOCEntryLengthOption}, and \Macro{DefineTOCEntryNumberOption}
should be used to define options for the attributes of the
\PName{style} inside \PName{initial code} only. If you use an \PName{option}
defined by one of these commands, a macro \Macro{\PName{prefix}\PName{entry
    level}\PName{postfix}} will be defined to be the assigned value or the
\PName{default value} of the option. Somehow special is
\Macro{DefineTOCEntryIfOption}. It defines \Macro{\PName{prefix}\PName{entry
    level}\PName{postfix}} as a command with two arguments. If the value to
the option is an activation value of \autoref{tab:truefalseswitch},
\autopageref{tab:truefalseswitch} the command expands to the first
argument. If the value to the option is a deactivation value, the command
expands to the second argument.

The \PName{description} should be a real short text that describes the sense
of the option with some catchwords. Package \Package{tocbasic} uses this text
in error messages, warnings or information output on the terminal and into the
\File{log}-file.

The most simple style of \Package{tocbasic}, \PValue{gobble}, is defined
using:
\begin{lstcode}[belowskip=\dp\strutbox plus 1pt]
  \DeclareTOCEntryStyle{gobble}{}%
\end{lstcode}
If you would define a entry level \PValue{dummy} using:
\begin{lstcode}[belowskip=\dp\strutbox plus 1pt]
  \DeclareTOCStyleEntry[level=1]{gobble}{dummy}
\end{lstcode}
among others this would do something like:
\begin{lstcode}[belowskip=\dp\strutbox plus 1pt]
  \def\dummytocdepth{1}
  \def\l@dummy#1#2{}
\end{lstcode}

Inside style \PValue{tocline} for example
\begin{lstcode}[belowskip=\dp\strutbox plus 1pt]
  \DefineTOCEntryCommandOption{linefill}[\TOCLineLeaderFill]%
  {scr@tso@}{@linefill}{filling between text and page number}%
\end{lstcode}
is used to define option \Option{linefill} with \PName{default value}
\Macro{TOCLineLeaderFill}. A call like:
\begin{lstcode}[belowskip=\dp\strutbox plus 1pt]
  \RedeclareTOCStyleEntry[linefill]{tocline}{part}
\end{lstcode}
would therefore result in a definition like:
\begin{lstcode}[belowskip=\dp\strutbox plus 1pt]
  \def\scr@tso@part@linefill{\TOCLineLeaderFill}
\end{lstcode}

Whoever likes to define his own styles is recommended to first study the
definition of style \PValue{dottedtocline}. If this definition is understood,
the much more complex definition of style \PValue{tocline} gives a lot of
hints of the correct usage of the described commands.

In many cases it will be enough to clone an existing style using
\Macro{CloneTOCEntryStyle} and to change the initial code of the new style
using \Macro{TOCEntryStyleInitCode} or \Macro{TOCEntryStyleStartInitCode}.

\Macro{DefineTOCEntryOption} is merely used to define the other
commands. It is not recommended to define options directly using
\Macro{DefineTOCEntryOption}. Normally this is even not needed. It is
alluded only for completeness.%
\EndIndex{Cmd}{DefineTOCEntryNumberOption}%
\EndIndex{Cmd}{DefineTOCEntryLengthOption}%
\EndIndex{Cmd}{DefineTOCEntryIfOption}%
\EndIndex{Cmd}{DefineTOCEntryCommandOption}%
\EndIndex{Cmd}{DefineTOCEntryBooleanOption}%
\EndIndex{Cmd}{DefineTOCEntryOption}%
\EndIndex{Cmd}{DeclareTOCEntryStyle}%

\begin{Declaration}
  \Macro{CloneTOCEntryStyle}\Parameter{style}\Parameter{new style}%
\end{Declaration}
\BeginIndex{Cmd}{CloneTOCEntryStyle}%
With\ChangedAt[2016/03]{v3.20}{\Package{tocbasic}} this command you can clone
an existing \PName{style}. This defines a \PName{new style} with the same
attributes and settings like the existing \PName{style}. The package itself
uses \Macro{CloneTOCEntryStyle} to declare style \PValue{default} as a clone
of \PValue{dottedtocline}. The \KOMAScript{} classes use the command to
declare the styles \PValue{part}, \PValue{section}, and \PValue{chapter} or
\PValue{subsection} as a clone of \PValue{tocline} and the style
\PValue{default} new as a clone of \PValue{section} or \PValue{subsection}.%
\EndIndex{Cmd}{CloneTOCEntryStyle}%

\begin{Declaration}
  \Macro{TOCEntryStyleInitCode}\Parameter{style}%
                               \Parameter{initial code}\\
  \Macro{TOCEntryStyleStartInitCode}\Parameter{style}%
                                    \Parameter{initial code}
\end{Declaration}
\BeginIndex{Cmd}{TOCEntryStyleInitCode}%
\BeginIndex{Cmd}{TOCEntryStyleStartInitCode}%
Every\ChangedAt[2016/03]{v3.20}{\Package{tocbasic}} TOC-entry style has an
initialisation code. This is used whenever a \PName{style} is assigned to an
TOC-entry using \Macro{DeclareTOCEntryStyle}. This \PName{initial code} should
never do anything global, because it is also used for local initialisation
inside other commands like \Macro{DeclareNewTOC}\IndexCmd{DeclareNewTOC}. The
\PName{initial code} not only defines all attributes of a \PName{style}. It
also should set the defaults for those attributes.

You can use \Macro{TOCEntryStyleStartInitCode} and
\Macro{TOCEntryStyleInitCode} to extend the already existing initialisation
code by \PName{initial code}. \Macro{TOCEntryStyleStartInitCode} adds
\PName{initial code} in front of the existing initialisation
code. \Macro{TOCEntryStyleInitCode} adds the \PName{initial code} at the end
of the existing initialisation code. The \KOMAScript{} classes, e.\,g., are
using \Macro{TOCEntryStyleStartInitCode} to change the filling, font and
vertical distances of style \PValue{part} that is a clone of
\PValue{tocline}. Class \Class{scrbook} and \Class{scrreprt} use
\begin{lstcode}[belowskip=\dp\strutbox plus 1pt]
  \CloneTOCEntryStyle{tocline}{section}
  \TOCEntryStyleStartInitCode{section}{%
    \expandafter\providecommand%
    \csname scr@tso@#1@linefill\endcsname
    {\TOCLineLeaderFill\relax}%
  }
\end{lstcode}
to declare \PValue{section} as a modified clone of \PValue{tocline}.%
\EndIndex{Cmd}{TOCEntryStyleStartInitCode}%
\EndIndex{Cmd}{TOCEntryStyleInitCode}%

\begin{Declaration}
  \Macro{LastTOCLevelWasHigher}\\
  \Macro{LastTOCLevelWasSame}\\
  \Macro{LastTOCLevelWasLower}
\end{Declaration}
\BeginIndex{Cmd}{LastTOCLevelWasHigher}%
\BeginIndex{Cmd}{LastTOCLevelWasSame}%
\BeginIndex{Cmd}{LastTOCLevelWasLower}%
At\ChangedAt[2016/03]{v3.20}{\Package{tocbasic}} the very beginning entries
with style \PValue{tocline} \Package{tocbasic} executes one of these three
commands depending on \Macro{lastpenalty}. \Macro{LastTOCLevelWasHigher} and
\Macro{LastTOCLevelWasSame} used in vertical mode add
\Macro{addpenalty}\PParameter{\Macro{@lowpenalty}} and therefore permit a
page break before an entry with same or higher hierarchical
position. \Macro{LastTOCLevelWasLower} is an empty command. Therefore page
break between an entry and its sub-entry is not permitted.

Users should not redefine these commands. Instead of a redefinition you should
change the behaviour of single entry levels using attributes
\PValue{onstartlowerlevel}, \PValue{onstartsamelevel}, and
\PValue{onstarthigherlevel}.%
\EndIndex{Cmd}{LastTOCLevelWasLower}%
\EndIndex{Cmd}{LastTOCLevelWasSame}%
\EndIndex{Cmd}{LastTOCLevelWasHigher}%

\begin{Declaration}
  \Macro{TOCLineLeaderFill}\OParameter{filling code}
\end{Declaration}
\BeginIndex{Cmd}{TOCLineLeaderFill}%
Command\ChangedAt[2016/03]{v3.20}{\Package{tocbasic}} has been made to be used
as value of option \Option{linefill} of assigning style \PName{tocline} to a
TOC-entry. It is a line filler between the end of the entry text and the entry
page number. The \PName{filling code} will be repeated with constant
distance. The default for this optional argument is a dot.

As implied by the name of the command it uses \Macro{leaders} to put the
\PName{filling code}. The distance is defined analogous to the \LaTeX{} kernel
command \Macro{@dottedtocline} by
\Macro{mkern}\Macro{@dotsep}\Unit{\texttt{mu}}.%
\EndIndex{Cmd}{TOCLineLeaderFill}%
%
\EndIndex{}{table of contents>entry}


\section{Internal Commands for Class and Package Authors}
\label{sec:tocbasic.internals}

Commands with prefix \Macro{tocbasic@} are internal but class and package
authors may use them. But even if you are a class or package author you
should not change them!

\begin{Declaration}
  \Macro{tocbasic@extend@babel}\Parameter{extension}
\end{Declaration}
\BeginIndex{Cmd}{tocbasic@extend@babel}%
The Package \Package{babel}\IndexPackage{babel} (see \cite{package:babel}),
or more specifically a \LaTeX{} kernel that has been extended by the language
management of \Package{babel} writes instructions to change the language
inside of the files with the file name extensions \File{toc}, \File{lof}, and
\File{lot} into those files at every change of the current language either at
the begin of the document or inside the document. Package \Package{tocbasic}
extends this mechanism with \Macro{tocbasic@extend@babel} to be used for other
file name extensions too. Argument \PName{extension} has to be expandable!
Otherwise the meaning of the argument may change until it will be used really.

Normally this command will be used by default for every file name
\PName{extension} that will be added to the list of known extensions using
\Macro{addtotoclist}. This may be suppressed using feature
\PValue{nobabel}\important{\PValue{nobabel}} (see \Macro{setuptoc},
\autoref{sec:tocbasic.toc}, \autopageref{desc:tocbasic.cmd.setuptoc}). For the
file name extensions \File{toc}, \File{lof}, and \File{lot} this will be done
automatically by \Package{tocbasic} to avoid double language switching in the
corresponding files.

Normally there isn't any reason to call this command yourself. But there may
by lists of something, that should not be under control of \Package{tocbasic},
and so are not in \Package{tocbasic}'s list of known file name extensions, but
nevertheless should be handled by the language change mechanism of
\Package{babel}. The command may be used explicitly for those files. But please
note that this should be done only once per file name extension!%
\EndIndex{Cmd}{tocbasic@extend@babel}%

\begin{Declaration}
  \Macro{tocbasic@starttoc}\Parameter{extension}
\end{Declaration}
\BeginIndex{Cmd}{tocbasic@starttoc}
This command is something like the \LaTeX{} kernel macro
\Macro{@starttoc}\IndexCmd{@starttoc}\important{\Macro{@starttoc}}.  It is the
command behind \Macro{listoftoc*} (see \autoref{sec:tocbasic.toc},
\autopageref{desc:tocbasic.cmd.listoftoc*}). Authors of classes or packages
who want to participate from the advantages of \Package{tocbasic} should at
least use this command. Nevertheless it is recommended to use
\Macro{listoftoc}. Command \Macro{tocbasic@starttoc} internally uses
\Macro{\@starttoc}, but sets
\Length{parskip}\IndexLength{parskip}\important{\Length{parskip}\\
  \Length{parindent}\\
  \Length{parfillskip}} and \Length {parindent}\IndexLength{parindent} to 0
and \Length{parfillskip} to 0 until infinite before. Moreover,
\Macro{@currext}\important{\Macro{@currext}}\IndexCmd{@currext} will be set to
the file name extension of the current TOC-file, so this will be available
while the execution of the hooks, that will be done before and after reading
the auxiliary files.

Because\textnote{Attention!} of \LaTeX{} will immediately open a new TOC-file
for writing after reading that file, the usage of \Macro{tocbasic@starttoc}
may result in an error message like
\begin{lstoutput}
  ! No room for a new \write .
  \ch@ck ...\else \errmessage {No room for a new #3}
                                                    \fi
\end{lstoutput}
if there are no more unused write handles. This may be solved, e.\,g., using
package
\Package{scrwfile}\important{\Package{scrwfile}}\IndexPackage{scrwfile}.
See \autoref{cha:scrwfile} for more information about that package.%
\EndIndex{Cmd}{tocbasic@starttoc}

\begin{Declaration}
  \Macro{tocbasic@@before@hook}\\
  \Macro{tocbasic@@after@hook}
\end{Declaration}
\BeginIndex{Cmd}{tocbasic@@before@hook}%
\BeginIndex{Cmd}{tocbasic@@after@hook}%
The hook \Macro{tocbasic@@before@hook} will be executed immediately before
reading a auxiliary file for a TOC even before execution of the instructions
of a \Macro{BeforeStartingTOC} command. It is permitted to extend this hook
using \Macro{g@addto@macro}\IndexCmd{g@addto@macro}.

Similarly \Macro{tocbasic@@after@hook} will be executed immediately after
reading such an auxiliary file and before execution of instructions of
\Macro{AfterStartingTOC}. It is permitted to extend this hook using
\Macro{g@addto@macro}\IndexCmd{g@addto@macro}.

\KOMAScript{} uses these hooks, to provide the automatic width calculation of
the place needed by heading numbers. Only classes and packages should use
these hooks. Users\textnote{Attention!} should really use
\Macro{BeforeStartingTOC} and \Macro{AfterStartingTOC} instead. Authors of
packages should also favour those commands! These hooks should not be used to
generate any output!

If neither\textnote{Attention!} \Macro{listofeachtoc} nor \Macro{listoftoc}
nor \Macro{listoftoc*} are used for the output of a TOC, the hooks should be
executed explicitly.%
\EndIndex{Cmd}{tocbasic@@before@hook}%
\EndIndex{Cmd}{tocbasic@@after@hook}%

\begin{Declaration}
  \Macro{tocbasic@\PName{extension}@before@hook}\\
  \Macro{tocbasic@\PName{extension}@after@hook}
\end{Declaration}
\BeginIndex{Cmd}{tocbasic@\PName{extension}@before@hook}%
\BeginIndex{Cmd}{tocbasic@\PName{extension}@after@hook}%
These hooks are processed after \Macro{tocbasic@@before@hook}, respectively
before \Macro{tocbasic@@after@hook} before and after loading the TOC-file with
the corresponding file \PName{extension}. Authors\textnote{Attention!}  of
classes and packages should never manipulate them! But if
neither\textnote{Attention!} \Macro{listofeachtoc} nor \Macro{listoftoc} nor
\Macro{listoftoc*} are used for the output of a TOC, the hooks should be
executed explicitly, if they are defined. Please note that they even can be
undefined.%
\EndIndex{Cmd}{tocbasic@\PName{extension}@after@hook}%
\EndIndex{Cmd}{tocbasic@\PName{extension}@before@hook}%

\begin{Declaration}
  \Macro{tocbasic@listhead}\Parameter{title}
\end{Declaration}
\BeginIndex{Cmd}{tocbasic@listhead}%
This command is used by \Macro{listoftoc} to set the heading of the TOC,
either the default heading or the individually defined heading. If you define
your own TOC-command not using \Macro{listoftoc} you may use
\Macro{tocbasic@listhead}. In this case you should define
\Macro{@currext}\important{\Macro{@currext}}\IndexCmd{@currext} to be the file
extension of the corresponding TOC-file before using
\Macro{tocbasic@listhead}.%
\EndIndex{Cmd}{tocbasic@listhead}%

\begin{Declaration}
  \Macro{tocbasic@listhead@\PName{extension}}\Parameter{title}
\end{Declaration}
\BeginIndex{Cmd}{tocbasic@listhead@\PName{extension}}%
This command is used in \Macro{tocbasic@listhead} to set the individual
headings, optional table of contents entry, and running head, if it was
defined. If it was not defined it will be defined and used in
\Macro{tocbasic@listhead} automatically.
\EndIndex{Cmd}{tocbasic@listhead@\PName{extension}}%

\begin{Declaration}
  \Macro{tocbasic@addxcontentsline}%
  \Parameter{extension}\Parameter{level}\Parameter{number}\Parameter{text}\\
  \Macro{nonumberline}
\end{Declaration}
\BeginIndex{Cmd}{tocbasic@addxcontentsline}%
\BeginIndex{Cmd}{nonumberline}%
Command\ChangedAt{v3.12}{\Package{tocbasic}} \Macro{tocbasic@addxcontentsline}
uses \Macro{addcontentsline} to either create a numbered or not numbered text
entry to the TOC-file with the given \PName{extension}. Note, all parameters
of \Macro{tocbasic@addxcontentsline} are mandatory. But you may use an empty
\PName{number} argument, if you do not want a number. In this case the
\PName{text} will be prefixed by \Macro{nonumberline} without any argument. In
the other case, if \PName{number} is not empty, \Macro{numberline} with
argument \PName{number} will used as usual.

Command \Macro{nonumberline} is redefined inside \Macro{listoftoc} (see
\autoref{sec:tocbasic.toc}, \autopageref{desc:tocbasic.cmd.listoftoc})
depending on feature \PValue{numberline} (see \autoref{sec:tocbasic.toc},
\autopageref{desc:tocbasic.cmd.setuptoc}). This guarantees that changes of the
feature results in changes of the corresponding TOC immediately at the next
\LaTeX{} run.%
\EndIndex{Cmd}{nonumberline}%
\EndIndex{Cmd}{tocbasic@addxcontentsline}%

\begin{Declaration}
  \Macro{tocbasic@DependOnPenaltyAndTOCLevel}\Parameter{entry level}\\
  \Macro{tocbasic@SetPenaltyByTOCLevel}\Parameter{entry level}
\end{Declaration}
\BeginIndex{Cmd}{tocbasic@DependOnPenaltyAndTOCLevel}%
\BeginIndex{Cmd}{tocbasic@SetPenaltyByTOCLevel}%
At\ChangedAt[2016/03]{v3.20}{\Package{tocbasic}} the end of TOC-entry style
\PValue{tocline} (see \autoref{sec:tocbasic.tocstyle}) \Macro{penalty} is set
to prohibit page breaks. The used penalty value depends on the \PName{entry
  level}. This is done by \Macro{tocbasic@SetPenaltyByTOCLevel}. At the very
beginning of an entry \Macro{tocbasic@DependOnPenaltyAndTOCLevel} is used to
execute the value of either style option \Option{onstartlowerlevel},
\Option{onstartsamelevel}, or \Option{onstarthigherlevel} depending on
\Macro{lastpenalty} and the current \PName{entry level}. The default of the
first and second option would be to permit a page break, if used in vertical
mode.

Developers of \PValue{tocline}-compatible styles should adapt this. To do so,
they are even allowed to copy the style option declarations of
\Option{onstartlowerlevel}, \Option{onstartsamelevel}, and
\Option{onstarthigherlevel}. These options should even use the same internal
macros \Macro{scr@tso@\PName{entry level}@LastTOCLevelWasHigher},
\Macro{scr@tso@\PName{entry level}@LastTOCLevelWasSame} and
\Macro{scr@tso@\PName{entry level}@LastTOCLevelWasLower} to store the current
values of the options.%
\EndIndex{Cmd}{tocbasic@SetPenaltyByTOCLevel}%
\EndIndex{Cmd}{tocbasic@DependOnPenaltyAndTOCLevel}%


\section{A Complete Example}
\seclabel{example}

This section will show you a complete example of a user defined floating
environment with list of that kind of floats and \KOMAScript{} integration
using \Package{tocbasic}. This example uses internal commands, that have a
``\texttt{@}'' in their name. This means\textnote{Attention}, that the code
has to be put into a package or class, or has to be placed between
\Macro{makeatletter}\important[i]{\Macro{makeatletter}\\\Macro{makeatother}}%
\IndexCmd{makeatletter} and \Macro{makeatother}\IndexCmd{makeatother}.

First\textnote{environment} of all, a new floating environment will be
needed. This could simply be done using:
\begin{lstcode}[belowskip=\dp\strutbox plus 1pt]
  \newenvironment{remarkbox}{%
    \@float{remarkbox}%
  }{%
    \end@float
  }
\end{lstcode}
To the new environment is named \Environment{remarkbox}.

Each\textnote{placement} floating environment has a default
placement. This is build by some of the well known placement options:
\begin{lstcode}[belowskip=\dp\strutbox plus 1pt]
  \newcommand*{\fps@remarkbox}{tbp}
\end{lstcode}
So, the new floating environment should be placed by default only either at
the top of a page, at the bottom of a page, or on a page on its own.

Floating\textnote{type} environments have a numerical floating
type. Environments with the same active bit at the floating type cannot change
their order. Figures and table normally use type~1 and 2. So a figure that
comes later at the source code than a table, may be output earlier than the
table and vica versa.
\begin{lstcode}[belowskip=\dp\strutbox plus 1pt]
  \newcommand*{\ftype@remarkbox}{4}
\end{lstcode}
The new environment has floating type~4, so it may pass figures and floats and
may be passed by those.

The\textnote{number} captions of floating environment also have numbers.
\begin{lstcode}[belowskip=\dp\strutbox plus 1pt]
  \newcounter{remarkbox}
  \newcommand*{\remarkboxformat}{%
    Remark~\theremarkbox\csname autodot\endcsname}
  \newcommand*{\fnum@remarkbox}{\remarkboxformat}
\end{lstcode}
Here first a new counter has been defined, that is independent from chapters
or the counters of other structural levels. \LaTeX{} itself also defines
\Macro{theremarkbox} with the default Arabic representation of the counter's
value. Afterwards this has been used defining the formatted output of the
counter. Last this formatted output has been used for the output of the
environment number of the \Macro{caption} command.

Floating\textnote{file name extension} environments have lists of themselves
and those need a auxiliary file with name \Macro{jobname} and a file name
extension, the TOC-file\Index{TOC-file}:
\begin{lstcode}[belowskip=\dp\strutbox plus 1pt]
  \newcommand*{\ext@remarkbox}{lor}
\end{lstcode}
The file name extension of the TOC-file for the list of
\Environment{remarkbox}es is ``\File{lor}''.

This was the definition of the floating environment. But the list of this new
environment's captions is still missing. To reduce the implementation effort
package \Package{tocbasic} will be used for this. This will be loaded using
\begin{lstcode}[belowskip=\dp\strutbox plus 1pt]
  \usepackage{tocbasic}
\end{lstcode}
inside of document preambles. Authors of classes or packages would use
\begin{lstcode}[belowskip=\dp\strutbox plus 1pt]
  \RequirePackage{tocbasic}
\end{lstcode}
instead.

Now\textnote{extension} we register the file name extension for package
\Package{tocbasic}:
\begin{lstcode}[belowskip=\dp\strutbox plus 1pt]
  \addtotoclist[float]{lor}
\end{lstcode}
Thereby the owner \PValue{float} has been used, to allude all further
\KOMAScript{} options for list of figures and list of tables also to the new
one.

Next\textnote{title} we define a title or heading for the list of
\Environment{remarkbox}es:
\begin{lstcode}[belowskip=\dp\strutbox plus 1pt]
  \newcommand*{\listoflorname}{List of Remarks}
\end{lstcode}
You may use package \Package{scrbase} to additionally support titles in other
languages than English.

Also\textnote{entry} a command is needed to define the layout of the entries
to the list of remarks:
\begin{lstcode}[belowskip=\dp\strutbox plus 1pt]
  \newcommand*{\l@remarkbox}{\l@figure}
\end{lstcode}
Here simply the entries to the list of remarks get the same layout like the
entries to the list of figures. This would be the easiest solution. A more
explicit would be, e.\,g.,
\begin{lstcode}[belowskip=\dp\strutbox plus 1pt]
  \DeclareTOCStyleEntry[level=1,indent=1em,numwidth=1.5em]%
                       {tocline}{remarkbox}
\end{lstcode}

Additionally\textnote{chapter entry} you may want structure the list of
remarks depending on chapters.
\begin{lstcode}[belowskip=\dp\strutbox plus 1pt]
  \setuptoc{lor}{chapteratlist}
\end{lstcode}
The \KOMAScript{} classes provide that feature and maybe other classes do so
too. Unfortunately the standard classes do not.

This\textnote{list of remarks} would already be enough. Now, users may already
select different kinds of headings either using the corresponding options of
the \KOMAScript{} classes, or \Macro{setuptoc}, e.\,g., with or without entry
in the table of contents, with or without number. But a simple
\begin{lstcode}[belowskip=\dp\strutbox plus 1pt]
  \newcommand*{\listofremarkboxes}{\listoftoc{lor}}
\end{lstcode}
may make the usage a little bit easier again.

As you've seen only five commands refers to the list of remarks. Only three of
them are necessary. Nevertheless the new list of remarks already provides
optional numbering of the heading and optional not numbered entry into the
table of contents. Optional even a lower document structure level may be used
for the heading. Running headers are provides with the \KOMAScript{} classes,
the standard classes, and all classes that explicitly support
\Package{tocbasic}. Supporting classes even pay attention to this new list of
remarks at every new \Macro{chapter}. Even changes of the current language are
handled inside the list of remarks like they will inside the list of figures
or inside the list of tables.

Moreover,\textnote{additional features} an author of a package may add more
features. For example, options to hide \Macro{setuptoc} from the users may be
added. On the other hand, the \Package{tocbasic} manual may be referenced to
describe the corresponding features. The advantage of this would be that user
would get information about new features provides by \Package{tocbasic}. But
if the user should be able to set the features of the remarks even without
knowledge about the file name extension \File{lor} a simple
\begin{lstcode}[belowskip=\dp\strutbox plus 1pt]
  \newcommand*{\setupremarkboxes}{\setuptoc{lor}}
\end{lstcode}
would be enough to use a list of features argument to
\Macro{setupremarkboxes} as list of features of file name extension \File{lor}.

\section{Everything with One Command Only}
\label{sec:tocbasic.highlevel}

The example from the previous section shows, that using \Package{tocbasic} to
define floating environments and lists with the captions of those floating
environments is easy. The following example will show, that it may be even
easier.

\begin{Declaration}
  \Macro{DeclareNewTOC}\OParameter{options}\Parameter{extension}
\end{Declaration}
\BeginIndex{Cmd}{DeclareNewTOC}%
This command declares\ChangedAt{v3.06}{\Package{tocbasic}} in one step only a
new TOC, the heading of that TOC, the term used for the TOC-entries, and to
manage the file name \PName{extension}. Additionally optional floating and
non-floating environments may be defined, and inside of both such environments
\Macro{caption}\important{\Macro{caption}}\IndexCmd{caption} may be used. The
additional features \Macro{captionabove}\important[i]{%
  \Macro{captionabove}\\
  \Macro{captionbelow}}, \Macro{captionbelow}, and \Environment{captionbeside}
of the \KOMAScript{} classes (see \autoref{sec:maincls.floats}) may also be
used inside of those environments.

Argument \PName{extension} is the file name extension of the TOC-file, that
represents the list of something. See \autoref{sec:tocbasic.basics} for more
information about this. This argument is mandatory and must not be empty!

Argument \PName{options} is a comma separated list, like you know it from,
e.\,g., \Macro{KOMAoptions} (see
\autoref{sec:typearea.options}). Nevertheless\textnote{Attention!}, those
options cannot be set using \Macro{KOMAoptions}\IndexCmd{KOMAoptions}! An
overview of all available options may be found in
\autoref{tab:tocbasic.DeclareNewTOC-options}.

If\ChangedAt[2015/12]{v3.20}{\Package{tocbasic}} option \Option{tocentrystyle}
is not used, style \PValue{default} will be used. For information about this
style see \autoref{sec:tocbasic.tocstyle}. If you do not want to define a
command for entries to the list of something, you can use an empty argument,
i.\,e., \OptionValue{tocentrystyle}{} or
\OptionValue{tocentrystyle}{\PParameter{}}. Nevertheless, this would contain
the risk to get a lot of errors while printing that list.

Depending\ChangedAt[2015/12]{v3.20}{\Package{tocbasic}}%
\ChangedAt[2016/06]{v3.21}{\Package{tocbasic}} on the style of the
entries to the list of something, you can setup all valid attributes of the
selected style as part of the \PName{options}. To do so you have to prefix the
names of the attributes given in \autoref{tab:tocbasic.tocstyle.attributes}
from \autopageref{tab:tocbasic.tocstyle.attributes} by prefix
\PValue{tocentry}. Later changes of the style of the entries can be made using
\Macro{DeclareTOCStyleEntry}%
\IndexCmd{DeclareTOCStyleEntry}\important{\Macro{DeclareTOCStyleEntry}}. See
\autoref{sec:tocbasic.tocstyle},
\autopageref{desc:tocbasic.cmd.DeclareTOCStyleEntry} for more information
about the styles.%
%
\begin{desclist}
  \renewcommand*{\abovecaptionskipcorrection}{-\normalbaselineskip}%
  \desccaption[{Options for command \Macro{DeclareNewTOC}}]{%
    Options for command
    \Macro{DeclareNewTOC}\ChangedAt{v3.06}{\Package{tocbasic}}%
    \label{tab:tocbasic.DeclareNewTOC-options}%
  }{%
    Options for command \Macro{DeclareNewTOC} (\emph{continuation})%
  }%
  \entry{\KOption{atbegin}\PName{instructions}%
    \ChangedAt{v3.09}{\Package{tocbasic}}}{%
    The \PName{instructions} will be executed at the begin of the floating or
    non-floating environment.%
  }%
  \entry{\KOption{atend}\PName{instructions}%
    \ChangedAt{v3.09}{\Package{tocbasic}}}{%
    The \PName{instructions} will be executed at the end of the floating or
    non-floating environment.%
  }%
  \entry{\KOption{counterwithin}\PName{\LaTeX{} counter}}{%
    If you define a float or non-float, the captions will be numbered and a
    counter \PName{type} (see option \Option{type}) will be defined. You may
    declare another counter to be the parent \LaTeX{} counter. In this case,
    the parent counter will be set before the float counter and the float
    counter will be reset whenever the parent counter is increased using
    \Macro{stepcounter} or \Macro{refstepcounter}.%
  }%
  \entry{\Option{float}}{%
    If set, float environments for that type will be defined. The names of the
    environments are the value of \PName{type} and for double column floats
    the value of \PName{type} with addendum ``\PValue{*}''.%
  }%
  \entry{\KOption{floatpos}\PName{float positions}}{%
    The default floating position of the float. If no float position was
    given, ``tbp'' will be used like the standard classes do for figures and
    tables.%
  }%
  \entry{\KOption{floattype}\PName{number}}{%
    The numerical float type of the defined floats. Float types with common
    bits cannot be reordered. At the standard classes figures has float type 1
    and tables has float type 2. If no float type was given, 16 will be used.%
  }%
  \entry{\Option{forcenames}}{%
    If set, the names will be even defined, if they where already defined
    before.%
  }%
  \entry{\KOption{hang}\PName{length}}{%
    \ChangedAt[2015/12]{v3.20}{\Package{tocbasic}}%
    \ChangedAt[2016/06]{v3.21}{\Package{tocbasic}}%
    This option is deprecated since \KOMAScript~3.20.  Now, the amount of the
    hanging indent of the entries for that list\index{table of contents>entry}
    depend on attributes of the TOC-entry style given by option
    \Option{tocentrystyle}. The styles of \KOMAScript{} provide an attribute
    \PValue{numwidth}. If the used style has such an attribute,
    \Macro{DeclareNewTOC} will initialise it with 1.5\Unit{em}. You can change
    the real \PName{value} using \KOption{tocentrynumwidth}\PName{value}. The
    \KOMAScript{} classed for example use
    \OptionValue{tocentrynumwidth}{2.3em}.%
  }%
  \entry{\KOption{indent}\PName{length}}{%
    \ChangedAt[2015/12]{v3.20}{\Package{tocbasic}}%
    \ChangedAt[2016/06]{v3.21}{\Package{tocbasic}}%
    This option is deprecated since \KOMAScript~3.20.  Now, the amount of
    indenting the entries of that list\index{table of contents>entry} depend
    on attributes of the TOC-entry style given by option
    \Option{tocentrystyle}. The styles of \KOMAScript{} provide an attribute
    \PValue{indent}.  If the used style has such an attribute,
    \Macro{DeclareNewTOC} will initialise it with 1\Unit{em}. You can change
    the real \PName{value} using \KOption{tocentryindent}\PName{value}. The
    \KOMAScript{} classed for example use
    \OptionValue{tocentrynumwidth}{1.5em}.%
  }%
  \entry{\KOption{level}\PName{number}}{%
    \ChangedAt[2015/12]{v3.20}{\Package{tocbasic}}%
    \ChangedAt[2016/06]{v3.21}{\Package{tocbasic}}%
    This option is deprecated since \KOMAScript~3.20.  Now, the level of the
    entries of that list\index{table of contents>entry} depend on attributes
    of the TOC-entry style given by option
    \Option{tocentrystyle}. Nevertheless all styles have an attribute
    \PValue{level} and \Macro{DeclareNewTOC} initialises it with 1. You can
    change the real \PName{value} using \KOption{tocentrylevel}\PName{value}.%
  }%
  \entry{\KOption{listname}\PName{string}}{%
    The name of the TOC. If not given the value of \PValue{types} with upper
    case first char using \Macro{MakeUppercase}\IndexCmd{MakeUppercase}
    prefixed by ``List of '' will be used.%
  }%
  \entry{\KOption{name}\PName{string}}{%
    The name of an element. If no name is given, the value of \PValue{type}
    with upper case first char will be used.%
  }%
  \entry{\Option{nonfloat}}{%
    If set, a non floating environment will be defined. The name of the
    environment is the value of \PName{type} with attached ``\PValue{-}''.%
  }%
  \entry{\KOption{owner}\PName{string}}{%
    The owner as described in the sections before. If no owner was given owner
    \PValue{float} will be used.%
  }%
  \entry{\KOption{tocentrystyle}\PName{TOC-entry style}}{%
    \ChangedAt[2015/12]{v3.20}{\Package{tocbasic}}%
    \PName{TOC-entry style} is the style that should be used for all entries
    into the TOC corresponding to the \PName{extension}. The name of the entry
    level is given by option \Option{type}. Additional to the options of this
    table all attributes of the \PName{TOC-entry style} can be used as
    options. To do so, you have to prefix the name of such an attribute by
    \PValue{toc}. For example, you can change the numerical level of the
    entries using option \Option{tocentrylevel}. For more information about
    the styles and their attributes see \autoref{sec:tocbasic.tocstyle} from
    \autopageref{sec:tocbasic.tocstyle}.%
  }%
  \entry{\KOption{type}\PName{string}}{%
    sets the type of the new declared TOC. The type will be used e.\,g. to
    defined a \Macro{listof}\PName{string}. If no type is set up the extension
    from the mandatory argument will be used.%
  }%
  \entry{\KOption{types}\PName{string}}{%
    the plural of the type. If no plural was given the value of \PValue{type}
    with attached ``s'' will be used.%
  }%
\end{desclist}

\begin{Example}
  Using \Macro{DeclareNewTOC} reduces the example from
  \autoref{sec:tocbasic.example} to:
\begin{lstcode}
  \DeclareNewTOC[%
    type=remarkbox,%
    types=remarkboxes,%
    float,% define a floating environment
    floattype=4,%
    name=Remark,%
    listname={List of Remarks}%
  ]{lor}
  \setuptoc{lor}{chapteratlist}
\end{lstcode}
  Beside environments \Environment{remarkbox} and \Environment{remarkbox*} the
  counter \Counter{remarkbox}, the commands \Macro{theremarkbox},
  \Macro{remarkboxname}, and \Macro{remarkboxformat} that are used for
  captions; the commands \Macro{listremarkboxnames} and
  \Macro{listofremarkboxes} that are used at the list of remarks; and some
  internal commands that depends on the file name extension \File{lor} are
  defined. If the package should use a default for the floating type, option
  Option{floattype} may be omitted. If option \Option{nonfloat} will be used
  additionally, then a non-floating environment \Environment{remarkbox-} will
  be also defined. You may use \Macro{caption}\IndexCmd{caption} inside of
  that non-floating environment as usual for floating environments.
  \hyperref[tab:tocbasic.comparison]{Figure~\ref*{tab:tocbasic.comparison}}
  shows a comparison of the commands, counters and environments of the
  example environment \Environment{remarkbox} and of the commands, counters
  and environments for figures.%
  \begin{table}
    \centering
    \caption{Comparison of example environment \Environment{remarkbox}
      and environment \Environment{figure}}
    \label{tab:tocbasic.comparison}
    \begin{tabularx}{\textwidth}{ll>{\raggedright}p{6em}X}
      \toprule
      \Environment{remarkbox} & \Environment{figure}
      & options of \Macro{DeclareNewTOC} & short description \\[1ex]
      \midrule
      \Environment{remarkbox} & \Environment{figure}
      & \Option{type}, \Option{float}
      & floating environments of the respective types\\[1ex]
      \Environment{remarkbox*} & \Environment{figure*}
      & \Option{type}, \Option{float}
      & columns spanning floating environments of the respective types\\[1ex]
      \Counter{remarkbox} & \Counter{figure}
      & \Option{type}, \Option{float}
      & counter used by \Macro{caption}\\[1ex]
      \Macro{theremarkbox} & \Macro{thefigure}
      & \Option{type}, \Option{float}
      & output command to the respective counters\\[1ex]
      \Macro{remarkboxformat} & \Macro{figureformat}
      & \Option{type}, \Option{float}
      & formatting command to the respective counters used by
        \Macro{caption}\\[1ex]
      \Macro{remarkboxname} & \Macro{figurename}
      & \Option{type}, \Option{float}, \Option{name}
      & names used in the label of \Macro{caption}\\[1ex]
      \Macro{listofremarkboxes} & \Macro{listoffigures}
      & \Option{types}, \Option{float}
      & command to show the list of the respective environments\\[1ex]
      \Macro{listremarboxname} & \Macro{listfigurename}
      & \Option{type}, \Option{float}, \Option{listname}
      & heading text of the respective list \\[1ex]
      \Macro{fps@remarkbox} & \Macro{fps@figure}
      & \Option{type}, \Option{float}, \Option{floattype}
      & numeric float type for order perpetuation\\[1ex]
      \File{lor} & \File{lof}
      &
      & file name extension of the TOC-file of the respective list \\
      \bottomrule
    \end{tabularx}
  \end{table}

  And now a possible usage of the example environment:
\begin{lstcode}
  \begin{remarkbox}
    \centering
    Equal should be typeset equally
    and with equal formatting.
    \caption{First theorem of typography}
    \label{rem:typo1}
  \end{remarkbox}
\end{lstcode}
  A segment of an example page with this environment could be:
  \begin{center}\footnotesize
    \begin{tabular}
      {|!{\hspace{.1\linewidth}}p{.55\linewidth}!{\hspace{.1\linewidth}}|}
      \\
      \centering
      Equal should be typeset equally
      and with equal formatting.\\[\abovecaptionskip]
      {%
        \usekomafont{caption}\footnotesize{\usekomafont{captionlabel}%
          Remark 1: }First theorem of typography
      }\\
    \end{tabular}%
  \end{center}%
\end{Example}

Users of \Package{hyperref} should always use option
\Option{listname}. Otherwise they may get an error message, because
\Package{hyperref} usually has a problem with the
\Macro{MakeUppercase}\IndexCmd{MakeUppercase} command that is used to change
the case of the first letter of \Option{types} in the name of the list.%
\EndIndex{Cmd}{DeclareNewTOC}%
\EndIndex{Package}{tocbasic}%
\EndIndex{}{table of contents}%
\EndIndex{}{list>of contents}%
\EndIndex{}{file>extension}%
\endinput

%%% Local Variables:
%%% mode: latex
%%% mode: flyspell
%%% coding: us-ascii
%%% ispell-local-dictionary: "en_GB"
%%% TeX-master: "../guide"
%%% End:


%  LocalWords:  Multiline multiline
