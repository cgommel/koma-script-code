% \CheckSum{134}
% \iffalse meta-comment
% ======================================================================
% scrkvars.dtx
% Copyright (c) Markus Kohm, 2006
%
% This file is part of the LaTeX2e KOMA-Script bundle.
%
% This work may be distributed and/or modified under the conditions of
% the LaTeX Project Public License, version 1.3b of the license.
% The latest version of this license is in
%   http://www.latex-project.org/lppl.txt
% and version 1.3b or later is part of all distributions of LaTeX
% version 2005/12/01 and of this work.
%
% This work has the LPPL maintenance status "author-maintained".
%
% The Current Maintainer and author of this work is Markus Kohm.
%
% This work consists of all files listed in manifest.txt.
% ----------------------------------------------------------------------
% scrkvars.dtx
% Copyright (c) Markus Kohm, 2006
%
% Dieses Werk darf nach den Bedingungen der LaTeX Project Public Lizenz,
% Version 1.3b.
% Die neuste Version dieser Lizenz ist
%   http://www.latex-project.org/lppl.txt
% und Version 1.3b ist Teil aller Verteilungen von LaTeX
% Version 2005/12/01 und dieses Werks.
%
% Dieses Werk hat den LPPL-Verwaltungs-Status "author-maintained"
% (allein durch den Autor verwaltet).
%
% Der Aktuelle Verwalter und Autor dieses Werkes ist Markus Kohm.
%
% Dieses Werk besteht aus den in manifest.txt aufgefuehrten Dateien.
% ======================================================================
% \fi
% \iffalse
%%% From File: scrkvars.dtx
%<*driver>
% \fi
\ProvidesFile{scrkvars.dtx}[2006/03/20 v3.0 KOMA-Script (koma variables)]
% \iffalse
\documentclass{scrdoc}
\usepackage[english,ngerman]{babel}
\usepackage[latin1]{inputenc}
\CodelineIndex
\RecordChanges
\GetFileInfo{scrkvars.dtx}
\title{\KOMAScript{} \partname\ \texttt{\filename}%
  \footnote{Dies ist Version \fileversion\ von Datei \texttt{\filename}.}}
\date{\filedate}
\author{Markus Kohm}

\begin{document}
  \maketitle
  \tableofcontents
  \DocInput{\filename}
\end{document}
%</driver>
% \fi
%
% \selectlanguage{ngerman}
%
% \changes{v3.0}{2006/03/20}{%
%   erste Version aus der Aufteilung von \textsf{scrclass.dtx}}
%
% \section{\KOMAScript{}-Variablen}
%
% Bei Briefen gibt es eine ganze Reihe von speziellen Variablen. An dieser
% Stelle sei als Beispiel nur das Feld "`Ihr Zeichen"' genannt. Es ist nahezu
% unm�glich alle diese Felder vorherzusehen. Daher wurde in der neuen
% Briefklasse ein neuer Mechanismus zur Definition solcher Felder geschaffen.
%
% \iffalse
%<*letter>
% \fi
%
% \StopEventually{\PrintIndex\PrintChanges}
%
% \iffalse
%<*option>
% \fi
%
% \subsection{Option}
% Der Mechanismus ist unabh�ngig von Optionen.
%
%
% \iffalse
%</option>
%<*body>
% \fi
%
% \subsection{Anweisungen zum Anlegen und Verwenden von Variablen}
%
%  \begin{macro}{\newkomavar}
%    \changes{v2.8q}{2002/01/14}{Neu}
% Es wird
% zwischen zwei Arten von Feldern unterschieden:
% \begin{itemize}
%   \item Felder f�r die Gesch�ftszeile
%   \item Felder f�r sonstige Verwendung
% \end{itemize}
% Felder k�nnen aber auch nachtr�glich zur Gesch�ftszeile
% hinzugef�gt werden. Die Sternvariante von \cs{newkomavar} f�hrt
% dies jedoch automatisch durch.
%  \begin{macro}{\@newkomavar}
% Da mit einer Sternvariante gearbeitet wird, wird ein Hilfsmakro
% ben�tigt. Das Makro erh�lt ein optionales und
% ein obligatorisches Argument. Das optionale Argument ist dabei die
% Voreinstellung f�r den \emph{Ausgabenamen} des Feldes, der
% ggf. ausgegeben werden soll. Das obligatorische Argument ist
% \emph{Name} des Feldes, �ber den das Feld angesprochen wird.
% Das Makro definiert zwei weitere Makros: \cs{\emph{scr@Name}@var} und
% \cs{scr@\emph{Name}@name}.
%    \begin{macrocode}
\newcommand*{\newkomavar}{%
  \@ifstar {\@tempswatrue\@newkomavar}{\@tempswafalse\@newkomavar}}
\newcommand*{\@newkomavar}[2][\relax]{%
  \@ifundefined{scr@#2@name}{%
    \@ifundefined{scr@#2@var}{%
      \begingroup
        \def\@tempa{#1}\def\@tempb{\relax}%
        \ifx\@tempa\@tempb\endgroup\else
        \endgroup\@namedef{scr@#2@name}{#1}%
      \fi%
      \expandafter\let\csname scr@#2@var\endcsname=\@empty
      \if@tempswa\addtoreffields{#2}\fi
      }{%
      \ClassError{%
        scrlttr2%
        }{%
        This should never happen%
        }{%
        The contents of the KOMA-Script variable `#2'\MessageBreak
        is undefined, but the name of the variable is
        defined.\MessageBreak
        This should never happen. So someone crashs me!%
        }%
      }%
    }{%
    \ClassError{%
      scrlttr2%
      }{%
      Variable `#2' already defined%
      }{%
      I'll ignore this command, if you'll continue.%
      }%
    }%
  }
%    \end{macrocode}
%  \end{macro}
%  \end{macro}
%
%  \begin{macro}{\setkomavar}
%    \changes{v2.8q}{2002/01/14}{Neu}
% Nat�rlich wird auch ein Befehl ben�tigt, um die Feldvariablen
% zu setzen und ihnen ggf. einen neuen Namen f�r die Ausgabe zu
% geben. Der \emph{Name} der Feldvariable wird als obligatorisches
% erstes Argument �bergeben.
%  \begin{macro}{\@setkomavar}
%    \changes{v2.8q}{2002/01/14}{Neu}
% Da dann zwischen einer Sternvariante und einer normalen
% unterschieden werden soll wird ein Hilfsmakro ben�tigt.
%  \begin{macro}{\@setkomaname}
%    \changes{v2.8q}{2002/01/14}{Neu}
% Die Sternvariante definiert nur den Ausgabename neu.
%  \begin{macro}{\@@setkomavar}
%    \changes{v2.8q}{2002/01/14}{Neu}
% Da in der normalen Variante nach dem Stern ein optionales Argument
% folgen kann, wird mit einem weiteren Hilfsmakro gearbeitet. Der
% \emph{Ausgabename} wird dabei als optionales, zweites Argument
% �bergeben. Der \emph{Inhalt} wird als obligatorisches, drittes
% Argument �bergeben.
%    \begin{macrocode}
\newcommand*{\setkomavar}{%
  \@ifstar {\@setkomaname}{\@setkomavar}}
\newcommand*{\@setkomavar}[1]{%
  \@ifnextchar [%]
  {\@@setkomavar{#1}}{\@@setkomavar{#1}[\relax]}}
\newcommand*{\@setkomaname}[2]{%
  \@ifundefined{scr@#1@var}{%
    \@komavar@err{set}{#1}%
    }{%
    \@namedef{scr@#1@name}{#2}%
    }%
  }
\newcommand*{\@@setkomavar}{}
\long\def\@@setkomavar#1[#2]#3{%
  \@ifundefined{scr@#1@var}{%
    \@komavar@err{set}{#1}%
    }{%
    \@namedef{scr@#1@var}{#3}%
    \begingroup
      \def\@tempa{#2}\def\@tempb{\relax}%
      \ifx\@tempa\@tempb\endgroup\else
      \endgroup\@namedef{scr@#1@name}{#2}%
    \fi
    }%
  }
%    \end{macrocode}
%  \end{macro}
%  \end{macro}
%  \end{macro}
%  \end{macro}
%
%  \begin{macro}{\@komavar@err}
%    \changes{v2.8q}{2002/01/14}{Neu}
% Dieses Makro wird intern in den Makros \cs{setkomavar} und
% \cs{usekomavar} verwendet.
%    \begin{macrocode}
\newcommand*{\@komavar@err}[2]{%
  \ClassError{%
    scrlttr2%
    }{%
    KOMA-Script variable not defined%
    }{%
    You've tried to #1 the not defined KOMA-Script variable
    `#2'.\MessageBreak
    You have to define the variable using \string\newkomavar\space
    before\MessageBreak
    you do this.%
    }%
  }
%    \end{macrocode}
%  \end{macro}
%
%  \begin{macro}{\usekomavar}
%    \changes{v2.8q}{2002/01/14}{Neu}
%    \changes{v2.9i}{2002/09/04}{robust}
% Selbstverst�ndlich wird auch ein Makro ben�tigt, um auf den
% Inhalt oder den Ausgabename einer Variablen zugreifen zu
% k�nnen. Mit der normalen Variante wird dabei der Inhalt, mit der
% Sternvariante der Name ausgegeben.
%  \begin{macro}{\@usekomavar}
%    \changes{v2.8q}{2002/01/14}{Neu}
%    \changes{v2.9i}{2002/09/04}{optionales Argument}
%    \changes{v2.9j}{2002/09/18}{\cs{@firstofone} als Defaultargument
%      eingef�gt}
%  \begin{macro}{\@usekomaname}
%    \changes{v2.8q}{2002/01/14}{Neu}
%    \changes{v2.9i}{2002/09/04}{optionales Argument}
%    \changes{v2.9j}{2002/09/18}{\cs{@firstofone} als Defaultargument
%      eingef�gt}
% Da mir einer Sternvariante gearbeitet wird, werden zwei Hilfsmakros
% ben�tigt. 
%    \begin{macrocode}
\DeclareRobustCommand*{\usekomavar}{%
  \@ifstar {\@usekomaname}{\@usekomavar}}
\newcommand*{\@usekomavar}[2][\@firstofone]{%
  \@ifundefined{scr@#2@var}{%
    \@komavar@err{use}{#2}%
    }{%
    #1{\@nameuse{scr@#2@var}}%
    }%
  }
\newcommand*{\@usekomaname}[2][\@firstofone]{%
  \@ifundefined{scr@#2@var}{%
    \@komavar@err{use}{#2}%
    }{%
    #1{\@nameuse{scr@#2@name}}%
    }%
  }
%    \end{macrocode}
%  \end{macro}
%  \end{macro}
%  \end{macro}
%
%  \begin{macro}{\ifkomavarempty}
%    \changes{v2.8q}{2002/01/14}{Neu}
%    \changes{v2.9i}{2002/09/04}{Leerzeichen rausgeworfen}
%    \changes{v2.9i}{2002/09/04}{robust}
% Es gibt auch noch ein Makro, um zu testen, ob eine KOMA-Script
% Variable oder deren Name leer ist. Die Sternvariante ist wieder f�r
% den Namen zust�ndig.
%  \begin{macro}{\if@komavarempty}
%    \changes{v2.8q}{2002/01/14}{Neu}
% Da mit einer Sternvariante gearbeitet wird, wird wieder ein
% Hilfsmakro verwendet.
%    \begin{macrocode}
\DeclareRobustCommand*{\ifkomavarempty}{%
  \begingroup
  \@ifstar {\@tempswatrue\@ifkomavarempty}%
           {\@tempswafalse\@ifkomavarempty}%
  }
\newcommand{\@ifkomavarempty}[3]{%
  \@ifundefined{scr@#1@var}{%
    \endgroup\@komavar@err{use}{#1}%
    }{%
    \if@tempswa\endgroup
      \@ifundefined{scr@#1@name}{#2}{%
        \expandafter\ifx\csname scr@#1@name\endcsname\@empty
          #2%
        \else
          #3%
        \fi
        }%
    \else\endgroup
      \expandafter\ifx\csname scr@#1@var\endcsname\@empty
        #2%
      \else
        #3%
      \fi
    \fi
    }%
  }
%    \end{macrocode}
%  \end{macro}
%  \end{macro}
%
%
% \iffalse
%</body>
% \fi
%
% \iffalse
%</letter>
% \fi
%
% \Finale
%
\endinput
%
% end of file `scrkvars.dtx'
%%% Local Variables:
%%% mode: doctex
%%% TeX-master: t
%%% End:
