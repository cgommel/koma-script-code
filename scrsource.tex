% ======================================================================
% scrsource.tex
% Copyright (c) Markus Kohm, 2002
%
% This file is part of the LaTeX2e KOMA-Script-Bundle
%
% This file can be redistributed and/or modified under the terms
% of the LaTeX Project Public License Version 1.0 distributed 
% together with this file. See LEGAL.TXT or LEGALDE.TXT.
% ----------------------------------------------------------------------
% scrsource.tex
% Copyright (c) Markus Kohm, 2002
%
% Diese Datei ist Teil des LaTeX2e KOMA-Script-Pakets.
%
% Diese Datei kann nach den Regeln der LaTeX Project Public
% Licence Version 1.0, wie sie zusammen mit dieser Datei verteilt
% wird, weiterverbreitet und/oder modifiziert werden. Siehe dazu
% auch LEGAL.TXT oder LEGALDE.TXT.
% ======================================================================

% Try to get the version of KOMA-Script from scrkvers.dtx.
\begingroup
  \def\ProvidesFile#1[#2]{\csname endinput\endcsname}%
  \input scrkvers.dtx
  \newcommand*{\ParseKOMAScriptVersion}{}
  \def\ParseKOMAScriptVersion#1/#2/#3 v#4 #5\ParseKOMAScriptVersion{%
    \gdef\filedate{#1\kern1pt-\kern0pt#2\kern1pt-\kern0pt#3}%
    \gdef\fileversion{#4}%
  }
  \expandafter
  \ParseKOMAScriptVersion\KOMAScriptVersion\ParseKOMAScriptVersion
  \gdef\filename{\jobname}
\endgroup

\ProvidesFile{scrsource.tex}[\KOMAScriptVersion (source)]
\documentclass{scrdoc}
\usepackage[english,german]{babel}
\usepackage[latin1]{inputenc}
\CodelineIndex
\RecordChanges\setcounter{GlossaryColumns}{1}
\title{\KOMAScript{} -- The Source\footnote{Ich wei� nat�rlich, dass ein
    Englischer Titel eines vorwiegend deutschsprachigen Dokuments ein
    wenig eigent�mlich ist, aber "`\KOMAScript{} -- Der
    Quelltext"' wollte mir einfach nicht gefallen.}%
}
\date{\filedate\\[1ex] Version \fileversion}
\author{Markus Kohm}

\begin{document}
  \maketitle
  \begin{abstract}\noindent
    In diesem Dokument finden Sie \emph{nicht} die Anleitung zu
    \KOMAScript{}. Wenn Sie diese suchen, dann suchen Sie bitte nach
    \texttt{scrguide} (Deutsch) oder \texttt{scrguien} (Englisch). In
    diesem Dokument ist die Implementierung von \KOMAScript{} -- in
    erster Linie der \KOMAScript{}-Klassen -- dokumentiert.
  \end{abstract}
  \tableofcontents

  \clearpage
  \addsec{Vorwort}
  
  Sie finden im Folgenden die Dokumentation der Implementierung von
  \KOMAScript.  Diese kann f�r Paketautoren von Interesse sein. Bevor
  ein Paketautor ein Makro umdefiniert oder ein internes Makro
  verwendet, sollte er jedoch R�cksprache mit dem Autor von
  \KOMAScript{} halten, damit so weit wie m�glich sichergestellt ist,
  dass bei zuk�nftigen Versionen von \KOMAScript{} die notwendige
  Kompatibilit�t erhalten bleibt oder der Paketautor �ber notwendige
  �nderungen an den entsprechenden Makros informiert wird.

  Teile der Dokumentation sind in Englisch. Der gr��te Teil ist jedoch
  in Deutsch.

  Der Quelltext ist in logische Gruppen eingeteilt und in dieser Form
  dokumentiert. Die Reihenfolge in der Dokumentation entspricht jedoch
  \emph{nicht} der Reihenfolge des Codes in den Klassen und Paketen.

  \DocInclude{scrkvers}
  \DocInclude{scrkbase}
  \DocInclude{scrkmisc}
  \DocInclude{scrklang} 
  \DocInclude{scrkfont}
  \DocInclude{scrktare} 
  \DocInclude{scrkfloa}
  \DocInclude{scrkftn} 
  \DocInclude{scrkpage}
  \DocInclude{scrkpar}% TODO: Hier geht es weiter 
  \DocInclude{scrktitl}
  \DocInclude{scrksect} 
  \DocInclude{scrkliof}
  \DocInclude{scrkbib} 
  \DocInclude{scrkidx}
  \DocInclude{scrklist} 
  \DocInclude{scrlfile} 
  \DocInclude{scrlogo}

  \PrintIndex
  \PrintChanges
\end{document}
%
% end of file `scrsource.tex'
%%% Local Variables:
%%% mode: latex
%%% mode: font-lock
%%% TeX-master: t
%%% End:
