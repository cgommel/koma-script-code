% \CheckSum{1312}
% \iffalse meta-comment
% ======================================================================
% tocbasic.dtx
% Copyright (c) Markus Kohm, 2008-2013
%
% This file is part of the LaTeX2e KOMA-Script bundle.
%
% This work may be distributed and/or modified under the conditions of
% the LaTeX Project Public License, version 1.3c of the license.
% The latest version of this license is in
%   http://www.latex-project.org/lppl.txt
% and version 1.3c or later is part of all distributions of LaTeX
% version 2005/12/01 or later and of this work.
%
% This work has the LPPL maintenance status "author-maintained".
%
% The Current Maintainer and author of this work is Markus Kohm.
%
% This work consists of files `tocbasic.dtx' and `scrlogo.dtx' at least.
%
% This work consists of all files listed in manifest.txt.
% ----------------------------------------------------------------------
% tocbasic.dtx
% Copyright (c) Markus Kohm, 2008-2013
%
% Dieses Werk darf nach den Bedingungen der LaTeX Project Public Lizenz,
% Version 1.3c, verteilt und/oder veraendert werden.
% Die neuste Version dieser Lizenz ist
%   http://www.latex-project.org/lppl.txt
% und Version 1.3c ist Teil aller Verteilungen von LaTeX
% Version 2005/12/01 oder spaeter und dieses Werks.
%
% Dieses Werk hat den LPPL-Verwaltungs-Status "author-maintained"
% (allein durch den Autor verwaltet).
%
% Der Aktuelle Verwalter und Autor dieses Werkes ist Markus Kohm.
%
% Dieses Werk besteht aus den in manifest.txt aufgefuehrten Dateien.
% ======================================================================
% \fi
%
% \CharacterTable
%  {Upper-case    \A\B\C\D\E\F\G\H\I\J\K\L\M\N\O\P\Q\R\S\T\U\V\W\X\Y\Z
%   Lower-case    \a\b\c\d\e\f\g\h\i\j\k\l\m\n\o\p\q\r\s\t\u\v\w\x\y\z
%   Digits        \0\1\2\3\4\5\6\7\8\9
%   Exclamation   \!     Double quote  \"     Hash (number) \#
%   Dollar        \$     Percent       \%     Ampersand     \&
%   Acute accent  \'     Left paren    \(     Right paren   \)
%   Asterisk      \*     Plus          \+     Comma         \,
%   Minus         \-     Point         \.     Solidus       \/
%   Colon         \:     Semicolon     \;     Less than     \<
%   Equals        \=     Greater than  \>     Question mark \?
%   Commercial at \@     Left bracket  \[     Backslash     \\
%   Right bracket \]     Circumflex    \^     Underscore    \_
%   Grave accent  \`     Left brace    \{     Vertical bar  \|
%   Right brace   \}     Tilde         \~}
%
% \iffalse
%%% From File: $Id$
%<*dtx>
\ifx\ProvidesFile\undefined\def\ProvidesFile#1[#2]{}\fi
\ProvidesFile{tocbasic.dtx}[%
%</dtx>
%<package>\NeedsTeXFormat{LaTeX2e}[1995/12/01]
%<doc>\ProvidesFile{tocbasic.tex}[%
%<package>\ProvidesPackage{tocbasic}[%
%<*dtx|package|driver|doc>
%!KOMAScriptVersion
  package
%</dtx|package|driver|doc>
%<driver>  (driver)%
%<package|doc>  (handling toc-files)%
%<*dtx>
  (dtx)%
%</dtx>
%<*dtx|package|driver|doc>
]
%</dtx|package|driver|doc>
%<*dtx>
\ifx\documentclass\undefined
  \input docstrip.tex
  \@@input scrdocstrip.tex
  \@@input scrkvers.dtx
  \@@input scrstrip.inc
  \KOMAdefVariable{COPYRIGHTFROM}{2008}
  \generate{\usepreamble\defaultpreamble
    \useWarning\@gobbletwo
    \file{tocbasic.tex}{%
      \from{tocbasic.dtx}{doc}%
    }%
  }
  \generate{\usepreamble\defaultpreamble
    \file{tocbasic.sty}{%
      \from{tocbasic.dtx}{package}%
      \from{scrlogo.dtx}{logo}%
    }%
  }%
  \@@input scrstrop.inc
\else
  \let\endbatchfile\relax
\fi
\endbatchfile
%</dtx>
%<*driver>
\documentclass[halfparskip-]{scrdoc}
\usepackage[latin1]{inputenc}
\usepackage[english]{babel}
\usepackage{array}
\usepackage{listings}
\lstset{breaklines,prebreak=\mbox{$\hookleftarrow$},language=[LaTeX]TeX,
  basicstyle=\small}
\lstnewenvironment{lstcode}[1][]{%
  \lstset{breaklines,prebreak=\mbox{$\hookleftarrow$},language=[LaTeX]TeX,
    basicstyle=\ttfamily\small,#1}%
}{}
\lstnewenvironment{lstoutput}[1][]{}{}
\CodelineIndex
\RecordChanges
\GetFileInfo{tocbasic.dtx}
\sloppy
\DocInput{\filename}
\end{document}
%</driver>
% \fi
%
% \providecommand*{\DescribeCounter}{\DescribeMacro}
% \let\Macro\cs
% \let\Class\textsf
% \let\Package\textsf
% \let\File\texttt
% \let\Environment\texttt
% \let\Parameter\marg
% \let\OParameter\oarg
% \let\PName\meta
% \let\PValue\texttt
% \let\Option\texttt
% \newenvironment{Example}{\quote
%   \hspace*{-\leftmargin}\textbf{Example:}\enskip\ignorespaces
% }{\endquote}
% \newlength\descwidth
% \newenvironment{desctabular}{
%   \setlength{\descwidth}{\linewidth}
%   \addtolength{\descwidth}{-1em}
%   \addtolength{\descwidth}{-2\tabcolsep}
%   \tabular{@{}lp{\descwidth}@{}}
%   \hline
% }{%
%   \hline
%   \endtabular
% }
% \newcommand{\pventry}[2]{\multicolumn{2}{@{}l@{}}{\PValue{#1}}\\
%   ~ & #2 \\ }
% \providecommand*{\autoref}[1]{\expandafter\AUTOREF#1:}
% \newcommand*{\AUTOREF}{}
% \makeatletter
% \def\AUTOREF#1:#2:{\edef\@tempa{#1}\edef\@tempb{tab}\ifx\@tempa\@tempb 
%   table~\fi
%   \edef\@tempb{sec}\ifx\@tempa\@tempb section~\fi
%   \ref{#1:#2}}
% \makeatother
% \providecommand*{\BeginIndex}[2]{}
% \providecommand*{\IndexPackage}[2][]{\index{#2 (package)}}
% \providecommand*{\IndexCmd}[2][]{\index{\cs{#2}}}
% \providecommand*{\Index}[1][]{\index}
% \providecommand*{\IndexOption}[2][]{\index{#2 (option)}}
% \providecommand*{\IndexLength}[2][]{\index{\cs{#2} (length)}}
% \providecommand*{\textnote}[2][]{}
% \providecommand*{\important}[2][]{}
% \providecommand*{\KOption}[1]{\texttt{#1=}}
% \providecommand*{\OptionValue}[2]{\texttt{#1=#2}}
% \providecommand*{\ChangedAt}[2]{}
%
% \begin{document}
% \title{\KOMAScript{} \partname\ \texttt{\filename}%
%   \thanks{Diese Datei ist Version \fileversion\ von \texttt{\filename}.}}
% \date{\filedate}
% \author{Markus Kohm\thanks{mailto:komascript(at)gmx.info}}
% \maketitle
% \begin{abstract}
% \iffalse
%<*doc>

\translator{Markus Kohm}

% Date of the translated German file: 2012/03/14

\chapter[{Management of Tables and Lists of Contents Using
  \Package{tocbasic}}]
  {Management of Tables and Lists of Contents Using
  \Package{tocbasic}\protect\footnote{This chapter has been generated from the
 source of the package. It's not a 1-to-1 translation of the German
 manual. Currently translation has not been finished}}
\labelbase{tocbasic}

\BeginIndex{Package}{tocbasic}%
\BeginIndex{}{table of contents}%
\BeginIndex{}{list>of contents}%
\BeginIndex{}{file>extension}%
%</doc>
%<*doc|dtx>
% \fi
The main purpose of package \Package{tocbasic} is to provide features for
authors of classes and packages to create own tables or lists of contents like
the list of figures and the list of tables and thereby allow other classes or
packages some types of controll about these. For examples package
\Package{tocbasic} delegates language control of all these tables and lists of
contents to package \Package{babel}\IndexPackage{babel}%^^A
%\iffalse^^A
(see \cite{package:babel})%^^A
%\fi^^A
. So automatic change of language will be provides inside
all these tables and lists of contents. Using \Package{tocbasic} will
exculpate authors of classes and packages from implementation of such
features.

\KOMAScript{} itself uses \Package{tocbasic} not only for the table of
contents but also for the already mentioned lists of figures and tables.
% \iffalse
%<*dtx>
% \fi
% \end{abstract}
%
% \tableofcontents
% 
% \section{Legal Note}
% \label{sec:tocbasic.legalnote}
% 
% You are allowed to destribute this part of \KOMAScript{} without the main
% part of \KOMAScript{}. The files ``\File{scrlogo.dtx}'' and
% ``\File{tocbasic.dtx}'' may be distributed together under the conditions
% of the \LaTeX{} Project Public License, either version~1.3c of this license
% or (at your option) any later version.
%
% The latest version of this license is in
% \mbox{http://www.latex-project.org/lppl.txt} and version~1.3c or later is
% part of all distributions of \LaTeX{} version~2005/12/01 or later.
%
% The \KOMAScript{} bundle may be found at
% CTAN:/\mbox{macros}/\linebreak[3]\mbox{latex}/\linebreak[3]\mbox{contrib}/%
% \linebreak[3]\mbox{koma-script}/. 
% ``CTAN:'' is a shortcut for every ``tex-archive'' directory at every
% CTAN-server or CTAN-mirror. See \mbox{http://www.ctan.org} for a list of all
% those servers and mirrors.
%
% \section{Using Package \Package{tocbasic}}
% \label{sec:tocbasic.usage}
%
% This package was made to be used by class and package authors.  Because of
% this the package has no options.  If different packages would load it with
% different options a option clash would be the result.  So using options
% wouldn't be a good idea.
%
% This package may be used by by other class and package authors.  It is also
% allowed to distribute it without the main part of \KOMAScript{} (see
% section~\ref{sec:tocbasic.legalnote}.  Because of this, it does not use any
% \KOMAScript{} package, that isn't allowed to be distributed without the main
% part of \KOMAScript{}, too.  Currently \Package{tocbasic} does only need
% package \Package{keyval} from the graphics bundle.
%
% There are two kind of commands.  The first kind are basic command.  Those
% are used to inform other packages about the extensions used for files that
% represent a list of something.  Classes or packages may use this information
% e.\,g, for putting something to every of those files.  Packages may also ask,
% if an extension is already in use.  This does even work, if \Macro{nofiles}
% was used. The second kind are commands to create the list of something.
% \iffalse
%</dtx>
% \fi

\section{Basic Commands}
\label{sec:tocbasic.basics}

Basic commands are used to handle a list of all extensions\textnote{file
  extension, table or list of contents} known for files representing a table
of contents or list of something.  Entries to such files are written using
\Macro{addtocontents}\important{\Macro{addtocontents},
  \Macro{addcontentsline}} or \Macro{addcontentsline} typically.  There are
also commands to do something for all known extensions.  And there are
commands to set or unset features of an extension or the file represented by
the extension.  Typically an extension also has an owner\textnote{file owner}.
This owner may be a class or package or a term decided by the author of the
class or package using \Package{tocbasic}, e.\,g., \KOMAScript{} uses the
owner \texttt{float} for list of figures and list of tables, and the file name
of the class file as owner for the table of contents.

% \iffalse
\begin{Declaration}
  \Macro{ifattoclist}\Parameter{extension}\Parameter{true
    part}\Parameter{false part}
\end{Declaration}
\BeginIndex{Cmd}{ifattoclist}%
This command
%<*dtx>
%\fi
\DescribeMacro{\ifattoclist}
Command \Macro{ifattoclist}\Parameter{extension}\Parameter{true
  instructions}\Parameter{false instructions}
% \iffalse
%</dtx>
% \fi
may be used to ask, wether or not a \PName{extension} is already a known
extension.  If the \PName{extension} is already known the \PName{true
  instructions} will be used, otherwise the \PName{false instructions} will be
used.
\begin{Example}
  Maybe you want to know if the extension ``\File{foo}'' is already in use to
  report an error, if you can not use it:
\begin{lstcode}
  \ifattoclist{foo}{%
    \PackageError{bar}{%
      extension `foo' already in use%
    }{%
      Each extension may be used only 
      once.\MessageBreak
      The class or another package already 
      uses extension `foo'.\MessageBreak
      This error is fatal!\MessageBreak  
      You should not continue!}%
  }{%
    \PackageInfo{bar}{using extension `foo'}%
  }
\end{lstcode}
\end{Example}
% \iffalse
\EndIndex{Cmd}{ifattoclist}%

\begin{Declaration}
  \Macro{addtotoclist}\OParameter{owner}\Parameter{extension}
\end{Declaration}
\BeginIndex{Cmd}{addtotoclist}%
This command
%<*dtx>
%\fi
\DescribeMacro{\addtotoclist}
Command \Macro{addtotoclist}\OParameter{owner}\Parameter{extension}
% \iffalse
%</dtx>
% \fi
adds the \PName{extension} to the list of known extensions. But if the
\PName{extension} is a known one already, then an error will be reported to
avoid double usage of the same \PName{extension}.

If the optional argument, \OParameter{owner}, was given this \PName{owner}
will be stored to be the owner of the \PName{extension}.  If the optional
argument has been omitted, \Package{tocbasic} tries to find out the file name
of the current processed class or package and stores this as owner.
This\textnote{Attention!} will fail if \Macro{addtotoclist} was not used,
loading a class or package but using a command of a class or package after
loading this class or package.  In this case the owner will be set to
``\PValue{.}''.  

Please note\textnote{Attention!} that an empty \PName{owner} is not the same
like omitting the optional argument with the braces. An empty argument would
result in an empty owner.
\begin{Example}
  You want to add the extension ``\File{foo}'' to the list of known extension,
  while loading your package with file name ``\File{bar.sty}'':
\begin{lstcode}
  \addtotoclist{foo}
\end{lstcode}%
  This will add the extension ``\PValue{foo}'' with owner ``\PValue{bar.sty}''
  to the list of known extensions, if it was not already at the list of known
  extensions. If the class or another package already added the extension you
  will get the error:
\begin{lstoutput}
  Package tocbasic Error: file extension `foo' cannot be used twice

  See the tocbasic package documentation for explanation.
  Type  H <return>  for immediate help.
\end{lstoutput}
  and after typing \texttt{H} and pressing the return key you will get the
  help:
\begin{lstoutput}
  File extension `foo' is already used by a toc-file, while bar.sty
  tried to use it again for a toc-file.
  This may be either an incompatibility of packages, an error at a package,
  or a mistake by the user.
\end{lstoutput}
  
  Maybe you package has a command, that creates list of files dynamically.  In
  this case you should use the optional argument of \Macro{addtotoclist} to
  set the owner.
\begin{lstcode}
  \newcommand*{\createnewlistofsomething}[1]{%
    \addtotoclist[bar.sty]{#1}%
    % Do something more to make this list of something available
  }
\end{lstcode}
  If the user calls now, e.\,g.,
\begin{lstcode}
  \createnewlistofsomething{foo}
\end{lstcode}
  this would add the extension ``\PValue{foo}'' with the owner
  ``\PValue{bar.sty}'' to the list of known extension or report an error, if the
  extension is already in use.
\end{Example}
You may use any owner you want.  But it should be unique!  So, if you would
be, e.\,g., the author of package \Package{float} you could use for example
owner ``\PValue{float}'' instead of owner ``\PValue{float.sty}'', so the
\KOMAScript{} options for list of figure and list of table will also handle
the lists of this package, that are already added to the known extensions,
when the option is used. This is because \KOMAScript{} already registers file
extension ``\PValue{lof}'' for the list of figures and file extension
``\PValue{lot}'' for the list of tables with owner ``\PValue{float}'' and sets
options for this owner.
% \iffalse
\EndIndex{Cmd}{addtotoclist}%
% \fi

% \iffalse
\begin{Declaration}
  \Macro{AtAddToTocList}\OParameter{owner}\Parameter{instructions}
\end{Declaration}
\BeginIndex{Cmd}{AtAddToTocList}%
This command
%<*dtx>
%\fi
\DescribeMacro{\AtAddtoTocList}
Command \Macro{AtAddToTocList}\OParameter{owner}\Parameter{instructions}
% \iffalse
%</dtx>
% \fi
adds the \PName{instructions} to a internal list of instructions, that will be
processed, whenever a file extension with the given \PName{owner} will be
added to the list of known extensions using \Macro{addtotoclist}.  The
optional argument is handled in the same kind as with command
\Macro{addtotoclist}. With an empty \PName{owner} you may
add \Parameter{instructions}, that will be processed at every successful
\Macro{addtotoclist}, after processing the instructions for the individual
owner.  While processing the \PValue{instructions},
\Macro{@currext}\IndexCmd{@currext}\important{\Macro{@currext}} will be set to
the extension of the currently added extension.
\begin{Example}
  \Package{tocbasic} itself uses
\begin{lstcode}
  \AtAddToTocList[]{%
    \expandafter\tocbasic@extend@babel
    \expandafter{\@currext}%
  }
\end{lstcode}
  to add every extension to the \Package{tocbasic}-internal babel handling of
  files.
\end{Example}

The two \Macro{expandafter} commands are needed, because the argument of
\Macro{tocbasic@extend@babel} has to be expanded!  See the description of
\Macro{tocbasic@extend@babel} at \autoref{sec:tocbasic.internals}%
% \iffalse
, \autopageref{desc:tocbasic.cmd.tocbasic@extend@babel}
% \fi
for more information.
% \iffalse
\EndIndex{Cmd}{AtAddToTocList}%
% \fi

% \iffalse
\begin{Declaration}
  \Macro{removefromtoclist}\OParameter{owner}\Parameter{extension}
\end{Declaration}
\BeginIndex{Cmd}{removefromtoclist}%
This command
%<*dtx>
%\fi
\DescribeMacro{\removefromtoclist}
Command \Macro{removefromtotoclist}\OParameter{owner}\Parameter{extension}
% \iffalse
%</dtx>
% \fi
removes the \PName{extension} from the list of known extensions.  If the
optional argument, \OParameter{owner}, was given the \PName{extension} will
only be removed, if it was added by this \PName{owner}. See description of
\Macro{addtotoclist} for information of omitting optional argument. Note that
an empty \PName{owner} is not the same like omitting the optional argument,
but removes the \PName{extension} without any owner test.%
% \iffalse
\EndIndex{Cmd}{removefromtoclist}%
% \fi

% \iffalse
\begin{Declaration}
  \Macro{doforeachtocfile}\OParameter{owner}\Parameter{instructions}
\end{Declaration}
\BeginIndex{Cmd}{doforeachtocfile}%
%<*dtx>
% \fi
\DescribeMacro{\doforeachtocfile}
% \iffalse
%</dtx>
% \fi
Until now you've learned to know commands, that result in more safety in
handling file extensions, but also needs some additional effort. With
\Macro{doforeachtocfile}%^^A
% \iffalse
%<*dtx>
% \fi
\OParameter{owner}\Parameter{instructions}
% \iffalse
%</dtx>
% \fi
you'll win for this. The command provides to processes \PName{instructions}
for every known file extension of the given \PName{owner}.  While processing
the \PName{instructions} \Macro{@currext} is the extension of the current
file.  If you omit the optional argument, \OParameter{owner}, every known file
extensions independent from the owner will be used. If the optional argument
is empty, only file extensions with an empty owner will be processed.
\begin{Example}
  If you want to type out all known extensions, you may simply write:
\begin{lstcode}
  \doforeachtocfile{\typeout{\@currext}}
\end{lstcode}
  and if only the extensions of owner ``\PValue{foo}'' should be typed out:
\begin{lstcode}
  \doforeachtocfile[foo]{\typeout{\@currext}}
\end{lstcode}
\end{Example}
% \iffalse
\EndIndex{Cmd}{doforeachtocfile}%
% \fi

% \iffalse
%<*dtx>
% \fi
\DescribeMacro{\tocbasicautomode}
Command
% \iffalse
%</dtx>
\begin{Declaration}
% \fi
  \Macro{tocbasicautomode}
% \iffalse
\end{Declaration}
\BeginIndex{Cmd}{tocbasicautomode}%
This command
% \fi
redefines \LaTeX{} kernel macro \Macro{@starttoc} to add all not yet added
extensions to the list of known extensions and use \Macro{tocbasic@starttoc}
instead of \Macro{@starttoc}. See \autoref{sec:tocbasic.internals}%^^A
% \iffalse
, \autopageref{desc:tocbasic.cmd.tocbasic@starttoc}
% \fi
for more information about \Macro{tocbasic@starttoc} and \Macro{@starttoc}.

This means, that after using \Macro{tocbasicautomode} every table of contents
or list of something, that will be generated using \Macro{@starttoc} will be
at least partial under control of \Package{tocbasic}. Whether or not this will
make the wanted result, depends on the individual table of contents and lists
of something. At least the \Package{babel} control extension for
all those tables of contents and lists of something will work. Nevertheless,
it would be better, if the author of the corresponding class or package will
use \Package{tocbasic} explicitly. In that case additional advantages of
\Package{tocbasic} may be used, that will be described at the following
sections.%
% \iffalse
\EndIndex{Cmd}{tocbasicautomode}%
% \fi

\section{Creating a Table of Contents or List of Something}
\label{sec:tocbasic.toc}

At the previous section you've seen commands to handle a list of known
extensions and to trigger commands while adding a new extension to this
list. You've also seen a command to do something for all known extensions or
all known extensions of one owner. In this section you will see commands to
handle the file corresponding with an extension or the list of known
extensions.

% \iffalse
\begin{Declaration}
  \Macro{addtoeachtocfile}\OParameter{owner}\Parameter{content}
\end{Declaration}
\BeginIndex{Cmd}{addtoeachtocfile}%
This command
%<*dtx>
% \fi
\DescribeMacro{\addtoeachtocfile}
Command \Macro{addtoeachtocfile}\OParameter{owner}\Parameter{content}
% \iffalse
%</dtx>
% \fi
writes \PName{content} to the files of every known file extension of
\PName{owner} using \LaTeX{} kernel command \Macro{addtocontents}. If you omit
the optional argument, \PName{content} it written to the files of every known
file extension. Bay the way: The practical file name is build from
\Macro{jobname} and the file extension. While writing the \PName{content},
\Macro{@currext}\IndexCmd{@currext}\important{\Macro{@currext}} is the
extension of the currently handled file.
\begin{Example}
  You may add a vertical space of one text line to all toc-files.
\begin{lstcode}
    \addtoeachtocfile{%
      \protect\addvspace{\protect\baselineskip}%
    }
\end{lstcode}
  And if you want to do this, only for the toc-files of owner
  ``\PValue{foo}'':
\begin{lstcode}
    \addtoeachtocfile[foo]{%
      \protect\addvspace{\protect\baselineskip}%
    }
\end{lstcode}
\end{Example}
Commands, that shouldn't be expanded while writing, should be prefixed by
\Macro{protect} in the same way like they should be in the argument of
\Macro{addtocontents}.
% \iffalse
\EndIndex{Cmd}{addtoeachtocfile}%
% \fi


%\iffalse^^A meta-comment
\begin{Declaration}
%\fi^^A meta-comment
% \DescribeMacro{\addxcontentsline}
% The command
  \Macro{addxcontentsline}%
  \Parameter{extension}\Parameter{level}\OParameter{number}%^^A
  \Parameter{text}
%\iffalse^^A meta-comment
\end{Declaration}
\BeginIndex{Cmd}{addxcontentsline}%
The command \Macro{addxcontentsline}
%\fi^^A meta-comment
adds an entry of given \PName{level} to toc-file with \PName{extension}. If
the \Parameter{number} is empty or omitted the entry won't have a number for
the entry with the given \PName{text}. Entries without number may be left
aligned to the number of the numbered entries of the same \PName{level} or
indented like the text of the numbered entries of the same \PName{level},
depending on the \PValue{numberline} feature.

\begin{Example}
  Maybe you are not using a \KOMAScript{} class but need a not numbered
  chapter with entry to the table of contents. This may be done using
\begin{lstcode}
  \cleardoublepage
  \csname phantomsection\endcsname
  \addxcontentsline{toc}{chapter}
             {Chapters without Numbers}
  \chapter*{Chapters without Numbers}
  \markboth{Chapters without Numbers}{}
\end{lstcode}
  As you can see, you simply have to replace usual \Macro{addcontentsline} by
  \Macro{addxcontentsline} to support the \Package{tocbasic} feature
  \PValue{numberline}.
\end{Example}

Note, that \Macro{addxcontentsline} uses
\Macro{add\PName{level}\PName{extension}entry} if such a macro exists and
\Macro{tocbasic@addxcontentsline} otherwise. Therefore you cannot define
a macro \Macro{add\PName{level}\PName{extension}entry} using
\Macro{addxcontentsline} but \Macro{tocbasic@addxcontentsline}.

It is recommended to use \Macro{addxcontentsline} instead of
\Macro{addcontentsline} whenever possible.%
%\iffalse^^A meta-comment
\EndIndex{Cmd}{addxcontentsline}
%\fi^^A meta-comment


% \iffalse
\begin{Declaration}
% \fi
% \DescribeMacro{\addcontentslinetoeachfile}
% \DescribeMacro{\addxcontentslinetoeachfile}
% The command
  \Macro{addcontentslinetoeachtocfile}\OParameter{owner}\Parameter{level}%^^A
  \Parameter{contentsline}\\
% \iffalse
  \Macro{addxcontentslinetoeachtocfile}\OParameter{owner}%^^A
  \Parameter{level}\OParameter{number}\Parameter{text}
\end{Declaration}
\BeginIndex{Cmd}{addcontentslinetoeachtocfile}%
\BeginIndex{Cmd}{addxcontentslinetoeachtocfile}%
The first command
% \fi
is something like \Macro{addcontentsline} from \LaTeX{} kernel. In difference
to that it writes the \PName{contentsline} not only
into one file, but into all files of all known file extensions or of all known
file extensions of a given owner.

The Command \Macro{addxcontentslinetoeachtocfile}%^^A
% \OParameter{owner}\Parameter{level}\OParameter{number}\Parameter{text}%^^A
\ is similar but uses
\Macro{addxcontentsline} instead of \Macro{addcontentsline} and
therefore supports \Package{tocbasic} feature \PValue{numberline}. 
\begin{Example}
  You are a class author and want to write the chapter entry not only to the
  table of contents toc-file but to all toc-files, while \texttt{\#1} is the
  title, that should be written to the files.
\begin{lstcode}
    \addxcontentslinetoeachtocfile
              {chapter}[\thechapter]{#1}%
\end{lstcode}
  In this case the current chapter number should be expanded while writing
  into the file. So it isn't protected from expansion using \Macro{protect}.
\end{Example}
While writing \Macro{@currext}\IndexCmd{@currext}\important{\Macro{@currext}}
is the file extension of the file into which \PName{contentsline} will be
written.

It is recommended to use \Macro{addxcontentslinetoeachtocfile} instead
of \Macro{addcontentslinetoeachtocfile} whenever possible.
% \iffalse
\EndIndex{Cmd}{addxcontentslinetoeachtocfile}%
\EndIndex{Cmd}{addcontentslinetoeachtocfile}%
% \fi


% \iffalse
%<*dtx>
% \fi
\DescribeMacro{\listoftoc}%^^A
\DescribeMacro{\listoftoc*}%^^A
\DescribeMacro{\listofeachtoc}%^^A
Commands
% \iffalse
%</dtx>
\begin{Declaration}
% \fi
  \Macro{listoftoc}\OParameter{list of title}\Parameter{extension}%^^A
% \iffalse
\\
%<*dtx>
% \fi
,
% \iffalse
%</dtx>
% \fi
  \Macro{listoftoc*}\Parameter{extension}%^^A
% \iffalse
\\
%<*dtx>
% \fi
,
% \iffalse
%</dtx>
% \fi
  \Macro{listofeachtoc}\OParameter{owner}%^^A
% \iffalse
\\
%<*dtx>
% \fi
, and
% \iffalse
%</dtx>
% \fi
  \Macro{listof\PName{file-extension}name}
% \iffalse
\end{Declaration}
\BeginIndex{Cmd}{listoftoc*}%
\BeginIndex{Cmd}{listoftoc}%
\BeginIndex{Cmd}{listofeachtoc}%
\BeginIndex{Cmd}{listof\PName{file-extension}name}%
These commands
% \fi
may be used to set the ``list of'' corresponding to file
extension. The\important{\Macro{listoftoc*}} star version \Macro{listoftoc*}
needs only one argument, the extension of the file. It does setup the vertical
and horizontal spacing of paragraphs, calls before hooks, reads the file, and
last but not least calls the after hooks.  You may interpret it as direct
replacement of the \LaTeX{} kernel macro
\Macro{@starttoc}\IndexCmd{@starttoc}\important{\Macro{@starttoc}}.

The\important{\Macro{listoftoc}} version without star, sets the whole file
with title, optional table of contents entry, and running heads. If the
optional argument \OParameter{list of title} was given, it will be used as
title term, optional table of contents entry and running head. Please
note\textnote{Attention!}: If the optional argument is empty, this term will
be empty, too! If you omit the optional argument, but
\Macro{listof\PName{extension}name} was defined, that will be used. If that is
also not defined, a standard ersatz name will be used and reported by a
warning message.

The\important{\Macro{listofeachtoc}} command \Macro{listofeachtoc} outputs all
lists of something of the given \PName{owner} or of all known file
extensions. Thereby\textnote{Attention!}
\Macro{listof\PName{file-extension}name} should be defined to get the correct
titles.

It\textnote{Hint!} is recommended to define
\Macro{listof\PName{file-extension}name} for all used file extensions, because
the user itself may use \Macro{listofeachtoc}.
\begin{Example}
  Assumed, you have a new ``list of algorithms'' with extension \PValue{loa}
  and want to show it:
\begin{lstcode}
  \listoftoc[List of Algorithms]{loa}
\end{lstcode}
  will do it for you. But maybe the ``list of algorithms'' should not be set
  with a title. So you may use
\begin{lstcode}
  \listof*{loa}
\end{lstcode}
  Note that in this case no entry at the table of contents will be created,
  even if you'd used the setup command above.
  See command \Macro{setuptoc}
%\iffalse
  at \autopageref{desc:tocbasic.cmd.setuptoc}
%\fi
  for more information about the
  attribute of generating entries into the table of contents using
  \Macro{setuptoc}.

  If you've defined
\begin{lstcode}
  \newcommand*{\listofloaname}{%
    List of Algorithms%
  }
\end{lstcode}
  before, then
\begin{lstcode}
  \listoftoc{loa}
\end{lstcode}
  would be enough to print the list of algorithms with the wanted heading. For
  the user it may be easier to operate, if you'd define
\begin{lstcode}
  \newcommand*{\listofalgorithms}{\listoftoc{loa}}
\end{lstcode}
  additionally.
\end{Example}

Because\textnote{Attention!} \LaTeX{} normally opens a new file for each of
those lists of something immediately, the call of each of those commands may
result in an error like:
\begin{lstoutput}
  ! No room for a new \write .
  \ch@ck ...\else \errmessage {No room for a new #3}
                                                    \fi   
\end{lstoutput}
if there are no more write handles left. Loading package
\Package{scrwfile}\important{\Package{scrwfile}}\IndexPackage{scrwfile}
%\iffalse
(see \autoref{cha:scrwfile})
%\fi
may solve this problem.
%\iffalse
\EndIndex{Cmd}{listof\PName{Dateierweiterung}name}%
\EndIndex{Cmd}{listofeachtoc}%
\EndIndex{Cmd}{listoftoc*}%
\EndIndex{Cmd}{listoftoc}%
%\fi

% \iffalse
%<*dtx>
% \fi
\DescribeMacro{\BeforeStartingTOC}
\DescribeMacro{\AfterStartingTOC}
The Commands
% \iffalse
%</dtx>
\begin{Declaration}
% \fi
  \Macro{BeforeStartingTOC}\OParameter{extension}\Parameter{instructions}
% \iffalse
\\
%<*dtx>
% \fi
and
% \iffalse
%</dtx>
% \fi
  \Macro{AfterStartingTOC}\OParameter{extension}\Parameter{instructions}
% \iffalse
\end{Declaration}
\BeginIndex{Cmd}{BeforeStartingTOC}%
\BeginIndex{Cmd}{AfterStartingTOC}%
Sometimes it's useful, to process \PName{instructions} immediately before
reading the helper file of a list of something.
These commands
% \fi
may be used to process \PName{instructions} before or after loading the file
with given \PName{extension} using \Macro{listoftoc*}, \Macro{listoftoc}, or
\Macro{listofeachtoc}.  If you omit the optional argument (or set an empty
one) the general hooks will be set. The general before hook will be called
before the individuel one and the general after hook will be called after the
individuel one. While calling the hooks
\Macro{@currext}\IndexCmd{@currext}\important{\Macro{@currext}} is the
extension of the toc-file and should not be changed.
% \iffalse

An example\textnote{Example} for usage of \Macro{AfterStartingTOC} may be
found in \autoref{sec:scrwfile.clonefilefeature} at
\autopageref{example:scrwfile.AfterStartingTOC}.
\EndIndex{Cmd}{AfterStartingTOC}%
\EndIndex{Cmd}{BeforeStartingTOC}%
% \fi

% \iffalse
%<*dtx>
% \fi
\DescribeMacro{\BeforeTOCHead}
\DescribeMacro{\AfterTOCHead}
The Commands
% \iffalse
%</dtx>
\begin{Declaration}
% \fi
  \Macro{BeforeTOCHead}\OParameter{file extension}\Parameter{instructions}
% \iffalse
\\
%<*dtx>
% \fi
and
% \iffalse
%</dtx>
% \fi
  \Macro{AfterTOCHead}\OParameter{file extension}\Parameter{instructions}
% \iffalse
\end{Declaration}
\BeginIndex{Cmd}{BeforeTOCHead}%
\BeginIndex{Cmd}{AfterTOCHead}%
This commands
% \fi
may be used to process \PName{instructions} before or after setting the title
of a list of something corresponding to given \PName{file extension} using
\Macro{listoftoc*} or \Macro{listoftoc}. If you omit the optional argument (or
set an empty one) the general hooks will be set. The general before hook will
be called before the individuel one and the general after hook will be called
after the individuel one. While calling the hooks
\Macro{@currext}IndexCmd{@currext}\important{\Macro{@currext}} is the
extension of the corresponding file and should not be changed.
% \iffalse
\EndIndex{Cmd}{AfterTOCHead}%
\EndIndex{Cmd}{BeforeTOCHead}%
% \fi

% \iffalse
%<*dtx>
% \fi
\DescribeMacro{\MakeMarkcase}
% \iffalse
%</dtx>
\begin{Declaration}
  \Macro{MakeMarkcase}\Parameter{text}
\end{Declaration}
\BeginIndex{Cmd}{MakeMarkcase}%
% \fi
Whenever \Package{tocbasic} sets a mark for a running head, The text of the
mark will be an argument of \Macro{MakeMarkcase}%^^A
%\Parameter{text}^^A
. This command may be used, to change the case of the letters at the running
head if wanted. The default is, to use
\Macro{@firstofone}\IndexCmd{@firstofone}\important{\Macro{@firstofone}} for
\KOMAScript{} classes. This means the text of the running head will be set
without change of case.
\Macro{MakeUppercase}\IndexCmd{MakeUppercase}\important{\Macro{MakeUppercase}}
will be used for all other classes. If you are the class author you may define
\Macro{MakeMarkcase} on your own. If \Package{scrpage2} or another package,
that defines \Macro{MakeMarkcase} will be used, \Package{tocbasic} will not
overwrite that definition.
\begin{Example}
  For incomprehensible reasons, you want to set the running heads in lower
  case letters only. To make this automatically for all running heads, that
  will be set by \Package{tocbasic}, you define:
\begin{lstcode}
  \let\MakeMarkcase\MakeLowercase
\end{lstcode}
\end{Example}
Please\textnote{Hint!} allow me some words about \Macro{MakeUppercase}, First
of all this command isn't fully expandable. This means, that problems may
occur using it in the context of other commands. Beyond that typographers
accord, that whenever setting whole words or phrases in capitals, letter
spacing is absolutely necessary. But correct letter spacing of capitals
shouldn't be done with a fix white space between all letters. Different pairs
of letters need different space between each other. Additional some letters
build holes in the text, that have to be taken into account. Packages like
\Package{ulem} or \Package{soul} doesn't provide this and
\Macro{MakeUppercase} doesn't do anything like this. Also automatic letter
spacing using package \Package{microtype} is only one step to a less-than-ideal
solution, because it cannot recognize and take into account the glyphs of the
letters. Because of this\textnote{Attention!} typesetting whole words and
phrases is expert work and almost ever must be hand made. So average users are
recommended to not do that or to use it only spare and not at exposed places
like running heads.%
% \iffalse
\EndIndex{Cmd}{MakeMarkcase}%
% \fi

% \DescribeMacro{\deftocheading}
% \iffalse
\begin{Declaration}
  \Macro{deftocheading}\Parameter{file extension}\Parameter{definition}
\end{Declaration}
\BeginIndex{Cmd}{deftocheading}%
% \fi
The package \Package{tocbasic} contains a standard definition for typesetting
headings of tables of contents or lists of something. This standard definition
is configurable by several features, described at \Macro{setuptoc} next. But
if all those features are not enough, an alternative heading command may be
defined using \Macro{deftocheading}%^^A
%\Parameter{file extension}\Parameter{definition}%^^A
. Thereby \PName{file extension} is the file extension of the corresponding
helper file. The \PName{definition} of the heading command may use one single
parameter \PValue{\#1}. While calling the newly defined command inside of
\Macro{listoftoc} or \Macro{listofeachtoc} that \PValue{\#1} will be replaced by the
corresponding heading term.
% \iffalse
\EndIndex{Cmd}{deftocheading}%
% \fi

% \iffalse
%<*dtx>
% \fi
\DescribeMacro{\setuptoc}
\DescribeMacro{\unsettoc}
The Commands
% \iffalse
%</dtx>
\begin{Declaration}
% \fi
  \Macro{setuptoc}\Parameter{file extension}\Parameter{feature list}
% \iffalse
\\
%<*dtx>
% \fi
and
% \iffalse
%</dtx>
% \fi
  \Macro{unsettoc}\Parameter{file extension}\Parameter{feature list}
% \iffalse
\end{Declaration}
\BeginIndex{Cmd}{setuptoc}%
\BeginIndex{Cmd}{unsettoc}%
This commands
% \fi
set up and unset features bound to an \PName{file extension}. The
\PName{feature list} is a comma seperated list of single
features. \Package{tocbasic} does know following features:
\begin{description}
\item[\texttt{leveldown}] uses not the top level heading below
  \Macro{part}\,---\,\Macro{chapter} if available, \Macro{section}
  otherwise\,---\,but the first sub level. This feature will be evaluated by
  the internal heading command. On the other hand, if an user defined heading
  command has been made with \Macro{deftocheading}, that user is responsible
  for the evaluation of the feature. The \KOMAScript{} classes set this
  feature using option
  \OptionValue{listof}{leveldown}important{\OptionValue{listof}{leveldown}}%
  \IndexOption{listof~=\PValue{leveldown}} for all file extensions of the
  owner \PValue{float}.
\item[\texttt{nobabel}] prevents usage of the language switch of
  \Package{babel}\IndexPackage{babel} at the helper file with the
  corresponding \PName{file extension}. This feature should be used only for
  helper files, that contain text in one language only. Changes of the
  language inside of the document will not longer regarded at the helper
  file. Package
  \Package{scrwfile}\important{\Package{scrwfile}}\IndexPackage{scrwfile} uses
  this feature also for clone destinations, because those will get the
  language change from the clone source already.
\item[\texttt{noprotrusion}] prevents\ChangedAt{v3.10}{\Package{tocbasic}}
  disabling character protrusion at the lists of something. Character
  protrusion at the lists will be disabled by default if package
  \Package{microtype}\IndexPackage{microtype} or another package, that
  supports \Macro{microtypesetup}\IndexCmd{microtypesetup}, was loaded. So if
  you want protrusion at the lists, you have to set this feature. But
  note\textnote{Attention!}, that with character protrusion entries at the
  list may be set wrong. This is a known issue of character protrusion.
\item[\texttt{numbered}] uses a numbered heading for the table of contents or
  list of something and because of this also generates an entry to the table
  of contents. This feature will be evaluated by the internal heading
  command. On the other hand, if an user defined heading command has been made
  with \Macro{deftocheading}, that user is responsible for the evaluation of
  the feature. The \KOMAScript{} classes set this feature using option
  \OptionValue{listof}{numbered}\important{\OptionValue{listof}{numbered}}%
  \IndexOption{listof~=\PValue{numbered}} for all file extensions of the owner
  \PValue{float}.
\item[\texttt{numberline}] \leavevmode\ChangedAt{v3.12}{\Package{tocbasic}}%
  redefines \Macro{nonumberline} to use \Macro{numberline}. With this the not
  numbered entries generated by \KOMAScript{} or using \Macro{nonumberline} at
  the very beginning of the last argument of \Macro{addcontentline} will also
  be indented like numbered entries of the same type.
\item[\texttt{onecolumn}] \leavevmode\ChangedAt{v3.01}{\Package{tocbasic}}%
  typesets the corresponding table of contents or list of something with
  internal one column mode of
  \Macro{onecolumn}\IndexCmd{onecolumn}. This\textnote{Attention!} will be
  done only, if the corresponding table of contents or list of something
  doesn't use feature \PValue{leveldown}\important{\PValue{leveldown}}. The
  \KOMAScript{} classes \Class{scrbook} and \Class{scrreprt} activate this
  feature with \Macro{AtAddToTocList} (see \autoref{sec:tocbasic.basics}%^^A
% \iffalse
  , \autopageref{desc:tocbasic.cmd.AtAddToTocList}%^^A
% \fi
  ) for all lists of something with owner \PValue{float} or with themselves as
  owner. With this, e.\,g., the table of contents, the list of figures and the
  list of tables of both classes will be single columned automatically. The
  multiple-column-mode of package \Package{multicol}\IndexPackage{multicol}
  will not be recognized or changed by this option.
\item[\texttt{totoc}] writes the title of the corresponding table of contents
  or the list of something to the table of contents. This feature will be
  evaluated by the internal heading command. On the other hand, if an user
  defined heading command has been made with \Macro{deftocheading}, that user
  is responsible for the evaluation of the feature. The \KOMAScript{} classes
  set this feature using option
  \OptionValue{listof}{totoc}\important{\OptionValue{listof}{totoc}}%
  \IndexOption{listof~=\PValue{totoc}} for all file extensions of the owner
  \PValue{float}.
\end{description}
Classes and packages may know features, too, e.\,g, the \KOMAScript{} classes
know following additional features:
\begin{description}
\item[\texttt{chapteratlist}] activates special code to be put into the list
  at start of a new chapter. This code may either be vertical space or the
  heading of the chapter.
% \iffalse
  See \Option{listof}\IndexOption{listof}\important{\Option{listof}} in
  \autoref{sec:maincls.floats}, \autopageref{desc:maincls.option.listof} for
  more information about such features.
% \fi
\end{description}
\begin{Example}
  Because \KOMAScript{} classes use \Package{tocbasic} for the list of figures
  and list of tables, there's one more way to remove chapter structuring at
  those:
\begin{lstcode}
  \unsettoc{lof}{chapteratlist}
  \unsettoc{lot}{chapteratlist}
\end{lstcode}

  And if you want to have the chapter structuring of the \KOMAScript{} classes 
  at your own list of algorithms with \PName{file extension} ``\PValue{load}''
  from the previous examples, you may use
\begin{lstcode}
  \setuptoc{loa}{chapteratlist}
\end{lstcode}
  And if classes with \Macro{chapter} should also force single column mode for
  the list of algorithms you may use
\begin{lstcode}
  \ifundefinedorrelax{chapter}{}{%
    \setuptoc{loa}{onecolumn}%
  }
\end{lstcode}
  Usage of \Macro{ifundefinedorrelax} presumes package \Package{scrbase}%^^A
%\iffalse
  \ (see \autoref{sec:scrbase.if},
  \autopageref{desc:scrbase.cmd.ifundefinedorrelax})%^^A
%\fi
  .

  It\textnote{Hint!} doesn't matter if you're package would be used with
  another class. You should never the less set this feature. And if the other
  class would also recognize the feature your package would automatically use
  the feature of that class.
\end{Example}
As you may see, packages, that use \Package{tocbasic}, already provide several
interesting features, without the need of a lot of implementation effort. Such
an effort would be needed only without \Package{tocbasic} and because of this,
most packages currently lack of such features.%
% \iffalse
\EndIndex{Cmd}{unsettoc}%
\EndIndex{Cmd}{setuptoc}%
% \fi

% \iffalse
%<*dtx>
% \fi
\DescribeMacro{\iftocfeature}
Command
% \iffalse
%</dtx>
\begin{Declaration}
% \fi
  \Macro{iftocfeature}\Parameter{file extension}\Parameter{feature}%^^A
  \Parameter{true-instructions}%^^A
%\linebreak[1]^^A
  \Parameter{false-instructions}
% \iffalse
\end{Declaration}
\BeginIndex{Cmd}{iftocfeature}%
This command
% \fi
may be used, to test, if a \PName{feature} was set for \PName{file
  extension}. If so the \PName{true-instructions} will be processed, otherwise
the \PName{false-instruction} will be. This may be useful, e.\,g., if you
define your own heading command using \Macro{deftocheading} but want to
support the features \PValue{totoc}, \PValue{numbered} or \PValue{leveldown}.
% \iffalse
\EndIndex{Cmd}{iftocfeature}%
% \fi


\section{Internal Commands for Class and Package Authors}
\label{sec:tocbasic.internals}

Commands with prefix \Macro{tocbasic@} are internal but class and package
authors may use them. But even if you are a class or package author you
should not change them!

% \iffalse
%<*dtx>
% \fi
\DescribeMacro{\tocbasic@extend@babel}
% \iffalse
%</dtx>
\begin{Declaration}
  \Macro{tocbasic@extend@babel}\Parameter{file extension}
\end{Declaration}
\BeginIndex{Cmd}{tocbasic@extend@babel}%
% \fi
The Package \Package{babel}\IndexPackage{babel} 
% \iffalse
(see \cite{package:babel})
% \fi
respectively a \LaTeX{} kernel that has been extended by the language
management of \Package{babel} writes instructions to change the language
inside of the files with the file extensions \File{toc}, \File{lof}, and
\File{lot} into those files at every change of the current language
either at the begin of the document or inside the document. Package
\Package{tocbasic} extends this mechanism with
\Macro{tocbasic@extend@babel}
%\unskip\Parameter{file extension}
to be used for other file extensions too. Argument \PName{file extension} has
to be expandable! Otherwise the meaning of the argument may change until it
will be used really.

Normally this command will be used by default for every \PName{file extension}
that will be added to the list of known extensions using
\Macro{addtotoclist}. The may be suppressed using feature
\PValue{nobabel}\important{\PValue{nobabel}} (see \Macro{setuptoc},
\autoref{sec:tocbasic.toc}%^^A
% \iffalse
, \autopageref{desc:tocbasic.cmd.setuptoc}%^^A
% \fi
). For the file extensions \File{toc}, \File{lof}, and \File{lot} this will be
done automatically by \Package{tocbasic} to avoid double language switching in
the corresponding files.

Normally there isn't any reason to call this command yourself. But there may
by lists of something, that should not be under control of \Package{tocbasic},
and to are not in \Package{tocbasic}'s list of known file extensions, but
nevertheless should be handled by the language change mechanism of
\Package{babel}. The command may be used explicitly for those files. But pleas
note, that this should be done only once per file extension!%
% \iffalse
\EndIndex{Cmd}{tocbasic@extend@babel}%
% \fi

% \iffalse
%<*dtx>
% \fi
\DescribeMacro{\tocbasic@starttoc}
Command
% \iffalse
%</dtx>
\begin{Declaration}
% \fi
  \Macro{tocbasic@starttoc}\Parameter{extension}
% \iffalse
\end{Declaration}
\BeginIndex{Cmd}{tocbasic@starttoc}
This command
% \fi
is something like the \LaTeX{} kernel macro
\Macro{@starttoc}\IndexCmd{@starttoc}\important{\Macro{@starttoc}}.  It's the
command behind \Macro{listoftoc*} (siehe \autoref{sec:tocbasic.toc}%^^A
% \iffalse
, \autopageref{desc:tocbasic.cmd.listoftoc*}%^^A
% \fi
). Authors of classes or packages who want to participate from the advantages
of \Package{tocbasic} should at least use this command. Nevertheless it's
recommended to use \Macro{listoftoc}. Command \Macro{tocbasic@starttoc}
internally uses \Macro{\@starttoc}, but sets
% \expandafter\Macro\iffalse
\Length
% \fi
{parskip}\IndexLength{parskip}\important{\Length{parskip}\\
  \Length{parindent}\\
  \Length{parfillskip}} and 
% \expandafter\Macro\iffalse
\Length
% \fi
{parindent}\IndexLength{parindent} to 0 and
% \expandafter\Macro\iffalse
\Length
% \fi
{parfillskip} to 0 until infinite before. Moreover,
\Macro{@currext}\important{\Macro{@currext}}\IndexCmd{@currext} will be set to
the file extension of the current helper file, so this will be available while
the execution of the hooks, that will be done before and after reading the
helper files.

Because\textnote{Attention!} of \LaTeX{} will immediately open a new helper
file for writing after reading that file, the usage of
\Macro{tocbasic@starttoc} may result in an error message like
\begin{lstoutput}
  ! No room for a new \write .
  \ch@ck ...\else \errmessage {No room for a new #3}
                                                    \fi   
\end{lstoutput}
if there are no more unused write handles. This may be solved, e.\,g., using
package
\Package{scrwfile}\important{\Package{scrwfile}}\IndexPackage{scrwfile}.
% \iffalse
See \autoref{cha:scrwfile} for more information about that package.%
\EndIndex{Cmd}{tocbasic@starttoc}
% \fi
%

% \iffalse
%<*dtx>
% \fi
\DescribeMacro{\tocbasic@@before@hook}
\DescribeMacro{\tocbasic@@after@hook}
% \iffalse
%</dtx>
\begin{Declaration}
  \Macro{tocbasic@@before@hook}\\
  \Macro{tocbasic@@after@hook}
\end{Declaration}
\BeginIndex{Cmd}{tocbasic@@before@hook}%
\BeginIndex{Cmd}{tocbasic@@after@hook}%
% \fi
The hook \Macro{tocbasic@@before@hook} will be executed immediately before
reading a helper file for a table of contents or list of something even
before execution of the instructions of a \Macro{BeforeStartingTOC}
command. It is permitted to extend this hook using
\Macro{g@addto@macro}\IndexCmd{g@addto@macro}.

Similarly \Macro{tocbasic@@after@hook} will be executed immediately after
reading such a helper file and before execution of instructions of
\Macro{AfterStartingTOC}. It is permitted to extend this hook using
\Macro{g@addto@macro}\IndexCmd{g@addto@macro}.

\KOMAScript{} uses these hooks, to provide the automatic width calculation of
the place needed by heading numbers. Only classes and packages should use
these hooks. Users\textnote{Attention!} should really use
\Macro{BeforeStartingTOC} and \Macro{AfterStartingTOC} instead. Authors of
packages should also favor those commands! These hooks shouldn't be used to
generate any output!

If neither\textnote{Attention!} \Macro{listofeachtoc} nor \Macro{listoftoc}
nor \Macro{listoftoc*} are used for the output of a table of contents or list
of something, the hooks should be executed explicitly.%
% \iffalse
\EndIndex{Cmd}{tocbasic@@before@hook}%
\EndIndex{Cmd}{tocbasic@@after@hook}%
% \fi

% \iffalse
%<*dtx>
% \fi
\DescribeMacro{\tocbasic@\PName{extension}@before@hook}
\DescribeMacro{\tocbasic@\PName{extension}@after@hook}
% \iffalse
%</dtx>
\begin{Declaration}
  \Macro{tocbasic@\PName{extension}@before@hook}\\
  \Macro{tocbasic@\PName{extension}@after@hook}
\end{Declaration}
\BeginIndex{Cmd}{tocbasic@\PName{extension}@before@hook}%
\BeginIndex{Cmd}{tocbasic@\PName{extension}@after@hook}%
% \fi
These hooks are processed after \Macro{tocbasic@@before@hook}, respectively
before \Macro{tocbasic@@after@hook} before and after loading the helper file
with the corresponding file \PName{extension}. Authors\textnote{Attention!}
of classes and packages should never manipulate them! But if
neither\textnote{Attention!} \Macro{listofeachtoc} nor \Macro{listoftoc} nor
\Macro{listoftoc*} are used for the output of a table of contents or list of
something, the hooks should be executed explicitly, if they are
defined. Please note, that they even can be undefined.%
% \iffalse
\EndIndex{Cmd}{tocbasic@\PName{extension}@after@hook}%
\EndIndex{Cmd}{tocbasic@\PName{extension}@before@hook}%
% \fi

% \iffalse
%<*dtx>
% \fi
\DescribeMacro{\tocbasic@listhead}
Command
% \iffalse
%</dtx>
\begin{Declaration}
% \fi
  \Macro{tocbasic@listhead}\Parameter{title}
% \iffalse
\end{Declaration}
\BeginIndex{Cmd}{tocbasic@listhead}%
This command
% \fi
is used by \Macro{listoftoc} to set the heading of the list, either the
default heading or the individually defined heading. If you define your own
list command not using \Macro{listoftoc} you may use
\Macro{tocbasic@listhead}. In this case you should define
\Macro{@currext}\important{\Macro{@currext}}\IndexCmd{@currext} to be the file
extension of the corresponding helper file before using
\Macro{tocbasic@listhead}.
% \iffalse
\EndIndex{Cmd}{tocbasic@listhead}%
% \fi

% \iffalse
%<*dtx>
% \fi
\DescribeMacro{\tocbasic@listhead@\PName{extension}}
Command
% \iffalse
%</dtx>
\begin{Declaration}
% \fi
  \Macro{tocbasic@listhead@\PName{extension}}\Parameter{title}
% \iffalse
\end{Declaration}
\BeginIndex{Cmd}{tocbasic@listhead@\PName{extension}}%
This command
% \fi
is used in \Macro{tocbasic@listhead} to set the individual headings, optional
toc-entry, and running head, if it was defined. If it was not defined it will
be defined and used in \Macro{tocbasic@listhead} automatically.
% \iffalse
\EndIndex{Cmd}{tocbasic@listhead@\PName{extension}}%
% \fi

%\iffalse^^A meta-comment
\begin{Declaration}
%\fi^^A meta-comment
% \DescribeMacro{\tocbasic@addxcontentsline}
% The command
  \Macro{tocbasic@addxcontentsline}%
  \Parameter{extension}\Parameter{level}\Parameter{number}\Parameter{text}
%\iffalse^^A meta-comment
\end{Declaration}
\BeginIndex{Cmd}{tocbasic@addxcontentsline}%
This command
%\fi^^A meta-comment
uses \Macro{addcontentsline} to either create a numbered or not numbered text
entry to toc-file with given \PName{extension}. Note, that all parameters of
\Macro{tocbasic@addxcontentsline} are mandatory. But you may use an
empty \PName{number} argument, if you don't want a number.
%\iffalse^^A meta-comment
\EndIndex{Cmd}{tocbasic@addxcontentsline}
%\fi^^A meta-comment


\section{A Complete Example}
\label{sec:tocbasic.example}

This section will show you a complete example of a user defined floating
environment with list of that kind of floats and \KOMAScript{} integration
using \Package{tocbasic}. This example uses internal commands, that have a
``\texttt{@}'' in their name. This means\textnote{Attention}, that the code
has to be put into a own package or class, or has to be placed between
\Macro{makeatletter}\important[i]{\Macro{makeatletter}\\\Macro{makeatother}}%
\IndexCmd{makeatletter} and \Macro{makeatother}\IndexCmd{makeatother}.

First\textnote{environment} of all, a new floating environment will be
needed. This could simply be done using:
\begin{lstcode}[belowskip=\dp\strutbox plus 1pt]
  \newenvironment{remarkbox}{%
    \@float{remarkbox}%
  }{%
    \end@float
  }
\end{lstcode}
To the new environment is named \Environment{remarkbox}.

Each\textnote{placement} floating environment has a default
placement. This is build by some of the well known placement options:
\begin{lstcode}[belowskip=\dp\strutbox plus 1pt]
  \newcommand*{\fps@remarkbox}{tbp}
\end{lstcode}
So, the new floating environment should be placed by default only either at
the top of a page, at the bottom of a page, or on a page on its own.

Floating\textnote{type} environments have a numerical floating
type. Environments with the same active bit at the floating type cannot change
their order. Figures and table normally use type~1 and 2. So a figure that
comes later at the source code than a table, may be output earlier than the
table and vica versa.
\begin{lstcode}[belowskip=\dp\strutbox plus 1pt]
  \newcommand*{\ftype@remarkbox}{4}
\end{lstcode}
The new environment has floating type~4, so it may pass figures and floats and
may be passed by those.

The\textnote{number} captions of floating environment also have numbers.
\begin{lstcode}[belowskip=\dp\strutbox plus 1pt]
  \newcounter{remarkbox}
  \newcommand*{\remarkboxformat}{%
    Remark~\theremarkbox\csname autodot\endcsname}
  \newcommand*{\fnum@remarkbox}{\remarkboxformat}
\end{lstcode}
Here first a new counter has been defined, that is independent from chapters
or the counters of other structural levels. \LaTeX{} itself also defines
\Macro{theremarkbox} with the default Arabic representation of the counter's
value. Afterwards this has been used defining the formatted output of the
counter. Last this formatted output has been used for the output of the
environment number of the \Macro{caption} command.

Floating\textnote{file extension} environments have lists of themselves and
those need a helper file with name \Macro{jobname} and a file extension.
\begin{lstcode}[belowskip=\dp\strutbox plus 1pt]
  \newcommand*{\ext@remarkbox}{lor}
\end{lstcode}
The file extension of the helper file for the list of
\Environment{remarkbox}es is ``\File{lor}''.

This was the definition of the floating environment. But the list of this new
environment's captions is still missing. To reduce the implementation effort
package \Package{tocbasic} will be used for this. This will be loaded using
\begin{lstcode}[belowskip=\dp\strutbox plus 1pt]
  \usepackage{tocbasic}
\end{lstcode}
inside of document preambles. Authors of classes or packages would use
\begin{lstcode}[belowskip=\dp\strutbox plus 1pt]
  \RequirePackage{tocbasic}
\end{lstcode}
instead.

Now\textnote{extension} we register the file name extension for package
\Package{tocbasic}:
\begin{lstcode}[belowskip=\dp\strutbox plus 1pt]
  \addtotoclist[float]{lor}
\end{lstcode}
Thereby the owner \PValue{float} has been used, to allude all further
\KOMAScript{} options for lists of something also to the new one.

Next\textnote{title} we define a title or heading for the list of
\Environment{remarkbox}es:
\begin{lstcode}[belowskip=\dp\strutbox plus 1pt]
  \newcommand*{\listoflorname}{List of Remarks}
\end{lstcode}
You may use package \Package{scrbase} to additionally support titles in other
languages than English.

Also\textnote{entry} a command is needed to define the layout of the entries
to the list of remarks:
\begin{lstcode}[belowskip=\dp\strutbox plus 1pt]
  \newcommand*{\l@remarkbox}{\l@figure}
\end{lstcode}
Here simply the entries to the list of remarks get the same layout like the
entries to the list of figure. This would be the easiest solution. A more
explicit would be, e.\,g.,
\begin{lstcode}[belowskip=\dp\strutbox plus 1pt]
  \newcommand*{\l@remarkbox}{\@dottedtocline{1}{1em}{1.5em}}
\end{lstcode}

Additionally\textnote{chapter entry} you may want structure the list of
remarks depending on chapters.
\begin{lstcode}[belowskip=\dp\strutbox plus 1pt]
  \setuptoc{lor}{chapteratlist}
\end{lstcode}
The \KOMAScript{} classes provide that feature and may other classes do so
too. Unfortunately the standard classes do not.

This\textnote{list of remarks} would already be enough. Now, users may already
select different kinds of headings either using the corresponding options of
the \KOMAScript{} classes, or \Macro{setuptoc}, e.\,g., with or without entry
in the table of contents, with or without number. But a simply
\begin{lstcode}[belowskip=\dp\strutbox plus 1pt]
  \newcommand*{\listofremarkboxes}{\listoftoc{lor}}
\end{lstcode}
may make the usage a little bit easier again.

As you've seen only five commands refers to the list of remarks. Only three of
them are necessary. Nevertheless the new list of remarks already provides
optional numbering of the heading and optional not numbered entry into the
table of contents. Optional even a lower document structure level may be used
for the heading. Running headers are provides with the \KOMAScript{} classes,
the standard classes, and all classes that explicitly support
\Package{tocbasic}. Supporting classes even pay attention to this new list of
remarks at every new \Macro{chapter}. Even changes of the current language are
handled inside the list of remarks like they will inside the list of figures
or inside the list of tables.

Moreover.\textnote{additional features} an author of a package may add more
features. For example, options to hide \Macro{setuptoc} from the users may be
added. On the other hand, the \Package{tocbasic} manual may be referenced to
describe the corresponding features. The advantage of this would be that user
would get information about new features provides by \Package{tocbasic}. But
if the user should be able to set the features of the remarks even without
knowledge about the file extension \File{lor} a simple
\begin{lstcode}[belowskip=\dp\strutbox plus 1pt]
  \newcommand*{\setupremarkboxes}{\setuptoc{lor}}
\end{lstcode}
would be enough to use a list of features argument to
\Macro{setupremarkboxes} as list of features of file extension \File{lor}.


\section{Everything with One Command Only}
\label{sec:tocbasic.highlevel}

The example from the previous section shows, that using \Package{tocbasic} to
define floating environments and lists with the captions of those floating
environments is easy. The following example will show, that is may be even
easier.

% \iffalse
%<*dtx>
% \fi
\DescribeMacro{\DeclareNewTOC}
Command
% \iffalse
%</dtx>
\begin{Declaration}
% \fi
  \Macro{DeclareNewTOC}\OParameter{options}\Parameter{extension}
% \iffalse
\end{Declaration}
\BeginIndex{Cmd}{DeclareNewTOC}%
This command
% \fi
declares\ChangedAt{v3.06}{\Package{tocbasic}} in one step only a new list of
something, the heading of that list, the term used for the entries to the
list, and to manage the file name \PName{extension}. Additionally optional
floating and non-floating environments may be defined, and inside of both such
environments \Macro{caption}\important{\Macro{caption}}\IndexCmd{caption} may
be used. The additional features \Macro{captionabove}\important[i]{%
  \Macro{captionabove}\\
  \Macro{captionbelow}}, \Macro{captionbelow}, and \Environment{captionbeside}
of the \KOMAScript{} classes
% \iffalse
(see \autoref{sec:maincls.floats})
% \fi
may also be used inside of those environments.

Argument \PName{extension} is the file name extension of the helper file, that
represents the list of something. See \autoref{sec:tocbasic.basics} for more
information about this. This argument is mandatory and must not be empty!

Argument \PName{options} is a comma separated list, like you know it from,
e.\,g., \Macro{KOMAoptions}%
% \iffalse
\ (see \autoref{sec:typearea.options})%^^A
% \fi
. Nevertheless\textnote{Attention!}, those options cannot be set using
\Macro{KOMAoptions}\IndexCmd{KOMAoptions}! An overview of all available
options
% is shown next.
% \iffalse
may be found in \autoref{tab:tocbasic.DeclareNewTOC-options}.
% \fi

% \iffalse
\begin{desclist}
  \renewcommand*{\abovecaptionskipcorrection}{-\normalbaselineskip}%
  \desccaption[{Options for command \Macro{DeclareNewTOC}}]{%
    Options for command
    \Macro{DeclareNewTOC}\ChangedAt{v3.06}{\Package{tocbasic}}%
    \label{tab:tocbasic.DeclareNewTOC-options}%
  }{%
    Options for command \Macro{DeclareNewTOC} (continuation)%
  }%
% \fi
%\begin{description}
%  \newcommand{\entry}[2]{\item[\normalfont#1]\hfill\\* #2}
  \entry{\KOption{atbegin}\PName{instructions}}{%
    The \PName{instructions} will be executed at the begin of the floating or
    non-floating environment.%
  }%
  \entry{\KOption{atend}\PName{instructions}}{%
    The \PName{instructions} will be executed at the end of the floating or
    non-floating environment.%
  }%
  \entry{\KOption{counterwithin}\PName{\LaTeX{} counter}}{%
    If you define a float or non-float, the captions will be numbered and a
    counter \PName{type} (see option \Option{type}) will be defined. You may
    declare another counter to be the parent \LaTeX{} counter. In this case,
    the parent counter will be set before the float counter and the float
    counter will be reset whenever the parent counter is increased using
    \Macro{stepcounter} or \Macro{refstepcounter}.%
  }%
%^^A not yet implemented:
%^^A \entry{\KOption{captionpos}\PName{position}}{%
%^^A   If \PName{position} is \PValue{top} or \PValue{above} the spacing above
%^^A   and below the caption will be done for captions above objects. If
%^^A   \PName{position} is \PValue{bot}, \PValue{bottom}, or \PValue{below} the
%^^A   spacing above and below the caption will be done for captions below
%^^A   objects.
%^^A }
  \entry{\Option{float}}{%
    If set, float environments for that type will be defined. The names of the
    environments are the value of \PName{type} and for double column floats
    the value of \PName{type} with addendum ``\PValue{*}''.%
  }%
  \entry{\KOption{floatpos}\PName{float positions}}{%
    The default floating position of the float. If no float position was
    given, ``tbp'' will be used like the standard classes do for figures and
    tables.%
  }%
  \entry{\KOption{floattype}\PName{number}}{%
    The numerical float type of the defined floats. Float types with common
    bits cannot be reordered. At the standard classes figures has float type 1
    and tables has float type 2. If no float type was given, 16 will be used.%
  }%
  \entry{\Option{forcenames}}{%
    If set, the names will be even defined, if they where already defined
    before.%
  }%
  \entry{\KOption{hang}\PName{length}}{%
    The amount of the hanging indent of the entries for that list. If not
    given, 1.5\,em will be used like standard classes use for entries to list
    of figures or list of tables.%
  }%
  \entry{\KOption{indent}\PName{length}}{%
    The indent value for the entries of that list. If not given, 1\,em will be
    used like standard classes use for entries to list of figures or list of
    tables.%
  }%
  \entry{\KOption{level}\PName{number}}{%
    The level of the entries of that list. If not given level 1 will be used
    like standard classes use for entries to list of figures or list of
    tables.%
  }%
  \entry{\KOption{listname}\PName{string}}{%
    The name of the list of foo. If not given the value of \PValue{types} with
    upper case first char prefixed by ``List of '' will be used.%
  }%
  \entry{\KOption{name}\PName{string}}{%
    The name of an element. If no name is given, the value of \PValue{type}
    with upper case first char will be used.%
  }%
  \entry{\Option{nonfloat}}{%
    If set, a non floating environment will be defined. The name of the
    environment is the value of \PName{type} with postfix ``\PValue{-}''.%
  }%
  \entry{\KOption{owner}\PName{string}}{%
    The owner as described in the sections before. If no owner was given owner
    \PValue{float} will be used.%
  }%
  \entry{\KOption{type}\PName{string}}{%
    sets the type of the new declared list. The type will be used e.\,g. to
    defined a \Macro{listof}\PName{string}. If no type is set up the extension
    from the mandatory argument will be used.%
  }%
  \entry{\KOption{types}\PName{string}}{%
    the plural of the type. If no plural was given the value of \PValue{type}
    with addendum ``s'' will be used.%
  }%
% \end{description}
% \iffalse
\end{desclist}
% \fi

\begin{Example}
  Using \Macro{DeclareNewTOC} reduces the example from
  \autoref{sec:tocbasic.example} to:
\begin{lstcode}
  \DeclareNewTOC[%
    type=remarkbox,%
    types=remarkboxes,%
    float,% define a floating environment
    floattype=4,%
    name=Remark,%
    listname={List of Remarks}%
  ]{lor}
  \setuptoc{lor}{chapteratlist}
\end{lstcode}
  Beside environments \Environment{remarkbox} and \Environment{remarkbox*} the
  counter 
% \expandafter\texttt\iffalse
\Counter
% \fi
  {remarkbox}, the commands \Macro{theremarkbox},
  \Macro{remarkboxname}, and \Macro{remarkboxformat} that are used for
  captions; the commands \Macro{listremarkboxnames} and
  \Macro{listofremarkboxes} that are used at the list of remarks; and some
  internal commands that depends on the file name extension \File{lor} are
  defined. If the package should use a default for the floating type, option
  Option{floattype} may be omitted. If option \Option{nonfloat} will be used
  additionally, then a non-floating environment \Environment{remarkbox-} will
  be also defined. You may use \Macro{caption}\IndexCmd{caption} inside of
  that non-floating environment as usual for floating environments.
%\iffalse
  \hyperref[tab:tocbasic.comparison]{Figure~\ref*{tab:tocbasic.comparison}}
  showes a comparison of the commands, counters and environments of the
  example environment \Environment{remarkbox} and of the commands, counters
  and environments for figures.%
  \begin{table}
    \centering
    \caption{Comparison of example environment \Environment{remarkbox}
      and environment \Environment{figure}}
    \label{tab:tocbasic.comparison}
    \begin{tabularx}{\textwidth}{ll>{\raggedright}p{6em}X}
      \toprule
      \Environment{remarkbox} & \Environment{figure}
      & options of \Macro{DeclareNewTOC} & short description \\[1ex]
      \midrule
      \Environment{remarkbox} & \Environment{figure} 
      & \Option{type}, \Option{float} 
      & floating environments of the respective types\\[1ex]
      \Environment{remarkbox*} & \Environment{figure*} 
      & \Option{type}, \Option{float} 
      & columns spanning floating environments of the respective types\\[1ex]
      \Counter{remarkbox} & \Counter{figure} 
      & \Option{type}, \Option{float} 
      & counter used by \Macro{caption}\\[1ex]
      \Macro{theremarkbox} & \Macro{thefigure} 
      & \Option{type}, \Option{float} 
      & output command to the respective counters\\[1ex]
      \Macro{remarkboxformat} & \Macro{figureformat}
      & \Option{type}, \Option{float}
      & formating command to the respective counters used by
        \Macro{caption}\\[1ex]
      \Macro{remarkboxname} & \Macro{figurename}
      & \Option{type}, \Option{float}, \Option{name}
      & names used in the label of \Macro{caption}\\[1ex]
      \Macro{listofremarkboxes} & \Macro{listoffigures}
      & \Option{types}, \Option{float}
      & command to show the list of the respective environments\\[1ex]
      \Macro{listremarboxname} & \Macro{listfigurename}
      & \Option{type}, \Option{float}, \Option{listname}
      & heading text of the respective list \\[1ex]
      \Macro{fps@remarkbox} & \Macro{fps@figure}
      & \Option{type}, \Option{float}, \Option{floattype}
      & numeric float type for order perputation\\[1ex]
      \File{lor} & \File{lof}
      &
      & file extension of the helper file of the respective list \\
      \bottomrule
    \end{tabularx}
  \end{table}
%\fi

  And now a possible usage of the example environment:
\begin{lstcode}
  \begin{remarkbox}
    \centering
    Equal should be typeset equally
    and with equal formatting.
    \caption{First theorem of typography}
    \label{rem:typo1}
  \end{remarkbox}
\end{lstcode}
  A segment of an example page with this environment could be:
  \begin{center}\footnotesize
    \begin{tabular}
      {|!{\hspace{.1\linewidth}}p{.55\linewidth}!{\hspace{.1\linewidth}}|}
      \\
      \centering
      Equal should be typeset equally
      and with equal formatting.\\[\abovecaptionskip]
      {%
        \usekomafont{caption}\footnotesize{\usekomafont{captionlabel}%
          Remark 1: }First theorem of typography
      }\\
    \end{tabular}%
  \end{center}%
\end{Example}
% \iffalse
\EndIndex{Cmd}{DeclareNewTOC}%
%
\EndIndex{Package}{tocbasic}%
\EndIndex{}{table of contents}%
\EndIndex{}{list>of contents}%
\EndIndex{}{file>extension}%
% \fi
%
% \iffalse
%</doc|dtx>
% \fi
%
% \StopEventually{\PrintIndex\PrintChanges}
%
%
% \section{Implementation}
% \label{sec:implementation}
%
% All macros with prefix \texttt{tb@} or \texttt{@} are internal macros and
% should not be used by package and class authors. Macros with prefix
% \texttt{tocbasic@} are internal macros, that may be used by class and
% packages authors. Macros without \texttt{@} are interface macros and may be
% used by class and package authors and users.
%
% \iffalse
%<*package>
% \fi
%
% \subsection{Options}
% \label{sec:options}
%
% There are no options because the package should be used by class and package
% authors not by users. So the package will be loaded using
% \Macro{RequiresPackage}. Using different options by different packages would
% result in an option clash.
%    \begin{macrocode}
\ProcessOptions\relax
%    \end{macrocode}
%
%
% \subsection{Required Packages}
% \label{sec:packages}
%
% Currently only graphics bundle's package \Package{keyval} is needed.
%    \begin{macrocode}
\RequirePackage{keyval}
%    \end{macrocode}
% We load it after processing the options, because we don't need it for
% processing options.
%
%
% \subsection{Having a List of All Tocs}
% \label{sec:listoftocs}
%
% If we have a list of all toc-files we may do commands for all
% toc-files. Somethimes it may be usefull to known the package, that created
% the toc-file, so this information will be stored additionally.
%
% \begin{macro}{\tb@listoftocs}
%   This is the list of toc-files. The list will be:
%   \begin{quote}
%   \cs{do}\marg{extension}\marg{class or
%   package}\cs{do}\marg{extension}\marg{class or package}\dots
%   \end{quote}
%   With this, adding and processing the list will be very fast but removing
%   an element will be very slow.
%
%   The initial state of the list will be \emph{empty}.
%    \begin{macrocode}
\newcommand*{\tb@listoftocs}{}
%    \end{macrocode}
% \end{macro}
%
% \begin{macro}{\ifattoclist}
%   This command tests, if an extension is already at the list of
%   toc-files. The extension has to be the first argument. The second argument
%   will be done, if the extension is already at the list of toc-files. The
%   third argument will be done, if the extension is at the list of toc-files
%   not yet.
%    \begin{macrocode}
\newcommand{\ifattoclist}[1]{%
  \begingroup
    \def\do##1##2{%
      \edef\reserved@a{##1}%
      \ifx\reserved@a\reserved@b\@tempswatrue\fi
    }%
    \edef\reserved@b{#1}\@tempswafalse\tb@listoftocs
    \if@tempswa\aftergroup\@firstoftwo\else\aftergroup\@secondoftwo\fi
  \endgroup
}
%    \end{macrocode}
% \end{macro}
%
% \begin{macro}{\addtotoclist}
%   \changes{v1.0a}{2008/11/11}{internal \cs{@addtotoclist} renamed to
%     \cs{scr@addtotoclist} because of package \textsf{floatfram}}.%
%   \changes{v1.0a}{2008/11/11}{internal \cs{@@addtotoclist} renamed to
%     \cs{scr@@addtotoclist} because of package \textsf{floatfram}}.%
%   \changes{v3.06a}{2010/09/14}{expand the arguments while adding them to the
%     internal list}
%   This command adds an extension to the list of toc-files. The first,
%   optional argument is the class or package name with the corresponding
%   extension of class or package files. If this argument was omitted
%   \textsf{tocbasic} tries to get it automatically. This should be
%   successfull while loading a class or package but not while processing any
%   command of a class or package after loading the class or package. The
%   second, mandatory argument is the extension of the toc-file. NOTE: An
%   empty first argument is not the same like omitting the first argument!
%    \begin{macrocode}
\newcommand*{\addtotoclist}{%
  \@ifnextchar [%]
    \scr@@addtotoclist\scr@addtotoclist
}
\newcommand*{\scr@addtotoclist}{%
  \ifx\@currname\@empty
    \def\reserved@a{\scr@@addtotoclist[]}%
  \else
    \edef\reserved@a{\noexpand\scr@@addtotoclist[\@currname.\@currext]}%
  \fi
  \reserved@a
}
\newcommand*{\scr@@addtotoclist}[2][]{%
  \ifattoclist{#2}{%
    \PackageError{tocbasic}{%
      file extension `#2' cannot be used twice
    }{%
      File extension `#2' is already used by a toc-file, while
      \ifx\relax#1\relax someone\else #2\fi\MessageBreak
      tried to use it again for a toc-file.\MessageBreak
      This may be either an incompatibility of packages, an error at a
      package,\MessageBreak
      or a mistake by the user.\MessageBreak
    }%
  }{%
    \begingroup
      \protected@edef\reserved@a{%
        \noexpand\g@addto@macro\noexpand\tb@listoftocs{%
          \noexpand\do{#2}{#1}}}\reserved@a
    \endgroup
    \ifx\relax #1\relax\else
      \@ifundefined{tb@#1@add@hook}{}{%
        \edef\reserved@a{%
          \noexpand\def\noexpand\@currext{#2}%
          \noexpand\@nameuse{tb@#1@add@hook}%
          \noexpand\def\noexpand\@currext{\@currext}%
        }\reserved@a
      }%
    \fi
    \edef\reserved@a{%
      \noexpand\def\noexpand\@currext{#2}%
      \noexpand\@nameuse{tb@@add@hook}%
      \noexpand\def\noexpand\@currext{\@currext}%
    }\reserved@a
  }%
}
%    \end{macrocode}
% \end{macro}
%
% \begin{macro}{\AtAddToTocList}
%   Action to be done, when adding a toc-file of a known owner. The first
%   optional argument is the owner. The second, mandatory argument is the
%   action. While doing the action \cs{@currext} will be the extension of the
%   added toc-file. If the first argument was omitted \textsf{tocbasic} tries
%   to get it automatically. This should be successfull while loading a class
%   or package but not while processing any command of a class or package
%   after loading the class or package. The second, mandatory argument is the
%   extension of the toc-file. NOTE: An empty first argument is not the same
%   like omitting the first argument, but set's up the general hook. The
%   general hook will be done for every adding after the individual hook (this
%   means, you may dactivate features at the individual hook before processing
%   the general hook). But the individual hook will not be processed for
%   toc-files without any owner!
%    \begin{macrocode}
\newcommand*{\AtAddToTocList}{%
  \@ifnextchar [%]
    \@@AtAddToTocList\@AtAddToTocList
}
\newcommand*{\@AtAddToTocList}{%
  \ifx\@currname\@empty
    \def\reserved@a{\@@AtAddToTocList[]}%
  \else
    \edef\reserved@a{\noexpand\@@AtAddToTocList[\@currname.\@currext]}%
  \fi
  \reserved@a
}
\newcommand*{\@@AtAddToTocList}[1][]{%
  \@ifundefined{tb@#1@add@hook}{\@namedef{tb@#1@add@hook}{}}{}%
  \expandafter\g@addto@macro\csname tb@#1@add@hook\endcsname
}
%    \end{macrocode}
% \end{macro}
%
% \begin{macro}{\tocbasic@extend@babel}
%   NOTE: To extend the original \cs{bbl@set@language} the argument of
%   \cs{tocbasic@extend@babel} must not be a local macro (like \cs{@currext})
%    \begin{macrocode}
\newcommand*{\tocbasic@extend@babel}[1]{%
  \begingroup\expandafter\expandafter\expandafter\endgroup
  \ifx\csname bbl@set@language\endcsname\relax
    \PackageInfo{tocbasic}{babel extension for `#1' omitted\MessageBreak
      because of missing \string\bbl@set@language}%
  \else
    \iftocfeature{#1}{nobabel}{%
      \PackageInfo{tocbasic}{omitting babel extension for `#1'\MessageBreak
        because of feature `nobabel' available\MessageBreak
        for `#1'}%
    }{%
      \PackageInfo{tocbasic}{setting babel extension for `#1'}%
      \expandafter\gdef\expandafter\bbl@set@language\expandafter##\expandafter1%
      \expandafter{%
        \bbl@set@language{##1}%
        \addtocontents{#1}{\xstring\select@language{\languagename}}%
      }%
    }%
  \fi
}
%    \end{macrocode}
%   This feature should be used for all toc-files (unless feature
%   \texttt{nobabel} was set for the toc-file.
%    \begin{macrocode}
\AtAddToTocList[]{\expandafter\tocbasic@extend@babel\expandafter{\@currext}}
%    \end{macrocode}
% \end{macro}
%
% \begin{macro}{\removefromtoclist}
%   This command will remove an extension from the list of toc-files. If the
%   first, optional argument is given, the extension will only be removed, if
%   it was added by the given owner. NOTE: An empty first argument is not the
%   same like omitting the first argument!
%    \begin{macrocode}
\newcommand*{\removefromtoclist}{%
  \@ifnextchar [%]
    \@removefromtoclist\@@removefromtoclist
}
\newcommand*{\@removefromtoclist}[2][]{%
  \begingroup
    \let\tb@oldlist\tb@listoftocs
    \def\do##1##2{%
      \edef\reserved@a{##1}%
      \ifx\reserved@a\reserved@b
        \begingroup
          \edef\reserved@a{##2}%
          \edef\reserved@b{#1}%
          \ifx\reserved@a\reserved@b\else
            \g@addto@macro\tb@listoftocs{\do{##1}{##2}}%
          \fi
        \endgroup
      \else
        \g@addto@macro\tb@listoftocs{\do{##1}{##2}}%
      \fi
    }%
    \edef\reserved@b{#2}\let\tb@listoftocs\@empty
    \tb@oldlist
  \endgroup
}
\newcommand*{\@@removefromtoclist}[1]{%
  \begingroup
    \let\tb@oldlist\tb@listoftocs
    \def\do##1##2{%
      \edef\reserved@a{##1}%
      \ifx\reserved@a\reserved@b\else
        \g@addto@macro\tb@listoftocs{\do{##1}{##2}}%
      \fi
    }%
    \edef\reserved@b{#1}\let\tb@listoftocs\@empty
    \tb@oldlist
  \endgroup
}
%    \end{macrocode}
% \end{macro}
%
% \begin{macro}{\doforeachtocfile}
%   \changes{v3.09}{2011/03/01}{\cs{@currext} defined using \cs{edef} instead
%     of \cs{def}}
%   This command does the second, mandatory argument for each toc-file at the
%   list of toc-files. If the first, optional argument was given this will be
%   done only for the toc-files of that owner. NOTE: An empty first argument
%   is not the same like omitting the first argument!
%
%   While processing the second argument \cs{@currext} is the extension of
%   the toc-file. The second argument will be processed with increased group
%   level!
%
%   See \cs{addtoeachtocfile} for an example of usage of
%   \cs{doforeachtocfile}.
%    \begin{macrocode}
\newcommand{\doforeachtocfile}{%
  \@ifnextchar [%]
    \@doforeachtocfile\@@doforeachtocfile
}
\newcommand{\@doforeachtocfile}[2][]{%
  \def\do##1##2{%
    \edef\reserved@a{#1}\edef\reserved@b{##2}\ifx\reserved@a\reserved@b
      \edef\@currext{##1}#2%
    \fi
  }%
  \edef\reserved@a{%
    \noexpand\tb@listoftocs
    \noexpand\def\noexpand\@currext{\@currext}%
  }\reserved@a
  \let\do\relax
}
\newcommand{\@@doforeachtocfile}[1]{%
  \def\do##1##2{%
    \edef\@currext{##1}#1%
  }%
  \edef\reserved@a{%
    \noexpand\tb@listoftocs
    \noexpand\def\noexpand\@currext{\@currext}%
  }\reserved@a
  \let\do\relax
}
%    \end{macrocode}
% \end{macro}
%
% \begin{macro}{\addtoeachtocfile}
% \begin{macro}{\@addtoeachtocfile}
% \begin{macro}{\@@addtoeachtocfile}
%   \changes{v3.03b}{2009/06/08}{typo fix at usage of \cs{doforeachtocfile}}
%   This command calls \cs{addtocontents} with the section, mandatory
%   argument for each toc-file at the list of toc-files. If the first,
%   optional argument was given this will be done only for the toc-files of
%   that owener. NOTE: An empty first argument is not the same like omitting
%   the first argument! And don't forget to protect the commands, that should
%   not be expanded, at the mandatory argument.
%    \begin{macrocode}
\newcommand{\addtoeachtocfile}{%
  \@ifnextchar [%]
    \@addtoeachtocfile\@@addtoeachtocfile
}
\newcommand{\@addtoeachtocfile}[2][]{%
  \doforeachtocfile[{#1}]{\addtocontents{\@currext}{#2}}%
}
\newcommand{\@@addtoeachtocfile}[1]{%
  \doforeachtocfile{\addtocontents{\@currext}{#1}}%
}
%    \end{macrocode}
% \end{macro}
% \end{macro}
% \end{macro}
%
% \subsection{Adding Entries into Toc-Files}
%
% We have a new general command to add entries to several toc-files at
% once. And we have some special commands for the table of contents.
%
% \begin{macro}{\addcontentslinetoeachtocfile}
%   Something like a combination of \cs{addtoeachtocfile} and
%   \cs{addcontentsline}.
%    \begin{macrocode}
\newcommand{\addcontentslinetoeachtocfile}{%
  \@ifnextchar [%]
    \@addcontentslinetoeachtocfile\@@addcontentslinetoeachtocfile
}
\newcommand{\@addcontentslinetoeachtocfile}[3][]{%
  \doforeachtocfile[{#1}]{\addcontentsline{\@currext}{#2}{#3}}%
}
\newcommand{\@@addcontentslinetoeachtocfile}[2]{%
  \doforeachtocfile{\addcontentsline{\@currext}{#1}{#2}}%
}
%    \end{macrocode}
% \end{macro}
%
% \begin{macro}{\addxcontentsline}
%   \changes{v3.12}{2013/09/24}{new}
%   \changes{v3.12}{2013/09/24}{Take care of new feature \texttt{numberline}}
% \begin{macro}{\@addxcontentsline}
%   \changes{v3.12}{2013/09/24}{new (internal)}
% This will be used instead of \cs{addcontentsline} to generate numbered or
% not numbered entries to a toc-file. First argument is the toc-file, second
% the entry type, e.g., \texttt{chapter}, the third is the entry number or
% empty and the fourth and last is the entry text.
%    \begin{macrocode}
\newcommand*{\addxcontentsline}[2]{%
  \@ifnextchar [%]
  {\@addxcontentsline{#1}{#2}}%
  {\@addxcontentsline{#1}{#2}[]}%
}
\newcommand*{\@addxcontentsline}{}
\def\@addxcontentsline#1#2[#3]#4{%
  \begingroup\expandafter\expandafter\expandafter\endgroup
  \expandafter\ifx\csname add#2#1entry\endcsname
    \@nameuse{add#2#1entry}[{#3}]{#4}%
  \else
    \tocbasic@addxcontentsline{#1}{#2}{#3}{#4}%
  \fi
}
%    \end{macrocode}
% \begin{macro}{\nonumberline}
%   \changes{v3.12}{2013/09/24}{new}
% Either \cs{relax} or \cs{numberline} depending on the feature
% \texttt{numberline}.
%    \begin{macrocode}
\newcommand*{\nonumberline}{}%
%    \end{macrocode}
% \end{macro}
% \begin{macro}{\addxcontentsline}
% This may be used to either use a heading type command or the general default
% one. You are not allowed to use it in the definition of a heading type
% command like \cs{addchaptertocentry}! You may use
% \cs{tocbasic@addxcontentsline} at the definition of those commands.
%    \begin{macrocode}
\newcommand*{\tocbasic@addxcontentsline}[4]{%
  \if\relax\detokenize{#3}\relax
    \addcontentsline{#1}{#2}{\protect\nonumberline#4}%
  \else
    \addcontentsline{#1}{#2}{\protect\numberline{#3}#4}%
  \fi
}
%    \end{macrocode}
% \end{macro}
% \end{macro}
% \end{macro}
%
% \begin{macro}{\addxcontentslinetoeachtocfile}
%   \changes{v3.12}{2013/09/24}{new}
% \begin{macro}{\@addxcontentslinetoeachtocfile}
%   \changes{v3.12}{2013/09/24}{new (internal)}
% \begin{macro}{\@@addxcontentslinetoeachtocfile}
%   \changes{v3.12}{2013/09/24}{new (internal)}
% \begin{macro}{\@@@addxcontentslinetoeachtocfile}
%   \changes{v3.12}{2013/09/24}{new (internal)}
% \begin{macro}{\@@@@addxcontentslinetoeachtocfile}
%   \changes{v3.12}{2013/09/24}{new (internal)}
% Similar to \cs{addcontentslinetoeachtocfile} using
% \cs{addxcontentsline} instead of \cs{addcontentsline}.
%    \begin{macrocode}
\newcommand{\addxcontentslinetoeachtocfile}{%
  \@ifnextchar [%]
    \@addxcontentslinetoeachtocfile\@@addxcontentslinetoeachtocfile
}
\newcommand{\@addxcontentslinetoeachtocfile}[2][]{%
  \@ifnextchar [%]
    {\@@@addxcontentslinetoeachtocfile[{#1}]{#2}}%
    {\@@@addxcontentslinetoeachtocfile[{#1}]{#2}[]}
}
\newcommand*{\@@@addxcontentslinetoeachtocfile}{}
\def\@@@addxcontentslinetoeachtocfile[#1]#2[#3]#4{%
  \doforeachtocfile[{#1}]{\addxcontentsline{\@currext}{#2}[#3]{#4}}%
}
\newcommand{\@@addxcontentslinetoeachtocfile}[1]{%
  \@ifnextchar [%]
    {\@@@@addxcontentslinetoeachtocfile{#1}}%
    {\@@@@addxcontentslinetoeachtocfile{#1}[]}%
}
\newcommand*{\@@@@addxcontentslinetoeachtocfile}{}
\def\@@@@addxcontentslinetoeachtocfile#1[#2]#3{%
  \doforeachtocfile{\addxcontentsline{\@currext}{#1}[{#2}]{#3}}%
}
%    \end{macrocode}
% \end{macro}
% \end{macro}
% \end{macro}
% \end{macro}
% \end{macro}
%
%
% \subsection{Show List of Toc-File}
% \label{sec:showlistoftocfile}
%
% If you have a toc-file you want a list-of-command for this toc-file,
% too. Here are basics and high level commands for this.
%
% \begin{macro}{\tocbasic@starttoc}
%   Some basics are done like setting up \cs{parskip}, \cs{parindent} and
%   \cs{parfillskip}, a general hook will be called, an individual hook will
%   be called, the toc will be started, an individual hook will be called, an
%   general hook wil be called, that's it.
%    \begin{macrocode}
\newcommand*{\tocbasic@starttoc}[1]{%
  \begingroup
    \setlength{\parskip}{\z@}%
    \setlength{\parindent}{\z@}%
    \setlength{\parfillskip}{\z@\@plus 1fil}%
    \edef\@currext{#1}%
    \csname tocbasic@@before@hook\endcsname
    \csname tb@#1@before@hook\endcsname
    \@starttoc{#1}%
    \csname tb@#1@after@hook\endcsname
    \csname tocbasic@@after@hook\endcsname
  \endgroup
}
%    \end{macrocode}
% \begin{macro}{\tocbasic@@before@hook}
% \begin{macro}{\tocbasic@@after@hook}
%   These are the general hooks. They may be used by classes and packages for
%   commands, that should be used for all lists of not only the own lists of,
%   e.\,g., \KOMAScript{} may use it to handle option \texttt{tocleft}.
%    \begin{macrocode}
\newcommand*{\tocbasic@@before@hook}{}
\newcommand*{\tocbasic@@after@hook}{}
%    \end{macrocode}
% \end{macro}
% \end{macro}
% \begin{macro}{\BeforeStartingTOC}
%   \changes{v3.04b}{2010/01/05}{fix: define the hook if not already defined}
%   \changes{v3.06}{2010/06/09}{fix: using \cs{@ifundefined} instead of
%     undefined \cs{scr@ifundefinedorrelax}}
% \begin{macro}{\AfterStartingTOC}
%   \changes{v3.04b}{2010/01/05}{fix:define the hook if not already defined}
%   \changes{v3.06}{2010/06/09}{fix: using \cs{@ifundefined} instead of
%     undefined \cs{scr@ifundefinedorrelax}}
%   These are the commands to add code to the general or individual hooks. If
%   the first, optional argument was given, the second, mandatory argument
%   will be added to the individual hook, otherwise the general hook will be
%   extended.
%    \begin{macrocode}
\newcommand{\BeforeStartingTOC}[2][]{%
  \ifx\relax#1\relax
    \g@addto@macro\tocbasic@@before@hook{#2}%
  \else
    \@ifundefined{tb@#1@before@hook}{%
      \PackageInfo{tocbasic}{defining new hook before starting `#1'}%
      \expandafter\global\expandafter\let\csname tb@#1@before@hook\endcsname
      \@empty
    }{}%
    \expandafter\g@addto@macro\csname tb@#1@before@hook\endcsname{#2}%
  \fi
}
\newcommand{\AfterStartingTOC}[2][]{%
  \ifx\relax#1\relax
    \g@addto@macro\tocbasic@@after@hook{#2}%
  \else
    \@ifundefined{tb@#1@after@hook}{%
      \PackageInfo{tocbasic}{defining new hook after starting `#1'}%
      \expandafter\global\expandafter\let\csname tb@#1@after@hook\endcsname
      \@empty
    }{}%
    \expandafter\g@addto@macro\csname tb@#1@after@hook\endcsname{#2}%
  \fi
}
%    \end{macrocode}
% \end{macro}
% \end{macro}
% \end{macro}
%
% \begin{macro}{\listoftoc}
% \begin{macro}{\listoftoc*}
%   \changes{v1.0a}{2008/11/12}{\cs{twocolumn} fixed}%
%   \changes{v1.01}{2008/11/13}{new feature \texttt{onecolumn}}
%   \changes{v3.09}{2011/03/01}{\cs{@currext} defined using \cs{edef} instead
%     of \cs{def}}
%   Command to handle the hole list of something. There are additional hooks
%   for this. The first optional argument is the title for this list. If the
%   optional argument was omitted \cs{listof\#2name} will be used. The star
%   version does not set up a heading or switch the column number!
%    \begin{macrocode}
\newcommand*{\listoftoc}{%
  \@ifstar \tocbasic@starttoc\@listoftoc
}
\newcommand*{\@listoftoc}[2][\list@fname]{%
  \begingroup
    \@ifundefined{listof#2name}{%
      \let\list@fname\relax
      \ifx\relax#1\relax
        \PackageWarning{tocbasic}{%
          You should either define \expandafter\string\csname
          listof#2name\endcsname\MessageBreak
          or use the optional argument of \string\listoftoc\space\MessageBreak
          to set the term to be used for the\MessageBreak
          heading of list of #2}%
        \def\list@fname{\listofname~#2}%
      \fi
    }{%
      \expandafter\let\expandafter\list@fname\csname listof#2name\endcsname
    }%
    \edef\@currext{#2}%
    \iftocfeature{\@currext}{onecolumn}{%
      \iftocfeature{\@currext}{leveldown}{}{%
        \if@twocolumn
          \aftergroup\twocolumn\onecolumn
        \fi
      }%
    }{}%
    \iftocfeature{\@currext}{numberline}{\def\nonumberline{\numberline{}}}{}%
    \tocbasic@listhead{#1}%
    \begingroup\expandafter\expandafter\expandafter\endgroup
    \expandafter\ifx\csname microtypesetup\endcsname\relax
    \else
      \iftocfeature{\@currext}{noprotrusion}{}{%
        \microtypesetup{protrusion=false}%
        \PackageInfo{tocbasic}{character protrusion at \@currext\space deactivated}%
      }%
    \fi
    \tocbasic@starttoc{#2}%
  \endgroup
}
%    \end{macrocode}
% \end{macro}
% \end{macro}
%
% \begin{macro}{\listofname}
%   While this is only an emergancy command, we don't support languages.
%    \begin{macrocode}
\newcommand*{\listofname}{List of}
%    \end{macrocode}
% \end{macro}
%
% \begin{macro}{\listofeachtoc}
% \begin{macro}{\@listofeachtoc}
% \begin{macro}{\@@listofeachtoc}
%   \changes{v3.03b}{2009/06/08}{typo fix at usage of \cs{doforeachtocfile}}
%   Another example of using \cs{doforeachtocfile}.
%    \begin{macrocode}
\newcommand*{\listofeachtoc}{%
  \@ifnextchar [%]
    \@listofeachtoc\@@listofeachtoc
}
\newcommand{\@listofeachtoc}[1][]{%
  \doforeachtocfile[{#1}]{\listoftoc{\@currext}}%
}
\newcommand{\@@listofeachtoc}[1]{%
  \doforeachtocfile{\listoftoc{\@currext}}%
}
%    \end{macrocode}
% \end{macro}
% \end{macro}
% \end{macro}
%
% \begin{macro}{\tocbasic@listhead}
%   \changes{v3.02a}{2009/01/20}{no explicite marks if ``numbered''}%
%   \changes{v3.09a}{2011/04/12}{again no explicite marks if ``numbered''}%
%   \changes{v3.12}{2013/09/24}{usage of new
%     \cs{tocbasic@addxcontentsline}}%
%   Setting the headings of a list of something. The heading is the only
%   argument.
%    \begin{macrocode}
\newcommand*{\tocbasic@listhead}[1]{%
  \@ifundefined{tocbasic@listhead@\@currext}{%
    \begingroup\expandafter\expandafter\expandafter\endgroup
    \expandafter\ifx\csname chapter\endcsname\relax
      \expandafter\def\csname tocbasic@listhead@\@currext\endcsname##1{%
        \iftocfeature{\@currext}{leveldown}{%
          \iftocfeature{\@currext}{numbered}{%
            \subsection[##1]{##1}%
          }{%
            \subsection*{##1}%
            \ifx\@mkboth\@gobbletwo\else\markright{\MakeMarkcase{##1}}\fi
            \iftocfeature{\@currext}{totoc}{%
              \addxcontentsline{toc}{subsection}{##1}%
            }{}%
          }%
        }{%
          \iftocfeature{\@currext}{numbered}{%
            \section[##1]{##1}%
          }{%
            \section*{##1}%
            \@mkboth{\MakeMarkcase{##1}}{\MakeMarkcase{##1}}%
            \iftocfeature{\@currext}{totoc}{%
              \addxcontentsline{toc}{section}{##1}%
            }{}%
          }%
        }%
      }%
    \else
      \expandafter\def\csname tocbasic@listhead@\@currext\endcsname##1{%
        \iftocfeature{\@currext}{leveldown}{%
          \iftocfeature{\@currext}{numbered}{%
            \section{##1}%
          }{%
            \section*{##1}%
            \iftocfeature{\@currext}{totoc}{%
              \addxcontentsline{toc}{section}{##1}%
            }{}%
            \ifx\@mkboth\@gobbletwo\else\markright{\MakeMarkcase{##1}}\fi
          }%
        }{%
          \iftocfeature{\@currext}{numbered}{%
            \chapter[##1]{##1}%
          }{%
            \chapter*{##1}%
            \@mkboth{\MakeMarkcase{##1}}{\MakeMarkcase{##1}}%
            \iftocfeature{\@currext}{totoc}{%
              \addxcontentsline{toc}{chapter}{##1}%
            }{}%
          }%
        }%
      }%
    \fi
  }{}%
  \csname tb@@beforehead@hook\endcsname
  \csname tb@\@currext @beforehead@hook\endcsname
  \csname tocbasic@listhead@\@currext\endcsname{#1}%
  \csname tb@\@currext @afterhead@hook\endcsname
  \csname tb@@afterhead@hook\endcsname
}
%    \end{macrocode}
% \begin{macro}{\BeforeTOCHead}
%   \changes{v3.04b}{2010/01/05}{fix: define the hook if not already defined}
%   \changes{v3.06}{2010/06/09}{fix: using \cs{@ifundefined} instead of
%     undefined \cs{scr@ifundefinedorrelax}}
% \begin{macro}{\AfterTOCHead}
%   \changes{v3.04b}{2010/01/05}{fix: define the hook if not already defined}
%   \changes{v3.06}{2010/06/09}{fix: using \cs{@ifundefined} instead of
%     undefined \cs{scr@ifundefinedorrelax}}
%   These are the commands to add code to the general or individual hooks. If
%   the first, optional argument was given, the second, mandatory argument
%   will be added to the individual hook, otherwise the general hook will be
%   extended.
%    \begin{macrocode}
\newcommand{\BeforeTOCHead}[2][]{%
  \@ifundefined{tb@#1@beforehead@hook}{%
    \PackageInfo{tocbasic}{defining new hook before heading of `#1'}%
    \expandafter\global\expandafter\let\csname tb@#1@beforehead@hook\endcsname
    \@empty
  }{}%
  \expandafter\g@addto@macro\csname tb@#1@beforehead@hook\endcsname{#2}%
}
\newcommand{\AfterTOCHead}[2][]{%
  \@ifundefined{tb@#1@afterhead@hook}{%
    \PackageInfo{tocbasic}{defining new hook after heading of `#1'}%
    \expandafter\global\expandafter\let\csname tb@#1@afterhead@hook\endcsname
    \@empty
  }{}%
  \expandafter\g@addto@macro\csname tb@#1@afterhead@hook\endcsname{#2}%
}
%    \end{macrocode}
% \end{macro}
% \end{macro}
% \end{macro}
%
% \begin{macro}{\MakeMarkcase}
%   Use upper-case or not?
%    \begin{macrocode}
\AtBeginDocument{%
  \@ifundefined{MakeMarkcase}{%
    \begingroup\expandafter\expandafter\expandafter\endgroup
    \expandafter\ifx\csname KOMAClassName\endcsname\relax
      \let\MakeMarkcase\MakeUppercase
    \else
      \let\MakeMarkcase\@firstofone
    \fi
  }{}%
}
%    \end{macrocode}
% \end{macro}
%
% \begin{macro}{\deftocheading}
%   Define a toc headings command with one argument (the title).
%    \begin{macrocode}
\newcommand*{\deftocheading}[1]{%
  \@namedef{tocbasic@listhead@#1}##1}
%    \end{macrocode}
% \end{macro}
%
% \begin{macro}{\setuptoc}
% \begin{macro}{\unsettoc}
%   Known features are:
%   \begin{description}
%   \item[\texttt{totoc}] writes the title of the list of to the table of
%     contents
%   \item[\texttt{numberline}] redefines \cs{nonumberline} to use
%     \cs{numberline}
%   \item[\texttt{numbered}] uses a numbered headings for the list of
%   \item[\texttt{leveldown}] uses not the top level heading (e.\,g.,
%     \cs{chapter} with book) but the first sub level (e.\,g., \cs{section} with
%     book).
%   \item[\texttt{nobabel}] prevents the extension to be added to the babel
%     handling of toc-files.  To make this work, you have to set the feature
%     before adding the extension to the list of known extension.
%   \item[[\texttt{noprotrusion}]] prevents disabling character protrusion at
%     the toc.
%   \end{description}
%   Other features may be package dependent. You may test the feature using:
%   \begin{quote}
%   \cs{@ifundefined}\texttt{\{tocbasic@\meta{toc}@feature@\meta{feature}\}}\\
%   \phantom{\cs{@ifundefined}}\marg{do if feature not set}\\
%   \phantom{\cs{@ifundefined}}\marg{do if feature set}
%   \end{quote}
%   See \cs{tocbasic@listhead} for an example of this.
%    \begin{macrocode}
\newcommand*{\setuptoc}[2]{%
  \@for\@tempa:=#2\do{%
    \expandafter\tb@@sp@def\expandafter\@tempa\expandafter{\@tempa}%
    \@namedef{tocbasic@#1@feature@\@tempa}{}%
  }%
}
\newcommand*{\unsettoc}[2]{%
  \@for\@tempa:=#2\do{%
    \expandafter\tb@@sp@def\expandafter\@tempa\expandafter{\@tempa}%
    \expandafter\let\csname tocbasic@#1@feature@\@tempa\endcsname\relax
  }%
}
\def\@tempa#1{%
  \def\tb@@sp@def##1##2{%
    \futurelet\tb@sp@tempa\tb@@sp@d##2\@nil\@nil#1\@nil\relax##1}%
  \def\tb@@sp@d{%
    \ifx\tb@sp@tempa\@sptoken
      \expandafter\tb@@sp@b
    \else
      \expandafter\tb@@sp@b\expandafter#1%
    \fi}%
  \def\tb@@sp@b#1##1 \@nil{\tb@@sp@c##1}%
}
\@tempa{ }
\def\tb@@sp@c#1\@nil#2\relax#3{\@temptokena{#1}\edef#3{\the\@temptokena}}
%    \end{macrocode}
%   Do not add the files, that babel handles by default.
%    \begin{macrocode}
\setuptoc{toc}{nobabel}
\setuptoc{lof}{nobabel}
\setuptoc{lot}{nobabel}
%    \end{macrocode}
% \begin{macro}{\iftocfeature}
%   We need a test for the features
%    \begin{macrocode}
\newcommand*{\iftocfeature}[2]{%
  \begingroup\expandafter\expandafter\expandafter\endgroup
  \expandafter\ifx\csname tocbasic@#1@feature@#2\endcsname\relax
    \expandafter\@secondoftwo
  \else
    \expandafter\@firstoftwo
  \fi
}
%    \end{macrocode}
% \end{macro}
% \end{macro}
% \end{macro}
%
% \begin{macro}{\tocbasicautomode}
%   \changes{v1.0a}{2008/11/11}{use of \cs{addtotoclist} instead of internal}%
%   \changes{v3.07a}{2010/11/25}{fix: added missing
%     \cs{let}\cs{@starttoc}\cs{tocbasic@starttoc}}%
% \Package{tocbasic} can overtake \Macro{@starttoc} to automaticly add all
% used extensions to the list of known extensions and use
% \Macro{tocbasic@starttoc} instead of \Macro{@starttoc} from \LaTeX{}
% kernel. Please note, that we don't need a fix for the unfriendly
% redefinition of \Macro{@starttoc} by \Package{hyperref}, because hyperref
% does it only at the first \LaTeX{} run. So this action of \Package{hyperref}
% may only result in the need of one more \LaTeX{} run, but not in a permanent
% mistake.
%    \begin{macrocode}
\newcommand*{\tocbasicautomode}{%
  \let\tb@saved@starttoc\@starttoc
  \let\@starttoc\tocbasic@starttoc
  \BeforeStartingTOC{%
    \let\@starttoc\tb@saved@starttoc
    \expandafter\ifattoclist\expandafter{\@currext}{}{%
      \addtotoclist[tocbasicautomode]{\@currext}%
    }%
  }%
}
%    \end{macrocode}
% \end{macro}
%
%
% \section{High Level Interface for Generating New TOCs and Floats}
% \label{sec:newfloats}
%
% \begin{macro}{\DeclareNewTOC}
%   \changes{v3.06}{2010/06/01}{new high level command}%
%   This command defines only a new TOC and a new caption command for this
%   TOC.  It does not define a new foat unless option float was set!  It does
%   not define a new nonfloat unless option nonfloat was set!  The optional
%   argument is used to set up several settings using \Package{keyval}.  The
%   mandatory argument is the shortcut.  The default type is the shortcut.
%   The extension is the shortcut.  The default owner is ``\texttt{float}''.
%
%   But first of all, we define the options used by the optional argument:
%   The type of list is used for environment and counter names and as part of
%   several internal and user macros.
%    \begin{macrocode}
\define@key{tocbasic}{type}{%
  \tb@ifvalueisnotempty{type}{#1}{\edef\tb@nt@type{#1}}%
}
%    \end{macrocode}
% \begin{macro}{\tb@nt@counterwithin}
%   If a counter will be defined, this counter may depend on another counter
%   (like figure counter depends on chapter counter at book classes). If the
%   value is empty, the counter does not depend on any other counter.
%    \begin{macrocode}
\define@key{tocbasic}{counterwithin}{%
  \edef\tb@nt@counterwithin{#1}%
}
%    \end{macrocode}
% \end{macro}
% \begin{macro}{\tb@nt@types}
%   Sometimes the plural of the type is needed and may be defined by a
%   seperate option.
%    \begin{macrocode}
\define@key{tocbasic}{types}{%
  \tb@ifvalueisnotempty{types}{#1}{\edef\tb@nt@types{#1}}%
}
%    \end{macrocode}
% \end{macro}
% \begin{macro}{\tb@nt@ext}
%   Don't tell the user, that the mandatory argument of the command may be
%   overloaded by this option. This was only done to simplify the definition.
%    \begin{macrocode}
\define@key{tocbasic}{extension}{%
  \tb@ifvalueisnotempty{extension}{#1}{\edef\tb@nt@ext{#1}}%
}
%    \end{macrocode}
% \end{macro}
% \begin{macro}{\tb@nt@owner}
%   The owner of the new list extension.
%    \begin{macrocode}
\define@key{tocbasic}{owner}{%
  \tb@ifvalueisnotempty{owner}{#1}{\edef\tb@nt@owner{#1}}%
}
%    \end{macrocode}
% \end{macro}
% \begin{macro}{\tb@nt@name}
%   We need a name/term for the entries of the new list. This may e.\,g. be used
%   for |caption| output. A |\<type>name| will also be defined.
%    \begin{macrocode}
\define@key{tocbasic}{name}{%
  \tb@ifvalueisnotempty{name}{#1}{\edef\tb@nt@name{#1}}%
}
%    \end{macrocode}
% \end{macro}
% \begin{macro}{\tb@nt@listname}
%   The list itself has a heading. A |\list<type>name| will also be defined.
%    \begin{macrocode}
\define@key{tocbasic}{listname}{%
  \tb@ifvalueisnotempty{listname}{#1}{\edef\tb@nt@listname{#1}}%
}
%    \end{macrocode}
% \end{macro}
% \begin{macro}{\iftb@forcenames}
%   \changes{v3.06}{2010/06/01}{new internal}
%   The |\...name| commands will only be defined, if they are not already
%   defined. But some users may want to define their own names even if there
%   are already definitions. So this boolean option may switch to force the
%   definition.
%    \begin{macrocode}
\newif\iftb@forcenames
\define@key{tocbasic}{forcenames}[true]{%
  \lowercase{\tb@boolkey{#1}}{forcenames}%
}
%    \end{macrocode}
% \end{macro}
% \begin{macro}{\tb@nt@level}
%   Each entry to a TOC/list of has a numeric level. You may set up your own
%   level.
%    \begin{macrocode}
\define@key{tocbasic}{level}{%
  \tb@ifvalueisnotempty{level}{#1}{\tb@numkey\tb@nt@level{#1}}%
}
%    \end{macrocode}
% \end{macro}
% \begin{macro}{\tb@nt@indent}
%   Earch entry of a TOC/list of is idented on the left side. This is the
%   amount of the indention of the number part.
%    \begin{macrocode}
\define@key{tocbasic}{indent}{%
  \tb@ifvalueisnotempty{indent}{#1}{\tb@dimkey\tb@nt@indent{#1}}%
}
%    \end{macrocode}
% \end{macro}
% \begin{macro}{\tb@nt@hang}
%   Earch entry of a TOC/list of is idented on the left side. This is the
%   amount of the indention of the text of the entry.
%    \begin{macrocode}
\define@key{tocbasic}{hang}{%
  \tb@ifvalueisnotempty{hang}{#1}{\tb@dimkey\tb@nt@hang{#1}}%
}
%    \end{macrocode}
% \end{macro}
% \begin{macro}{\iftb@float}
%   \changes{v3.06}{2010/06/01}{new internal}
%   This boolean indicates, that float environments should also be defined.
%    \begin{macrocode}
\newif\iftb@float
\define@key{tocbasic}{float}[true]{%
  \lowercase{\tb@boolkey{#1}}{float}%
}
%    \end{macrocode}
% \end{macro}
% \begin{macro}{\iftb@nonfloat}
%   \changes{v3.06}{2010/06/01}{new internal}
%   This boolean indicates, that non-float environment should also be
%   defined. The environment has the postfix ``-''.
%    \begin{macrocode}
\newif\iftb@nonfloat
\define@key{tocbasic}{nonfloat}[true]{%
  \lowercase{\tb@boolkey{#1}}{nonfloat}%
}
%    \end{macrocode}
% \end{macro}
% \begin{macro}{\tb@nt@floattype}
%   Simply the value for the |\ftype@...| macro. It has to be
%   $1<=\textrm{value}<=31$.
%    \begin{macrocode}
\define@key{tocbasic}{floattype}{%
  \tb@ifvalueisnotempty{floattype}{#1}{\tb@numkey\tb@nt@floattype{#1}}%
}
%    \end{macrocode}
% \end{macro}
% \begin{macro}{\tb@nt@floatpos}
%   Simply the value for the |\fps@...| macro.
%    \begin{macrocode}
\define@key{tocbasic}{floatpos}{%
  \tb@ifvalueisnotempty{floatpos}{#1}{\edef\tb@nt@floatpos{#1}}%
}
%    \end{macrocode}
% \end{macro}
% \begin{macro}{\tb@nt@atbegin}
%   \changes{v3.09}{2011/03/01}{new optional identifier \texttt{atbegin}}
%   \changes{v3.11c}{2013/02/01}{\texttt{atbegin} argument may be empty}
%   Additional code executed at the begin of a new defined environment.
%    \begin{macrocode}
\define@key{tocbasic}{atbegin}{%
  \def\tb@nt@atbegin{#1}%
}
%    \end{macrocode}
% \end{macro}
% \begin{macro}{\tb@nt@atend}
%   \changes{v3.09}{2011/03/01}{new optional identifier \texttt{atbend}}
%   \changes{v3.11c}{2013/02/01}{\texttt{atend} argument may be empty}
%   Additional code executed at the end of a new defined environment.
%    \begin{macrocode}
\define@key{tocbasic}{atend}{%
  \def\tb@nt@atend{#1}%
}
%    \end{macrocode}
% \end{macro}
%  Now the command to define a new list and maybe float or nonfloat.
%    \begin{macrocode}
\newcommand*{\DeclareNewTOC}[2][]{%
  \tb@floatfalse
  \tb@nonfloatfalse
  \tb@forcenamesfalse
  \def\tb@nt@floattype{\tb@nt@autofloattype}%
  \def\tb@nt@floatpos{tbp}%
  \def\tb@nt@ext{#2}%
  \def\tb@nt@type{#2}%
  \def\tb@nt@types{\tb@nt@type s}%
  \def\tb@nt@owner{float}%
  \def\tb@nt@name{\expandafter\MakeUppercase\tb@nt@type}%
  \def\tb@nt@listname{List of \expandafter\MakeUppercase\tb@nt@types}%
  \def\tb@nt@level{1}%
  \def\tb@nt@indent{1em}%
  \def\tb@nt@hang{1.5em}%
  \let\tb@nt@counterwithin\@empty
  \let\tb@nt@atbegin\@empty
  \let\tb@nt@atend\@empty
  \setkeys{tocbasic}{#1}%
%    \end{macrocode}
%   First of all, every TOC needs a extension, that should be added to the
%   list of known extensions.
%    \begin{macrocode}
  \expandafter\newcommand\csname ext@\tb@nt@type\endcsname{}%
  \expandafter\let\csname ext@\tb@nt@type\endcsname\tb@nt@ext
  \addtotoclist[\tb@nt@owner]{\csname ext@\tb@nt@type\endcsname}%
%    \end{macrocode}
% Then we need the TOC itself with a name:
%    \begin{macrocode}
  \@ifundefined{\tb@nt@type name}{%
    \expandafter\let\csname \tb@nt@type name\endcsname\tb@nt@name
  }{%
    \iftb@forcenames
      \expandafter\let\csname \tb@nt@type name\endcsname\tb@nt@name
    \fi
  }%
  \@ifundefined{list\tb@nt@type name}{%
    \expandafter\let\csname list\tb@nt@type name\endcsname\tb@nt@listname
  }{%
    \iftb@forcenames
      \expandafter\let\csname list\tb@nt@type name\endcsname\tb@nt@listname
    \fi
  }%
  \expandafter\newcommand\csname listof\tb@nt@ext name\expandafter\endcsname
    \expandafter{\csname list\tb@nt@type name\endcsname}%
  \begingroup\edef\@tempa{\endgroup
    \noexpand\newcommand*\csname listof\tb@nt@types\endcsname{%
      \noexpand\listoftoc{\tb@nt@ext}%
    }%
  }\@tempa
%    \end{macrocode}
% The list entry:
%    \begin{macrocode}
  \begingroup\edef\@tempa{\endgroup
    \noexpand\newcommand*\csname l@\tb@nt@type\endcsname{%
      \noexpand\@dottedtocline{\tb@nt@level}{\tb@nt@indent}{\tb@nt@hang}%
    }%
  }\@tempa
%    \end{macrocode}
% \changes{v3.12}{2013/05/03}{fixed: definition of
%   \cs{listof\meta{type}entryname} added as explained in the manual}
%    \begin{macrocode}
  \@ifundefined{listof\tb@nt@ext entryname}{%
    \expandafter\let\csname listof\tb@nt@ext entryname\endcsname\tb@nt@name
  }{}%
%    \end{macrocode}
% The setup command:
%    \begin{macrocode}
  \begingroup\edef\@tempa{\endgroup
    \noexpand\newcommand*\csname setup\tb@nt@types\endcsname{%
      \noexpand\setuptoc{\tb@nt@ext}%
    }%
  }\@tempa
  \@tempswafalse
%    \end{macrocode}
% The float environments:
% \changes{v3.09a}{2011/05/30}{fixed: floats may have an optional argument}
%    \begin{macrocode}
  \iftb@float
    \expandafter\newcommand\expandafter*\csname fps@\tb@nt@type\expandafter
    \endcsname{\tb@nt@floatpos}%
    \expandafter\newcommand\expandafter*\csname ftype@\tb@nt@type\expandafter
    \endcsname{\tb@nt@floattype}%
    \begingroup
      \edef\@tempa{%
        \noexpand\endgroup
        \noexpand\newenvironment{\tb@nt@type}{%
          \noexpand\@ifnextchar[%]
          {\noexpand\tb@atbegin@after\noexpand\@float{\tb@nt@type}}%
          {\noexpand\tb@atbegin@after\noexpand\@float{\tb@nt@type}%
            [\expandafter\noexpand\csname fps@\tb@nt@type\endcsname]}%
        }{%
          \noexpand\csname \tb@nt@type @atend\noexpand\endcsname
          \noexpand\end@float
        }%
        \noexpand\newenvironment{\tb@nt@type*}{%
          \noexpand\@ifnextchar[%]
          {\noexpand\tb@atbegin@after\noexpand\@dblfloat{\tb@nt@type}}%
          {\noexpand\tb@atbegin@after\noexpand\@dblfloat{\tb@nt@type}%
            [\expandafter\noexpand\csname fps@\tb@nt@type\endcsname]}%
        }{%
          \noexpand\csname \tb@nt@type @atend\noexpand\endcsname
          \noexpand\end@dblfloat
        }%
      }%
    \@tempa
    \@tempswatrue
  \fi
%    \end{macrocode}
% The nonfloat environment:
%    \begin{macrocode}
  \iftb@nonfloat
    \begingroup
      \edef\@tempa{%
        \noexpand\endgroup
        \noexpand\newenvironment{\tb@nt@type-}{%
          \noexpand\trivlist\noexpand\item\noexpand\relax
          \noexpand\minipage{\noexpand\linewidth}%
          \noexpand\def\noexpand\@captype{\tb@nt@type}%
          \noexpand\csname \tb@nt@type @atbegin\noexpand\endcsname
        }{%
          \noexpand\csname \tb@nt@type @atend\noexpand\endcsname
          \noexpand\endminipage\noexpand\endtrivlist
        }%
      }%
    \@tempa
    \@tempswatrue
  \fi
%    \end{macrocode}
% The counter and hooks of float and nonfloat environments:
%   \changes{v3.10b}{2012/04/04}{\cs{noexpand} added to fix problems using
%     \cs{autodot}}
%    \begin{macrocode}
  \if@tempswa
    \@ifundefined{c@\tb@nt@type}{%
      \newcounter{\tb@nt@type}%
    }{%
      \PackageWarning{tocbasic}{using already defined counter `\tb@nt@type'}%
    }%
    \ifx\tb@nt@counterwithin\@empty\else
      \@addtoreset{\tb@nt@type}{\tb@nt@counterwithin}%
      \begingroup\edef\@tempa{%
        \noexpand\endgroup
        \noexpand\renewcommand*\expandafter\noexpand
        \csname the\tb@nt@type\endcsname{%
          \expandafter\noexpand\csname the\tb@nt@counterwithin\endcsname
          .\noexpand\arabic{\tb@nt@type}%
        }%
      }\@tempa
    \fi
    \begingroup
      \edef\@tempa{%
        \noexpand\endgroup
        \noexpand\newcommand*\expandafter\noexpand
        \csname \tb@nt@type format\endcsname{%
          \expandafter\noexpand\csname \tb@nt@type name\endcsname
          \noexpand\nobreakspace
          \expandafter\noexpand\csname the\tb@nt@type\endcsname
          \noexpand\csname autodot\endcsname
        }%
        \noexpand\newcommand*\expandafter\noexpand
        \csname fnum@\tb@nt@type\endcsname{%
          \expandafter\noexpand\csname \tb@nt@type format\endcsname
        }%
      }%
    \@tempa
    \expandafter\newcommand\expandafter*\csname \tb@nt@type @atbegin\expandafter
    \endcsname\expandafter{\tb@nt@atbegin}%
    \expandafter\newcommand\expandafter*\csname \tb@nt@type @atend\expandafter
    \endcsname\expandafter{\tb@nt@atend}%
  \fi
}
%    \end{macrocode}
% \begin{macro}{\tb@atbegin@after}
%   \changes{v3.09a}{2011/05/30}{new internal}%
%   Processes command \#1 with argument \#2 and optional argument \#3 and than
%   \verb|\csname #2@atbegin\endcsname|. This is used to add
%   \verb|csname #2@atbegin\endcsname| at begin part of a float (but after
%   optional argument of the float).
%    \begin{macrocode}
\newcommand*{\tb@atbegin@after}[3]{}
\def\tb@atbegin@after#1#2[#3]{%
  #1{#2}[#3]%
  \csname #2@atbegin\endcsname
}
%    \end{macrocode}
% \end{macro}
% \begin{macro}{\tb@ifvalueisnotempty}
%   \changes{v3.06}{2010/05/21}{new internal}%
%   This is only a helper for defining some of the keys.
%    \begin{macrocode}
\newcommand*{\tb@ifvalueisnotempty}[2]{%
  \begingroup
    \edef\@tempa{#2}\ifx\@tempa\@empty
      \PackageError{tocbasic}{empty value for key `#1'}{%
        The argument of key `#1' may not by empty!}%
      \aftergroup\@gobble
    \else
      \aftergroup\@firstofone
    \fi
  \endgroup
}
%    \end{macrocode}
% \end{macro}
% \begin{macro}{\tb@boolkey}
%   \changes{v3.06}{2010/05/21}{new internal}%
%   This is only a helper for boolean keys.
%    \begin{macrocode}
\newcommand*{\tb@boolkey}[2]{%
  \csname tb@#2\ifx\relax#1\relax true\else#1\fi\endcsname
}
%    \end{macrocode}
% \end{macro}
% \begin{macro}{\tb@dimkey}
%   \changes{v3.06}{2010/05/21}{new internal}%
%   This is only a helper for dimension keys.
%    \begin{macrocode}
\newcommand*{\tb@dimkey}[2]{%
  \begingroup
    \@defaultunits\@tempdima#2pt\relax\@nnil
    \edef\@tempa{\noexpand\endgroup\noexpand\def\noexpand#1{\the\@tempdima}}
  \@tempa
}
%    \end{macrocode}
% \end{macro}
% \begin{macro}{\tb@dimkey}
%   \changes{v3.06}{2010/05/21}{new internal}%
%   This is only a helper for numerical keys.
%    \begin{macrocode}
\newcommand*{\tb@numkey}[2]{%
  \begingroup
    \afterassignment\remove@to@nnil\@tempcnta#2\relax\@nnil
    \edef\@tempa{\noexpand\endgroup\noexpand\def\noexpand#1{\the\@tempcnta}}
  \@tempa
}
%    \end{macrocode}
% \end{macro}
% \begin{macro}{\tb@nt@autofloattype}
%   \changes{v3.06}{2010/06/01}{new internal}
%   This macro saves the float type for auto generated float types.
%    \begin{macrocode}
\newcommand*{\tb@nt@autofloattype}{16}
%    \end{macrocode}
% \end{macro}
% \end{macro}
%
%
%
% \iffalse
%</package>
%<*load>
% Used at \KOMAScript{} packages and classes to load the package.
%    \begin{macrocode}
\RequirePackage{tocbasic}[%
%!KOMAScriptVersion
]
%    \end{macrocode}
%
%<*class>
% \KOMAScript{} classes need some additional initialization.
%
% \changes{v1.00}{auto-activation of feature \texttt{onecolumn}}
% Set feature \texttt{onecolumn} for every toc file if feature
% \texttt{leveldown} was not used.
%    \begin{macrocode}
%<*chapter>
\AtAddToTocList[\@currname.\@currext]{\setuptoc{\@currext}{onecolumn}}%
\AtAddToTocList[float]{\setuptoc{\@currext}{onecolumn}}%
%</chapter>
%    \end{macrocode}
% Tell the package, what files are used:
%    \begin{macrocode}
\addtotoclist{toc}
\addtotoclist[float]{lof}
\addtotoclist[float]{lot}
%    \end{macrocode}
%</class>
%</load>
% \fi
%
% \Finale
%
\endinput
%
% end of file `tocbasic.dtx'
%%% Local Variables:
%%% mode: doctex
%%% coding: iso-latin-1
%%% TeX-master: t
%%% End:
