% \CheckSum{0}
% \iffalse meta-comment
% tocbasic.dtx
%
% Copyright (c) Markus Kohm, 2007-2008
% This is a alpha verion, therefor: All rights reserved.
% \fi
%
% \CharacterTable
%  {Upper-case    \A\B\C\D\E\F\G\H\I\J\K\L\M\N\O\P\Q\R\S\T\U\V\W\X\Y\Z
%   Lower-case    \a\b\c\d\e\f\g\h\i\j\k\l\m\n\o\p\q\r\s\t\u\v\w\x\y\z
%   Digits        \0\1\2\3\4\5\6\7\8\9
%   Exclamation   \!     Double quote  \"     Hash (number) \#
%   Dollar        \$     Percent       \%     Ampersand     \&
%   Acute accent  \'     Left paren    \(     Right paren   \)
%   Asterisk      \*     Plus          \+     Comma         \,
%   Minus         \-     Point         \.     Solidus       \/
%   Colon         \:     Semicolon     \;     Less than     \<
%   Equals        \=     Greater than  \>     Question mark \?
%   Commercial at \@     Left bracket  \[     Backslash     \\
%   Right bracket \]     Circumflex    \^     Underscore    \_
%   Grave accent  \`     Left brace    \{     Vertical bar  \|
%   Right brace   \}     Tilde         \~}
%
% \iffalse
%%% From File: tocbasic.dtx
%<*dtx>
\def\LaTeXformat{LaTeX2e}
\ifx\fmtname\LaTeXformat\def\MainBodyWork{%
\ProvidesFile{tocbasic.dtx}
%</dtx>
%<package>\NeedsTeXFormat{LaTeX2e}[1995/06/01]
%<driver>\ProvidesFile{tocbasic.drv}
%<package>\ProvidesPackage{tocbasic}
%<*dtx|package|driver>
                [2008/03/13 v0.1 alpha
%</dtx|package|driver>
%<driver>                 (driver)%
%<package>                (package)%
%<*dtx>
                 (dtx)%
%</dtx>
%<*dtx|package|driver>
                ]
%</dtx|package|driver>
%<*dtx|driver>
\IfFileExists{scrdoc.cls}{%
  \documentclass[halfparskip-]{scrdoc}
}{%
  \documentclass{ltxdoc}
}
\providecommand*{\DescribeOption}{\DescribeMacro}
\providecommand*{\DescribeCounter}{\DescribeMacro}
\usepackage[latin1]{inputenc}
\usepackage[ngerman]{babel}
\CodelineIndex
\RecordChanges
\GetFileInfo{tocbasic.dtx}
\DocInput{\filename}
\end{document}
%</dtx|driver>
%<*dtx>
}\else\let\MainBodyWork\relax\fi

\MainBodyWork

\def\batchfile{tocbasic.dtx}
%</dtx>
%<ins>\def\batchfile{tocbasic.ins}
%<*dtx|ins>
\input docstrip.tex

\let\ifbeta=\iftrue

\ifToplevel{\ifx\generate\undefined
    \errhelp{Install a new version of docstrip.}
    \errmessage{Old docstrip in input path}
    \batchmode
    \csname @@end\endcsname
  \fi
  \Msg{************************************************************}
  \Msg{*}
  \Msg{* KOMA-Script}
  \Msg{* a versatile LaTeX2e bundle}
  \Msg{*}
  \Msg{* This is `\batchfile', a batchfile to unpack some or all}
  \Msg{* parts of KOMA-Script. See `liesmich.txt' (german) or}
  \Msg{* `readme.txt' (english) for additional information.}
  \Msg{*}
  \Msg{* Files of an old KOMA-Script installation may be}
  \Msg{* overwritten without asking!}
  \Msg{*}
  \Msg{************************************************************}
}

% ---------- some docstrip switchs -------------------------------------

\ifToplevel{%
  \keepsilent
  \askforoverwritefalse
}

% ---------- defining preambles ----------------------------------------

\preamble

Copyright (c) 2007-2008
Markus Kohm and any individual authors listed elsewhere in this file.

This file was generated from file(s) of the KOMA-Script bundle.
---------------------------------------------------------------

This work may be distributed and/or modified under the conditions of
the LaTeX Project Public License, version 1.3b of the license.
The latest version of this license is in
  http://www.latex-project.org/lppl.txt
and version 1.3b or later is part of all distributions of LaTeX
version 2005/12/01 or later and of this work.

This work has the LPPL maintenance status "author-maintained".

The Current Maintainer and author of this work is Markus Kohm.

This file may only be distributed together with the files
`scrlogo.dtx' and `tocbasic.dtx'. You may however distribute the files
`scrlogo.dtx' and `tocbasic.dtx' without this file. 
See also `tocbasic.dtx' for additional information.

If this file is a beta version, you are not allowed to distribute it.

Currently there is only a short english manual at `tocbasic.dtx', that
should also be found as `tocbasic.pdf'.

The KOMA-Script bundle (but maybe not this file) was based upon the
LaTeX2.09 Script family created by Frank Neukam 1993 and the LaTeX2e
standard classes created by The LaTeX3 Project 1994-1996.

\endpreamble

% ---------- File generation -------------------------------------------

\generate{\usepreamble\defaultpreamble
  \file{tocbasic.ins}{%
    \from{tocbasic.dtx}{ins}%
  }%
%  \file{tocbasic.drv}{%
%    \from{tocbasic.dtx}{driver}%
%  }%
  \file{tocbasic.sty}{%
    \from{tocbasic.dtx}{package}%
    \from{scrlogo.dtx}{logo}%
  }%
}%

% ---------- end of docstrip process -----------------------------------

\ifToplevel{%
  \Msg{************************************************************}
  \Msg{*}
  \Msg{* KOMA-Script}
  \Msg{* a versatile LaTeX2e bundle}
  \Msg{*}
  \ifbeta
    \Msg{* THIS IS AN ALPHA VERSION. YOU SHOULD NOT INSTALL OR USE IT!}
    \Msg{* THERE MAY BE A LOT OF BUGS AT THIS VERSION!}
%    \Msg{* PLEASE INSTALL THE RELEASE YOU CAN FIND AT CTAN!}
  \else
    \Msg{* To finish the installation you have to move some}
    \Msg{* files into a directory searched by TeX.}
    \Msg{* See INSTALL.TXT (english) or INSTALLD.TXT (german)}
    \Msg{* from KOMA-Script bundle for additional information.}
    \Msg{*}
    \Msg{* You may also produce the implementation documentation}
    \Msg{* including also a short version of the english manual.}
    \Msg{* To produce it, do}
    \Msg{* \space\space pdflatex tocbasic.dtx}
    \Msg{* \space\space mkindex tocbasic}
    \Msg{* \space\space pdflatex tocbasic.dtx}
    \Msg{* after finishing the installation.}
    \Msg{*}
    \Msg{* Happy TeXing}
  \fi
  \Msg{*}
  \Msg{************************************************************}
}

\bye
%</dtx|ins>
% \fi
%
% \title{\KOMAScript{} \partname\ \texttt{\filename}%
%   \thanks{Diese Datei ist Version \fileversion\ von \texttt{\filename}.}}
% \date{\filedate}
% \author{Markus Kohm\thanks{mailto:komascript(at)gmx.info}}
% \begin{document}
% \maketitle
% \begin{abstract}
%   If a package creates it's list ``list of something''---something like ``list
%   of figures'', ``list of tables'', ``list of listings'', ``list of
%   algorithms'', etc. also known as \emph{toc-files}---have to do some
%   operations, that are equal for all those packages. Also it may be usefull
%   for classes and other packages to know about these additional
%   toc-files. This packages implements some basic functionality for all those
%   packages. Using this package will also improve compatibility with
%   \KOMAScript{} and---let us hope---other classes and packages.
% \end{abstract}
%
% \tableofcontents
% 
% \section{Legal Note}
% You are allowed to destribute this part of \KOMAScript{} without the main
% part of \KOMAScript{}. The files ``\texttt{scrlogo.dtx}'' and
% ``\texttt{tocbasic.dtx}'' may be distributed together under the conditions
% of the \LaTeX{} Project Public License, either version~1.3b of this license
% or (at your option) any later version.
%
% The latest version of this license is in
% \mbox{http://www.latex-project.org/lppl.txt} and version~1.3b or later is
% part of all distributions of \LaTeX{} version~2005/12/01 or later.
%
% \KOMAScript{} comes with a detailed manual in English and German. But for
% this package theres currently only this short English manual, because this
% package should be used by package authors, not users.
%
% The \KOMAScript{} bundle may be found at
% CTAN:/\mbox{macros}/""\mbox{latex}/""\mbox{contrib}/""\mbox{koma-script}/. 
% ``CTAN:'' is a shortcut for every ``tex-archive'' directory at every
% CTAN-server or CTAN-mirror. See \mbox{http://www.ctan.org} for a list of all
% those servers and mirrors.
%
% \section{Package ``tocbasic''}
%
% See the section~\ref{sec:implementation}. I think, class and package authors
% should be able to read the implemenation documentation.
%
%
% \StopEventually{\PrintIndex\PrintChanges}
%
% \section{Implementation}
% \label{sec:implementation}
%
% All macros with prefix \texttt{tb@} or \texttt{@} are internal macros and
% should not be used by package and class authors. Macros with prefix
% \texttt{tocfile@} are internal macros, that may be used by class and
% packages authors. Macros without \texttt{@} are interface macros and may be
% used by class and package authors and users.
%
% \iffalse
%<*package>
% \fi
%
% \subsection{Options}
% \label{sec:options}
%
% There are no options because the package should be used by class and package
% authors not by users. So the package will be loaded using
% \cs{RequiresPackage}. Using different options by different packages would
% result in an option clash.
%
%
% \subsection{Having a List of All Tocs}
% \label{sec:listoftocs}
%
% If we have a list of all toc-files we may do commands for all
% toc-files. Somethimes it may be usefull to known the package, that created
% the toc-file, so this information will be stored additionally.
%
% \begin{macro}{\tb@listoftocs}
%   This is the list of toc-files. The list will be:
%   \begin{quote}
%   \cs{do}\marg{extension}\marg{class or
%   package}\cs{do}\marg{extension}\marg{class or package}\dots
%   \end{quote}
%   With this, adding and processing the list will be very fast but removing
%   an element will be very slow.
%
%   The initial state of the list will be \emph{empty}.
%    \begin{macrocode}
\newcommand*{\tb@listoftocs}{}
%    \end{macrocode}
% \end{macro}
%
% \begin{macro}{\ifattoclist}
%   This command tests, if an extension is already at the list of
%   toc-files. The extension has to be the first argument. The second argument
%   will be done, if the extension is already at the list of toc-files. The
%   third argument will be done, of the extension is at the list of toc-files
%   not yet.
%    \begin{macrocode}
\newcommand{\ifattoclist}[1]{%
  \begingroup
    \def\do##1##2{%
      \edef\reserved@a{##1}%
      \ifx\reserved@a\reserved@b\@tempswatrue\fi
    }%
    \edef\reserved@b{#1}\@tempswafalse\tb@listoftocs
    \if@tempswa\aftergroup\@firstoftwo\else\aftergroup\@secondoftwo\fi
  \endgroup
}
%    \end{macrocode}
% \end{macro}
%
% \begin{macro}{\addtotoclist}
%   This command adds an extension to the list of toc-files. The first,
%   optional argument is the class or package name with the corresponding
%   extension of class or package files. If this argument was omitted
%   \textsf{tocbasic} tries to get it automatically. This should be
%   successfull while loading a class or package but not while processing any
%   command of a class or package after loading the class or package. The
%   second, mandatory argument is the extension of the toc-file. NOTE: An
%   empty first argument is not the same like omitting the first argument!
%    \begin{macrocode}
\newcommand*{addtotoclist}{%
  \@ifnextchar [%]
    \@@addtotoclist\@addtotoclist
}
\newcommand*{\@addtotoclist}{%
  \ifx\@currname\@empty
    \def\reserved@a{\@@addtotoclist[]}%
  \else
    \edef\reserved@a{\noexpand\@@addtotoclist[\@currname.\@currext]}%
  \fi
  \reserved@a
}
\newcommand*{\@@addtotoclist}[2][]{%
  \ifattoclist{#2}{%
    \PackageError{tocbasic}{%
      file extension `#2' cannot be used twice
    }{%
      File extension `#2' is already used by a toc-file, while
      \ifx\relax#1\relax someone\else #2\fi\MessageBreak
      tried to use it again for a toc-file.\MessageBreak
      This may be either an incompatibility of packages, an error at a
      package,\MessageBreak
      or a mistake by the user.\MessageBreak
    }%
  }{%
    \g@addto@macro\tb@listoftocs{\do{#2}{#1}}%
  }%
}
%    \end{macrocode}
% \end{macro}
%
% \begin{macro}{removefromtoclist}
%   This command will remove an extension from the list of toc-files. If the
%   first, optional argument is given, the extension will only be removed, if
%   it was added by the given owner. NOTE: An empty first argument is not the
%   same like omitting the first argument!
%    \begin{macrocode}
\newcommand*{\removefromtoclist}{%
  \@ifnextchar [%]
    \@removefromtoclist\@@removefromtoclist
}
\newcommand*{\@removefromtoclist}[2][]{%
  \begingroup
    \let\tb@oldlist\tb@listoftocs
    \def\do##1##2{%
      \edef\reserved@a{##1}%
      \ifx\reserved@a\reserved@b
        \begingroup
          \edef\reserved@a{##2}%
          \edef\reserved@b{#1}%
          \ifx\reserved@a\reserved@b\else
            \g@addto@macro\tb@listoftocs{\do{##1}{##2}}%
          \fi
        \endgroup
      \else
        \g@addto@macro\tb@listoftocs{\do{##1}{##2}}%
      \fi
    }%
    \edef\reserved@b{#2}\let\tb@listoftocs\@empty
    \tb@oldlist
  \endgroup
}
\newcommand*{\@@removefromtoclist}[1]{%
  \begingroup
    \let\tb@oldlist\tb@listoftocs
    \def\do##1##2{%
      \edef\reserved@a{##1}%
      \ifx\reserved@a\reserved@b\else
        \g@addto@macro\tb@listoftocs{\do{##1}{##2}}%
      \fi
    }%
    \edef\reserved@b{#1}\let\tb@listoftocs\@empty
    \tb@oldlist
  \endgroup
}
%    \end{macrocode}
% \end{macro}
%
% \begin{macro}{\doforeachtocfile}
%   This command does the second, mandatory argument for each toc-file at the
%   list of toc-files. If the first, optional argument was given this will be
%   done only for the toc-files of that owner. NOTE: An empty first argument
%   is not the same like omitting the first argument!
%
%   While processing the second argument \cs{\@currext} is the extension of
%   the toc-file. The second argument will be processed with increased group
%   level!
%
%   See \cs{addtoeachtocfile} for an example of usage of
%   \cs{doforeachtocfile}.
%    \begin{macrocode}
\newcommand{\doforeachtocfile}{%
  \@ifnextchar [%]
    \@doforeachtocfile\@@doforeachtocfile
}
\newcommand{\@doforeachtocfile}[2][]{%
  \begingroup
    \def\do##1##2{%
      \edef\reserved@a{#1}\edef\reserved@b{##2}\ifx\reserved@a\reserved@b
        \edef\@currext{##1}#2%
      \fi
    }%
    \tb@listoftocs
  \endgroup
}
\newcommand{\@@doforeachtocfile}[1]{%
  \begingroup
    \def\do##1##2{%
      \edef\@currext{##1}#1%
    }%
    \tb@listoftocs
  \endgroup
}
%    \end{macrocode}
% \end{macro}
%
% \begin{macro}{\addtoeachtocfile}
%   This command calls \cs{addtocontents} with the section, mandatory
%   argument for each toc-file at the list of toc-files. If the first,
%   optional argument was given this will be done only for the toc-files of
%   that owener. NOTE: An empty first argument is not the same like omitting
%   the first argument! And don't forget to protect the commands, that should
%   not be expanded, at the mandatory argument.
%    \begin{macrocode}
\newcommand{\addtoeachtocfile}{%
  \@ifnextchar [%]
    \@addtoeachtocfile\@@addtoeachtocfile
}
\newcommand{\@addtoeachtocfile}[2][]{%
  \doforeachtocfile[{#1}]{\addtocontents{\@currext}{#2}}%
}
\newcommand{\@@addtoeachtocfile}[1]{%
  \doforeachfocfile{\addtocontents{\@currext}{#1}}%
}
%    \end{macrocode}
% \end{macro}
%
% \begin{macro}{\addcontentslinetoeachtocfile}
%   Something like a combination of \cs{addtoeachtocfile} and
%   \cs{addcontentsline}.
%    \begin{macrocode}
\newcommand{\addcontentslinetoeachtocfile}{%
  \@ifnextchar [%]
    \@addcontentslinetoeachtocfile\@@addcontentslinetoeachtocfile
}
\newcommand{\@addcontentslinetoeachtocfile}[3][]{%
  \doforeachtocfile[{#1}]{\addcontentsline{\@currext}{#2}{#3}}%
}
\newcommand{\@addcontentslinetoeachtocfile}[2]{%
  \doforeachtocfile{\addcontentsline{\@currext}{#1}{#2}}%
}
%    \end{macrocode}
% \end{macro}
%
%
% \subsection{Show List of Toc-File}
% \label{sec:showlistoftocfile}
%
% If you have a toc-file you want a list-of-command for this toc-file,
% too. Here are basics and high level commands for this.
%
% \begin{macro}{\tocfile@starttoc}
%   Some basics are done like setting up \cs{parskip}, \cs{parindent} and
%   \cs{parfillskip}, a general hook will be called, an individual hook will
%   be called, the toc will be started, an individual hook will be called, an
%   general hook wil be called, that's it.
%    \begin{macrocode}
\newcommand*{\tocfile@starttoc}[1]{%
  \begingroup
    \setlength{\parskip}{\z@}%
    \setlength{\parindent}{\z@}%
    \setlength{\parfillskip}{\z@\@plus 1fil}%
    \tocfile@@before@hook
    \csname tb@#1@before@hook\endcsname
    \@starttoc{#1}%
    \csname tb@#1@after@hook\endcsname
    \tocfile@@after@hook
  \endgroup
}
%    \end{macrocode}
% \begin{macro}{\tocfile@@before@hook}
% \begin{macro}{\tocfile@@after@hook}
%   These are the general hooks. They may be used by classes and packages for
%   commands, that should be used for all lists of not only the own lists of,
%   e.g., \KOMAScript{} may use it to handle option \texttt{tocleft}.
%    \begin{macrocode}
\newcommand*{\tocfile@@before@hook}{}
\newcommand*{\tocfile@@after@hook}{}
%    \end{macrocode}
% \end{macro}
% \end{macro}
% \begin{macro}{\BeforeStartingTOC}
% \begin{macro}{\AfterStartingTOC}
%   These are the commands to add code to the general or individual hooks. If
%   the first, optional argument was given, the second, mandatory argument
%   will be added to the individual hook, otherwise the general hook will be
%   extended.
\newcommand{\BeforeStartingTOC}[2][]{%
  \expandafter\g@addto@macro\csname tb@#1@before@hook\endcsname{#2}%
}
\newcommand{\AfterStartingTOC}[2][]{%
  \expandafter\g@addto@macro\csname tb@#1@before@hook\endcsname{#2}%
}
% \end{macro}
% \end{macro}
% \end{macro}
%
% \begin{macro}{\listoftoc}
% \begin{macro}{\listoftoc*}
%   Command to handle the hole list of something. There are additional hooks
%   for this. The first optional argument is the title for this list. If the
%   optional argument was omitted \cs{listof\#2name} will be used. The star
%   version does not set up a heading or switch the column number!
%    \begin{macrocode}
\newcommand*{\listoftoc}{%
  \@ifstar \tocfile@starttoc\@listoftoc
}
\newcommand*{\@listoftoc}[2][\list@fname]{%
  \begingroup
    \@ifundefined{listof#2name}{%
      \PackageWarning{tocbasic}{%
        You should either define \expandafter\string\csname
        listof#2name\endcsname\MessageBreak
        or use the optional argument of \string\listoftoc\space\MessageBreak
        to set the term to be used for the\MessageBreak
        heading of list of #2}%
      \def\list@fname{\listofname~#2}%
    }{%
      \expandafter\let\expandafter\list@fname\csname listof#2name\endcsname
    }%
    \if@twocolumn
      \aftergroup\twocolumne\onecolumn
    \fi
    \tocfile@listhead{\list@fname}%
    \tocfile@starttoc{#2}%
  \endgroup
}
%    \end{macrocode}
% \end{macro}
% \end{macro}
%
% \begin{macro}{\listofname}
%   While this is only an emergancy command, we don't support languages.
%    \begin{macrocode}
\newcommand*{\listofname}{List of}
%    \end{macrocode}
% \end{macro}
%
% \begin{macro}{\listofeachtoc}
%   Another example of using \cs{doforeachtoc}.
%    \begin{macrocode}
\newcommand*{\listofeachtoc}{%
  \@ifnextchar [%]
    \@listofeachtoc\@@listofeachtoc
}
\newcommand{\@listofeachtoc}[1][]{%
  \doforeachtocfile[{#1}]{\listoftoc{\@currext}}%
}
\newcommand{\@@listofeachtoc}[1]{%
  \doforeachfocfile{\listoftoc{\@currext}}%
}
%    \end{macrocode}
% \end{macro}
%
% \begin{macro}{\tocfile@listhead}
%   Setting the headings of a list of something. The heading is the only
%   argument.
%    \begin{macrocode}
\newcommand*{\tocfile@listhead}[1]{%
  \@ifundefined{tocfile@listhead@#1}{%
    \begingroup\expandafter\expandafter\expandafter\endgroup
    \expandafter\ifx\csname chapter\endcsname\relax
      \expandafter\def\csname tocfile@listhead@#1\endcsname##1{%
        \@ifundefined{tocfile@#1@feature@leveldown}{%
          \@ifundefined{tocfile@#1@numbered}{%
            \section*{##1}%
            \@ifundefined{tocfile@#1@feature@totoc}{}{%
              \addcontentsline{toc}{section}{##1}%
            }%
          }{%
            \section[##1]{##1\@mkboth{##1}{##1}}%
          }%
          \@mkboth{\MakeMarkcase{##1}}{\MakeMarkcase{##2}}%
        }{%
          \@ifundefined{tocfile@#1@feature@numbered}{%
            \subsection*{##1}%
            \@ifundefined{tocfile@#1@feature@totoc}{}{%
              \addcontentsline{toc}{subsection}{##1}%
            }%
          }{%
            \subsection[##1]{##1}%
          }%
          \ifx\@mkboth\@gobbletwo\else\markright{\MakeMarkcase{##1}}\fi
        }%
      }%
    \else
      \expandafter\def\csname tocfile@listhead@#1\endcsname##1{%
        \@ifundefined{tocfile@#1@feature@leveldown}{%
          \@ifundefined{tocfile@#1@feature@numbered}{%
            \chapter*{##1}%
            \@ifundefined{tocfile@#1@feature@totoc}{}{%
              \addcontentsline{toc}{chapter}{##1}%
            }%
          }{%
            \chapter[##1]{##1\@mkboth{##1}{##1}}%
          }%
          \@mkboth{\MakeMarkcase{##1}}{\MakeMarkcase{##1}}%
        }{%
          \@ifundefined{tocfile@#1@feature@numbered}{%
            \section*{##1}%
            \@ifundefined{tocfile@#1@feature@totoc}{}{%
              \addcontentsline{toc}{section}{##1}%
            }%
          }{%
            \section{##1}%
          }%
          \ifx\@mkboth\@gobbletwo\else\markright{\MakeMarkcase{##1}}\fi
        }%
      }%
    \fi
  }%
  \tb@@beforehead@hook
  \csname tb@#1@beforehead@hook\endcsname
  \csname tocfile@listhead@#1\endcsname{#1}%
  \csname tb@#1@afterhead@hook\endcsname
  \tb@@afterhead@hook
}
%    \end{macrocode}
% \end{macro}
%
% \begin{macro}{\MakeMarkcase}
%   Use upper-case or not?
%    \begin{macrocode}
\AtBeginDocument{%
  \@ifundefined{MakeMarkcase}{%
    \begingroup\expandafter\expandafter\expandafter\endgroup
    \expandafter\ifx\csname KOMAClassName\endcsname\relax
      \let\MakeMarkcase\MakeUppercase
    \else
      \let\MakeMarkcase\@firstofone
    \fi
  }{}%
}
%    \end{macrocode}
% \end{macro}
%
% \begin{macro}{\deftocheading}
%   Define a toc headings command with one argument (the title).
%    \begin{macrocode}
\newcommand*{\deftocheading}[1]{%
  \@namedef{tocfile@listhead@#1}##1}
%    \end{macrocode}
% \end{macro}
%
% \begin{macro}{\setuptoc}
%   Known features are:
%   \begin{description}
%   \item[\texttt{totoc}] writes the title of the list of to the table of
%     contents
%   \item[\texttt{numbered}] uses a numbered headings for the list of
%   \item[\texttt{leveldown}] uses not the top level heading (e.g.,
%     \cs{chapter} with book) but the first sub level (e.g., \cs{section} with
%     book).
%   \end{description}
%   Other features may be package dependent. You may test the feature using:
%   \begin{quote}
%   \cs{@ifundefined}\texttt{\{tocfile@\meta{toc}@feature@\meta{feature}\}}\\
%   \phantom{\cs{@ifundefined}}\marg{do if feature not set}\\
%   \phantom{\cs{@ifundefined}}\marg{do if feature set}
%   \end{quote}
%   See \cs{tocfile@listhead} for an example of this.
%    \begin{macrocode}
\newcommand*{\setuptoc}[2]{%
  \@for\@tempa:=#2\do{%
    \expandafter\tb@@sp@def\expandafter\@tempa\expandafter{\@tempa}%
    \@namedef{tocfile@#1@feature@\@tempa}{}%
  }%
}
\def\@tempa#1{%
  \def\tb@@sp@def##1##2{%
    \futurelet\tb@sp@tempa\tb@@sp@d##2\@nil\@nil#1\@nil\relax##1}%
  \def\tb@@sp@d{%
    \ifx\tb@sp@tempa\@sptoken
      \expandafter\tb@@sp@b
    \else
      \expandafter\tb@@sp@b\expandafter#1%
    \fi}%
  \def\tb@@sp@b#1##1 \@nil{\tb@@sp@c##1}%
}
\@tempa{ }
\def\tb@@sp@c#1\@nil#2\relax#3{\@temptokena{#1}\edef#3{\the\@temptokena}}
%    \end{macrocode}
% \end{macro}
%
% \iffalse
%</package>
% \fi
%
% \Finale
%
\endinput
%
% end of file `tocbasic.dtx'
%%% Local Variables:
%%% mode: doctex
%%% coding: iso-latin-1
%%% TeX-master: t
%%% End:
